% Capitolo 24: Best Practices

\chapter{Best Practices per Sviluppo con SLIPS}
\label{cap:best_practices}

\section{Progettazione di Regole Efficienti}

\subsection{Principio della Specificità}

\begin{infobox}[Regola d'Oro]
Pattern più specifici riducono il conflict set e migliorano performance.
\end{infobox}

\textbf{Cattivo esempio}:
\begin{lstlisting}[language=CLIPS]
(defrule troppo-generica
  (persona)  ; Matcha TUTTE le persone!
  =>
  ...)
\end{lstlisting}

\textbf{Buon esempio}:
\begin{lstlisting}[language=CLIPS]
(defrule specifica
  (persona (eta ?e&:(>= ?e 18)) (citta "Roma"))  ; Molto selettiva
  =>
  ...)
\end{lstlisting}

\subsection{Ordinamento Pattern}

\textbf{Euristica}: Pattern più selettivi prima.

\begin{lstlisting}[language=CLIPS]
; BAD: pattern generico prima
(defrule bad-order
  (persona (nome ?n))           ; 10000 match
  (vip (nome ?n))               ; 10 match
  =>
  ...)

; GOOD: pattern selettivo prima
(defrule good-order
  (vip (nome ?n))               ; 10 match
  (persona (nome ?n))           ; Ridotto a 10 check
  =>
  ...)
\end{lstlisting}

\textbf{Risparmio}: da $10000 \times 10 = 100k$ a $10 \times 1 = 10$ confronti!

\subsection{Uso delle Costanti}

Costanti nei pattern attivano ottimizzazioni RETE:

\begin{lstlisting}[language=CLIPS]
; Senza costanti: alpha node generale
(defrule generic
  (item (type ?t) (value ?v))
  =>
  ...)

; Con costanti: alpha node specializzato
(defrule specific
  (item (type "premium") (value ?v&:(> ?v 1000)))
  =>
  ...)
\end{lstlisting}

Alpha node con costanti può usare hash index per match $O(1)$.

\section{Gestione della Working Memory}

\subsection{Minimizzare Fatti Ridondanti}

\textbf{Problema}: WM cresce senza controllo.

\begin{lstlisting}[language=CLIPS]
; BAD: accumula fatti
(defrule process-item
  (item ?i)
  =>
  (assert (processed ?i))
  (assert (timestamp (now)))   ; Nuovo fatto ogni volta!
  ...)
\end{lstlisting}

\textbf{Soluzione}: Retract fatti temporanei.

\begin{lstlisting}[language=CLIPS]
; GOOD: pulizia esplicita
(defrule process-item
  ?f <- (item ?i)
  =>
  (retract ?f)                  ; Rimuovi item processato
  (assert (processed ?i))
  ...)
\end{lstlisting}

\subsection{Fact Temporal Validity}

Per fatti temporanei, usa pattern:

\begin{lstlisting}[language=CLIPS]
(deftemplate event
  (slot type)
  (slot data)
  (slot timestamp)
  (slot ttl (default 100)))  ; Time-to-live

(defrule expire-events
  ?e <- (event (timestamp ?ts) (ttl ?ttl))
  (test (> (- (now) ?ts) ?ttl))
  =>
  (retract ?e))
\end{lstlisting}

\section{Modularizzazione e Riuso}

\subsection{Design Pattern: Utility Modules}

\begin{lstlisting}[language=CLIPS]
(defmodule STRING-UTILS
  (export deffunction starts-with)
  (export deffunction ends-with)
  (export deffunction contains))

(deffunction starts-with (?str ?prefix)
  (eq (sub-string 1 (str-length ?prefix) ?str) ?prefix))

(deffunction ends-with (?str ?suffix)
  (bind ?len (str-length ?str))
  (bind ?slen (str-length ?suffix))
  (eq (sub-string (- ?len ?slen -1) ?len ?str) ?suffix))

;; Altri moduli importano
(defmodule VALIDATION
  (import STRING-UTILS deffunction starts-with))
\end{lstlisting}

\subsection{Design Pattern: Pipeline}

Moduli organizzati in pipeline:

\begin{lstlisting}[language=CLIPS]
(defmodule INPUT-PROCESSING
  (export deftemplate cleaned-data))

(defmodule VALIDATION
  (import INPUT-PROCESSING deftemplate cleaned-data)
  (export deftemplate validated-data))

(defmodule ENRICHMENT
  (import VALIDATION deftemplate validated-data)
  (export deftemplate enriched-data))

(defmodule OUTPUT
  (import ENRICHMENT deftemplate enriched-data))

; Main orchestra il flusso
(defmodule MAIN)
(defrule start
  =>
  (focus INPUT-PROCESSING VALIDATION ENRICHMENT OUTPUT))
\end{lstlisting}

\section{Performance Optimization}

\subsection{Profiling-Guided Optimization}

\begin{enumerate}
\item \textbf{Measure}: Usa Instruments per identificare bottleneck
\item \textbf{Analyze}: Determina causa (algoritmo? struttura dati?)
\item \textbf{Optimize}: Intervento mirato
\item \textbf{Verify}: Conferma miglioramento senza breaking changes
\end{enumerate}

\subsection{Common Hotspots}

\begin{table}[h]
\centering
\begin{tabular}{@{}lll@{}}
\toprule
\textbf{Hotspot} & \textbf{Causa} & \textbf{Fix} \\
\midrule
Join operations & Pattern generici & Aggiungi costanti/constraints \\
Alpha matching & Linear scan fatti & Hash indexing (già implementato) \\
Beta memory growth & Join produttivo & Limita prodotto cartesiano \\
Agenda operations & Troppi tie & Usa salience \\
\bottomrule
\end{tabular}
\caption{Hotspot comuni e soluzioni}
\label{tab:hotspots}
\end{table}

\subsection{Salience Strategy}

Usa salience per controllo esplicito:

\begin{lstlisting}[language=CLIPS]
; Alta priorità: cleanup urgente
(defrule cleanup-critical-error
  (declare (salience 1000))
  (error (severity critical))
  =>
  (halt))

; Priorità normale: processing
(defrule process-data
  (declare (salience 0))
  (data ?d)
  =>
  ...)

; Bassa priorità: logging
(defrule log-completion
  (declare (salience -1000))
  (all-done)
  =>
  (printout t "Completato" crlf))
\end{lstlisting}

\section{Memory Management}

\subsection{Evitare Memory Leaks}

In Swift con ARC, leaks tipicamente da:

\begin{enumerate}
\item \textbf{Reference Cycles}

\begin{lstlisting}[language=Swift]
// BAD: ciclo forte
class Node {
    var parent: Node?  // Strong reference
    var children: [Node] = []  // Strong references
}

// GOOD: weak parent
class Node {
    weak var parent: Node?  // Weak!
    var children: [Node] = []
}
\end{lstlisting}

\item \textbf{Closures Capturing Self}

\begin{lstlisting}[language=Swift]
// BAD: self captured strongly
class Handler {
    var callback: (() -> Void)?
    
    func setup() {
        callback = {
            self.doSomething()  // Strong capture!
        }
    }
}

// GOOD: weak self
func setup() {
    callback = { [weak self] in
        self?.doSomething()
    }
}
\end{lstlisting}
\end{enumerate}

\subsection{Monitoring con Instruments}

Usa \textbf{Leaks} instrument:

\begin{enumerate}
\item Run con Instruments
\item Esegui operazioni (assert/retract ciclo)
\item Verifica che memoria rimane costante
\item Indaga picchi o crescita continua
\end{enumerate}

\section{Code Style e Convenzioni}

\subsection{Swift Style Guide}

Seguiamo convenzioni Swift standard:

\begin{lstlisting}[language=Swift]
// Naming: camelCase per funzioni/variabili
func evaluateExpression(_ env: inout Environment, _ node: ExpressionNode) -> Value

// Naming: PascalCase per tipi
class AlphaNodeClass: ReteNode

// Indentazione: 4 spazi
func example() {
    if condition {
        doSomething()
    }
}

// Line length: < 120 caratteri (soft limit)
// Function length: < 50 linee (soft limit)

// Access control esplicito
public func publicAPI()
internal func internalHelper()
private func implementationDetail()
\end{lstlisting}

\subsection{Documentazione Inline}

\begin{lstlisting}[language=Swift]
/// Propaga assert di un fatto attraverso la rete RETE
///
/// Questa funzione implementa la logica di NetworkAssert da drive.c (CLIPS).
/// Quando un fatto viene asserito, viene propagato attraverso alpha nodes
/// che matchano il pattern, generando token che fluiscono attraverso
/// join nodes fino ai production nodes.
///
/// - Parameters:
///   - fact: Il fatto da propagare
///   - env: Environment (modificato in-place)
///
/// - Complexity: O(α × j × β) dove:
///   - α = numero alpha nodes matchanti
///   - j = numero join per alpha
///   - β = dimensione beta memory
///
/// - SeeAlso: NetworkRetract per operazione inversa
/// - Note: Riferimento C: drive.c linee 450-520
public static func propagateAssert(
    fact: Environment.FactRec,
    env: inout Environment
) {
    // Implementazione...
}
\end{lstlisting}

\section{Error Handling}

\subsection{Swift Error Model}

\begin{lstlisting}[language=Swift]
enum EvaluationError: Error {
    case typeError(String)
    case unboundVariable(String)
    case divisionByZero
    case templateNotFound(String)
    case constraintViolation(String)
}

func divide(_ a: Value, _ b: Value) throws -> Value {
    let x = try asDouble(a)
    let y = try asDouble(b)
    
    guard y != 0 else {
        throw EvaluationError.divisionByZero
    }
    
    return .float(x / y)
}
\end{lstlisting}

\subsection{Graceful Degradation}

Per funzioni built-in, preferire:

\begin{lstlisting}[language=Swift]
// Invece di crash:
func builtin_sqrt(_ env: inout Environment, _ args: [Value]) throws -> Value {
    guard args.count == 1 else {
        print("[ERROR] sqrt requires exactly 1 argument")
        return .none  // Graceful failure
    }
    
    guard case .float(let x) = args[0], x >= 0 else {
        print("[ERROR] sqrt requires non-negative number")
        return .none
    }
    
    return .float(sqrt(x))
}
\end{lstlisting}

\section{Contribuire a SLIPS}

\subsection{Workflow per Contributor}

\begin{enumerate}
\item \textbf{Fork} repository
\item \textbf{Clone} localmente
\item \textbf{Branch} per feature: \texttt{git checkout -b feature/my-feature}
\item \textbf{Implementa} con test
\item \textbf{Commit} con messaggi descrittivi
\item \textbf{Push} e apri Pull Request
\item \textbf{Code Review}
\item \textbf{Merge} dopo approvazione
\end{enumerate}

\subsection{Commit Message Convention}

Seguiamo Conventional Commits:

\begin{verbatim}
<type>(<scope>): <subject>

<body>

<footer>
\end{verbatim}

\textbf{Types}:
\begin{itemize}
\item \texttt{feat}: Nuova funzionalità
\item \texttt{fix}: Bug fix
\item \texttt{docs}: Documentazione
\item \texttt{test}: Aggiunta test
\item \texttt{refactor}: Refactoring
\item \texttt{perf}: Performance improvement
\end{itemize}

\textbf{Esempio}:
\begin{verbatim}
feat(modules): implementa sistema completo di moduli CLIPS

Fase 3 completata (95%):

- Defmodule parsing con import/export
- Focus stack LIFO
- 5 comandi builtin: focus, get/set-current-module, list-defmodules
- 22 test (100% pass)

Riferimenti CLIPS C:
- moduldef.h/c → Modules.swift
- modulbsc.c → functions.swift

Closes #123
\end{verbatim}

\section{Code Review Checklist}

Prima di PR, verifica:

\begin{itemize}
\item[$\square$] Build passa senza warning
\item[$\square$] Tutti i test passano (inclusi i nuovi)
\item[$\square$] Coverage non diminuisce
\item[$\square$] Documentazione aggiornata
\item[$\square$] Riferimenti a file C originali presenti
\item[$\square$] No force unwrap in codice pubblico
\item[$\square$] Error handling appropriato
\item[$\square$] Performance accettabili
\item[$\square$] Commit messages descrittivi
\item[$\square$] AGENTS.md rispettato
\end{itemize}

\section{Antipattern da Evitare}

\subsection{Force Unwrap in Production Code}

\begin{lstlisting}[language=Swift]
// BAD: crash se nil
let module = env.findDefmodule(name: moduleName)!

// GOOD: gestione esplicita
guard let module = env.findDefmodule(name: moduleName) else {
    print("[ERROR] Module \(moduleName) not found")
    return .boolean(false)
}
\end{lstlisting}

\subsection{Premature Optimization}

\begin{warningbox}[Knuth's Law]
"Premature optimization is the root of all evil" --- Donald Knuth
\end{warningbox}

Workflow corretto:
\begin{enumerate}
\item Implementa correttamente (equivalenza con CLIPS)
\item Misura performance (profiler)
\item Identifica bottleneck reali
\item Ottimizza dove necessario
\item Verifica miglioramento
\end{enumerate}

\subsection{God Objects}

\texttt{Environment} è intenzionalmente God Object per compatibilità C, ma:

\begin{itemize}
\item Non aggiungere responsabilità non-essenziali
\item Usa extension per organizzare logicamente
\item Considera refactor se cresce > 150 campi
\end{itemize}

\section{Sicurezza e Robustezza}

\subsection{Input Validation}

\begin{lstlisting}[language=Swift]
func builtin_assert(_ env: inout Environment, _ args: [Value]) throws -> Value {
    // 1. Valida numero argomenti
    guard args.count >= 1 else {
        print("[ERROR] assert requires at least 1 argument")
        return .int(-1)
    }
    
    // 2. Valida tipo argomenti
    guard case .symbol(let templateName) = args[0] else {
        print("[ERROR] assert first argument must be template name")
        return .int(-1)
    }
    
    // 3. Valida template exists
    guard env.templates[templateName] != nil else {
        print("[ERROR] Template \(templateName) not defined")
        return .int(-1)
    }
    
    // 4. Valida constraints
    // ...
    
    // 5. Esegui operazione
    let factID = createFact(template: templateName, slots: slots, env: &env)
    return .int(Int64(factID))
}
\end{lstlisting}

\subsection{Defensive Programming}

\begin{lstlisting}[language=Swift]
func join(leftToken: BetaToken, rightFact: FactRec) -> BetaToken? {
    // Precondizioni
    precondition(!leftToken.usedFacts.isEmpty, "Token deve contenere almeno 1 fatto")
    precondition(rightFact.id > 0, "Fact ID deve essere positivo")
    
    // Verifica no duplicati
    guard !leftToken.usedFacts.contains(rightFact.id) else {
        return nil  // Fact già usato in questo token
    }
    
    // Verifica consistenza join keys
    for key in joinKeys {
        guard let leftValue = leftToken.bindings[key],
              let rightValue = rightFact.slots[key] else {
            continue
        }
        
        guard leftValue == rightValue else {
            return nil  // Inconsistente
        }
    }
    
    // Crea nuovo token
    // ... postcondizioni
    return newToken
}
\end{lstlisting}

\section{Documentazione}

\subsection{Inline Documentation}

Ogni file public deve avere:

\begin{lstlisting}[language=Swift]
// File header
// SLIPS - Swift Language Implementation of Production Systems
// Copyright (c) 2025 SLIPS Contributors
// Licensed under the MIT License - see LICENSE file for details

import Foundation

// MARK: - Module Name
// Traduzione fedele da <file.c>, <file.h> (CLIPS 6.4.2)
// Riferimenti C:
// - FunctionName (file.c linee 123-145)
// - StructName (file.h linee 67-89)

/// Brief description of module
///
/// Longer description explaining purpose, usage, and relationship
/// with CLIPS C implementation.
///
/// Example:
/// ```swift
/// var env = Environment()
/// let result = SomeFunction(&env, param)
/// ```
///
/// - SeeAlso: RelatedModule.swift
/// - Note: Port of file.c from CLIPS 6.4.2
\end{lstlisting}

\subsection{README e Guide}

Ogni modulo significativo ha README:

```
Sources/SLIPS/Rete/README.md

# RETE Engine

Implementazione algoritmo RETE per pattern matching efficiente.

## Files

- AlphaNetwork.swift: Alpha nodes (pattern filtering)
- BetaEngine.swift: Beta network (join operations)
- DriveEngine.swift: Propagation (assert/retract)
- Nodes.swift: Explicit node classes
- NetworkBuilder.swift: Network construction

## References

CLIPS C files:
- drive.c: Network propagation
- reteutil.c: RETE utilities
- pattern.c: Pattern nodes
- network.c: Network structures

## Usage

See ReteTests.swift for examples.
```

\section{Conclusioni del Capitolo}

Best practices fondamentali:

\begin{enumerate}
\item \textbf{Testing}: TDD, alta coverage, proprietà verificate
\item \textbf{Performance}: Profiling-guided, pattern specifici
\item \textbf{Safety}: No force unwrap, validation input, error handling
\item \textbf{Style}: Convenzioni Swift, documentazione inline
\item \textbf{Workflow}: Git flow, code review, CI/CD
\end{enumerate}

\begin{successbox}[Quality Standards]
Aderendo a queste best practices, SLIPS mantiene qualità production-ready con:
\begin{itemize}
\item 97.8\% test pass rate
\item Zero unsafe code pubblico
\item Build sempre clean
\item Documentazione completa
\end{itemize}
\end{successbox}

