% Capitolo 6: Introduzione all'Algoritmo RETE

\chapter{L'Algoritmo RETE: Introduzione}
\label{cap:rete_intro}

\section{Il Problema del Pattern Matching Efficiente}

\subsection{Analisi del Problema}

Consideriamo un sistema a produzione con:
\begin{itemize}
\item $n$ regole nella production memory
\item $m$ fatti nella working memory
\item $k$ condizioni medie per regola
\end{itemize}

\subsubsection{Approccio Naïve}

L'approccio naïve ricalcola da zero ad ogni ciclo:

\begin{algorithm}[H]
\caption{Match Naïve}
\begin{algorithmic}[1]
\Function{MatchNaive}{$PM, WM$}
    \State $CS \gets \emptyset$
    \For{each $r \in PM$}
        \For{each combination $\langle w_1, \ldots, w_k \rangle$ of $k$ facts from $WM$}
            \If{$\langle w_1, \ldots, w_k \rangle$ matches $r.LHS$}
                \State $\theta \gets \text{extract\_bindings}(r.LHS, \langle w_1, \ldots, w_k \rangle)$
                \State $CS \gets CS \cup \{(r, \theta)\}$
            \EndIf
        \EndFor
    \EndFor
    \State \Return $CS$
\EndFunction
\end{algorithmic}
\end{algorithm}

\textbf{Complessità}:
\begin{equation}
T_{\text{naïve}} = O\left(n \cdot \binom{m}{k}\right) = O(n \cdot m^k)
\end{equation}

\begin{esempio}[Costo Computazionale]
Per un sistema realistico:
\begin{align*}
n &= 1000 \text{ regole}\\
m &= 10000 \text{ fatti}\\
k &= 3 \text{ condizioni/regola}
\end{align*}

Otteniamo:
\begin{equation}
T = 1000 \cdot 10000^3 = 10^{15} \text{ confronti}
\end{equation}

A 1 GHz (1 confronto/ns), servirebbero:
\begin{equation}
\frac{10^{15}}{10^9} = 10^6 \text{ secondi} \approx 11.6 \text{ giorni!}
\end{equation}
\end{esempio}

\subsection{L'Intuizione di Forgy}

Charles Forgy osservò due invarianti critici:

\begin{osservazione}[Continuità Temporale]
Tra un ciclo recognize-act e il successivo:
\begin{itemize}
\item La maggior parte dei fatti rimane invariata
\item Solo pochi fatti vengono asseriti o ritratti
\item Molti match parziali rimangono validi
\end{itemize}

Formalmente, se $WM_t$ è la working memory al ciclo $t$:
\begin{equation}
\frac{|WM_{t+1} \Delta WM_t|}{|WM_t|} \ll 1
\end{equation}

dove $\Delta$ denota differenza simmetrica.
\end{osservazione}

\begin{osservazione}[Similarità Strutturale]
Molte regole condividono pattern comuni:

\begin{lstlisting}[language=CLIPS]
(defrule r1
  (persona (nome ?n) (eta ?e))
  ...
  =>
  ...)

(defrule r2
  (persona (nome ?n) (eta ?e))
  ...
  =>
  ...)
\end{lstlisting}

Il pattern \texttt{(persona (nome ?n) (eta ?e))} è condiviso.
\end{osservazione}

\subsection{Idea Centrale di RETE}

L'algoritmo RETE sfrutta queste osservazioni costruendo una \textit{rete di nodi} che:

\begin{enumerate}
\item \textbf{Condivide} risultati di match parziali tra regole
\item \textbf{Memorizza} risultati intermedi per riuso
\item \textbf{Propaga} solo cambiamenti incrementali (delta)
\end{enumerate}

\begin{definizione}[Rete RETE]
Una rete RETE è un grafo diretto aciclico $G = (V, E)$ dove:
\begin{itemize}
\item $V$ è l'insieme dei nodi (alpha, beta, join, production)
\item $E$ è l'insieme degli archi (collegamenti parent-child)
\item Ogni nodo mantiene \textit{memoria locale} di match parziali
\item La propagazione è \textit{incrementale}: solo delta vengono processati
\end{itemize}
\end{definizione}

\section{Architettura della Rete}

\subsection{Tipologia di Nodi}

La rete RETE comprende quattro tipi principali di nodi:

\subsubsection{Nodi Alpha (Alpha Network)}

\begin{definizione}[Nodo Alpha]
Un nodo alpha $\alpha_i$ è associato a un singolo pattern $P_i$ e mantiene:
\begin{equation}
\text{memory}(\alpha_i) = \{w \in WM \mid w \text{ matcha } P_i\}
\end{equation}
\end{definizione}

Funzione: \textit{filtering} --- seleziona fatti che soddisfano un pattern.

\subsubsection{Nodi Beta (Beta Network)}

\begin{definizione}[Nodo Beta Memory]
Un nodo beta memory $\beta_i$ mantiene \textit{token}:
\begin{equation}
\text{memory}(\beta_i) = \{(w_1, \ldots, w_j) \mid \text{combinazione valida fino al pattern } j\}
\end{equation}
\end{definizione}

Funzione: \textit{memorizzazione} di match parziali multi-pattern.

\subsubsection{Nodi Join}

\begin{definizione}[Nodo Join]
Un nodo join $J_{i,j}$ combina:
\begin{itemize}
\item Input sinistro: token da beta memory $\beta_{i-1}$
\item Input destro: fatti da alpha node $\alpha_j$
\item Output: token estesi se join ha successo
\end{itemize}
\end{definizione}

Funzione: \textit{combinazione} di match parziali con nuovi fatti.

\subsubsection{Nodi Production}

\begin{definizione}[Nodo Production]
Un nodo production $\pi_r$ per la regola $r$:
\begin{itemize}
\item Riceve token completi (matchano tutto LHS)
\item Crea istanziazioni $(r, \theta)$ da aggiungere all'agenda
\item Non ha figli (nodo foglia)
\end{itemize}
\end{definizione}

\subsection{Struttura della Rete}

\begin{figure}[h]
\centering
\begin{tikzpicture}[
  scale=0.8,
  node distance=1.5cm,
  every node/.style={font=\small},
  alpha/.style={rectangle, draw=blue!50, fill=blue!10, thick, minimum width=2cm, minimum height=0.8cm},
  beta/.style={rectangle, draw=green!50, fill=green!10, thick, minimum width=2cm, minimum height=0.8cm},
  joinnode/.style={diamond, draw=orange!50, fill=orange!10, thick, minimum width=1.5cm, minimum height=1.5cm},
  prod/.style={ellipse, draw=red!50, fill=red!10, thick, minimum width=2cm, minimum height=0.8cm}
]

% Alpha network
\node[alpha] (a1) {$\alpha_1$: P1};
\node[alpha, right=of a1] (a2) {$\alpha_2$: P2};
\node[alpha, right=of a2] (a3) {$\alpha_3$: P3};

% Beta network - level 1
\node[joinnode, below=of a1] (j1) {$J_1$};
\node[beta, below=of j1] (b1) {$\beta_1$};

% Beta network - level 2
\node[joinnode, below=of b1] (j2) {$J_2$};
\node[beta, below=of j2] (b2) {$\beta_2$};

% Production node
\node[prod, below=of b2] (p1) {$\pi_r$: Rule R};

% Connections
\draw[->, thick] (a1) -- (j1);
\draw[->, thick] (a2) -- (j1);
\draw[->, thick] (j1) -- (b1);
\draw[->, thick] (b1) -- (j2);
\draw[->, thick] (a3) -- (j2);
\draw[->, thick] (j2) -- (b2);
\draw[->, thick] (b2) -- (p1);

% Labels
\node[left=0.5cm of a1, align=right] {\footnotesize Alpha\\Network};
\node[left=0.5cm of j1, align=right] {\footnotesize Beta\\Network};
\node[left=0.5cm of p1] {\footnotesize Production};

% WM connection
\node[above=0.5cm of a2] (wm) {\textbf{Working Memory}};
\draw[->, dashed, thick] (wm) -- (a1);
\draw[->, dashed, thick] (wm) -- (a2);
\draw[->, dashed, thick] (wm) -- (a3);

\end{tikzpicture}
\caption{Architettura generale rete RETE per regola con 3 pattern}
\label{fig:rete_overview}
\end{figure}

\section{Operazioni Fondamentali}

\subsection{Costruzione della Rete}

La rete viene costruita \textit{una volta} all'inizio, quando le regole vengono caricate:

\begin{algorithm}[H]
\caption{Costruzione Rete RETE}
\begin{algorithmic}[1]
\Function{BuildNetwork}{$PM$}
    \State $\alpha \gets \emptyset$ \Comment{Alpha nodes}
    \State $\beta \gets \emptyset$ \Comment{Beta nodes}
    \For{each rule $r \in PM$}
        \State $\text{prev} \gets \text{null}$
        \For{each pattern $P_i$ in $r.LHS$}
            \State $\alpha_i \gets \text{FindOrCreateAlpha}(P_i, \alpha)$
            \If{$\text{prev} = \text{null}$}
                \State $\text{prev} \gets \alpha_i$ \Comment{Primo pattern}
            \Else
                \State $J \gets \text{CreateJoin}(\text{prev}, \alpha_i)$
                \State $\beta_i \gets \text{CreateBetaMemory}()$
                \State $\text{prev} \gets \beta_i$
            \EndIf
        \EndFor
        \State $\pi_r \gets \text{CreateProduction}(r, \text{prev})$
    \EndFor
\EndFunction
\end{algorithmic}
\end{algorithm}

\subsection{Propagazione Assert}

Quando un fatto $w$ viene asserito:

\begin{algorithm}[H]
\caption{Propagazione Assert}
\begin{algorithmic}[1]
\Function{PropagateAssert}{$w, G$}
    \For{each alpha node $\alpha$ that matches $w$}
        \State $\alpha.\text{memory} \gets \alpha.\text{memory} \cup \{w\}$
        \State $\tau \gets \text{CreateToken}(w)$ \Comment{Token iniziale}
        \For{each child join $J$ of $\alpha$}
            \State $\text{PropagateTo}(J, \tau, \text{from-right})$
        \EndFor
    \EndFor
\EndFunction
\end{algorithmic}
\end{algorithm}

\subsection{Propagazione Retract}

Quando un fatto $w$ viene ritratto:

\begin{algorithm}[H]
\caption{Propagazione Retract}
\begin{algorithmic}[1]
\Function{PropagateRetract}{$w, G$}
    \For{each alpha node $\alpha$ containing $w$}
        \State $\alpha.\text{memory} \gets \alpha.\text{memory} \setminus \{w\}$
    \EndFor
    \For{each beta memory $\beta$ in $G$}
        \State $\text{affected} \gets \{\tau \in \beta.\text{memory} \mid w \in \tau\}$
        \State $\beta.\text{memory} \gets \beta.\text{memory} \setminus \text{affected}$
    \EndFor
    \State $A \gets A \setminus \{(r, \theta) \mid w \in \text{support}(r, \theta)\}$
\EndFunction
\end{algorithmic}
\end{algorithm}

\section{Analisi Preliminare di Complessità}

\subsection{Complessità Spaziale}

\subsubsection{Alpha Memory}

Ogni alpha node memorizza fatti:
\begin{equation}
\text{Space}(\alpha_i) = O(|WM_i|)
\end{equation}

dove $|WM_i|$ è il numero di fatti che matchano $P_i$.

Nel caso peggiore (pattern senza costanti): $|WM_i| = |WM|$

Totale alpha memory:
\begin{equation}
\text{Space}_\alpha = O(|\alpha| \cdot |WM|)
\end{equation}

\subsubsection{Beta Memory}

Ogni beta node al livello $j$ memorizza token di lunghezza $j$:
\begin{equation}
\text{Space}(\beta_j) = O(|WM|^j)
\end{equation}

Nel caso peggiore (tutti i fatti matchano):
\begin{equation}
\text{Space}_\beta = O\left(\sum_{j=1}^{k} |WM|^j\right) = O(|WM|^k)
\end{equation}

\begin{warningbox}[Esplosione Combinatoria]
La beta memory può crescere esponenzialmente! Questo è il \textit{beta memory blowup problem}.

In pratica, pattern ben progettati con costanti e join constraints limitano drasticamente la crescita.
\end{warningbox}

\subsection{Complessità Temporale}

\subsubsection{Ciclo Singolo}

Per un singolo ciclo recognize-act:

\textbf{Caso medio} (con $c$ fatti cambiati):
\begin{equation}
T_{\text{RETE}} = O(c \cdot n)
\end{equation}

\textbf{Caso peggiore} (tutti i fatti cambiano):
\begin{equation}
T_{\text{worst}} = O(m \cdot n)
\end{equation}

\subsubsection{Confronto Asintotico}

\begin{table}[h]
\centering
\begin{tabular}{@{}lcc@{}}
\toprule
\textbf{Approccio} & \textbf{Caso Medio} & \textbf{Caso Peggiore} \\
\midrule
Naïve & $O(n \cdot m^k)$ & $O(n \cdot m^k)$ \\
RETE & $O(c \cdot n)$ & $O(m \cdot n)$ \\
\midrule
\textbf{Speedup} & $\mathbf{\frac{m^k}{c}}$ & $\mathbf{m^{k-1}}$ \\
\bottomrule
\end{tabular}
\caption{Confronto complessità Naïve vs RETE}
\label{tab:complexity_comparison}
\end{table}

Con $m=10000$, $k=3$, $c=10$:
\begin{equation}
\text{Speedup} = \frac{10000^3}{10} = 10^{11} \text{ volte più veloce!}
\end{equation}

\section{Invarianti Fondamentali}

\subsection{Invariante di Correttezza}

\begin{teorema}[Correttezza RETE]
\label{thm:rete_correctness}
Sia $CS_{\text{naïve}}$ il conflict set calcolato con approccio naïve e $CS_{\text{RETE}}$ quello calcolato con RETE. Allora:
\begin{equation}
CS_{\text{RETE}} = CS_{\text{naïve}}
\end{equation}

per ogni stato della working memory.
\end{teorema}

La dimostrazione verrà fornita nel Capitolo \ref{cap:rete_beta} dopo aver definito formalmente tutti i nodi.

\subsection{Invariante di Consistenza}

\begin{definizione}[Consistenza Alpha]
Per ogni alpha node $\alpha_i$ e working memory $WM$:
\begin{equation}
\alpha_i.\text{memory} = \{w \in WM \mid w \text{ matcha } \alpha_i.\text{pattern}\}
\end{equation}
\end{definizione}

\begin{definizione}[Consistenza Beta]
Per ogni beta node $\beta_j$ al livello $j$:
\begin{equation}
\beta_j.\text{memory} = \{\tau \mid \tau \text{ è un match valido dei primi } j \text{ pattern}\}
\end{equation}
\end{definizione}

Questi invarianti devono essere mantenuti dopo ogni operazione (assert/retract).

\section{Token e Partial Matches}

\subsection{Definizione di Token}

\begin{definizione}[Token]
Un token $\tau$ al livello $j$ è una sequenza:
\begin{equation}
\tau = \langle w_1, w_2, \ldots, w_j \rangle
\end{equation}

dove ogni $w_i \in WM$ e la sequenza matcha i primi $j$ pattern della regola.
\end{definizione}

\subsection{Binding e Consistenza}

Un token $\tau$ ha associato un environment di binding $\theta_\tau$:

\begin{equation}
\theta_\tau: \text{Var}(P_1, \ldots, P_j) \to \text{Val}(WM)
\end{equation}

\textbf{Condizione di consistenza}: tutte le occorrenze della stessa variabile devono avere lo stesso valore.

\begin{esempio}[Binding Consistency]
Pattern:
\begin{lstlisting}[language=CLIPS]
(persona (nome ?n) (eta ?e))
(esame (studente ?n) (voto ?v))
\end{lstlisting}

Se $\theta(?n) = \text{"Mario"}$ nel primo pattern, deve essere $\theta(?n) = \text{"Mario"}$ anche nel secondo.
\end{esempio}

\subsection{Join Keys}

\begin{definizione}[Join Keys]
Le \textit{join keys} per un nodo join sono le variabili condivise tra:
\begin{itemize}
\item Token del ramo sinistro (beta memory precedente)
\item Fatto del ramo destro (alpha node corrente)
\end{itemize}
\end{definizione}

Formalmente:
\begin{equation}
\text{JoinKeys}(J) = \text{Var}(\text{left}) \cap \text{Var}(\text{right})
\end{equation}

\section{Propagazione Incrementale}

\subsection{Assert Incrementale}

Quando viene asserito $w_{\text{new}}$:

\begin{enumerate}
\item Trova alpha nodes che matchano: $A = \{\alpha \mid \alpha.\text{pattern} \text{ matcha } w_{\text{new}}\}$
\item Per ogni $\alpha \in A$:
   \begin{enumerate}
   \item Aggiungi $w_{\text{new}}$ a $\alpha.\text{memory}$
   \item Crea token iniziale $\tau_0 = \langle w_{\text{new}} \rangle$
   \item Propaga $\tau_0$ ai join children di $\alpha$
   \end{enumerate}
\item I join tentano combinazioni con token esistenti nel ramo opposto
\item Token validi vengono propagati verso production nodes
\end{enumerate}

\textbf{Chiave}: solo il \textit{nuovo} fatto viene processato, non tutti i fatti.

\subsection{Retract Incrementale}

Quando viene ritratto $w_{\text{old}}$:

\begin{enumerate}
\item Rimuovi $w_{\text{old}}$ da tutti gli alpha nodes che lo contenevano
\item Trova tutti i token che includono $w_{\text{old}}$:
   \begin{equation}
   T_{\text{affected}} = \{\tau \in \bigcup_\beta \beta.\text{memory} \mid w_{\text{old}} \in \tau\}
   \end{equation}
\item Rimuovi $T_{\text{affected}}$ dalle beta memories
\item Rimuovi istanziazioni dipendenti dall'agenda:
   \begin{equation}
   A' = A \setminus \{(r, \theta) \mid w_{\text{old}} \in \text{support}(r, \theta)\}
   \end{equation}
\end{enumerate}

\section{Esempio Completo}

\subsection{Scenario}

Consideriamo un sistema semplice per rilevare coppie di amici:

\textbf{Regola}:
\begin{lstlisting}[language=CLIPS]
(defrule trova-coppia-amici
  (persona (nome ?n1) (hobby ?h))
  (persona (nome ?n2&~?n1) (hobby ?h))
  =>
  (printout t ?n1 " e " ?n2 " condividono hobby: " ?h crlf))
\end{lstlisting}

\textbf{Working Memory Iniziale}:
\begin{align*}
w_1 &= \text{persona}(\text{nome}: \text{"Alice"}, \text{hobby}: \text{"tennis"})\\
w_2 &= \text{persona}(\text{nome}: \text{"Bob"}, \text{hobby}: \text{"tennis"})\\
w_3 &= \text{persona}(\text{nome}: \text{"Carol"}, \text{hobby}: \text{"golf"})
\end{align*}

\subsection{Costruzione Rete}

\begin{enumerate}
\item \textbf{Alpha node} $\alpha_1$ per pattern \texttt{(persona (nome ?n1) (hobby ?h))}:
\begin{equation}
\alpha_1.\text{memory} = \{w_1, w_2, w_3\} \quad \text{(tutti matchano)}
\end{equation}

\item \textbf{Alpha node} $\alpha_2$ per pattern \texttt{(persona (nome ?n2) (hobby ?h))}:
\begin{equation}
\alpha_2.\text{memory} = \{w_1, w_2, w_3\}
\end{equation}

\item \textbf{Join node} $J$ con:
\begin{itemize}
\item Join key: \texttt{?h} (hobby condiviso)
\item Test: \texttt{?n2 \~{} ?n1} (nomi diversi)
\end{itemize}

\item \textbf{Beta memory} $\beta$ memorizza coppie valide

\item \textbf{Production node} $\pi$ crea istanziazioni per agenda
\end{enumerate}

\subsection{Esecuzione Passo-Passo}

\textbf{Inizializzazione}: Asserisci $w_1$, $w_2$, $w_3$

\begin{enumerate}
\item $\alpha_1.\text{memory} = \{w_1, w_2, w_3\}$
\item $\alpha_2.\text{memory} = \{w_1, w_2, w_3\}$
\item Join $J$ combina:
   \begin{itemize}
   \item $w_1$ con $w_2$: $\theta_1 = \{?n1 \mapsto \text{"Alice"}, ?n2 \mapsto \text{"Bob"}, ?h \mapsto \text{"tennis"}\}$ $\checkmark$
   \item $w_1$ con $w_3$: hobby diversi $\times$
   \item $w_2$ con $w_1$: $\theta_2 = \{?n1 \mapsto \text{"Bob"}, ?n2 \mapsto \text{"Alice"}, ?h \mapsto \text{"tennis"}\}$ $\checkmark$
   \item Altri: falliscono per test $?n2 \neq ?n1$ o hobby diversi
   \end{itemize}
\item $\beta.\text{memory} = \{\langle w_1, w_2 \rangle, \langle w_2, w_1 \rangle\}$
\item Agenda: 2 istanziazioni
\end{enumerate}

\textbf{Assert} $w_4 = \text{persona}(\text{nome}: \text{"David"}, \text{hobby}: \text{"tennis"})$:

\begin{enumerate}
\item $\alpha_1.\text{memory} \gets \alpha_1.\text{memory} \cup \{w_4\}$
\item $\alpha_2.\text{memory} \gets \alpha_2.\text{memory} \cup \{w_4\}$
\item Join propaga solo combinazioni con $w_4$:
   \begin{itemize}
   \item $w_4$ con $w_1$: $\checkmark$
   \item $w_4$ con $w_2$: $\checkmark$
   \item $w_1$ con $w_4$: $\checkmark$
   \item $w_2$ con $w_4$: $\checkmark$
   \end{itemize}
\item $\beta.\text{memory}$ cresce da 2 a 6 token
\item Agenda: +4 nuove istanziazioni
\end{enumerate}

\textbf{Nota critica}: Solo le \textit{nuove} combinazioni vengono calcolate, non tutte da capo!

\section{Ottimizzazioni Fondamentali}

\subsection{Alpha Node Sharing}

Pattern identici condividono lo stesso alpha node:

\begin{lstlisting}[language=CLIPS]
(defrule r1
  (persona (eta ?e))
  ...
  =>
  ...)

(defrule r2
  (persona (eta ?e))
  ...
  =>
  ...)
\end{lstlisting}

Entrambe le regole usano lo stesso $\alpha_{\text{persona}}$.

\textbf{Beneficio}: Spazio e tempo di match risparmiati.

\subsection{Join Test Inlining}

Test semplici vengono eseguiti inline durante il join:

\begin{equation}
\text{JoinTest}(\tau_{\text{left}}, w_{\text{right}}) = \bigwedge_{v \in \text{JoinKeys}} \tau_{\text{left}}[v] = w_{\text{right}}[v]
\end{equation}

\textbf{Beneficio}: Fallimento rapido senza creazione token.

\subsection{Hash Indexing}

Beta memories usano hash table per lookup efficiente:

\begin{equation}
H(\tau) = \text{hash}\left(\bigoplus_{v \in \text{JoinKeys}} \tau[v]\right)
\end{equation}

Join diventa:
\begin{enumerate}
\item Calcola $h = H(w_{\text{right}})$
\item Cerca bucket $\beta.\text{hashTable}[h]$
\item Testa solo token in quel bucket
\end{enumerate}

\textbf{Complessità}: $O(1)$ atteso invece di $O(|\beta.\text{memory}|)$

\section{Varianti dell'Algoritmo}

\subsection{TREAT (Miranker, 1987)}

TREAT (Temporal RETE) elimina le beta memories:

\begin{itemize}
\item \textbf{Pro}: Spazio $O(|WM|)$ invece di $O(|WM|^k)$
\item \textbf{Contro}: Tempo peggiore, ricalcola join ad ogni ciclo
\end{itemize}

\textbf{Trade-off}: Spazio vs Tempo

\subsection{RETE-II}

Estensioni moderne includono:
\begin{itemize}
\item Parallel matching su multi-core
\item Incremental delete più efficiente
\item Garbage collection di nodi inutilizzati
\item Adaptive heuristics
\end{itemize}

\section{Conclusioni del Capitolo}

Abbiamo introdotto:

\begin{itemize}
\item L'architettura generale di RETE
\item I quattro tipi di nodi (alpha, beta, join, production)
\item Il meccanismo di propagazione incrementale
\item Analisi preliminare di complessità
\item Ottimizzazioni fondamentali
\end{itemize}

Nei prossimi capitoli approfondiremo:

\begin{itemize}
\item \textbf{Capitolo 7}: Alpha network in dettaglio
\item \textbf{Capitolo 8}: Beta network e join algorithm
\item \textbf{Capitolo 9}: Dimostrazione formale di correttezza e complessità
\item \textbf{Capitolo 10}: Ottimizzazioni avanzate
\end{itemize}

\begin{successbox}[Punti Chiave]
\begin{itemize}
\item RETE riduce complessità da $O(n \cdot m^k)$ a $O(c \cdot n)$ sfruttando continuità e condivisione
\item La rete è costruita una volta, poi usata incrementalmente
\item Invarianti di correttezza garantiscono equivalenza con match naïve
\item Trade-off spazio/tempo gestiti con ottimizzazioni (hashing, sharing)
\end{itemize}
\end{successbox}

