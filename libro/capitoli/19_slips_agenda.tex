% Capitolo 19: Implementazione Agenda in SLIPS

\chapter{SLIPS Agenda Implementation}
\label{cap:slips_agenda}

\section{Introduzione}

L'agenda di SLIPS gestisce il conflict set con le strategie di CLIPS tradotte in Swift idiomatico.

\section{Activation Structure}

\begin{lstlisting}[language=Swift]
public struct Activation: Hashable, Identifiable {
    public let id: UUID = UUID()
    public let rule: Defrule
    public let token: Token
    public let salience: Int
    public let timetag: UInt64
    public let randomID: UInt32
    
    init(rule: Defrule, token: Token, timetag: UInt64) {
        self.rule = rule
        self.token = token
        self.timetag = timetag
        self.salience = rule.salience
        self.randomID = UInt32.random(in: 0..<UInt32.max)
    }
    
    public func hash(into hasher: inout Hasher) {
        hasher.combine(id)
    }
    
    public static func == (lhs: Activation, rhs: Activation) -> Bool {
        return lhs.id == rhs.id
    }
}
\end{lstlisting}

\section{Conflict Resolution Strategies}

\begin{lstlisting}[language=Swift]
public enum ConflictStrategy {
    case depth
    case breadth
    case simplicity
    case complexity
    case lex
    case mea
    case random
}

protocol ActivationComparator {
    func compare(_ a1: Activation, _ a2: Activation) -> ComparisonResult
}

class DepthComparator: ActivationComparator {
    func compare(_ a1: Activation, _ a2: Activation) -> ComparisonResult {
        // 1. Salience (higher first)
        if a1.salience != a2.salience {
            return a1.salience > a2.salience ? .orderedAscending : .orderedDescending
        }
        
        // 2. Recency (higher timetag first)
        if a1.timetag != a2.timetag {
            return a1.timetag > a2.timetag ? .orderedAscending : .orderedDescending
        }
        
        // 3. Specificity
        let spec1 = a1.rule.specificity
        let spec2 = a2.rule.specificity
        if spec1 != spec2 {
            return spec1 > spec2 ? .orderedAscending : .orderedDescending
        }
        
        // 4. Rule order
        return a1.rule.definitionOrder < a2.rule.definitionOrder 
            ? .orderedAscending 
            : .orderedDescending
    }
}

class BreadthComparator: ActivationComparator {
    func compare(_ a1: Activation, _ a2: Activation) -> ComparisonResult {
        // Like depth, but recency reversed
        if a1.salience != a2.salience {
            return a1.salience > a2.salience ? .orderedAscending : .orderedDescending
        }
        
        // Older facts first (opposite of depth)
        if a1.timetag != a2.timetag {
            return a1.timetag < a2.timetag ? .orderedAscending : .orderedDescending
        }
        
        // Rest is same
        let spec1 = a1.rule.specificity
        let spec2 = a2.rule.specificity
        if spec1 != spec2 {
            return spec1 > spec2 ? .orderedAscending : .orderedDescending
        }
        
        return a1.rule.definitionOrder < a2.rule.definitionOrder 
            ? .orderedAscending 
            : .orderedDescending
    }
}

class RandomComparator: ActivationComparator {
    func compare(_ a1: Activation, _ a2: Activation) -> ComparisonResult {
        // Salience still matters
        if a1.salience != a2.salience {
            return a1.salience > a2.salience ? .orderedAscending : .orderedDescending
        }
        
        // Random for same salience
        return a1.randomID < a2.randomID ? .orderedAscending : .orderedDescending
    }
}
\end{lstlisting}

\section{Agenda Implementation}

\begin{lstlisting}[language=Swift]
public class Agenda {
    private var activations: [Activation] = []
    private var comparator: ActivationComparator
    private var strategy: ConflictStrategy
    
    init(strategy: ConflictStrategy = .depth) {
        self.strategy = strategy
        self.comparator = Self.createComparator(for: strategy)
    }
    
    func add(_ activation: Activation) {
        // Insert maintaining sorted order
        let insertionIndex = activations.firstIndex { existing in
            comparator.compare(activation, existing) == .orderedAscending
        } ?? activations.endIndex
        
        activations.insert(activation, at: insertionIndex)
    }
    
    func remove(_ activation: Activation) {
        activations.removeAll { $0.id == activation.id }
    }
    
    func next() -> Activation? {
        return activations.first
    }
    
    func removeAndReturnNext() -> Activation? {
        guard !activations.isEmpty else { return nil }
        return activations.removeFirst()
    }
    
    func setStrategy(_ strategy: ConflictStrategy) {
        self.strategy = strategy
        self.comparator = Self.createComparator(for: strategy)
        reorder()
    }
    
    private func reorder() {
        activations.sort { a1, a2 in
            comparator.compare(a1, a2) == .orderedAscending
        }
    }
    
    func clear() {
        activations.removeAll()
    }
    
    var count: Int {
        return activations.count
    }
    
    var all: [Activation] {
        return activations
    }
}
\end{lstlisting}

\section{Module-Aware Agenda}

\begin{lstlisting}[language=Swift]
public class ModuleAgenda {
    private var agendas: [Defmodule: Agenda] = [:]
    private var focusStack: FocusStack
    
    init(focusStack: FocusStack) {
        self.focusStack = focusStack
    }
    
    func add(_ activation: Activation) {
        let module = activation.rule.module
        let agenda = agendas[module, default: Agenda()]
        agenda.add(activation)
        agendas[module] = agenda
    }
    
    func next() -> Activation? {
        // Try current focus
        if let activation = agendas[focusStack.current]?.next() {
            return activation
        }
        
        // Pop and try next
        focusStack.pop()
        
        if focusStack.isEmpty {
            return nil  // Quiescence
        }
        
        return next()  // Recursive
    }
    
    func removeAndReturnNext() -> Activation? {
        guard let activation = next() else { return nil }
        agendas[focusStack.current]?.remove(activation)
        return activation
    }
}
\end{lstlisting}

\section{Dynamic Salience}

\begin{lstlisting}[language=Swift]
extension Defrule {
    func evaluateSalience(with token: Token, env: Environment) -> Int {
        guard let dynamicExpr = dynamicSalience else {
            return salience  // Static
        }
        
        let bindings = token.bindings
        let result = env.evaluate(dynamicExpr, bindings: bindings)
        
        if case .integer(let value) = result {
            return value
        }
        
        return salience  // Fallback
    }
}

// Usage in Activation creation
func createActivation(rule: Defrule, token: Token, env: Environment) -> Activation {
    let evaluatedSalience = rule.evaluateSalience(with: token, env: env)
    
    return Activation(
        rule: rule,
        token: token,
        timetag: env.currentTimetag,
        overrideSalience: evaluatedSalience
    )
}
\end{lstlisting}

\section{Salience Evaluation Modes}

\begin{lstlisting}[language=Swift]
public enum SalienceEvaluation {
    case whenDefined      // At rule compilation
    case whenActivated    // When activation created
    case everyCycle       // Before each rule selection
}

public class SalienceManager {
    var mode: SalienceEvaluation = .whenDefined
    
    func evaluateSalience(
        for activation: Activation,
        env: Environment
    ) -> Int {
        switch mode {
        case .whenDefined:
            return activation.salience  // Already computed
            
        case .whenActivated, .everyCycle:
            return activation.rule.evaluateSalience(
                with: activation.token,
                env: env
            )
        }
    }
}
\end{lstlisting}

\section{Refresh and Reorder}

\begin{lstlisting}[language=Swift]
extension Agenda {
    func refresh(rule: Defrule, env: Environment) {
        // Remove existing activations for this rule
        activations.removeAll { $0.rule.name == rule.name }
        
        // Regenerate from production node
        if let prodNode = rule.productionNode {
            for token in prodNode.tokens {
                let activation = Activation(
                    rule: rule,
                    token: token,
                    timetag: env.currentTimetag
                )
                add(activation)
            }
        }
    }
    
    func refreshAll(env: Environment) {
        let rules = Set(activations.map(\.rule))
        for rule in rules {
            refresh(rule: rule, env: env)
        }
    }
}
\end{lstlisting}

\section{Conclusioni del Capitolo}

\subsection{Punti Chiave}

\begin{enumerate}
\item Agenda usa \textbf{sorted array} per efficienza
\item \textbf{Strategy pattern} per conflict resolution
\item \textbf{Module-aware} agenda con focus stack
\item \textbf{Dynamic salience} supportata
\item Refresh e reorder per flessibilità
\end{enumerate}

\subsection{Prossimi Passi}

Cap.~\ref{cap:slips_pattern_matching} mostra pattern matching avanzato.

\subsection{Letture Consigliate}

\begin{itemize}
\item CLIPS Source - \texttt{agenda.c}
\item Swift Collections - Sorted Arrays
\end{itemize}
