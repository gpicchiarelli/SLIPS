% Appendice A: Riferimento API Completo

\chapter{Riferimento API Completo}
\label{app:api}

\section{API Pubblica CLIPS}

\subsection{Gestione Environment}

\begin{lstlisting}[language=Swift]
@MainActor
public enum CLIPS {
    /// Crea nuovo environment
    /// - Returns: Environment inizializzato
    public static func createEnvironment() -> Environment
    
    /// Carica file .clp
    /// - Parameter path: Percorso file
    /// - Throws: IOError se file non trovato
    public static func load(_ path: String) throws
    
    /// Reset environment (clear + deffacts)
    public static func reset()
    
    /// Accesso environment corrente
    public static var currentEnvironment: Environment? { get }
}
\end{lstlisting}

\subsection{Esecuzione Regole}

\begin{lstlisting}[language=Swift]
extension CLIPS {
    /// Esegue regole fino a esaurimento o limite
    /// - Parameter limit: Numero massimo regole (nil = infinito)
    /// - Returns: Numero regole eseguite
    @discardableResult
    public static func run(limit: Int?) -> Int
}
\end{lstlisting}

\subsection{Gestione Fatti}

\begin{lstlisting}[language=Swift]
extension CLIPS {
    /// Asserisce fatto
    /// - Parameter fact: Espressione CLIPS
    /// - Returns: ID del fatto (-1 se errore)
    @discardableResult
    public static func assert(fact: String) -> Int
    
    /// Ritrae fatto
    /// - Parameter id: ID del fatto da ritrarre
    public static func retract(id: Int)
}
\end{lstlisting}

\subsection{Valutazione Espressioni}

\begin{lstlisting}[language=Swift]
extension CLIPS {
    /// Valuta espressione CLIPS
    /// - Parameter expr: Espressione S-expression
    /// - Returns: Valore risultante
    @discardableResult
    public static func eval(expr: String) -> Value
}
\end{lstlisting}

\section{Built-in Functions}

\subsection{Matematica}

\begin{table}[h]
\centering
\small
\begin{tabular}{@{}llp{6cm}@{}}
\toprule
\textbf{Funzione} & \textbf{Args} & \textbf{Descrizione} \\
\midrule
\texttt{+} & $n \geq 1$ & Somma argomenti \\
\texttt{-} & $n \geq 1$ & Sottrazione (unario: negazione) \\
\texttt{*} & $n \geq 1$ & Prodotto \\
\texttt{/} & $n \geq 1$ & Divisione \\
\texttt{div} & 2 & Divisione intera \\
\texttt{mod} & 2 & Modulo \\
\texttt{abs} & 1 & Valore assoluto \\
\texttt{min} & $n \geq 1$ & Minimo \\
\texttt{max} & $n \geq 1$ & Massimo \\
\texttt{sqrt} & 1 & Radice quadrata \\
\texttt{pow} & 2 & Potenza \\
\texttt{exp} & 1 & Esponenziale \\
\texttt{log} & 1 & Logaritmo naturale \\
\texttt{log10} & 1 & Logaritmo base 10 \\
\bottomrule
\end{tabular}
\caption{Funzioni matematiche}
\label{tab:api_math}
\end{table}

\subsection{Logiche}

\begin{table}[h]
\centering
\small
\begin{tabular}{@{}llp{6cm}@{}}
\toprule
\textbf{Funzione} & \textbf{Args} & \textbf{Descrizione} \\
\midrule
\texttt{and} & $n \geq 1$ & AND logico \\
\texttt{or} & $n \geq 1$ & OR logico \\
\texttt{not} & 1 & NOT logico \\
\texttt{eq} & 2+ & Uguaglianza valore \\
\texttt{neq} & 2+ & Disuguaglianza \\
\texttt{=} & 2+ & Uguaglianza numerica \\
\texttt{<>} & 2+ & Disuguaglianza numerica \\
\texttt{<} & 2+ & Minore \\
\texttt{<=} & 2+ & Minore o uguale \\
\texttt{>} & 2+ & Maggiore \\
\texttt{>=} & 2+ & Maggiore o uguale \\
\bottomrule
\end{tabular}
\caption{Funzioni logiche}
\label{tab:api_logic}
\end{table}

\subsection{Facts e Rules}

\begin{table}[h]
\centering
\small
\begin{tabular}{@{}llp{6cm}@{}}
\toprule
\textbf{Funzione} & \textbf{Args} & \textbf{Descrizione} \\
\midrule
\texttt{assert} & 1+ & Asserisce fatto \\
\texttt{retract} & 1+ & Ritrae fatto (per ID) \\
\texttt{modify} & 2+ & Modifica fatto \\
\texttt{duplicate} & 2+ & Duplica fatto \\
\texttt{facts} & 0-1 & Lista fatti [modulo] \\
\texttt{rules} & 0-1 & Lista regole [modulo] \\
\texttt{agenda} & 0-1 & Lista agenda [modulo] \\
\texttt{clear} & 0 & Pulisce environment \\
\texttt{reset} & 0 & Reset + assert deffacts \\
\texttt{run} & 0-1 & Esegue regole [limit] \\
\bottomrule
\end{tabular}
\caption{Funzioni facts e rules}
\label{tab:api_facts}
\end{table}

\subsection{Moduli}

\begin{table}[h]
\centering
\small
\begin{tabular}{@{}llp{5cm}@{}}
\toprule
\textbf{Funzione} & \textbf{Args} & \textbf{Descrizione} \\
\midrule
\texttt{focus} & 1+ & Imposta focus su moduli \\
\texttt{get-current-module} & 0 & Ritorna modulo corrente \\
\texttt{set-current-module} & 1 & Imposta modulo corrente \\
\texttt{list-defmodules} & 0 & Stampa lista moduli \\
\texttt{get-defmodule-list} & 0 & Ritorna multifield moduli \\
\bottomrule
\end{tabular}
\caption{Funzioni moduli}
\label{tab:api_modules}
\end{table}

\section{Value Type}

\begin{lstlisting}[language=Swift]
public enum Value: Codable, Equatable {
    case int(Int64)
    case float(Double)
    case string(String)
    case symbol(String)
    case boolean(Bool)
    case multifield([Value])
    case none
}
\end{lstlisting}

\section{Template e Pattern}

\subsection{Pattern Test Types}

\begin{lstlisting}[language=Swift]
public struct PatternTest: Codable {
    public enum Kind: Codable {
        case constant(Value)
        case variable(String)
        case mfVariable(String)
        case predicate(ExpressionNode)
        case sequence([PatternTest])
    }
    public let kind: Kind
}
\end{lstlisting}

\section{Pattern Matching API}

\subsection{Constraint Builders}

\begin{lstlisting}[language=Swift]
public class PatternBuilder {
    public func pattern(_ template: String, 
                       @ConstraintBuilder _ constraints: () -> [Constraint]) -> Pattern {
        Pattern(template: template, constraints: constraints())
    }
}

@resultBuilder
public struct ConstraintBuilder {
    public static func buildBlock(_ components: Constraint...) -> [Constraint] {
        Array(components)
    }
}
\end{lstlisting}

\section{Error Handling}

\subsection{Error Types}

\begin{lstlisting}[language=Swift]
public enum CLIPSError: Error {
    case parseError(String, line: Int, column: Int)
    case runtimeError(String)
    case undefinedTemplate(String)
    case undefinedRule(String)
    case invalidSlot(String)
    case typeMismatch(expected: ValueType, got: ValueType)
    case fileNotFound(String)
}
\end{lstlisting}

\section{Esempi Completi}

Vedere Appendice~\ref{app:esempi} per esempi d'uso completi e casi di studio.

