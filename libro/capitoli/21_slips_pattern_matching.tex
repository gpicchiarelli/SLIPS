% Capitolo 21: Pattern Matching Avanzato in SLIPS

\chapter{Pattern Matching Avanzato}
\label{cap:slips_pattern_matching}

\section{Introduzione}

SLIPS implementa tutte le funzionalità avanzate di pattern matching di CLIPS, con particolare attenzione a multifield e constraint complessi.

\section{Constraint System}

\begin{lstlisting}[language=Swift]
public enum Constraint {
    case equals(Value)
    case notEquals(Value)
    case variable(String)
    case multifieldVariable(String)
    case predicate(PredicateFunction)
    case conjunction([Constraint])
    case disjunction([Constraint])
    case negation(Constraint)
    
    func evaluate(_ value: Value, bindings: [String: Value]) -> BindingResult {
        switch self {
        case .equals(let expected):
            return value == expected ? .success([:]) : .failure
            
        case .variable(let name):
            if let bound = bindings[name] {
                return value == bound ? .success([:]) : .failure
            }
            return .success([name: value])
            
        case .predicate(let fn):
            return fn(value, bindings) ? .success([:]) : .failure
            
        case .conjunction(let constraints):
            return evaluateConjunction(constraints, value, bindings)
            
        case .disjunction(let constraints):
            return evaluateDisjunction(constraints, value, bindings)
            
        case .negation(let inner):
            let result = inner.evaluate(value, bindings: bindings)
            return result.isSuccess ? .failure : .success([:])
            
        default:
            return .failure
        }
    }
}

public enum BindingResult {
    case success([String: Value])
    case failure
    
    var isSuccess: Bool {
        if case .success = self { return true }
        return false
    }
}
\end{lstlisting}

\section{Multifield Matching}

\subsection{Multifield Pattern}

\begin{lstlisting}[language=Swift]
struct MultifieldPattern {
    var segments: [Segment]
    
    enum Segment {
        case single(Constraint)
        case multifield(String?)  // Variable name or anonymous
    }
    
    func match(_ values: [Value], bindings: [String: Value]) -> [BindingResult] {
        return matchSegments(segments, values: values, bindings: bindings)
    }
    
    private func matchSegments(
        _ segments: [Segment],
        values: [Value],
        bindings: [String: Value],
        offset: Int = 0
    ) -> [BindingResult] {
        guard !segments.isEmpty else {
            return offset == values.count ? [.success(bindings)] : []
        }
        
        let first = segments.first!
        let rest = Array(segments.dropFirst())
        
        switch first {
        case .single(let constraint):
            guard offset < values.count else { return [] }
            let result = constraint.evaluate(values[offset], bindings: bindings)
            guard case .success(let newBindings) = result else { return [] }
            var combined = bindings
            combined.merge(newBindings) { $1 }
            return matchSegments(rest, values: values, bindings: combined, offset: offset + 1)
            
        case .multifield(let varName):
            // Try all possible lengths for multifield
            var results: [BindingResult] = []
            let minRest = minimumLength(rest)
            let maxLength = values.count - offset - minRest
            
            for length in 0...maxLength {
                let segment = Array(values[offset..<(offset + length)])
                var newBindings = bindings
                if let name = varName {
                    newBindings[name] = .multifield(segment)
                }
                results.append(contentsOf: matchSegments(
                    rest,
                    values: values,
                    bindings: newBindings,
                    offset: offset + length
                ))
            }
            
            return results
        }
    }
}
\end{lstlisting}

\subsection{Example Usage}

\begin{lstlisting}[language=CLIPS]
;; Pattern: (lista $?start 10 $?end)
;; Fact: (lista 1 2 10 3 4)
\end{lstlisting}

\begin{lstlisting}[language=Swift]
let pattern = MultifieldPattern(segments: [
    .multifield("start"),
    .single(.equals(.integer(10))),
    .multifield("end")
])

let fact = [.integer(1), .integer(2), .integer(10), .integer(3), .integer(4)]

let matches = pattern.match(fact, bindings: [:])
// Result: [
//   .success(["start": .multifield([1, 2]), "end": .multifield([3, 4])])
// ]
\end{lstlisting}

\section{Unification}

\begin{lstlisting}[language=Swift]
class Unifier {
    func unify(
        pattern: Pattern,
        fact: Fact,
        bindings: [String: Value] = [:]
    ) -> BindingResult {
        var currentBindings = bindings
        
        for (slotName, constraint) in pattern.constraints {
            guard let factValue = fact.slots[slotName] else {
                return .failure  // Missing slot
            }
            
            let result = constraint.evaluate(factValue, bindings: currentBindings)
            
            guard case .success(let newBindings) = result else {
                return .failure
            }
            
            // Merge bindings
            for (key, value) in newBindings {
                if let existing = currentBindings[key] {
                    if existing != value {
                        return .failure  // Conflict
                    }
                } else {
                    currentBindings[key] = value
                }
            }
        }
        
        return .success(currentBindings)
    }
}
\end{lstlisting}

\section{Test Nodes}

\subsection{Test CE}

\begin{lstlisting}[language=CLIPS]
(defrule example
  (a ?x)
  (b ?y)
  (test (> ?x ?y))  ; Inter-element test
  =>
  ...)
\end{lstlisting}

\begin{lstlisting}[language=Swift]
class TestNode: ReteNode {
    let testExpression: Expression
    
    func activate(token: Token) {
        let result = evaluate(testExpression, bindings: token.bindings)
        
        if case .symbol("TRUE") = result {
            // Pass through unchanged
            for child in children {
                child.activate(token: token)
            }
        }
        // Else filter out
    }
}
\end{lstlisting}

\section{Function Calls in Patterns}

\begin{lstlisting}[language=CLIPS]
(defrule check-range
  (value ?v&:(numberp ?v)&:(> ?v 10)&:(< ?v 100))
  =>
  ...)
\end{lstlisting>

\begin{lstlisting}[language=Swift]
typealias PredicateFunction = (Value, [String: Value]) -> Bool

let constraint = Constraint.conjunction([
    .predicate { value, _ in
        if case .integer = value { return true }
        if case .float = value { return true }
        return false
    },
    .predicate { value, _ in
        guard case .integer(let i) = value else { return false }
        return i > 10
    },
    .predicate { value, _ in
        guard case .integer(let i) = value else { return false }
        return i < 100
    }
])
\end{lstlisting}

\section{Exists and Forall}

\subsection{Exists}

\begin{lstlisting}[language=CLIPS]
(defrule has-order
  (exists (order (status pending)))
  =>
  (printout t "Has pending orders" crlf))
\end{lstlisting>

\begin{lstlisting}[language=Swift]
class ExistsNode: ReteNode {
    private var counters: [Token: Int] = [:]
    
    func leftActivate(token: Token) {
        counters[token] = 0
        checkAndPropagate(token)
    }
    
    func rightActivate(fact: Fact) {
        for token in leftParent!.tokens {
            if testPass(token, fact) {
                let wasZero = (counters[token] == 0)
                counters[token]! += 1
                
                if wasZero {
                    // Now exists - propagate
                    propagate(token)
                }
            }
        }
    }
    
    func rightRetract(fact: Fact) {
        for token in leftParent!.tokens {
            if testPass(token, fact) {
                counters[token]! -= 1
                
                if counters[token] == 0 {
                    // No longer exists - remove
                    removeFromChildren(token)
                }
            }
        }
    }
}
\end{lstlisting}

\section{Pattern Optimization}

\subsection{Compile-Time Optimization}

\begin{lstlisting}[language=Swift]
class PatternOptimizer {
    func optimize(_ pattern: Pattern) -> Pattern {
        var optimized = pattern
        
        // 1. Constant folding
        optimized = foldConstants(optimized)
        
        // 2. Predicate simplification
        optimized = simplifyPredicates(optimized)
        
        // 3. Reorder constraints by selectivity
        optimized = reorderConstraints(optimized)
        
        return optimized
    }
    
    private func foldConstants(_ pattern: Pattern) -> Pattern {
        // Replace (test (> 10 5)) with (test TRUE)
        // ...
    }
    
    private func reorderConstraints(_ pattern: Pattern) -> Pattern {
        // Put most selective constraints first
        let sorted = pattern.constraints.sorted { c1, c2 in
            estimateSelectivity(c1) < estimateSelectivity(c2)
        }
        return Pattern(template: pattern.template, constraints: sorted)
    }
}
\end{lstlisting>

\section{Conclusioni del Capitolo}

\subsection{Punti Chiave}

\begin{enumerate}
\item \textbf{Constraint system} flessibile con enum
\item \textbf{Multifield matching} con backtracking
\item \textbf{Unification} preserva semantica CLIPS
\item \textbf{Test nodes} per constraint inter-elemento
\item \textbf{Exists/forall} con counter-based logic
\end{enumerate}

\subsection{Fine Parte IV}

Con questo si conclude la Parte IV sull'implementazione SLIPS. La Parte V copre sviluppo, performance e futuro.

\subsection{Letture Consigliate}

\begin{itemize}
\item CLIPS Source - \texttt{pattern.c}, \texttt{prcdrpsr.c}
\item Swift Pattern Matching
\end{itemize}
