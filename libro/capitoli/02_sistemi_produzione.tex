% Capitolo 2: Sistemi a Produzione

\chapter{Sistemi a Produzione}
\label{cap:sistemi_produzione}

\section{Introduzione ai Production Systems}

Un \textit{sistema a produzione} (production system) è un modello computazionale per la rappresentazione e l'esecuzione della conoscenza basato su regole. Inventato da Allen Newell e Herbert Simon alla fine degli anni '50, rappresenta uno dei paradigmi fondamentali dell'intelligenza artificiale simbolica.

\subsection{Definizione Formale}

\begin{definizione}[Sistema a Produzione]
Un sistema a produzione è una quadrupla $\mathcal{P} = \langle WM, PM, CS, \sigma \rangle$ dove:
\begin{itemize}
\item $WM$ (Working Memory) è l'insieme dei \textit{fatti} attualmente noti
\item $PM$ (Production Memory) è l'insieme delle \textit{regole} di produzione
\item $CS$ (Conflict Set) è l'insieme delle regole applicabili
\item $\sigma$ (Conflict Resolution Strategy) è la strategia di selezione
\end{itemize}
\end{definizione}

\subsubsection{Working Memory}

La working memory $WM$ è un insieme dinamico di \textit{working memory elements} (WME):

\begin{equation}
WM = \{w_1, w_2, \ldots, w_m\}
\end{equation}

dove ogni $w_i$ è un fatto atomico della forma:

\begin{equation}
w_i = \text{predicato}(\text{arg}_1, \text{arg}_2, \ldots, \text{arg}_n)
\end{equation}

\begin{esempio}[Fatti in Working Memory]
In un sistema di gestione universitaria:
\begin{align*}
w_1 &= \text{studente}(\text{id}: 12345, \text{nome}: \text{"Mario Rossi"}, \text{anno}: 3)\\
w_2 &= \text{esame}(\text{studente}: 12345, \text{corso}: \text{"AI"}, \text{voto}: 28)\\
w_3 &= \text{corso}(\text{nome}: \text{"AI"}, \text{crediti}: 9, \text{anno}: 3)
\end{align*}
\end{esempio}

\subsubsection{Production Memory}

La production memory $PM$ è un insieme statico di regole:

\begin{equation}
PM = \{r_1, r_2, \ldots, r_n\}
\end{equation}

Ogni regola $r_i$ ha la forma:

\begin{equation}
r_i: \text{LHS}_i \Rightarrow \text{RHS}_i
\end{equation}

dove:
\begin{itemize}
\item $\text{LHS}_i$ (Left-Hand Side) è la \textit{condizione} o \textit{pattern}
\item $\text{RHS}_i$ (Right-Hand Side) è l'\textit{azione} da eseguire
\end{itemize}

\begin{esempio}[Regola di Produzione]
Regola per assegnare la lode:
\begin{equation*}
\begin{split}
\text{LHS}: &\quad \exists s \in WM: \text{studente}(s.\text{id}, s.\text{nome}, s.\text{anno})\\
&\quad \land \exists e \in WM: \text{esame}(e.\text{studente} = s.\text{id}, e.\text{corso}, e.\text{voto} \geq 30)\\
\text{RHS}: &\quad \text{assert}(\text{lode}(s.\text{id}, e.\text{corso}))
\end{split}
\end{equation*}
\end{esempio}

\subsection{Il Ciclo Recognize-Act}

L'esecuzione di un sistema a produzione segue il \textit{ciclo recognize-act}:

\begin{algorithm}[H]
\caption{Ciclo Recognize-Act}
\label{alg:recognize_act}
\begin{algorithmic}[1]
\Require Working Memory $WM$, Production Memory $PM$, Strategia $\sigma$
\State $halt \gets \text{false}$
\While{$\neg halt$}
    \State $CS \gets \text{Match}(WM, PM)$ \Comment{Phase: Match}
    \If{$CS = \emptyset$}
        \State $halt \gets \text{true}$ \Comment{Nessuna regola applicabile}
    \Else
        \State $r^* \gets \sigma(CS)$ \Comment{Phase: Conflict Resolution}
        \State $\text{Execute}(r^*.RHS)$ \Comment{Phase: Act}
        \State $WM \gets \text{Update}(WM, r^*.RHS)$ \Comment{Modifica WM}
    \EndIf
\EndWhile
\end{algorithmic}
\end{algorithm}

\subsubsection{Fase di Match}

La fase di match determina il \textit{conflict set}:

\begin{equation}
CS = \{(r, \theta) \mid r \in PM \land \theta \text{ unifica } r.LHS \text{ con } WM\}
\end{equation}

dove $\theta$ è una \textit{sostituzione} (binding) che mappa variabili in $r.LHS$ a valori in $WM$.

\begin{definizione}[Istanziazione]
Un'istanziazione è una coppia $(r, \theta)$ dove:
\begin{itemize}
\item $r$ è una regola in $PM$
\item $\theta: Var(r.LHS) \to Val(WM)$ è una sostituzione
\item $\theta(r.LHS)$ è vero in $WM$
\end{itemize}
\end{definizione}

\subsubsection{Fase di Conflict Resolution}

La strategia $\sigma$ seleziona una singola istanziazione da $CS$:

\begin{equation}
\sigma: 2^{PM \times \Theta} \to PM \times \Theta
\end{equation}

Strategie comuni includono:

\begin{enumerate}
\item \textbf{Depth}: LIFO --- ultima regola matchata viene eseguita per prima
\item \textbf{Breadth}: FIFO --- prima regola matchata viene eseguita per prima  
\item \textbf{Simplicity}: preferisce regole con meno condizioni
\item \textbf{Complexity}: preferisce regole con più condizioni
\item \textbf{LEX}: (Least Recently Activated) ordina per novità dei fatti
\item \textbf{MEA}: (Most Recently Activated) preferisce fatti nuovi
\end{enumerate}

\subsubsection{Fase di Act}

L'esecuzione del RHS può:

\begin{itemize}
\item \textbf{Asserire} nuovi fatti: $WM \gets WM \cup \{w_{\text{new}}\}$
\item \textbf{Ritrarre} fatti esistenti: $WM \gets WM \setminus \{w_{\text{old}}\}$
\item \textbf{Modificare} fatti: combinazione di retract e assert
\item \textbf{Eseguire} side effects (I/O, chiamate funzioni)
\end{itemize}

\section{Semantica Formale}

\subsection{Stati e Transizioni}

Formalizziamo la semantica operazionale come sistema di transizioni:

\begin{definizione}[Stato del Sistema]
Uno stato è una coppia $s = (WM, A)$ dove:
\begin{itemize}
\item $WM$ è la working memory corrente
\item $A$ è l'agenda (insieme ordinato di istanziazioni attive)
\end{itemize}
\end{definizione}

\begin{definizione}[Relazione di Transizione]
La relazione $\to \subseteq S \times S$ definisce le transizioni:
\begin{equation}
(WM, A) \xrightarrow{r, \theta} (WM', A')
\end{equation}
significa che eseguendo l'istanziazione $(r, \theta)$ si passa da stato $(WM, A)$ a $(WM', A')$.
\end{definizione}

\subsection{Regole di Inferenza}

Definiamo le regole che governano le transizioni:

\paragraph{Regola MATCH}

\begin{equation}
\frac{
  r \in PM \quad \theta \vDash r.LHS[WM] \quad (r, \theta) \notin A
}{
  (WM, A) \to (WM, A \cup \{(r, \theta)\})
}
\end{equation}

Significato: se una regola $r$ matcha con sostituzione $\theta$ e non è già nell'agenda, viene aggiunta.

\paragraph{Regola FIRE}

\begin{equation}
\frac{
  (r, \theta) = \max_\sigma A \quad WM' = \text{exec}(r.RHS, \theta, WM)
}{
  (WM, A) \to (WM', A \setminus \{(r, \theta)\})
}
\end{equation}

Significato: l'istanziazione con priorità massima secondo $\sigma$ viene eseguita, modificando $WM$ e venendo rimossa da $A$.

\paragraph{Regola RETRACT}

\begin{equation}
\frac{
  w \in WM \quad A' = \{(r, \theta) \in A \mid w \notin \text{support}(r, \theta)\}
}{
  (WM \setminus \{w\}, A) \to (WM \setminus \{w\}, A')
}
\end{equation}

Significato: ritrarre un fatto $w$ rimuove dall'agenda tutte le istanziazioni che dipendevano da $w$.

\subsection{Terminazione e Correttezza}

\begin{teorema}[Terminazione]
Un sistema a produzione termina se e solo se esiste un $k \in \mathbb{N}$ tale che dopo $k$ passi:
\begin{equation}
\text{Match}(WM_k, PM) = \emptyset
\end{equation}
\end{teorema}

\begin{proof}
($\Rightarrow$) Se il sistema termina, per definizione nessuna regola è applicabile nell'ultimo stato.

($\Leftarrow$) Se $CS = \emptyset$, l'algoritmo \ref{alg:recognize_act} imposta $halt = \text{true}$ e termina.
\end{proof}

\begin{warningbox}[Non Determinismo]
In generale, sistemi a produzione possono essere \textbf{non-deterministici}: l'ordine di esecuzione dipende dalla strategia $\sigma$ e può influenzare il risultato finale.
\end{warningbox}

\begin{teorema}[Confluenza]
Un sistema a produzione è \textit{confluente} se per ogni coppia di esecuzioni $e_1, e_2$ partendo dallo stesso stato iniziale:
\begin{equation}
e_1(s_0) = WM_1 \land e_2(s_0) = WM_2 \Rightarrow WM_1 = WM_2
\end{equation}
\end{teorema}

\begin{osservazione}
La confluenza è una proprietà desiderabile ma NON garantita in generale. CLIPS offre meccanismi (salience, strategie) per controllare il comportamento.
\end{osservazione}

\section{Pattern Matching e Unificazione}

\subsection{Pattern e Template}

Un \textit{pattern} è un'espressione che può contenere:

\begin{itemize}
\item \textbf{Costanti}: valori fissi che devono matchare esattamente
\item \textbf{Variabili}: simboli che vengono legati (bound) a valori
\item \textbf{Wildcard}: segnaposto che matchano qualsiasi valore
\item \textbf{Predicati}: test condizionali sui valori
\end{itemize}

\begin{esempio}[Pattern CLIPS]
\begin{lstlisting}[language=CLIPS]
(persona (nome ?n) (eta ?e&:(>= ?e 18)))
\end{lstlisting}

Questo pattern matcha ogni fatto \texttt{persona} dove:
\begin{itemize}
\item \texttt{nome} viene legato alla variabile \texttt{?n}
\item \texttt{eta} viene legato a \texttt{?e}, con vincolo $\text{eta} \geq 18$
\end{itemize}
\end{esempio}

\subsection{Unificazione}

\begin{definizione}[Unificazione]
Date due espressioni $e_1$ ed $e_2$, l'\textit{unificazione} è una sostituzione $\theta$ tale che:
\begin{equation}
\theta(e_1) = \theta(e_2)
\end{equation}

Se tale $\theta$ esiste, $e_1$ ed $e_2$ sono \textit{unificabili}.
\end{definizione}

L'algoritmo classico di unificazione di Robinson procede ricorsivamente:

\begin{algorithm}[H]
\caption{Unify($e_1$, $e_2$, $\theta$)}
\begin{algorithmic}[1]
\If{$e_1 = e_2$}
    \State \Return $\theta$ \Comment{Identici}
\ElsIf{$e_1$ e' variabile}
    \State \Return $\text{unify\_var}(e_1, e_2, \theta)$
\ElsIf{$e_2$ e' variabile}
    \State \Return $\text{unify\_var}(e_2, e_1, \theta)$
\ElsIf{$e_1 = f(a_1, \ldots, a_n)$ e $e_2 = g(b_1, \ldots, b_m)$}
    \If{$f \neq g$ o $n \neq m$}
        \State \Return $\bot$ \Comment{Fallimento}
    \Else
        \For{$i = 1$ to $n$}
            \State $\theta \gets \text{Unify}(a_i, b_i, \theta)$
            \If{$\theta = \bot$}
                \State \Return $\bot$
            \EndIf
        \EndFor
        \State \Return $\theta$
    \EndIf
\Else
    \State \Return $\bot$ \Comment{Tipo incompatibile}
\EndIf
\end{algorithmic}
\end{algorithm}

\subsection{Multi-Pattern Matching}

Una regola con $k$ condizioni richiede match simultaneo:

\begin{equation}
\text{LHS} = C_1 \land C_2 \land \cdots \land C_k
\end{equation}

dove ogni $C_i$ è un pattern. Una sostituzione $\theta$ soddisfa LHS se:

\begin{equation}
\forall i \in [1, k]: \exists w \in WM: \theta(C_i) = w
\end{equation}

\begin{osservazione}[Complessità Naïve]
Enumerare tutte le possibili combinazioni richiede:
\begin{equation}
O\left(\binom{|WM|}{k}\right) = O\left(\frac{|WM|^k}{k!}\right) \approx O(|WM|^k)
\end{equation}
confronti. Con $|WM| = 10000$ e $k = 5$, otteniamo $10^{20}$ operazioni!
\end{osservazione}

\section{Controllo del Flusso}

\subsection{Forward Chaining}

CLIPS implementa \textit{forward chaining} (data-driven):

\begin{equation}
\text{Fatti} + \text{Regole} \xRightarrow{\text{inferenza}} \text{Nuovi Fatti}
\end{equation}

Il processo parte dai dati osservati e applica regole per derivare conclusioni.

\begin{esempio}[Forward Chaining]
Dato:
\begin{itemize}
\item Fatto: "Piove"
\item Regola: "Se piove $\Rightarrow$ la strada è bagnata"
\end{itemize}

Il sistema inferisce: "La strada è bagnata"
\end{esempio}

\subsection{Backward Chaining (Cenni)}

Per completezza, menzioniamo il \textit{backward chaining} (goal-driven):

\begin{equation}
\text{Goal} + \text{Regole} \xRightarrow{\text{ricerca}} \text{Fatti Necessari}
\end{equation}

CLIPS non implementa backward chaining nativamente, ma può essere simulato.

\subsection{Refraction}

\begin{definizione}[Refraction]
Una regola già eseguita con un dato binding $\theta$ non viene rieseguita con lo stesso $\theta$ finché i fatti che la supportano non cambiano.
\end{definizione}

Implementazione:
\begin{equation}
\text{fired} = \{(r, \theta) \mid (r, \theta) \text{ è stata eseguita}\}
\end{equation}

\begin{equation}
CS' = CS \setminus \text{fired}
\end{equation}

\section{Salience e Priorità}

\subsection{Definizione di Salience}

In CLIPS, ogni regola ha una \textit{salience} (salienza):

\begin{equation}
\text{salience}: PM \to \mathbb{Z}
\end{equation}

dove valori più alti indicano priorità maggiore.

\begin{esempio}[Dichiarazione Salience]
\begin{lstlisting}[language=CLIPS]
(defrule regola-urgente
  (declare (salience 100))
  (condizione-critica ?x)
  =>
  (azione-immediata ?x))
\end{lstlisting}
\end{esempio}

\subsection{Ordinamento nell'Agenda}

L'agenda ordina le istanziazioni secondo:

\begin{equation}
(r_1, \theta_1) \prec_A (r_2, \theta_2) \Leftrightarrow 
\begin{cases}
\text{salience}(r_1) > \text{salience}(r_2) & \text{o}\\
\text{salience}(r_1) = \text{salience}(r_2) \land \sigma((r_1, \theta_1), (r_2, \theta_2)) & 
\end{cases}
\end{equation}

dove $\sigma$ è la strategia di conflict resolution.

\section{Conditional Elements}

\subsection{NOT Conditional Element}

Il CE \texttt{not} implementa negazione per assenza:

\begin{equation}
\text{not}(P) \text{ è vero in } WM \Leftrightarrow \nexists w \in WM: w \text{ matcha } P
\end{equation}

\begin{esempio}[Uso di NOT]
\begin{lstlisting}[language=CLIPS]
(defrule nessun-esame-superato
  (studente (id ?s))
  (not (esame (studente ?s) (voto ?v&:(>= ?v 18))))
  =>
  (printout t "Studente " ?s " non ha superato esami" crlf))
\end{lstlisting}
\end{esempio}

\subsection{EXISTS Conditional Element}

Il CE \texttt{exists} implementa quantificatore esistenziale:

\begin{equation}
\text{exists}(P) \text{ è vero in } WM \Leftrightarrow \exists w \in WM: w \text{ matcha } P
\end{equation}

\textbf{Differenza con pattern normale}: \texttt{exists} non introduce binding, verifica solo l'esistenza.

\subsection{OR Conditional Element}

Il CE \texttt{or} implementa disgiunzione:

\begin{equation}
\text{or}(P_1, P_2, \ldots, P_n) \Leftrightarrow P_1 \lor P_2 \lor \cdots \lor P_n
\end{equation}

CLIPS espande \texttt{or} in regole multiple (una per branch):

\begin{esempio}[Espansione OR]
\begin{lstlisting}[language=CLIPS]
(defrule check
  (or (tipo-A (id ?x))
      (tipo-B (id ?x)))
  (altro (id ?x))
  =>
  (azione ?x))
\end{lstlisting}

Viene espanso in:
\begin{lstlisting}[language=CLIPS]
(defrule check-A
  (tipo-A (id ?x))
  (altro (id ?x))
  =>
  (azione ?x))

(defrule check-B
  (tipo-B (id ?x))
  (altro (id ?x))
  =>
  (azione ?x))
\end{lstlisting}
\end{esempio}

\section{Vantaggi e Svantaggi}

\subsection{Vantaggi dei Production Systems}

\begin{enumerate}
\item \textbf{Modularità}
\begin{itemize}
\item Regole indipendenti e componibili
\item Facile aggiungere/rimuovere conoscenza
\item Manutenzione incrementale
\end{itemize}

\item \textbf{Trasparenza}
\begin{itemize}
\item Regole leggibili da esperti del dominio
\item Spiegazione del ragionamento (trace)
\item Debugging facilitato
\end{itemize}

\item \textbf{Separazione Conoscenza-Controllo}
\begin{itemize}
\item Conoscenza: nelle regole (dichiarativo)
\item Controllo: nel motore (procedurale)
\item Modifica senza riprogrammazione
\end{itemize}

\item \textbf{Scalabilità (con RETE)}
\begin{itemize}
\item Match incrementale efficiente
\item Gestione di grandi basi di conoscenza
\item Performance prevedibili
\end{itemize}
\end{enumerate}

\subsection{Svantaggi e Limiti}

\begin{enumerate}
\item \textbf{Problema del Match}
\begin{itemize}
\item Complessità intrinseca elevata
\item Richiede ottimizzazioni sofisticate (RETE)
\item Consumo di memoria per nodi intermedi
\end{itemize}

\item \textbf{Opacità del Controllo}
\begin{itemize}
\item Difficile predire ordine di esecuzione
\item Debugging complesso per interazioni regole
\item Possibile non-determinismo
\end{itemize}

\item \textbf{Rappresentazione Limitata}
\begin{itemize}
\item Difficoltà con conoscenza probabilistica
\item Mancanza di apprendimento automatico
\item Incertezza gestita in modo ad-hoc
\end{itemize}

\item \textbf{Knowledge Acquisition Bottleneck}
\begin{itemize}
\item Estrazione conoscenza da esperti è costosa
\item Validazione e testing complessi
\item Manutenzione nel tempo
\end{itemize}
\end{enumerate}

\section{Confronto con Altri Paradigmi}

\subsection{Production Systems vs Sistemi Procedurali}

\begin{table}[h]
\centering
\begin{tabular}{@{}lll@{}}
\toprule
\textbf{Aspetto} & \textbf{Production System} & \textbf{Procedurale} \\
\midrule
Controllo & Data-driven & Control-flow esplicito \\
Modularità & Alta (regole indipendenti) & Bassa (chiamate funzioni) \\
Ordine & Determinato da engine & Determinato da programmatore \\
Manutenibilità & Alta (regole isolate) & Media (dipendenze) \\
Performance & Variabile (dipende da RETE) & Prevedibile \\
Debugging & Complesso (emergent behavior) & Diretto (stack trace) \\
\bottomrule
\end{tabular}
\caption{Confronto Production Systems vs Programmazione Procedurale}
\label{tab:prod_vs_proc}
\end{table}

\subsection{Production Systems vs Sistemi Logici (Prolog)}

\begin{table}[h]
\centering
\begin{tabular}{@{}lll@{}}
\toprule
\textbf{Aspetto} & \textbf{CLIPS} & \textbf{Prolog} \\
\midrule
Paradigma & Forward chaining & Backward chaining \\
Controllo & Data-driven & Goal-driven \\
Matching & RETE (ottimizzato) & Unificazione (naïve) \\
Backtracking & No & Sì (automatico) \\
Modifiche WM & Esplicite (assert/retract) & Implicite (fail) \\
Persistenza & Fatti persistono & Backtrack annulla \\
\bottomrule
\end{tabular}
\caption{Confronto CLIPS vs Prolog}
\label{tab:clips_vs_prolog}
\end{table}

\section{Domini Applicativi}

I sistemi a produzione sono particolarmente adatti per:

\subsection{Configurazione e Pianificazione}

\begin{itemize}
\item Configurazione di sistemi complessi (hardware, software)
\item Pianificazione di azioni (robotica, logistica)
\item Scheduling di risorse limitate
\end{itemize}

\textbf{Esempio}: XCON/R1 (Digital Equipment) configurava sistemi VAX con migliaia di componenti. Risparmi stimati: \$40M/anno negli anni '80.

\subsection{Diagnosi e Troubleshooting}

\begin{itemize}
\item Diagnosi medica (MYCIN, Internist)
\item Diagnosi guasti in sistemi tecnici
\item Analisi cause-radice (root cause analysis)
\end{itemize}

\subsection{Monitoraggio e Controllo}

\begin{itemize}
\item Monitoraggio processi industriali
\item Sistemi di allarme intelligenti
\item Controllo qualità in produzione
\end{itemize}

\subsection{Business Rules}

\begin{itemize}
\item Validazione transazioni finanziarie
\item Approvazione workflow
\item Compliance e audit
\item Pricing dinamico
\end{itemize}

\section{Evoluzione Storica}

\subsection{Timeline}

\begin{table}[h]
\centering
\small
\begin{tabular}{@{}ll@{}}
\toprule
\textbf{Anno} & \textbf{Milestone} \\
\midrule
1956 & Logic Theorist (Newell \& Simon) - primo sistema a regole \\
1972 & MYCIN (Stanford) - sistema esperto medico \\
1979 & Algoritmo RETE (Forgy) - breakthrough performance \\
1981 & OPS5 - primo sistema RETE pubblico \\
1984 & CLIPS - NASA inizia sviluppo \\
1985 & CLIPS 1.0 - prima release pubblica \\
1991 & CLIPS 5.0 - aggiunta orientazione a oggetti \\
2002 & CLIPS 6.2 - stabilizzazione architettura \\
2017 & CLIPS 6.4 - ultima major release \\
2025 & SLIPS 1.0 - traduzione Swift \\
\bottomrule
\end{tabular}
\caption{Timeline evoluzione sistemi a produzione}
\label{tab:timeline}
\end{table}

\subsection{Declino e Rinascita}

Dopo l'entusiasmo degli anni '80, i sistemi esperti subirono un declino ("AI winter") per:
\begin{itemize}
\item Aspettative non realistiche
\item Limiti nella rappresentazione di incertezza
\item Costi elevati di sviluppo e manutenzione
\item Avvento di machine learning
\end{itemize}

Tuttavia, nel XXI secolo si assiste a una rinascita come:
\begin{itemize}
\item \textbf{Business Rules Engines}: per compliance e governance
\item \textbf{Complex Event Processing}: in sistemi real-time
\item \textbf{Hybrid Systems}: combinati con ML per spiegabilità
\item \textbf{Sistemi Critici}: dove trasparenza e verificabilità sono essenziali
\end{itemize}

\section{Conclusioni del Capitolo}

In questo capitolo abbiamo:

\begin{itemize}
\item Definito formalmente i sistemi a produzione
\item Introdotto il ciclo recognize-act
\item Presentato pattern matching e unificazione
\item Analizzato il problema della complessità
\item Contestualizzato storicamente CLIPS e SLIPS
\end{itemize}

Nel prossimo capitolo approfondiremo i fondamenti logici necessari per comprendere la semantica formale dei sistemi a regole.

\vspace{1cm}

\begin{successbox}[Punti Chiave]
\begin{itemize}
\item I production systems separano conoscenza (regole) e controllo (engine)
\item Il pattern matching ha complessità $O(n \cdot m^k)$ naïve
\item L'algoritmo RETE riduce a $O(n \cdot m)$ con tecniche incrementali
\item CLIPS è lo standard de facto, SLIPS ne offre versione type-safe in Swift
\end{itemize}
\end{successbox}

