% Capitolo 18: Implementazione RETE in SLIPS

\chapter{SLIPS RETE: Network Implementation}
\label{cap:slips_rete}

\section{Introduzione}

Questo capitolo mostra l'implementazione Swift della rete RETE, cuore del pattern matching di SLIPS.

\section{Node Hierarchy}

\begin{lstlisting}[language=Swift]
protocol ReteNode: AnyObject {
    var id: Int { get }
    var children: [ReteNode] { get set }
    func activate(token: Token)
}

// Alpha Network
class AlphaNode: ReteNode {
    let id: Int
    var children: [ReteNode] = []
    var test: AlphaTest?
    
    func activate(token: Token) {
        guard evaluateTest(token) else { return }
        for child in children {
            child.activate(token: token)
        }
    }
}

class AlphaMemory: ReteNode {
    let id: Int
    var children: [ReteNode] = []
    var facts: Set<Fact> = []
    
    func add(_ fact: Fact) {
        facts.insert(fact)
        notifyBeta(fact)
    }
}

// Beta Network
class JoinNode: ReteNode {
    let id: Int
    var children: [ReteNode] = []
    weak var leftParent: BetaMemory?
    weak var rightParent: AlphaMemory?
    var joinTests: [JoinTest] = []
    
    func leftActivate(token: Token) {
        guard let right = rightParent else { return }
        for fact in right.facts {
            if testPass(token, fact) {
                let newToken = token.extend(with: fact)
                propagate(newToken)
            }
        }
    }
    
    func rightActivate(fact: Fact) {
        guard let left = leftParent else { return }
        for token in left.tokens {
            if testPass(token, fact) {
                let newToken = token.extend(with: fact)
                propagate(newToken)
            }
        }
    }
}

class BetaMemory: ReteNode {
    var tokens: Set<Token> = []
    
    func add(_ token: Token) {
        tokens.insert(token)
        for child in children {
            child.activate(token: token)
        }
    }
}

class ProductionNode: ReteNode {
    let rule: Defrule
    var activations: Set<Activation> = []
    
    func activate(token: Token) {
        let activation = Activation(rule: rule, token: token)
        activations.insert(activation)
        agenda.add(activation)
    }
}
\end{lstlisting}

\section{Network Builder}

\begin{lstlisting}[language=Swift]
class NetworkBuilder {
    private var alphaNodes: [String: AlphaNode] = [:]
    private var nextNodeID = 0
    
    func buildNetwork(for rule: Defrule) -> ProductionNode {
        var currentBeta: ReteNode = dummyTopNode
        
        for pattern in rule.patterns {
            // Build alpha part
            let alphaMemory = buildAlphaNetwork(for: pattern)
            
            // Build join
            let joinNode = createJoinNode(
                left: currentBeta as! BetaMemory,
                right: alphaMemory,
                tests: extractJoinTests(pattern)
            )
            
            // Beta memory after join
            let betaMemory = BetaMemory(id: nextNodeID)
            nextNodeID += 1
            joinNode.children.append(betaMemory)
            
            currentBeta = betaMemory
        }
        
        // Production node
        let prodNode = ProductionNode(id: nextNodeID, rule: rule)
        nextNodeID += 1
        currentBeta.children.append(prodNode)
        
        return prodNode
    }
    
    private func buildAlphaNetwork(for pattern: Pattern) -> AlphaMemory {
        let key = pattern.template
        
        // Get or create type node
        if let existing = alphaNodes[key] {
            return findOrCreateAlphaMemory(under: existing, pattern: pattern)
        }
        
        let typeNode = AlphaNode(id: nextNodeID)
        nextNodeID += 1
        alphaNodes[key] = typeNode
        
        return buildAlphaChain(typeNode, pattern: pattern)
    }
    
    private func buildAlphaChain(_ node: AlphaNode, pattern: Pattern) -> AlphaMemory {
        var current = node
        
        // Add test nodes for each constraint
        for constraint in pattern.constraints where constraint.isIntraElement {
            let testNode = AlphaNode(id: nextNodeID)
            nextNodeID += 1
            testNode.test = AlphaTest(constraint: constraint)
            current.children.append(testNode)
            current = testNode
        }
        
        // Alpha memory at end
        let memory = AlphaMemory(id: nextNodeID)
        nextNodeID += 1
        current.children.append(memory)
        
        return memory
    }
}
\end{lstlisting}

\section{Hash Join Optimization}

\begin{lstlisting}[language=Swift]
class HashJoinNode: JoinNode {
    private var leftIndex: [Value: Set<Token>] = [:]
    private var rightIndex: [Value: Set<Fact>] = [:]
    private let joinVariable: String
    
    override func leftActivate(token: Token) {
        guard let value = token.binding[joinVariable] else { return }
        
        // Add to index
        leftIndex[value, default: []].insert(token)
        
        // Lookup in right index
        if let rightMatches = rightIndex[value] {
            for fact in rightMatches {
                let newToken = token.extend(with: fact)
                propagate(newToken)
            }
        }
    }
    
    override func rightActivate(fact: Fact) {
        guard let value = fact.slots[joinVariable] else { return }
        
        // Add to index
        rightIndex[value, default: []].insert(fact)
        
        // Lookup in left index
        if let leftMatches = leftIndex[value] {
            for token in leftMatches {
                let newToken = token.extend(with: fact)
                propagate(newToken)
            }
        }
    }
}
\end{lstlisting}

\section{Negative Nodes}

\begin{lstlisting}[language=Swift]
class NegativeNode: ReteNode {
    private var counters: [Token: Int] = [:]
    
    func leftActivate(token: Token) {
        counters[token] = 0
        
        // Check right memory
        guard let right = rightParent else { return }
        for fact in right.facts {
            if testPass(token, fact) {
                counters[token]! += 1
            }
        }
        
        // Propagate if count = 0
        if counters[token] == 0 {
            propagate(token)
        }
    }
    
    func rightActivate(fact: Fact) {
        guard let left = leftParent else { return }
        for token in left.tokens {
            if testPass(token, fact) {
                counters[token]! += 1
                if counters[token] == 1 {
                    // Era 0, ora non più - ritira
                    removeFromChildren(token)
                }
            }
        }
    }
    
    func rightRetract(fact: Fact) {
        guard let left = leftParent else { return }
        for token in left.tokens {
            if testPass(token, fact) {
                counters[token]! -= 1
                if counters[token] == 0 {
                    // Ora soddisfatto - propaga
                    propagate(token)
                }
            }
        }
    }
}
\end{lstlisting}

\section{Token Management}

\begin{lstlisting}[language=Swift]
struct Token: Hashable {
    let facts: [Fact]
    let bindings: [String: Value]
    
    func extend(with fact: Fact) -> Token {
        var newFacts = facts
        newFacts.append(fact)
        
        var newBindings = bindings
        // Extract new bindings from fact
        // (logic depends on pattern variables)
        
        return Token(facts: newFacts, bindings: newBindings)
    }
    
    func hash(into hasher: inout Hasher) {
        hasher.combine(facts.map(\.id))
    }
    
    static func == (lhs: Token, rhs: Token) -> Bool {
        return lhs.facts.map(\.id) == rhs.facts.map(\.id)
    }
}

// Token Pool for performance
class TokenPool {
    private var pool: [Token] = []
    private let maxSize = 1000
    
    func acquire(facts: [Fact], bindings: [String: Value]) -> Token {
        if let token = pool.popLast() {
            // Reuse (would need mutable token)
            return Token(facts: facts, bindings: bindings)
        }
        return Token(facts: facts, bindings: bindings)
    }
    
    func release(_ token: Token) {
        guard pool.count < maxSize else { return }
        pool.append(token)
    }
}
\end{lstlisting}

\section{Propagation Engine}

\begin{lstlisting}[language=Swift]
class PropagationEngine {
    func assertFact(_ fact: Fact, in network: AlphaNetwork) {
        let typeNode = network.getTypeNode(for: fact.template.name)
        propagateAssert(fact, through: typeNode)
    }
    
    private func propagateAssert(_ fact: Fact, through node: AlphaNode) {
        // Evaluate test
        if let test = node.test {
            guard test.evaluate(fact) else { return }
        }
        
        // Propagate to children
        for child in node.children {
            if let alphaMemory = child as? AlphaMemory {
                alphaMemory.add(fact)
            } else if let alphaNode = child as? AlphaNode {
                propagateAssert(fact, through: alphaNode)
            }
        }
    }
    
    func retractFact(_ fact: Fact, from network: AlphaNetwork) {
        // Find alpha memories containing fact
        for memory in fact.alphaMemories {
            memory.remove(fact)
        }
        
        // Find and remove tokens containing fact
        for token in fact.tokens {
            removeToken(token)
        }
    }
    
    private func removeToken(_ token: Token) {
        // Traverse beta network removing token
        // and dependent tokens/activations
    }
}
\end{lstlisting}

\section{Performance Optimization}

\subsection{Node Sharing}

\begin{lstlisting}[language=Swift]
class SharedNodeRegistry {
    private var alphaNodes: [AlphaNodeKey: AlphaNode] = [:]
    private var joinNodes: [JoinNodeKey: JoinNode] = [:]
    
    func getOrCreateAlphaNode(
        type: String,
        test: AlphaTest?
    ) -> AlphaNode {
        let key = AlphaNodeKey(type: type, test: test)
        
        if let existing = alphaNodes[key] {
            return existing
        }
        
        let node = AlphaNode(id: nextNodeID)
        nextNodeID += 1
        node.test = test
        alphaNodes[key] = node
        return node
    }
}

struct AlphaNodeKey: Hashable {
    let type: String
    let test: AlphaTest?
}
\end{lstlisting}

\section{Conclusioni del Capitolo}

\subsection{Punti Chiave}

\begin{enumerate}
\item RETE implementato con \textbf{protocol-oriented design}
\item \textbf{Weak references} per evitare cicli
\item \textbf{Hash join} per ottimizzazione
\item \textbf{Node sharing} riduce duplicazione
\item Token pool per performance
\end{enumerate}

\subsection{Prossimi Capitoli}

Capitolo~\ref{cap:slips_agenda} mostra l'implementazione dell'agenda in Swift.

\subsection{Letture Consigliate}

\begin{itemize}
\item CLIPS Source - \texttt{drive.c}, \texttt{reteutil.c}
\item Swift Performance - Protocol-Oriented Programming
\end{itemize}
