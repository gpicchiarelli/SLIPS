% Appendice B: Catalogo Built-in Functions

\chapter{Catalogo Completo Built-in Functions}
\label{app:builtin}

\section{Organizzazione}

SLIPS 1.0 implementa 87+ funzioni built-in, organizzate per categoria.

\section{Lista Completa Funzioni Implementate}

\subsection{Matematiche (20 funzioni)}

\texttt{+}, \texttt{-}, \texttt{*}, \texttt{/}, \texttt{div}, \texttt{mod}, \texttt{abs}, \texttt{min}, \texttt{max}, \texttt{sqrt}, \texttt{pow}, \texttt{exp}, \texttt{log}, \texttt{log10}, \texttt{sin}, \texttt{cos}, \texttt{tan}, \texttt{asin}, \texttt{acos}, \texttt{atan}

\subsection{Logiche e Confronto (15 funzioni)}

\texttt{and}, \texttt{or}, \texttt{not}, \texttt{eq}, \texttt{neq}, \texttt{=}, \texttt{<>}, \texttt{<}, \texttt{<=}, \texttt{>}, \texttt{>=}, \texttt{eq*}, \texttt{neq*}, \texttt{<*}, \texttt{>*}

\subsection{Facts Management (12 funzioni)}

\texttt{assert}, \texttt{retract}, \texttt{modify}, \texttt{duplicate}, \texttt{facts}, \texttt{ppfact}, \texttt{fact-index}, \texttt{fact-relation}, \texttt{fact-slot-value}, \texttt{get-fact-list}, \texttt{fact-existp}, \texttt{save-facts}

\subsection{Rules Management (10 funzioni)}

\texttt{rules}, \texttt{ppdefrule}, \texttt{undefrule}, \texttt{refresh}, \texttt{get-defrule-list}, \texttt{matches}, \texttt{list-focus-stack}, \texttt{pop-focus}, \texttt{get-focus}, \texttt{clear-focus-stack}

\subsection{Templates (8 funzioni)}

\texttt{deftemplate}, \texttt{undeftemplate}, \texttt{ppdeftemplate}, \texttt{list-deftemplates}, \texttt{get-deftemplate-list}, \texttt{deftemplate-slot-names}, \texttt{deftemplate-slot-types}, \texttt{deftemplate-slot-range}

\subsection{Modules (5 funzioni)}

\texttt{defmodule}, \texttt{focus}, \texttt{get-current-module}, \texttt{set-current-module}, \texttt{list-defmodules}, \texttt{get-defmodule-list}

\subsection{Agenda e Strategie (10 funzioni)}

\texttt{agenda}, \texttt{run}, \texttt{halt}, \texttt{set-strategy}, \texttt{get-strategy}, \texttt{refresh-agenda}, \texttt{reorder}, \texttt{get-salience-evaluation}, \texttt{set-salience-evaluation}, \texttt{clear}

\subsection{I/O (7 funzioni)}

\texttt{printout}, \texttt{read}, \texttt{readline}, \texttt{format}, \texttt{open}, \texttt{close}, \texttt{get-char}

\subsection{Multifield (10 funzioni)}

\texttt{create\$}, \texttt{length\$}, \texttt{nth\$}, \texttt{rest\$}, \texttt{first\$}, \texttt{insert\$}, \texttt{delete\$}, \texttt{replace\$}, \texttt{subseq\$}, \texttt{member\$}

\subsection{Stringhe (8 funzioni)}

\texttt{str-cat}, \texttt{sym-cat}, \texttt{str-length}, \texttt{str-index}, \texttt{sub-string}, \texttt{str-compare}, \texttt{upcase}, \texttt{lowcase}

\subsection{Controllo Flusso (8 funzioni)}

\texttt{bind}, \texttt{progn}, \texttt{if}, \texttt{while}, \texttt{foreach}, \texttt{break}, \texttt{return}, \texttt{switch}

\subsection{Utility (8 funzioni)}

\texttt{gensym}, \texttt{eval}, \texttt{build}, \texttt{load}, \texttt{save}, \texttt{watch}, \texttt{unwatch}, \texttt{dribble-on}

\subsection{Type Predicates (10 funzioni)}

\texttt{numberp}, \texttt{integerp}, \texttt{floatp}, \texttt{stringp}, \texttt{symbolp}, \texttt{multifieldp}, \texttt{evenp}, \texttt{oddp}, \texttt{pointerp}, \texttt{lexemep}

\section{Dettagli Funzioni Chiave}

\subsection{Multifield Operations}

\begin{lstlisting}[language=CLIPS]
;; create$ - Crea multifield
(create$ 1 2 3)          ; => (1 2 3)

;; length$ - Lunghezza
(length$ (create$ a b c))  ; => 3

;; nth$ - Accesso per indice (1-based)
(nth$ 2 (create$ a b c))   ; => b

;; insert$ - Inserimento
(insert$ (create$ a c) 2 b)  ; => (a b c)

;; delete$ - Cancellazione
(delete$ (create$ a b c) 2 2)  ; => (a c)

;; subseq$ - Sotto-sequenza
(subseq$ (create$ a b c d) 2 3)  ; => (b c)

;; member$ - Ricerca
(member$ b (create$ a b c))  ; => 2 (posizione)
\end{lstlisting}

\subsection{String Operations}

\begin{lstlisting}[language=CLIPS]
;; str-cat - Concatenazione
(str-cat "Hello" " " "World")  ; => "Hello World"

;; sub-string - Sotto-stringa (1-based)
(sub-string 2 5 "Hello")  ; => "ello"

;; str-index - Ricerca
(str-index "lo" "Hello")  ; => 4

;; upcase/lowcase
(upcase "hello")  ; => "HELLO"
(lowcase "WORLD")  ; => "world"
\end{lstlisting}

\textit{Per riferimenti completi, vedere Appendice~\ref{app:api} e documentazione online.}

