% SLIPS - Swift Language Implementation of Production Systems
% Libro Accademico - Versione 1.0
% Copyright (c) 2025 - Creative Commons BY-SA 4.0

\documentclass[12pt,a4paper,twoside,openright]{book}

% Pacchetti essenziali
\usepackage[utf8]{inputenc}
\usepackage[italian]{babel}
\usepackage[T1]{fontenc}
\usepackage{lmodern}

% Layout e geometria
\usepackage[a4paper,top=3cm,bottom=3cm,left=3.5cm,right=2.5cm,headheight=15pt]{geometry}
\usepackage{setspace}
\onehalfspacing

% Matematica
\usepackage{amsmath,amssymb,amsthm}
\usepackage{mathtools}

% Figure e grafica
\usepackage{graphicx}
\usepackage{tikz}
\usetikzlibrary{shapes,arrows,positioning,calc,trees,graphs}
\usepackage{pgfplots}
\pgfplotsset{compat=1.18}

% Codice sorgente
\usepackage{listings}
\usepackage{listingsutf8}
\usepackage{xcolor}

% Definizione colori
\definecolor{swiftKeyword}{RGB}{170,13,145}
\definecolor{swiftString}{RGB}{196,26,22}
\definecolor{swiftComment}{RGB}{87,166,74}
\definecolor{swiftNumber}{RGB}{28,0,207}
\definecolor{cKeyword}{RGB}{0,0,255}
\definecolor{codeBackground}{RGB}{248,248,248}

% Configurazione Swift
\lstdefinelanguage{Swift}{
  keywords={class,struct,enum,func,var,let,if,else,switch,case,return,public,private,internal,static,final,override,protocol,extension,import,guard,for,while,repeat,try,catch,throw,throws,inout,mutating,self,init,deinit,nil,true,false,where,in,is,as,async,await},
  keywordstyle=\color{swiftKeyword}\bfseries,
  sensitive=true,
  comment=[l]{//},
  morecomment=[s]{/*}{*/},
  commentstyle=\color{swiftComment}\itshape,
  stringstyle=\color{swiftString},
  string=[b]",
  numbers=left,
  numberstyle=\tiny\color{gray},
  stepnumber=1,
  numbersep=8pt,
  showstringspaces=false,
  breaklines=true,
  frame=lines,
  backgroundcolor=\color{codeBackground},
  basicstyle=\small\ttfamily,
  captionpos=b
}

% Configurazione C
\lstdefinelanguage{C}{
  keywords={auto,break,case,char,const,continue,default,do,double,else,enum,extern,float,for,goto,if,int,long,register,return,short,signed,sizeof,static,struct,switch,typedef,union,unsigned,void,volatile,while},
  keywordstyle=\color{cKeyword}\bfseries,
  sensitive=true,
  comment=[l]{//},
  morecomment=[s]{/*}{*/},
  commentstyle=\color{swiftComment}\itshape,
  stringstyle=\color{swiftString},
  string=[b]",
  numbers=left,
  numberstyle=\tiny\color{gray},
  stepnumber=1,
  numbersep=8pt,
  showstringspaces=false,
  breaklines=true,
  frame=lines,
  backgroundcolor=\color{codeBackground},
  basicstyle=\small\ttfamily,
  captionpos=b
}

% Configurazione CLIPS
\lstdefinelanguage{CLIPS}{
  morekeywords={defrule,deftemplate,deffacts,defmodule,assert,retract,printout,bind,if,then,else,while,and,or,not,exists,test,declare,salience,import,export,focus},
  keywordstyle=\color{swiftKeyword}\bfseries,
  sensitive=false,
  comment=[l]{;},
  commentstyle=\color{swiftComment}\itshape,
  stringstyle=\color{swiftString},
  string=[b]",
  showstringspaces=false,
  breaklines=true,
  frame=lines,
  backgroundcolor=\color{codeBackground},
  basicstyle=\small\ttfamily,
  captionpos=b,
  extendedchars=true,
  inputencoding=utf8
}

\lstset{
  language=Swift,
  extendedchars=true,
  inputencoding=utf8,
  literate=
    {à}{{\`a}}1
    {è}{{\`e}}1
    {é}{{\'e}}1
    {ì}{{\`{\i}}}1
    {ò}{{\`o}}1
    {ù}{{\`u}}1
    {À}{{\`A}}1
    {È}{{\`E}}1
    {É}{{\'E}}1
    {Ì}{{\`I}}1
    {Ò}{{\`O}}1
    {Ù}{{\`U}}1
}

% Tabelle
\usepackage{booktabs}
\usepackage{tabularx}
\usepackage{multirow}
\usepackage{longtable}

% Collegamenti
\usepackage[hidelinks]{hyperref}
\usepackage{url}

% Bibliografia
\usepackage[style=alphabetic,backend=biber]{biblatex}
\addbibresource{bibliografia.bib}

% Indice analitico
\usepackage{makeidx}
\makeindex

% Intestazioni personalizzate
\usepackage{fancyhdr}
\pagestyle{fancy}
\fancyhf{}
\fancyhead[LE]{\nouppercase{\leftmark}}
\fancyhead[RO]{\nouppercase{\rightmark}}
\fancyfoot[C]{\thepage}
\renewcommand{\headrulewidth}{0.4pt}

% Teoremi e definizioni
\theoremstyle{definition}
\newtheorem{definizione}{Definizione}[chapter]
\newtheorem{teorema}{Teorema}[chapter]
\newtheorem{proposizione}{Proposizione}[chapter]
\newtheorem{lemma}{Lemma}[chapter]
\newtheorem{corollario}{Corollario}[chapter]

\theoremstyle{remark}
\newtheorem{osservazione}{Osservazione}[chapter]
\newtheorem{esempio}{Esempio}[chapter]
\newtheorem{nota}{Nota}[chapter]

% Algoritmi
\usepackage{algorithm}
\usepackage{algpseudocode}
\renewcommand{\algorithmicrequire}{\textbf{Input:}}
\renewcommand{\algorithmicensure}{\textbf{Output:}}

% Box colorati
\usepackage[most]{tcolorbox}
\newtcolorbox{infobox}[1][]{
  colback=blue!5!white,
  colframe=blue!75!black,
  fonttitle=\bfseries,
  title=#1
}

\newtcolorbox{warningbox}[1][]{
  colback=orange!5!white,
  colframe=orange!75!black,
  fonttitle=\bfseries,
  title=#1
}

\newtcolorbox{successbox}[1][]{
  colback=green!5!white,
  colframe=green!75!black,
  fonttitle=\bfseries,
  title=#1
}

% Metadati
\title{%
  {\Huge\textbf{SLIPS}}\\[0.5cm]
  {\Large Swift Language Implementation of Production Systems}\\[1cm]
  {\large Teoria, Implementazione e Guida Tecnica}\\[0.5cm]
  {\large Un Sistema Esperto Moderno in Swift 6.2}
}
\author{Contributori SLIPS}
\date{Dicembre 2025 --- Versione 1.0}

% Informazioni PDF
\hypersetup{
  pdftitle={SLIPS - Swift Language Implementation of Production Systems},
  pdfauthor={Contributori SLIPS},
  pdfsubject={Sistemi Esperti, Algoritmo RETE, Swift Programming},
  pdfkeywords={SLIPS, CLIPS, Swift, Expert Systems, Production Systems, RETE Algorithm}
}

\begin{document}

% Frontespizio
\frontmatter
\maketitle

% Pagina copyright
\cleardoublepage
\thispagestyle{empty}
\vspace*{\fill}
\begin{center}
\begin{minipage}{0.8\textwidth}
\small

\textbf{SLIPS --- Swift Language Implementation of Production Systems}

Teoria, Implementazione e Guida Tecnica

\bigskip

Copyright \copyright\ 2025 Contributori SLIPS

\bigskip

\textbf{Licenza Creative Commons BY-SA 4.0}

Quest'opera è distribuita con Licenza Creative Commons Attribuzione - Condividi allo stesso modo 4.0 Internazionale.

Per leggere una copia della licenza visita:\\
\url{http://creativecommons.org/licenses/by-sa/4.0/}

\bigskip

\textbf{Codice Sorgente}

Il codice sorgente di SLIPS è disponibile su GitHub:\\
\url{https://github.com/gpicchiarelli/SLIPS}

Licenza del codice: MIT License

\bigskip

\textbf{Riferimenti}

Basato su CLIPS (C Language Integrated Production System) versione 6.4.2\\
Sviluppato originariamente dalla NASA\\
\url{https://www.clipsrules.net/}

\bigskip

Prima edizione: Dicembre 2025

\end{minipage}
\end{center}
\vspace*{\fill}

% Dedica
\cleardoublepage
\thispagestyle{empty}
\vspace*{\fill}
\begin{center}
\textit{A tutti coloro che credono che la conoscenza debba essere libera,\\
condivisa e accessibile a tutti.}

\bigskip

\textit{Ai pionieri dell'intelligenza artificiale simbolica\\
che hanno gettato le fondamenta su cui costruiamo oggi.}
\end{center}
\vspace*{\fill}

% Prefazione
\chapter*{Prefazione}
\addcontentsline{toc}{chapter}{Prefazione}

Questo libro nasce dall'ambizioso progetto di tradurre integralmente il motore di produzione CLIPS (C Language Integrated Production System), sviluppato dalla NASA negli anni '80, nel moderno linguaggio Swift 6.2. Non si tratta di una semplice wrapper o binding, ma di una traduzione semantica fedele, file per file, che preserva la logica e gli algoritmi originali adattandoli ai paradigmi di programmazione sicura e moderna del XXI secolo.

SLIPS ha raggiunto la versione 0.96 con 165 funzioni builtin implementate, un tasso di successo dei test del 97.8%, e supporto completo per il core engine con pattern matching avanzato, sistema di moduli, e architettura RETE dual (legacy + explicit). Il progetto rappresenta oltre 11.000 linee di codice Swift production-ready, mantenendo piena compatibilità semantica con CLIPS 6.4.2.

SLIPS (Swift Language Implementation of Production Systems) rappresenta più di un semplice port tecnologico: è un ponte tra due ere dell'informatica. Da un lato, la robustezza e l'efficienza dei sistemi esperti degli anni '80, dall'altro le garanzie di sicurezza della memoria e type safety delle tecnologie contemporanee.

\section*{Perché questo libro}

Esistono numerosi testi sui sistemi esperti e su CLIPS in particolare, ma manca una risorsa che unisca:

\begin{itemize}
\item La teoria formale dei sistemi a produzione
\item L'algoritmo RETE nella sua completezza matematica
\item Un'implementazione moderna e commentata
\item Una guida pratica all'uso e all'estensione
\item Best practices per lo sviluppo di sistemi esperti
\end{itemize}

Questo volume colma tale lacuna fornendo una trattazione completa che va dalla teoria formale all'implementazione pratica, passando per decisioni architetturali, ottimizzazioni e pattern di progettazione.

\section*{A chi si rivolge}

Il libro è pensato per:

\begin{itemize}
\item \textbf{Studenti} di informatica interessati all'intelligenza artificiale simbolica
\item \textbf{Ricercatori} che lavorano con sistemi a regole e ragionamento automatico
\item \textbf{Sviluppatori} Swift che desiderano comprendere sistemi complessi
\item \textbf{Ingegneri} che devono tradurre codice legacy in linguaggi moderni
\item \textbf{Architetti software} interessati a pattern di sistemi esperti
\end{itemize}

\section*{Struttura del libro}

Il volume è organizzato in cinque parti:

\textbf{Parte I: Fondamenti Teorici} introduce i sistemi esperti basati su regole, la logica proposizionale e del primo ordine, e i fondamenti matematici necessari.

\textbf{Parte II: L'Algoritmo RETE} presenta in dettaglio l'algoritmo di pattern matching inventato da Charles Forgy, con analisi della complessità e dimostrazione di correttezza.

\textbf{Parte III: Architettura di CLIPS} analizza il design originale del motore C, le strutture dati, e le decisioni implementative.

\textbf{Parte IV: Implementazione SLIPS} descrive la traduzione in Swift, le scelte architetturali, i pattern utilizzati e le ottimizzazioni.

\textbf{Parte V: Sviluppo e Manutenzione} fornisce guide pratiche per estendere SLIPS, aggiungere funzionalità, e contribuire al progetto.

\section*{Convenzioni utilizzate}

Nel corso del testo:
\begin{itemize}
\item Il codice C è mostrato in \texttt{font monospaziato blu}
\item Il codice Swift è mostrato in \texttt{font monospaziato viola}
\item Le formule matematiche seguono la notazione standard
\item I concetti chiave sono evidenziati in \textbf{grassetto}
\item Le definizioni formali sono in box dedicati
\end{itemize}

\section*{Ringraziamenti}

Si ringraziano:
\begin{itemize}
\item Gary Riley e il team della NASA per CLIPS
\item La comunità Swift per gli strumenti eccellenti
\item Tutti i contributori al progetto SLIPS
\item I revisori di questo volume
\end{itemize}

\vspace{1cm}
\begin{flushright}
I Contributori SLIPS\\
Dicembre 2025
\end{flushright}

\tableofcontents
\listoffigures
\listoftables
\lstlistoflistings

% Corpo del testo
\mainmatter

% PARTE I: FONDAMENTI TEORICI
\part{Fondamenti Teorici dei Sistemi Esperti}

% Capitolo 1: Introduzione

\chapter{Introduzione}
\label{cap:introduzione}

\section{Motivazione e Contesto Storico}

I sistemi esperti rappresentano una delle branche più rilevanti dell'intelligenza artificiale simbolica, nata negli anni '60 e sviluppatasi intensamente negli anni '70 e '80. A differenza degli approcci sub-simbolici moderni basati su reti neurali, i sistemi esperti si fondano sulla rappresentazione esplicita della conoscenza attraverso regole logiche e fatti.

\subsection{L'Era d'Oro dei Sistemi Esperti}

Negli anni '80, i sistemi esperti promettevano di rivoluzionare numerosi settori:

\begin{itemize}
\item \textbf{Medicina}: MYCIN (Stanford, 1976) per la diagnosi di infezioni batteriche
\item \textbf{Chimica}: DENDRAL per l'identificazione di strutture molecolari
\item \textbf{Configurazione}: XCON/R1 (Digital Equipment Corporation) per sistemi informatici
\item \textbf{Finanza}: sistemi di trading e valutazione del rischio
\item \textbf{Industria}: controllo processi e manutenzione predittiva
\end{itemize}

Il successo di questi sistemi dipendeva criticamente dall'efficienza del \textit{pattern matching}, ovvero la capacità di confrontare velocemente migliaia di regole con migliaia di fatti per determinare quali regole fossero applicabili.

\subsection{Il Problema della Scalabilità}

Consideriamo un sistema con:
\begin{itemize}
\item $n$ regole, ciascuna con $k$ condizioni
\item $m$ fatti nel working memory
\end{itemize}

Un approccio naïve richiederebbe $O(n \cdot m^k)$ confronti ad ogni ciclo. Per un sistema realistico con $n=1000$, $m=10000$, $k=3$, si avrebbero $10^{15}$ operazioni --- chiaramente impraticabile.

\begin{infobox}[Il Problema del Match]
Data una base di conoscenza con $n$ regole e $m$ fatti, trovare efficientemente tutte le istanziazioni valide delle regole è un problema computazionalmente complesso che cresce esponenzialmente con il numero di condizioni per regola.
\end{infobox}

\subsection{La Soluzione: L'Algoritmo RETE}

Nel 1979, Charles L. Forgy, allora dottorando alla Carnegie Mellon University, sviluppò l'algoritmo RETE (dal latino \textit{rete}, rete) che riduceva drasticamente la complessità del pattern matching sfruttando due intuizioni fondamentali:

\begin{enumerate}
\item \textbf{Continuità temporale}: tra un ciclo e il successivo, la maggior parte dei fatti rimane invariata
\item \textbf{Similarità strutturale}: molte regole condividono pattern comuni
\end{enumerate}

L'algoritmo RETE costruisce una rete di nodi che memorizza risultati intermedi di match, evitando di rifare da zero i confronti ad ogni ciclo. Questa tecnica, nota come \textit{incremental pattern matching}, riduce la complessità a $O(n \cdot m)$ nel caso medio.

\section{CLIPS: Un'Implementazione di Riferimento}

\subsection{Origini e Sviluppo}

CLIPS (C Language Integrated Production System) fu sviluppato a partire dal 1984 presso il Johnson Space Center della NASA. L'obiettivo era creare un sistema esperto:

\begin{itemize}
\item \textbf{Portabile}: scritto in C ANSI, eseguibile su qualsiasi piattaforma
\item \textbf{Efficiente}: basato sull'algoritmo RETE ottimizzato
\item \textbf{Estendibile}: architettura modulare e API per funzioni utente
\item \textbf{Completo}: supporto per programmazione procedurale, a oggetti e a regole
\item \textbf{Gratuito}: dominio pubblico, liberamente utilizzabile
\end{itemize}

\subsection{Adozione e Impatto}

CLIPS è diventato lo standard de facto per sistemi esperti grazie a:

\begin{itemize}
\item Ampia diffusione accademica (insegnamento AI)
\item Uso in missioni spaziali NASA
\item Adozione industriale in settori critici
\item Comunità attiva e documentazione estesa
\item Oltre 30 anni di sviluppo e manutenzione
\end{itemize}

Alla versione 6.4.2 (2017), CLIPS comprende circa 150.000 linee di codice C, 300+ funzioni built-in, e supporta paradigmi multipli.

\section{SLIPS: Motivazioni del Progetto}

\subsection{Perché una Traduzione in Swift}

Nonostante l'eccellenza di CLIPS, esistono motivazioni valide per una traduzione moderna:

\begin{enumerate}
\item \textbf{Memory Safety}
\begin{itemize}
\item C richiede gestione manuale della memoria (malloc/free)
\item Rischi di buffer overflow, use-after-free, memory leak
\item Swift offre ARC (Automatic Reference Counting) e type safety
\end{itemize}

\item \textbf{Interoperabilità con Ecosistema Apple}
\begin{itemize}
\item Integrazione nativa con iOS/macOS/watchOS
\item Accesso diretto a framework moderni (SwiftUI, Combine, etc.)
\item Performance ottimizzate per architetture Apple Silicon
\end{itemize}

\item \textbf{Espressività del Linguaggio}
\begin{itemize}
\item Enum con associated values per union types
\item Pattern matching nativo
\item Generics e protocolli per astrazione
\item Closures e higher-order functions
\end{itemize}

\item \textbf{Strumenti di Sviluppo}
\begin{itemize}
\item Xcode con debugging avanzato
\item Swift Package Manager per dipendenze
\item Testing framework integrato
\item Profiler Instruments per performance
\end{itemize}
\end{enumerate}

\subsection{Obiettivi di SLIPS}

Il progetto SLIPS si pone obiettivi ambiziosi:

\begin{definizione}[Equivalenza Semantica]
SLIPS deve produrre, per ogni programma CLIPS valido, lo stesso output e comportamento osservabile del motore C originale, preservando:
\begin{itemize}
\item Ordine di firing delle regole
\item Valori calcolati e fatti asseriti
\item Gestione dell'agenda e strategie
\item Semantica dei costrutti (deftemplate, defrule, etc.)
\end{itemize}
\end{definizione}

\begin{definizione}[Fedeltà Strutturale]
La traduzione deve mantenere una corrispondenza 1:1 tra file C e file Swift, preservando nomi di funzioni, strutture dati, e flusso algoritmico, adattando solo dove necessario per idiomi Swift.
\end{definizione}

\subsection{Non-Obiettivi}

È importante chiarire cosa SLIPS \textit{non} è:

\begin{itemize}
\item \textbf{Non è un wrapper}: non usa FFI o binding per chiamare CLIPS C
\item \textbf{Non è una riscrittura}: non semplifica o modernizza gli algoritmi
\item \textbf{Non è un'interpretazione}: non cambia la semantica di CLIPS
\item \textbf{Non è ottimizzato prematuramente}: preserva strutture anche se non idiomatiche
\end{itemize}

L'obiettivo è una \textit{traduzione conservativa} che permetta di:
\begin{enumerate}
\item Studiare il codice CLIPS con strumenti moderni
\item Verificare formalmente la correttezza della traduzione
\item Migrare applicazioni CLIPS esistenti
\item Estendere CLIPS con funzionalità Swift-native
\end{enumerate}

\section{Contributi di Questo Volume}

Questo libro offre diversi contributi originali:

\subsection{Contributi Teorici}

\begin{itemize}
\item \textbf{Formalizzazione matematica completa} dell'algoritmo RETE con dimostrazioni di correttezza e complessità
\item \textbf{Analisi comparativa} tra diverse varianti di RETE (TREAT, RETE-II, etc.)
\item \textbf{Caratterizzazione formale} della semantica operazionale di CLIPS
\item \textbf{Teoremi di equivalenza} tra implementazione C e Swift
\end{itemize}

\subsection{Contributi Implementativi}

\begin{itemize}
\item \textbf{Mappatura sistematica} C $\to$ Swift per ogni pattern comune
\item \textbf{Catalogo di pattern} di traduzione per:
  \begin{itemize}
  \item Union types $\to$ Enum con associated values
  \item Malloc/free $\to$ ARC e value semantics
  \item Puntatori $\to$ Reference types e Optional
  \item Macro $\to$ Computed properties e generics
  \end{itemize}
\item \textbf{Test suite} con equivalenza verificata
\item \textbf{Documentazione} del design space esplorato
\end{itemize}

\subsection{Contributi Pedagogici}

\begin{itemize}
\item \textbf{Spiegazione didattica} dell'algoritmo RETE con esempi completi
\item \textbf{Guida passo-passo} alla costruzione di un motore a regole
\item \textbf{Best practices} per sistemi esperti in Swift
\item \textbf{Casi di studio} reali con analisi dettagliata
\end{itemize}

\section{Metodologia}

\subsection{Approccio alla Traduzione}

La traduzione di SLIPS segue una metodologia rigorosa:

\begin{enumerate}
\item \textbf{Studio del Codice C}: Analisi approfondita del file sorgente CLIPS
\item \textbf{Identificazione Invarianti}: Determinazione delle proprietà da preservare
\item \textbf{Mappatura Tipi}: Traduzione strutture dati C in Swift idiomatico
\item \textbf{Traduzione Logica}: Conversione algoritmi con dimostrazione di equivalenza
\item \textbf{Testing}: Verifica comportamentale su test suite estesa
\item \textbf{Documentazione}: Annotazione con riferimenti al codice C originale
\end{enumerate}

\subsection{Criteri di Accettazione}

Ogni modulo tradotto deve soddisfare:

\begin{itemize}
\item \textbf{Compilazione}: Build clean senza warning
\item \textbf{Test funzionali}: Tutti i test verdi
\item \textbf{Test equivalenza}: Output identico a CLIPS C su suite di riferimento
\item \textbf{Documentazione}: Commenti con riferimenti a file C originale
\item \textbf{Code review}: Verifica da parte di almeno un altro contributore
\end{itemize}

\section{Struttura di Questo Volume}

\subsection{Parte I: Fondamenti Teorici}

Nei capitoli 2--4 introduciamo i fondamenti matematici e logici necessari:

\begin{itemize}
\item Logica proposizionale e del primo ordine
\item Sistemi di riscrittura e calcolo
\item Rappresentazione della conoscenza
\item Inferenza forward e backward
\end{itemize}

\subsection{Parte II: L'Algoritmo RETE}

I capitoli 5--10 costituiscono il cuore teorico del volume:

\begin{itemize}
\item Formulazione matematica dell'algoritmo
\item Rete alpha per filtering
\item Rete beta per join
\item Analisi di complessità e dimostrazioni
\item Ottimizzazioni e varianti
\end{itemize}

\subsection{Parte III: Architettura CLIPS}

I capitoli 11--15 analizzano il design di CLIPS C:

\begin{itemize}
\item Strutture dati fondamentali
\item Gestione della memoria
\item Sistema di agenda
\item Moduli e visibilità
\item Estensibilità e UDF
\end{itemize}

\subsection{Parte IV: Implementazione SLIPS}

I capitoli 16--22 descrivono l'implementazione Swift:

\begin{itemize}
\item Architettura generale
\item Core engine (Environment, Evaluator)
\item RETE network in Swift
\item Agenda e conflict resolution
\item Sistema di moduli
\item Pattern matching avanzato
\item Test e validazione
\end{itemize}

\subsection{Parte V: Guida allo Sviluppo}

I capitoli 23--27 forniscono guide pratiche:

\begin{itemize}
\item Estendere SLIPS con nuove funzioni
\item Best practices per regole efficienti
\item Ottimizzazione e profiling
\item Debugging e troubleshooting
\item Direzioni future e roadmap
\end{itemize}

\section{Come Leggere Questo Libro}

\subsection{Percorsi di Lettura Consigliati}

\subsubsection{Per lo Studente}

Se stai studiando sistemi esperti per la prima volta:
\begin{enumerate}
\item Leggi Parte I (Fondamenti) per acquisire background
\item Studia Parte II (RETE) per comprendere l'algoritmo
\item Esplora esempi in Parte IV per vedere applicazioni pratiche
\item Consulta Appendice C per esercizi
\end{enumerate}

\subsubsection{Per il Ricercatore}

Se ti interessa l'aspetto teorico:
\begin{enumerate}
\item Focus su Parte II (RETE) per formalizzazione matematica
\item Studio Capitolo 9 per analisi di complessità
\item Capitolo 10 per ottimizzazioni e varianti
\item Bibliografia per approfondimenti
\end{enumerate}

\subsubsection{Per lo Sviluppatore Swift}

Se vuoi usare o estendere SLIPS:
\begin{enumerate}
\item Panoramica Parte I e II per comprendere il dominio
\item Parte IV (Implementazione SLIPS) in dettaglio
\item Parte V (Sviluppo) per guide pratiche
\item Appendici A-B per riferimento API
\end{enumerate}

\subsubsection{Per l'Ingegnere di Traduzione}

Se stai traducendo altro codice C in Swift:
\begin{enumerate}
\item Capitolo 16 (Architettura SLIPS) per metodologia
\item Capitoli 17--22 per pattern di traduzione
\item Capitolo 24 (Best Practices) per linee guida
\item Studio dei file \texttt{.swift} commentati
\end{enumerate}

\subsection{Prerequisiti}

Il lettore ideale possiede:

\textbf{Prerequisiti Essenziali}:
\begin{itemize}
\item Programmazione: conoscenza di almeno un linguaggio (C, Swift, Java, Python)
\item Strutture dati: liste, alberi, grafi, hash table
\item Algoritmi: complessità computazionale, notazione Big-O
\end{itemize}

\textbf{Prerequisiti Utili}:
\begin{itemize}
\item Logica matematica: proposizionale e del primo ordine
\item Sistemi: compilatori, interpreti, macchine astratte
\item Swift: sintassi base, type system, memory model
\end{itemize}

\textbf{Non Richiesti} (spiegati nel testo):
\begin{itemize}
\item Esperienza con CLIPS
\item Conoscenza di AI simbolica
\item Background in sistemi esperti
\end{itemize}

\section{Notazione e Convenzioni}

\subsection{Notazione Matematica}

Nel corso del volume utilizziamo:

\begin{itemize}
\item $\mathbb{N}, \mathbb{Z}, \mathbb{R}$: insiemi numerici standard
\item $\langle x, y \rangle$: coppia ordinata
\item $f: A \to B$: funzione da $A$ a $B$
\item $x \in S$: appartenenza all'insieme
\item $S \subseteq T$: sottoinsieme
\item $|S|$: cardinalità dell'insieme
\item $\forall x \in S$: quantificatore universale
\item $\exists x \in S$: quantificatore esistenziale
\item $P \Rightarrow Q$: implicazione logica
\item $P \Leftrightarrow Q$: equivalenza logica
\end{itemize}

\subsection{Notazione per Complessità}

\begin{itemize}
\item $O(f(n))$: upper bound asintotico (caso peggiore)
\item $\Omega(f(n))$: lower bound asintotico
\item $\Theta(f(n))$: tight bound (upper e lower coincidono)
\item $O(f(n))$ ammortizzato: costo medio su sequenza di operazioni
\end{itemize}

\subsection{Convenzioni Tipografiche}

\begin{itemize}
\item \texttt{monospace}: codice, nomi di file, comandi
\item \textbf{grassetto}: concetti chiave, definizioni
\item \textit{corsivo}: enfasi, termini tecnici al primo uso
\item \textsf{sans-serif}: nomi di tool e applicazioni
\end{itemize}

\section{Risorse Online}

\subsection{Repository SLIPS}

Il codice sorgente completo è disponibile su:

\begin{center}
\url{https://github.com/gpicchiarelli/SLIPS}
\end{center}

Include:
\begin{itemize}
\item Codice Swift (35 file, 8000+ LOC)
\item Test suite (91 test)
\item Sorgenti CLIPS C di riferimento
\item Documentazione HTML
\item Issue tracker per bug e feature request
\end{itemize}

\subsection{Sito CLIPS Originale}

Documentazione e risorse CLIPS ufficiali:

\begin{center}
\url{https://www.clipsrules.net/}
\end{center}

Include:
\begin{itemize}
\item CLIPS Reference Manual (800+ pagine)
\item User's Guide
\item Tutorial e esempi
\item Mailing list e forum
\end{itemize}

\subsection{Documentazione Swift}

Risorse per il linguaggio Swift:

\begin{center}
\url{https://docs.swift.org/}
\end{center}

\section{Note Sulla Versione}

Questo libro documenta:

\begin{itemize}
\item \textbf{CLIPS}: versione 6.4.2 (ultima stabile)
\item \textbf{SLIPS}: versione 1.0 (prima release)
\item \textbf{Swift}: versione 6.2
\item \textbf{Platform}: macOS 15+ (Sequoia)
\end{itemize}

Le versioni future di SLIPS potrebbero divergere nei dettagli implementativi, ma i concetti teorici e architetturali rimangono validi.

\section{Organizzazione del Materiale}

Ogni capitolo è strutturato come segue:

\begin{enumerate}
\item \textbf{Introduzione}: overview e motivazione
\item \textbf{Teoria}: formalizzazione matematica
\item \textbf{Algoritmi}: pseudocodice e spiegazione
\item \textbf{Implementazione}: codice C e Swift commentato
\item \textbf{Analisi}: complessità, correttezza, ottimizzazioni
\item \textbf{Esempi}: casi d'uso pratici
\item \textbf{Esercizi}: problemi per il lettore (dove appropriato)
\end{enumerate}

\section{Ringraziamenti Estesi}

Si desidera ringraziare:

\textbf{Pionieri Teorici}:
\begin{itemize}
\item Charles L. Forgy per l'algoritmo RETE
\item Allen Newell e Herbert Simon per i production systems
\item Edward Feigenbaum per i sistemi esperti
\end{itemize}

\textbf{Team CLIPS}:
\begin{itemize}
\item Gary Riley (lead developer)
\item Brian Dantes
\item Il team NASA Johnson Space Center
\end{itemize}

\textbf{Comunità Swift}:
\begin{itemize}
\item Chris Lattner e il core team
\item La community open source
\end{itemize}

\textbf{Contributori SLIPS}:
\begin{itemize}
\item Tutti i developer che hanno contribuito codice
\item I reviewer che hanno verificato la traduzione
\item Gli utenti che hanno segnalato bug
\end{itemize}

\section{Feedback e Contributi}

Questo libro è un documento vivente. Feedback, correzioni e suggerimenti sono benvenuti:

\begin{itemize}
\item \textbf{Errata}: segnalare errori tecnici o refusi
\item \textbf{Miglioramenti}: suggerire chiarimenti o aggiunte
\item \textbf{Esempi}: proporre nuovi casi di studio
\item \textbf{Esercizi}: contribuire problemi e soluzioni
\end{itemize}

Contatti:
\begin{itemize}
\item GitHub Issues: \url{https://github.com/gpicchiarelli/SLIPS/issues}
\item Pull Request per correzioni
\item Discussioni: GitHub Discussions
\end{itemize}

\section{Roadmap del Volume}

Nei prossimi capitoli esploreremo:

\textbf{Capitolo 2} introduce i sistemi a produzione dal punto di vista formale, definendo working memory, production memory, e ciclo recognize-act.

\textbf{Capitolo 3} copre la logica formale necessaria per comprendere la semantica delle regole: logica proposizionale, del primo ordine, e unificazione.

\textbf{Capitolo 4} tratta la rappresentazione della conoscenza: frame, slot, template, e come codificare domini applicativi.

\textbf{Capitoli 5--10} costituiscono il cuore del volume, con la teoria completa di RETE: dall'intuizione alla formalizzazione matematica, dalle strutture dati agli algoritmi, dall'analisi di complessità alle ottimizzazioni avanzate.

\textbf{Capitoli 11--15} analizzano CLIPS C in dettaglio, preparando il terreno per la traduzione.

\textbf{Capitoli 16--22} presentano SLIPS: architettura, implementazione, testing, e validazione.

\textbf{Capitoli 23--27} forniscono guide pratiche per sviluppatori che vogliono usare, estendere, o contribuire a SLIPS.

Le \textbf{Appendici} offrono riferimenti rapidi, catalogo completo delle funzioni, esempi estesi, e benchmark di performance.

\vspace{1cm}

Iniziamo ora il nostro viaggio nel mondo affascinante dei sistemi a produzione.


% Capitolo 2: Sistemi a Produzione

\chapter{Sistemi a Produzione}
\label{cap:sistemi_produzione}

\section{Introduzione ai Production Systems}

Un \textit{sistema a produzione} (production system) è un modello computazionale per la rappresentazione e l'esecuzione della conoscenza basato su regole. Inventato da Allen Newell e Herbert Simon alla fine degli anni '50, rappresenta uno dei paradigmi fondamentali dell'intelligenza artificiale simbolica.

\subsection{Definizione Formale}

\begin{definizione}[Sistema a Produzione]
Un sistema a produzione è una quadrupla $\mathcal{P} = \langle WM, PM, CS, \sigma \rangle$ dove:
\begin{itemize}
\item $WM$ (Working Memory) è l'insieme dei \textit{fatti} attualmente noti
\item $PM$ (Production Memory) è l'insieme delle \textit{regole} di produzione
\item $CS$ (Conflict Set) è l'insieme delle regole applicabili
\item $\sigma$ (Conflict Resolution Strategy) è la strategia di selezione
\end{itemize}
\end{definizione}

\subsubsection{Working Memory}

La working memory $WM$ è un insieme dinamico di \textit{working memory elements} (WME):

\begin{equation}
WM = \{w_1, w_2, \ldots, w_m\}
\end{equation}

dove ogni $w_i$ è un fatto atomico della forma:

\begin{equation}
w_i = \text{predicato}(\text{arg}_1, \text{arg}_2, \ldots, \text{arg}_n)
\end{equation}

\begin{esempio}[Fatti in Working Memory]
In un sistema di gestione universitaria:
\begin{align*}
w_1 &= \text{studente}(\text{id}: 12345, \text{nome}: \text{"Mario Rossi"}, \text{anno}: 3)\\
w_2 &= \text{esame}(\text{studente}: 12345, \text{corso}: \text{"AI"}, \text{voto}: 28)\\
w_3 &= \text{corso}(\text{nome}: \text{"AI"}, \text{crediti}: 9, \text{anno}: 3)
\end{align*}
\end{esempio}

\subsubsection{Production Memory}

La production memory $PM$ è un insieme statico di regole:

\begin{equation}
PM = \{r_1, r_2, \ldots, r_n\}
\end{equation}

Ogni regola $r_i$ ha la forma:

\begin{equation}
r_i: \text{LHS}_i \Rightarrow \text{RHS}_i
\end{equation}

dove:
\begin{itemize}
\item $\text{LHS}_i$ (Left-Hand Side) è la \textit{condizione} o \textit{pattern}
\item $\text{RHS}_i$ (Right-Hand Side) è l'\textit{azione} da eseguire
\end{itemize}

\begin{esempio}[Regola di Produzione]
Regola per assegnare la lode:
\begin{equation*}
\begin{split}
\text{LHS}: &\quad \exists s \in WM: \text{studente}(s.\text{id}, s.\text{nome}, s.\text{anno})\\
&\quad \land \exists e \in WM: \text{esame}(e.\text{studente} = s.\text{id}, e.\text{corso}, e.\text{voto} \geq 30)\\
\text{RHS}: &\quad \text{assert}(\text{lode}(s.\text{id}, e.\text{corso}))
\end{split}
\end{equation*}
\end{esempio}

\subsection{Il Ciclo Recognize-Act}

L'esecuzione di un sistema a produzione segue il \textit{ciclo recognize-act}:

\begin{algorithm}[H]
\caption{Ciclo Recognize-Act}
\label{alg:recognize_act}
\begin{algorithmic}[1]
\Require Working Memory $WM$, Production Memory $PM$, Strategia $\sigma$
\State $halt \gets \text{false}$
\While{$\neg halt$}
    \State $CS \gets \text{Match}(WM, PM)$ \Comment{Phase: Match}
    \If{$CS = \emptyset$}
        \State $halt \gets \text{true}$ \Comment{Nessuna regola applicabile}
    \Else
        \State $r^* \gets \sigma(CS)$ \Comment{Phase: Conflict Resolution}
        \State $\text{Execute}(r^*.RHS)$ \Comment{Phase: Act}
        \State $WM \gets \text{Update}(WM, r^*.RHS)$ \Comment{Modifica WM}
    \EndIf
\EndWhile
\end{algorithmic}
\end{algorithm}

\subsubsection{Fase di Match}

La fase di match determina il \textit{conflict set}:

\begin{equation}
CS = \{(r, \theta) \mid r \in PM \land \theta \text{ unifica } r.LHS \text{ con } WM\}
\end{equation}

dove $\theta$ è una \textit{sostituzione} (binding) che mappa variabili in $r.LHS$ a valori in $WM$.

\begin{definizione}[Istanziazione]
Un'istanziazione è una coppia $(r, \theta)$ dove:
\begin{itemize}
\item $r$ è una regola in $PM$
\item $\theta: Var(r.LHS) \to Val(WM)$ è una sostituzione
\item $\theta(r.LHS)$ è vero in $WM$
\end{itemize}
\end{definizione}

\subsubsection{Fase di Conflict Resolution}

La strategia $\sigma$ seleziona una singola istanziazione da $CS$:

\begin{equation}
\sigma: 2^{PM \times \Theta} \to PM \times \Theta
\end{equation}

Strategie comuni includono:

\begin{enumerate}
\item \textbf{Depth}: LIFO --- ultima regola matchata viene eseguita per prima
\item \textbf{Breadth}: FIFO --- prima regola matchata viene eseguita per prima  
\item \textbf{Simplicity}: preferisce regole con meno condizioni
\item \textbf{Complexity}: preferisce regole con più condizioni
\item \textbf{LEX}: (Least Recently Activated) ordina per novità dei fatti
\item \textbf{MEA}: (Most Recently Activated) preferisce fatti nuovi
\end{enumerate}

\subsubsection{Fase di Act}

L'esecuzione del RHS può:

\begin{itemize}
\item \textbf{Asserire} nuovi fatti: $WM \gets WM \cup \{w_{\text{new}}\}$
\item \textbf{Ritrarre} fatti esistenti: $WM \gets WM \setminus \{w_{\text{old}}\}$
\item \textbf{Modificare} fatti: combinazione di retract e assert
\item \textbf{Eseguire} side effects (I/O, chiamate funzioni)
\end{itemize}

\section{Semantica Formale}

\subsection{Stati e Transizioni}

Formalizziamo la semantica operazionale come sistema di transizioni:

\begin{definizione}[Stato del Sistema]
Uno stato è una coppia $s = (WM, A)$ dove:
\begin{itemize}
\item $WM$ è la working memory corrente
\item $A$ è l'agenda (insieme ordinato di istanziazioni attive)
\end{itemize}
\end{definizione}

\begin{definizione}[Relazione di Transizione]
La relazione $\to \subseteq S \times S$ definisce le transizioni:
\begin{equation}
(WM, A) \xrightarrow{r, \theta} (WM', A')
\end{equation}
significa che eseguendo l'istanziazione $(r, \theta)$ si passa da stato $(WM, A)$ a $(WM', A')$.
\end{definizione}

\subsection{Regole di Inferenza}

Definiamo le regole che governano le transizioni:

\paragraph{Regola MATCH}

\begin{equation}
\frac{
  r \in PM \quad \theta \vDash r.LHS[WM] \quad (r, \theta) \notin A
}{
  (WM, A) \to (WM, A \cup \{(r, \theta)\})
}
\end{equation}

Significato: se una regola $r$ matcha con sostituzione $\theta$ e non è già nell'agenda, viene aggiunta.

\paragraph{Regola FIRE}

\begin{equation}
\frac{
  (r, \theta) = \max_\sigma A \quad WM' = \text{exec}(r.RHS, \theta, WM)
}{
  (WM, A) \to (WM', A \setminus \{(r, \theta)\})
}
\end{equation}

Significato: l'istanziazione con priorità massima secondo $\sigma$ viene eseguita, modificando $WM$ e venendo rimossa da $A$.

\paragraph{Regola RETRACT}

\begin{equation}
\frac{
  w \in WM \quad A' = \{(r, \theta) \in A \mid w \notin \text{support}(r, \theta)\}
}{
  (WM \setminus \{w\}, A) \to (WM \setminus \{w\}, A')
}
\end{equation}

Significato: ritrarre un fatto $w$ rimuove dall'agenda tutte le istanziazioni che dipendevano da $w$.

\subsection{Terminazione e Correttezza}

\begin{teorema}[Terminazione]
Un sistema a produzione termina se e solo se esiste un $k \in \mathbb{N}$ tale che dopo $k$ passi:
\begin{equation}
\text{Match}(WM_k, PM) = \emptyset
\end{equation}
\end{teorema}

\begin{proof}
($\Rightarrow$) Se il sistema termina, per definizione nessuna regola è applicabile nell'ultimo stato.

($\Leftarrow$) Se $CS = \emptyset$, l'algoritmo \ref{alg:recognize_act} imposta $halt = \text{true}$ e termina.
\end{proof}

\begin{warningbox}[Non Determinismo]
In generale, sistemi a produzione possono essere \textbf{non-deterministici}: l'ordine di esecuzione dipende dalla strategia $\sigma$ e può influenzare il risultato finale.
\end{warningbox}

\begin{teorema}[Confluenza]
Un sistema a produzione è \textit{confluente} se per ogni coppia di esecuzioni $e_1, e_2$ partendo dallo stesso stato iniziale:
\begin{equation}
e_1(s_0) = WM_1 \land e_2(s_0) = WM_2 \Rightarrow WM_1 = WM_2
\end{equation}
\end{teorema}

\begin{osservazione}
La confluenza è una proprietà desiderabile ma NON garantita in generale. CLIPS offre meccanismi (salience, strategie) per controllare il comportamento.
\end{osservazione}

\section{Pattern Matching e Unificazione}

\subsection{Pattern e Template}

Un \textit{pattern} è un'espressione che può contenere:

\begin{itemize}
\item \textbf{Costanti}: valori fissi che devono matchare esattamente
\item \textbf{Variabili}: simboli che vengono legati (bound) a valori
\item \textbf{Wildcard}: segnaposto che matchano qualsiasi valore
\item \textbf{Predicati}: test condizionali sui valori
\end{itemize}

\begin{esempio}[Pattern CLIPS]
\begin{lstlisting}[language=CLIPS]
(persona (nome ?n) (età ?e&:(>= ?e 18)))
\end{lstlisting}

Questo pattern matcha ogni fatto \texttt{persona} dove:
\begin{itemize}
\item \texttt{nome} viene legato alla variabile \texttt{?n}
\item \texttt{età} viene legato a \texttt{?e}, con vincolo $\text{età} \geq 18$
\end{itemize}
\end{esempio}

\subsection{Unificazione}

\begin{definizione}[Unificazione]
Date due espressioni $e_1$ ed $e_2$, l'\textit{unificazione} è una sostituzione $\theta$ tale che:
\begin{equation}
\theta(e_1) = \theta(e_2)
\end{equation}

Se tale $\theta$ esiste, $e_1$ ed $e_2$ sono \textit{unificabili}.
\end{definizione}

\begin{algoritmo}[Unificazione di Robinson]
L'algoritmo classico procede ricorsivamente:

\begin{algorithm}[H]
\caption{Unify($e_1$, $e_2$, $\theta$)}
\begin{algorithmic}[1]
\If{$e_1 = e_2$}
    \State \Return $\theta$ \Comment{Identici}
\ElsIf{$e_1$ è variabile}
    \State \Return $\text{unify\_var}(e_1, e_2, \theta)$
\ElsIf{$e_2$ è variabile}
    \State \Return $\text{unify\_var}(e_2, e_1, \theta)$
\ElsIf{$e_1 = f(a_1, \ldots, a_n)$ e $e_2 = g(b_1, \ldots, b_m)$}
    \If{$f \neq g$ o $n \neq m$}
        \State \Return $\bot$ \Comment{Fallimento}
    \Else
        \For{$i = 1$ to $n$}
            \State $\theta \gets \text{Unify}(a_i, b_i, \theta)$
            \If{$\theta = \bot$}
                \State \Return $\bot$
            \EndIf
        \EndFor
        \State \Return $\theta$
    \EndIf
\Else
    \State \Return $\bot$ \Comment{Tipo incompatibile}
\EndIf
\end{algorithmic}
\end{algorithm}
\end{algoritmo}

\subsection{Multi-Pattern Matching}

Una regola con $k$ condizioni richiede match simultaneo:

\begin{equation}
\text{LHS} = C_1 \land C_2 \land \cdots \land C_k
\end{equation}

dove ogni $C_i$ è un pattern. Una sostituzione $\theta$ soddisfa LHS se:

\begin{equation}
\forall i \in [1, k]: \exists w \in WM: \theta(C_i) = w
\end{equation}

\begin{osservazione}[Complessità Naïve]
Enumerare tutte le possibili combinazioni richiede:
\begin{equation}
O\left(\binom{|WM|}{k}\right) = O\left(\frac{|WM|^k}{k!}\right) \approx O(|WM|^k)
\end{equation}
confronti. Con $|WM| = 10000$ e $k = 5$, otteniamo $10^{20}$ operazioni!
\end{osservazione}

\section{Controllo del Flusso}

\subsection{Forward Chaining}

CLIPS implementa \textit{forward chaining} (data-driven):

\begin{equation}
\text{Fatti} + \text{Regole} \xRightarrow{\text{inferenza}} \text{Nuovi Fatti}
\end{equation}

Il processo parte dai dati osservati e applica regole per derivare conclusioni.

\begin{esempio}[Forward Chaining]
Dato:
\begin{itemize}
\item Fatto: "Piove"
\item Regola: "Se piove $\Rightarrow$ la strada è bagnata"
\end{itemize}

Il sistema inferisce: "La strada è bagnata"
\end{esempio}

\subsection{Backward Chaining (Cenni)}

Per completezza, menzioniamo il \textit{backward chaining} (goal-driven):

\begin{equation}
\text{Goal} + \text{Regole} \xRightarrow{\text{ricerca}} \text{Fatti Necessari}
\end{equation}

CLIPS non implementa backward chaining nativamente, ma può essere simulato.

\subsection{Refraction}

\begin{definizione}[Refraction]
Una regola già eseguita con un dato binding $\theta$ non viene rieseguita con lo stesso $\theta$ finché i fatti che la supportano non cambiano.
\end{definizione}

Implementazione:
\begin{equation}
\text{fired} = \{(r, \theta) \mid (r, \theta) \text{ è stata eseguita}\}
\end{equation}

\begin{equation}
CS' = CS \setminus \text{fired}
\end{equation}

\section{Salience e Priorità}

\subsection{Definizione di Salience}

In CLIPS, ogni regola ha una \textit{salience} (salienza):

\begin{equation}
\text{salience}: PM \to \mathbb{Z}
\end{equation}

dove valori più alti indicano priorità maggiore.

\begin{esempio}[Dichiarazione Salience]
\begin{lstlisting}[language=CLIPS]
(defrule regola-urgente
  (declare (salience 100))
  (condizione-critica ?x)
  =>
  (azione-immediata ?x))
\end{lstlisting}
\end{esempio}

\subsection{Ordinamento nell'Agenda}

L'agenda ordina le istanziazioni secondo:

\begin{equation}
(r_1, \theta_1) \prec_A (r_2, \theta_2) \Leftrightarrow 
\begin{cases}
\text{salience}(r_1) > \text{salience}(r_2) & \text{o}\\
\text{salience}(r_1) = \text{salience}(r_2) \land \sigma((r_1, \theta_1), (r_2, \theta_2)) & 
\end{cases}
\end{equation}

dove $\sigma$ è la strategia di conflict resolution.

\section{Conditional Elements}

\subsection{NOT Conditional Element}

Il CE \texttt{not} implementa negazione per assenza:

\begin{equation}
\text{not}(P) \text{ è vero in } WM \Leftrightarrow \nexists w \in WM: w \text{ matcha } P
\end{equation}

\begin{esempio}[Uso di NOT]
\begin{lstlisting}[language=CLIPS]
(defrule nessun-esame-superato
  (studente (id ?s))
  (not (esame (studente ?s) (voto ?v&:(>= ?v 18))))
  =>
  (printout t "Studente " ?s " non ha superato esami" crlf))
\end{lstlisting}
\end{esempio}

\subsection{EXISTS Conditional Element}

Il CE \texttt{exists} implementa quantificatore esistenziale:

\begin{equation}
\text{exists}(P) \text{ è vero in } WM \Leftrightarrow \exists w \in WM: w \text{ matcha } P
\end{equation}

\textbf{Differenza con pattern normale}: \texttt{exists} non introduce binding, verifica solo l'esistenza.

\subsection{OR Conditional Element}

Il CE \texttt{or} implementa disgiunzione:

\begin{equation}
\text{or}(P_1, P_2, \ldots, P_n) \Leftrightarrow P_1 \lor P_2 \lor \cdots \lor P_n
\end{equation}

CLIPS espande \texttt{or} in regole multiple (una per branch):

\begin{esempio}[Espansione OR]
\begin{lstlisting}[language=CLIPS]
(defrule check
  (or (tipo-A (id ?x))
      (tipo-B (id ?x)))
  (altro (id ?x))
  =>
  (azione ?x))
\end{lstlisting}

Viene espanso in:
\begin{lstlisting}[language=CLIPS]
(defrule check-A
  (tipo-A (id ?x))
  (altro (id ?x))
  =>
  (azione ?x))

(defrule check-B
  (tipo-B (id ?x))
  (altro (id ?x))
  =>
  (azione ?x))
\end{lstlisting}
\end{esempio}

\section{Vantaggi e Svantaggi}

\subsection{Vantaggi dei Production Systems}

\begin{enumerate}
\item \textbf{Modularità}
\begin{itemize}
\item Regole indipendenti e componibili
\item Facile aggiungere/rimuovere conoscenza
\item Manutenzione incrementale
\end{itemize}

\item \textbf{Trasparenza}
\begin{itemize}
\item Regole leggibili da esperti del dominio
\item Spiegazione del ragionamento (trace)
\item Debugging facilitato
\end{itemize}

\item \textbf{Separazione Conoscenza-Controllo}
\begin{itemize}
\item Conoscenza: nelle regole (dichiarativo)
\item Controllo: nel motore (procedurale)
\item Modifica senza riprogrammazione
\end{itemize}

\item \textbf{Scalabilità (con RETE)}
\begin{itemize}
\item Match incrementale efficiente
\item Gestione di grandi basi di conoscenza
\item Performance prevedibili
\end{itemize}
\end{enumerate}

\subsection{Svantaggi e Limiti}

\begin{enumerate}
\item \textbf{Problema del Match}
\begin{itemize}
\item Complessità intrinseca elevata
\item Richiede ottimizzazioni sofisticate (RETE)
\item Consumo di memoria per nodi intermedi
\end{itemize}

\item \textbf{Opacità del Controllo}
\begin{itemize}
\item Difficile predire ordine di esecuzione
\item Debugging complesso per interazioni regole
\item Possibile non-determinismo
\end{itemize}

\item \textbf{Rappresentazione Limitata}
\begin{itemize}
\item Difficoltà con conoscenza probabilistica
\item Mancanza di apprendimento automatico
\item Incertezza gestita in modo ad-hoc
\end{itemize}

\item \textbf{Knowledge Acquisition Bottleneck}
\begin{itemize}
\item Estrazione conoscenza da esperti è costosa
\item Validazione e testing complessi
\item Manutenzione nel tempo
\end{itemize}
\end{enumerate}

\section{Confronto con Altri Paradigmi}

\subsection{Production Systems vs Sistemi Procedurali}

\begin{table}[h]
\centering
\begin{tabular}{@{}lll@{}}
\toprule
\textbf{Aspetto} & \textbf{Production System} & \textbf{Procedurale} \\
\midrule
Controllo & Data-driven & Control-flow esplicito \\
Modularità & Alta (regole indipendenti) & Bassa (chiamate funzioni) \\
Ordine & Determinato da engine & Determinato da programmatore \\
Manutenibilità & Alta (regole isolate) & Media (dipendenze) \\
Performance & Variabile (dipende da RETE) & Prevedibile \\
Debugging & Complesso (emergent behavior) & Diretto (stack trace) \\
\bottomrule
\end{tabular}
\caption{Confronto Production Systems vs Programmazione Procedurale}
\label{tab:prod_vs_proc}
\end{table}

\subsection{Production Systems vs Sistemi Logici (Prolog)}

\begin{table}[h]
\centering
\begin{tabular}{@{}lll@{}}
\toprule
\textbf{Aspetto} & \textbf{CLIPS} & \textbf{Prolog} \\
\midrule
Paradigma & Forward chaining & Backward chaining \\
Controllo & Data-driven & Goal-driven \\
Matching & RETE (ottimizzato) & Unificazione (naïve) \\
Backtracking & No & Sì (automatico) \\
Modifiche WM & Esplicite (assert/retract) & Implicite (fail) \\
Persistenza & Fatti persistono & Backtrack annulla \\
\bottomrule
\end{tabular}
\caption{Confronto CLIPS vs Prolog}
\label{tab:clips_vs_prolog}
\end{table}

\section{Domini Applicativi}

I sistemi a produzione sono particolarmente adatti per:

\subsection{Configurazione e Pianificazione}

\begin{itemize}
\item Configurazione di sistemi complessi (hardware, software)
\item Pianificazione di azioni (robotica, logistica)
\item Scheduling di risorse limitate
\end{itemize}

\textbf{Esempio}: XCON/R1 (Digital Equipment) configurava sistemi VAX con migliaia di componenti. Risparmi stimati: \$40M/anno negli anni '80.

\subsection{Diagnosi e Troubleshooting}

\begin{itemize}
\item Diagnosi medica (MYCIN, Internist)
\item Diagnosi guasti in sistemi tecnici
\item Analisi cause-radice (root cause analysis)
\end{itemize}

\subsection{Monitoraggio e Controllo}

\begin{itemize}
\item Monitoraggio processi industriali
\item Sistemi di allarme intelligenti
\item Controllo qualità in produzione
\end{itemize}

\subsection{Business Rules}

\begin{itemize}
\item Validazione transazioni finanziarie
\item Approvazione workflow
\item Compliance e audit
\item Pricing dinamico
\end{itemize}

\section{Evoluzione Storica}

\subsection{Timeline}

\begin{table}[h]
\centering
\small
\begin{tabular}{@{}ll@{}}
\toprule
\textbf{Anno} & \textbf{Milestone} \\
\midrule
1956 & Logic Theorist (Newell \& Simon) - primo sistema a regole \\
1972 & MYCIN (Stanford) - sistema esperto medico \\
1979 & Algoritmo RETE (Forgy) - breakthrough performance \\
1981 & OPS5 - primo sistema RETE pubblico \\
1984 & CLIPS - NASA inizia sviluppo \\
1985 & CLIPS 1.0 - prima release pubblica \\
1991 & CLIPS 5.0 - aggiunta orientazione a oggetti \\
2002 & CLIPS 6.2 - stabilizzazione architettura \\
2017 & CLIPS 6.4 - ultima major release \\
2025 & SLIPS 1.0 - traduzione Swift \\
\bottomrule
\end{tabular}
\caption{Timeline evoluzione sistemi a produzione}
\label{tab:timeline}
\end{table}

\subsection{Declino e Rinascita}

Dopo l'entusiasmo degli anni '80, i sistemi esperti subirono un declino ("AI winter") per:
\begin{itemize}
\item Aspettative non realistiche
\item Limiti nella rappresentazione di incertezza
\item Costi elevati di sviluppo e manutenzione
\item Avvento di machine learning
\end{itemize}

Tuttavia, nel XXI secolo si assiste a una rinascita come:
\begin{itemize}
\item \textbf{Business Rules Engines}: per compliance e governance
\item \textbf{Complex Event Processing}: in sistemi real-time
\item \textbf{Hybrid Systems}: combinati con ML per spiegabilità
\item \textbf{Sistemi Critici}: dove trasparenza e verificabilità sono essenziali
\end{itemize}

\section{Conclusioni del Capitolo}

In questo capitolo abbiamo:

\begin{itemize}
\item Definito formalmente i sistemi a produzione
\item Introdotto il ciclo recognize-act
\item Presentato pattern matching e unificazione
\item Analizzato il problema della complessità
\item Contestualizzato storicamente CLIPS e SLIPS
\end{itemize}

Nel prossimo capitolo approfondiremo i fondamenti logici necessari per comprendere la semantica formale dei sistemi a regole.

\vspace{1cm}

\begin{successbox}[Punti Chiave]
\begin{itemize}
\item I production systems separano conoscenza (regole) e controllo (engine)
\item Il pattern matching ha complessità $O(n \cdot m^k)$ naïve
\item L'algoritmo RETE riduce a $O(n \cdot m)$ con tecniche incrementali
\item CLIPS è lo standard de facto, SLIPS ne offre versione type-safe in Swift
\end{itemize}
\end{successbox}


% Capitolo 3: Logica Formale per Sistemi a Produzione

\chapter{Fondamenti di Logica Formale}
\label{cap:logica_formale}

\section{Introduzione}

I sistemi a produzione si basano su solide fondamenta di logica formale. In questo capitolo esploreremo i principi logici che sottendono il ragionamento automatico, dal calcolo proposizionale alla logica del primo ordine, fornendo gli strumenti matematici necessari per comprendere la correttezza e la completezza dei sistemi esperti.

\subsection{Motivazione}

La logica formale fornisce:
\begin{itemize}
\item Un \textbf{linguaggio preciso} per esprimere conoscenza
\item \textbf{Regole di inferenza} per derivare nuova conoscenza
\item \textbf{Garanzie formali} di correttezza
\item Una \textbf{base teorica} per verificare proprietà del sistema
\end{itemize}

\section{Logica Proposizionale}

\subsection{Sintassi}

\begin{definizione}[Formula Proposizionale]
L'insieme delle formule proposizionali $\mathcal{L}_P$ è definito induttivamente:
\begin{align}
\varphi ::= &\; p \mid \bot \mid \top \mid \notag \\
            &\; \neg \varphi \mid (\varphi \land \varphi) \mid (\varphi \lor \varphi) \mid (\varphi \rightarrow \varphi) \mid (\varphi \leftrightarrow \varphi)
\end{align}
dove $p \in \mathcal{P}$ è una variabile proposizionale, $\bot$ rappresenta falso, $\top$ rappresenta vero.
\end{definizione}

\textbf{Connettivi logici}:
\begin{itemize}
\item $\neg$ (negazione): "non"
\item $\land$ (congiunzione): "e"
\item $\lor$ (disgiunzione): "o"
\item $\rightarrow$ (implicazione): "se... allora"
\item $\leftrightarrow$ (biimplicazione): "se e solo se"
\end{itemize}

\subsection{Semantica}

\begin{definizione}[Interpretazione]
Un'interpretazione (o valutazione) è una funzione:
\begin{equation}
\mathcal{I}: \mathcal{P} \rightarrow \{\text{vero}, \text{falso}\}
\end{equation}
che assegna un valore di verità a ogni variabile proposizionale.
\end{definizione}

\begin{definizione}[Tabella di Verità]
La semantica dei connettivi è definita dalle seguenti tabelle:

\begin{center}
\begin{tabular}{cc|c}
$p$ & $q$ & $p \land q$ \\
\hline
V & V & V \\
V & F & F \\
F & V & F \\
F & F & F
\end{tabular}
\quad
\begin{tabular}{cc|c}
$p$ & $q$ & $p \lor q$ \\
\hline
V & V & V \\
V & F & V \\
F & V & V \\
F & F & F
\end{tabular}
\quad
\begin{tabular}{cc|c}
$p$ & $q$ & $p \rightarrow q$ \\
\hline
V & V & V \\
V & F & F \\
F & V & V \\
F & F & V
\end{tabular}
\end{center}
\end{definizione}

\subsection{Concetti Fondamentali}

\begin{definizione}[Modello]
Un'interpretazione $\mathcal{I}$ è un \textit{modello} di una formula $\varphi$ (scritto $\mathcal{I} \models \varphi$) se $\varphi$ è vera sotto $\mathcal{I}$.
\end{definizione}

\begin{definizione}[Tautologia, Contraddizione, Contingenza]
Una formula $\varphi$ è:
\begin{itemize}
\item \textbf{Tautologia} se $\mathcal{I} \models \varphi$ per ogni interpretazione $\mathcal{I}$
\item \textbf{Contraddizione} se $\mathcal{I} \not\models \varphi$ per ogni interpretazione $\mathcal{I}$
\item \textbf{Contingenza} altrimenti
\end{itemize}
\end{definizione}

\textbf{Esempi}:
\begin{itemize}
\item Tautologia: $p \lor \neg p$ (legge del terzo escluso)
\item Contraddizione: $p \land \neg p$
\item Contingenza: $p \land q$
\end{itemize}

\subsection{Conseguenza Logica}

\begin{definizione}[Conseguenza Logica]
Una formula $\psi$ è conseguenza logica di un insieme di formule $\Gamma$ (scritto $\Gamma \models \psi$) se:
\begin{equation}
\forall \mathcal{I}: (\mathcal{I} \models \gamma \text{ per ogni } \gamma \in \Gamma) \Rightarrow \mathcal{I} \models \psi
\end{equation}
\end{definizione}

In altre parole: ogni modello di $\Gamma$ è anche modello di $\psi$.

\section{Logica del Primo Ordine (FOL)}

\subsection{Sintassi}

La logica del primo ordine estende quella proposizionale con:
\begin{itemize}
\item \textbf{Variabili}: $x, y, z, \ldots$
\item \textbf{Costanti}: $a, b, c, \ldots$
\item \textbf{Funzioni}: $f, g, h, \ldots$
\item \textbf{Predicati}: $P, Q, R, \ldots$
\item \textbf{Quantificatori}: $\forall$ (per ogni), $\exists$ (esiste)
\end{itemize}

\begin{definizione}[Termine]
L'insieme dei termini è definito induttivamente:
\begin{equation}
t ::= x \mid c \mid f(t_1, \ldots, t_n)
\end{equation}
\end{definizione}

\begin{definizione}[Formula FOL]
L'insieme delle formule FOL è definito induttivamente:
\begin{align}
\varphi ::= &\; P(t_1, \ldots, t_n) \mid \bot \mid \top \mid \notag \\
            &\; \neg \varphi \mid (\varphi \land \varphi) \mid (\varphi \lor \varphi) \mid \notag \\
            &\; (\varphi \rightarrow \varphi) \mid (\varphi \leftrightarrow \varphi) \mid \notag \\
            &\; \forall x. \varphi \mid \exists x. \varphi
\end{align}
\end{definizione}

\subsection{Semantica}

\begin{definizione}[Struttura]
Una struttura (o interpretazione) per FOL è una coppia $\mathcal{M} = \langle D, \mathcal{I} \rangle$ dove:
\begin{itemize}
\item $D$ è un insieme non vuoto (dominio)
\item $\mathcal{I}$ assegna:
  \begin{itemize}
  \item A ogni costante $c$ un elemento $\mathcal{I}(c) \in D$
  \item A ogni funzione $f$ di arietà $n$ una funzione $\mathcal{I}(f): D^n \rightarrow D$
  \item A ogni predicato $P$ di arietà $n$ una relazione $\mathcal{I}(P) \subseteq D^n$
  \end{itemize}
\end{itemize}
\end{definizione}

\subsection{Quantificatori}

\begin{definizione}[Semantica dei Quantificatori]
Data una struttura $\mathcal{M}$ e un'assegnazione $\sigma$ delle variabili:
\begin{itemize}
\item $\mathcal{M}, \sigma \models \forall x. \varphi$ sse $\mathcal{M}, \sigma[x \mapsto d] \models \varphi$ per ogni $d \in D$
\item $\mathcal{M}, \sigma \models \exists x. \varphi$ sse $\mathcal{M}, \sigma[x \mapsto d] \models \varphi$ per qualche $d \in D$
\end{itemize}
\end{definizione}

\textbf{Esempi}:
\begin{itemize}
\item $\forall x. \text{Umano}(x) \rightarrow \text{Mortale}(x)$ \\
  "Tutti gli umani sono mortali"
\item $\exists x. \text{Filosofo}(x) \land \text{Greco}(x)$ \\
  "Esiste un filosofo greco"
\end{itemize}

\section{Regole di Inferenza}

\subsection{Deduzione Naturale}

Le regole di inferenza permettono di derivare nuove formule da formule date.

\begin{teorema}[Modus Ponens]
\begin{equation}
\frac{\varphi \quad \varphi \rightarrow \psi}{\psi}
\end{equation}
Se $\varphi$ è vero e $\varphi$ implica $\psi$, allora $\psi$ è vero.
\end{teorema}

\begin{teorema}[Modus Tollens]
\begin{equation}
\frac{\varphi \rightarrow \psi \quad \neg \psi}{\neg \varphi}
\end{equation}
\end{teorema}

\begin{teorema}[Sillogismo Ipotetico]
\begin{equation}
\frac{\varphi \rightarrow \psi \quad \psi \rightarrow \chi}{\varphi \rightarrow \chi}
\end{equation}
\end{teorema}

\subsection{Regole per Quantificatori}

\textbf{Introduzione universale ($\forall$-I)}:
\begin{equation}
\frac{\varphi[x/a]}{\forall x. \varphi} \quad \text{(dove $a$ è arbitrario)}
\end{equation}

\textbf{Eliminazione universale ($\forall$-E)}:
\begin{equation}
\frac{\forall x. \varphi}{\varphi[x/t]} \quad \text{(per qualsiasi termine $t$)}
\end{equation}

\textbf{Introduzione esistenziale ($\exists$-I)}:
\begin{equation}
\frac{\varphi[x/t]}{\exists x. \varphi} \quad \text{(per qualsiasi termine $t$)}
\end{equation}

\textbf{Eliminazione esistenziale ($\exists$-E)}:
\begin{equation}
\frac{\exists x. \varphi \quad \varphi[x/a] \vdash \psi}{\psi} \quad \text{(dove $a$ è fresco)}
\end{equation}

\section{Sistemi a Produzione come Logica}

\subsection{Rappresentazione Logica delle Regole}

Una regola di produzione:
\begin{lstlisting}[language=CLIPS]
(defrule nome
  (pattern1)
  (pattern2)
  =>
  (azione))
\end{lstlisting}

può essere vista come un'implicazione logica:
\begin{equation}
\text{pattern1} \land \text{pattern2} \rightarrow \text{azione}
\end{equation}

\subsection{Forward Chaining come Modus Ponens}

Il forward chaining è l'applicazione ripetuta del modus ponens:

\begin{enumerate}
\item \textbf{Base di conoscenza}: $\{F_1, F_2, \ldots, F_n\}$ (fatti)
\item \textbf{Regola}: $P_1 \land P_2 \land \ldots \land P_k \rightarrow C$
\item \textbf{Matching}: Se $\{F_1, \ldots, F_n\} \models P_1 \land \ldots \land P_k$
\item \textbf{Firing}: Aggiungi $C$ alla base di conoscenza
\end{enumerate}

\begin{esempio}[Deduzione Sillogistica]
\begin{align*}
&\text{Fatto 1:} \quad \text{Socrate è un uomo} \\
&\text{Fatto 2:} \quad \text{Tutti gli uomini sono mortali} \\
&\text{Regola:} \quad \text{Uomo}(x) \rightarrow \text{Mortale}(x) \\
&\text{Conclusione:} \quad \text{Socrate è mortale}
\end{align*}
\end{esempio}

\subsection{Correttezza e Completezza}

\begin{definizione}[Correttezza]
Un sistema di inferenza è \textit{corretto} (sound) se:
\begin{equation}
\Gamma \vdash \varphi \Rightarrow \Gamma \models \varphi
\end{equation}
Ovvero: tutto ciò che è derivabile è anche vero.
\end{definizione}

\begin{definizione}[Completezza]
Un sistema di inferenza è \textit{completo} (complete) se:
\begin{equation}
\Gamma \models \varphi \Rightarrow \Gamma \vdash \varphi
\end{equation}
Ovvero: tutto ciò che è vero è anche derivabile.
\end{definizione}

\begin{teorema}[Teoremi di Gödel per FOL]
La logica del primo ordine è:
\begin{enumerate}
\item \textbf{Corretta}: le regole di inferenza preservano la verità
\item \textbf{Completa}: ogni conseguenza logica è derivabile
\item \textbf{Indecidibile}: non esiste algoritmo che determini se $\Gamma \models \varphi$ in tempo finito
\end{enumerate}
\end{teorema}

\section{Limiti della Logica Classica nei Sistemi Esperti}

\subsection{Ragionamento Non Monotono}

La logica classica è monotona:
\begin{equation}
\Gamma \models \varphi \Rightarrow \Gamma \cup \{\psi\} \models \varphi
\end{equation}

Aggiungere nuova informazione non invalida conclusioni precedenti.

\textbf{Problema}: Nel mondo reale spesso ragioniamo per \textit{default}:
\begin{itemize}
\item "Gli uccelli volano" (default)
\item "I pinguini sono uccelli"
\item "I pinguini \textbf{non} volano" (eccezione)
\end{itemize}

\subsection{Chiusura del Mondo (CWA)}

\begin{definizione}[Closed World Assumption]
Ciò che non è esplicitamente noto o derivabile è assunto falso:
\begin{equation}
\Gamma \not\vdash \varphi \Rightarrow \Gamma \vdash \neg \varphi
\end{equation}
\end{definizione}

Questa assunzione è usata nei database e in CLIPS per la negazione (\texttt{not}).

\subsection{Ragionamento Temporale}

I sistemi a produzione operano nel tempo:
\begin{itemize}
\item Lo stato della WM cambia ad ogni ciclo
\item Le regole hanno effetti temporali
\item L'ordine di firing può essere rilevante
\end{itemize}

Logiche modali temporali (LTL, CTL) sono necessarie per ragionare formalmente su proprietà temporali.

\section{Logiche Non Standard per Sistemi Esperti}

\subsection{Logica di Default}

\begin{definizione}[Regola di Default (Reiter)]
Una regola di default ha la forma:
\begin{equation}
\frac{\varphi : M\psi_1, \ldots, M\psi_n}{\chi}
\end{equation}
Significato: "Se $\varphi$ è vero e $\psi_1, \ldots, \psi_n$ sono consistenti con ciò che sappiamo, concludi $\chi$"
\end{definizione}

\subsection{Logica Modale}

Operatori modali:
\begin{itemize}
\item $\Box \varphi$ ("necessariamente $\varphi$")
\item $\Diamond \varphi$ ("possibilmente $\varphi$")
\end{itemize}

Utili per ragionare su credenze, conoscenza, obblighi.

\subsection{Logica Fuzzy}

Estende la logica classica a valori di verità nell'intervallo $[0, 1]$:
\begin{itemize}
\item $\mathcal{I}(\varphi \land \psi) = \min(\mathcal{I}(\varphi), \mathcal{I}(\psi))$
\item $\mathcal{I}(\varphi \lor \psi) = \max(\mathcal{I}(\varphi), \mathcal{I}(\psi))$
\item $\mathcal{I}(\neg \varphi) = 1 - \mathcal{I}(\varphi)$
\end{itemize}

Permette di gestire incertezza e vaghezza.

\section{Unificazione}

\subsection{Sostituzione}

\begin{definizione}[Sostituzione]
Una sostituzione è un insieme finito di coppie:
\begin{equation}
\theta = \{x_1 / t_1, x_2 / t_2, \ldots, x_n / t_n\}
\end{equation}
dove ogni $x_i$ è una variabile e ogni $t_i$ è un termine con $x_i \neq t_i$.
\end{definizione}

\textbf{Applicazione}: $\varphi\theta$ è la formula ottenuta sostituendo simultaneamente ogni $x_i$ con $t_i$ in $\varphi$.

\subsection{Unificatore}

\begin{definizione}[Unificatore]
Una sostituzione $\theta$ è un \textit{unificatore} di termini $t_1$ e $t_2$ se:
\begin{equation}
t_1\theta = t_2\theta
\end{equation}
\end{definizione}

\begin{definizione}[Unificatore Più Generale (MGU)]
$\theta$ è MGU di $t_1$ e $t_2$ se:
\begin{enumerate}
\item $\theta$ unifica $t_1$ e $t_2$
\item Per ogni altro unificatore $\sigma$ esiste $\lambda$ tale che $\sigma = \theta\lambda$
\end{enumerate}
\end{definizione}

\subsection{Algoritmo di Unificazione}

\begin{algorithm}
\caption{Algoritmo di Unificazione (Robinson)}
\begin{algorithmic}[1]
\Require Due termini $s$ e $t$
\Ensure MGU $\theta$ se esiste, altrimenti \texttt{fail}
\Function{Unify}{$s, t$}
  \If{$s = t$}
    \State \Return $\{\}$ \Comment{Identici}
  \ElsIf{$s$ è variabile}
    \If{$s$ appare in $t$}
      \State \Return \texttt{fail} \Comment{Occur check}
    \Else
      \State \Return $\{s/t\}$
    \EndIf
  \ElsIf{$t$ è variabile}
    \State \Return \Call{Unify}{$t, s$}
  \ElsIf{$s = f(s_1, \ldots, s_n)$ e $t = g(t_1, \ldots, t_m)$}
    \If{$f \neq g$ o $n \neq m$}
      \State \Return \texttt{fail}
    \EndIf
    \State $\theta \gets \{\}$
    \For{$i = 1$ to $n$}
      \State $\sigma \gets$ \Call{Unify}{$s_i\theta, t_i\theta$}
      \If{$\sigma = $ \texttt{fail}}
        \State \Return \texttt{fail}
      \EndIf
      \State $\theta \gets \theta \circ \sigma$
    \EndFor
    \State \Return $\theta$
  \EndIf
\EndFunction
\end{algorithmic}
\end{algorithm}

\textbf{Complessità}: $O(n)$ nel numero di simboli nei termini (quasi-lineare con tecniche di union-find).

\subsection{Unificazione in CLIPS}

L'unificazione è usata nel pattern matching:
\begin{lstlisting}[language=CLIPS]
;; Pattern con variabili
(persona (nome ?x) (eta ?y))

;; Fatto
(persona (nome "Mario") (eta 30))

;; Unificazione: {?x/"Mario", ?y/30}
\end{lstlisting}

\section{Risoluzione}

\subsection{Forma Normale Congiuntiva}

\begin{definizione}[Clausola]
Una clausola è una disgiunzione di letterali:
\begin{equation}
L_1 \lor L_2 \lor \ldots \lor L_n
\end{equation}
dove ogni $L_i$ è un letterale (atomo o sua negazione).
\end{definizione}

\begin{definizione}[CNF]
Una formula è in \textit{Forma Normale Congiuntiva} (CNF) se è una congiunzione di clausole:
\begin{equation}
(L_{11} \lor \ldots \lor L_{1n_1}) \land \ldots \land (L_{m1} \lor \ldots \lor L_{mn_m})
\end{equation}
\end{definizione}

\subsection{Regola di Risoluzione}

\begin{teorema}[Risoluzione Proposizionale]
Date due clausole:
\begin{align}
C_1 &= L \lor A_1 \lor \ldots \lor A_n \\
C_2 &= \neg L \lor B_1 \lor \ldots \lor B_m
\end{align}
la loro \textit{risolvente} è:
\begin{equation}
R = A_1 \lor \ldots \lor A_n \lor B_1 \lor \ldots \lor B_m
\end{equation}
\end{teorema}

\subsection{Risoluzione FOL}

Per FOL, combiniamo risoluzione e unificazione:

\begin{teorema}[Risoluzione con Unificazione]
Date clausole:
\begin{align}
C_1 &= L_1 \lor A \\
C_2 &= L_2 \lor B
\end{align}
se $\theta = \text{MGU}(L_1, \neg L_2)$ esiste, la risolvente è:
\begin{equation}
R = (A \lor B)\theta
\end{equation}
\end{teorema}

\subsection{Teorema di Completezza}

\begin{teorema}[Completezza della Risoluzione]
La risoluzione è completa per la refutazione: \\
$\Gamma \models \varphi$ se e solo se $\Gamma \cup \{\neg \varphi\}$ deriva la clausola vuota $\square$ per risoluzione.
\end{teorema}

\section{Connessione con CLIPS}

\subsection{Pattern come Formule}

Un pattern CLIPS:
\begin{lstlisting}[language=CLIPS]
(persona (nome ?n) (eta ?e&:(> ?e 18)))
\end{lstlisting}

corrisponde alla formula FOL:
\begin{equation}
\exists n, e. \text{Persona}(n, e) \land e > 18
\end{equation}

\subsection{Regole come Clausole di Horn}

\begin{definizione}[Clausola di Horn]
Una clausola con al più un letterale positivo:
\begin{equation}
\neg P_1 \lor \neg P_2 \lor \ldots \lor \neg P_n \lor Q
\end{equation}
equivalente a:
\begin{equation}
P_1 \land P_2 \land \ldots \land P_n \rightarrow Q
\end{equation}
\end{definizione}

Le regole CLIPS sono essenzialmente clausole di Horn.

\subsection{Limitazioni}

CLIPS non supporta nativamente:
\begin{itemize}
\item Quantificazione universale nelle LHS (solo esistenziale implicita)
\item Negazione di congiunzioni arbitrarie
\item Logica higher-order
\item Ragionamento probabilistico intrinseco
\end{itemize}

Queste limitazioni garantiscono decidibilità e efficienza.

\section{Conclusioni del Capitolo}

\subsection{Punti Chiave}

\begin{enumerate}
\item La logica formale fornisce le \textbf{fondamenta teoriche} dei sistemi a produzione
\item L'\textbf{unificazione} è l'operazione centrale per il pattern matching
\item La \textbf{risoluzione} offre un metodo di inferenza completo
\item I sistemi reali richiedono estensioni della logica classica
\item CLIPS usa un sottoinsieme decidibile ed efficiente di FOL
\end{enumerate}

\subsection{Implicazioni per SLIPS}

La traduzione C $\rightarrow$ Swift deve preservare:
\begin{itemize}
\item Semantica dell'unificazione
\item Ordine di valutazione dei pattern
\item Comportamento della negazione (CWA)
\item Corretta gestione delle variabili e sostituzioni
\end{itemize}

\subsection{Letture Consigliate}

\begin{itemize}
\item \textit{Mathematical Logic} - J. Shoenfield (1967)
\item \textit{Logic for Computer Science} - J. Gallier (1986)
\item \textit{Artificial Intelligence: A Modern Approach} - Russell \& Norvig (cap. 7-9)
\item \textit{Handbook of Logic in AI} - Vol. 1-2, Gabbay et al.
\item CLIPS Reference Manual - Sezione "Pattern Matching"
\end{itemize}

% Capitolo 4: Rappresentazione della Conoscenza

\chapter{Rappresentazione della Conoscenza}
\label{cap:rappresentazione_conoscenza}

\section{Introduzione}

La rappresentazione della conoscenza è il problema centrale dell'intelligenza artificiale simbolica: come codificare fatti, regole, relazioni e concetti in una forma che un computer possa manipolare per ragionare e prendere decisioni.

\subsection{Requisiti Fondamentali}

Un buon schema di rappresentazione deve essere:

\begin{infobox}[Proprietà Desiderabili]
\begin{itemize}
\item \textbf{Espressivo}: capace di rappresentare la conoscenza del dominio
\item \textbf{Sintetico}: conciso e leggibile
\item \textbf{Efficiente}: manipolabile computazionalmente
\item \textbf{Modulare}: organizzabile e componibile
\item \textbf{Incrementale}: estendibile senza ristrutturazioni
\item \textbf{Dichiarativo}: separazione tra cosa e come
\end{itemize}
\end{infobox}

\section{Paradigmi di Rappresentazione}

\subsection{Logica}

La rappresentazione più formale, basata su formule logiche.

\textbf{Vantaggi}:
\begin{itemize}
\item Semantica matematica precisa
\item Correttezza e completezza dimostrabili
\item Meccanismi di inferenza ben definiti
\end{itemize}

\textbf{Svantaggi}:
\begin{itemize}
\item Difficoltà di esprimere incertezza
\item Complessità computazionale elevata
\item Monotonia (difficoltà con eccezioni)
\end{itemize}

\subsection{Reti Semantiche}

Grafi diretti dove:
\begin{itemize}
\item \textbf{Nodi} rappresentano concetti
\item \textbf{Archi} rappresentano relazioni
\end{itemize}

\begin{figure}[h]
\centering
\begin{tikzpicture}[
  node distance=2.5cm,
  concept/.style={circle, draw, minimum size=1cm},
  relation/.style={->, >=stealth, thick}
]
  \node[concept] (uccello) {Uccello};
  \node[concept, right of=uccello] (pinguino) {Pinguino};
  \node[concept, below of=uccello] (animale) {Animale};
  \node[concept, right of=pinguino] (tweety) {Tweety};
  
  \draw[relation] (uccello) -- node[left] {is-a} (animale);
  \draw[relation] (pinguino) -- node[above] {is-a} (uccello);
  \draw[relation] (tweety) -- node[above] {instance-of} (pinguino);
  
  \node[right of=animale, xshift=1cm] (prop1) {\small respira};
  \node[right of=uccello, xshift=2cm] (prop2) {\small vola};
  \node[right of=pinguino, xshift=2.5cm] (prop3) {\small nuota};
  
  \draw[relation, dashed] (animale) -- (prop1);
  \draw[relation, dashed] (uccello) -- (prop2);
  \draw[relation, dashed] (pinguino) -- (prop3);
\end{tikzpicture}
\caption{Rete semantica gerarchica}
\label{fig:rete_semantica}
\end{figure}

\textbf{Ereditarietà}: Le proprietà si propagano lungo gli archi \textit{is-a}.

\subsection{Frame}

Strutture dati che raggruppano conoscenza su un concetto.

\begin{definizione}[Frame]
Un frame è una collezione di \textit{slot} (attributi) con valori, restrizioni e procedure associate.
\end{definizione}

\textbf{Esempio}:
\begin{verbatim}
Frame: Automobile
  Slots:
    - modello: [tipo: STRING]
    - anno: [tipo: INTEGER, range: 1900-2025]
    - proprietario: [tipo: Persona]
    - cilindrata: [tipo: FLOAT, default: 1600]
  Methods:
    - calcola_bollo()
    - verifica_revisione()
\end{verbatim}

\subsection{Regole di Produzione}

Il paradigma adottato da CLIPS e SLIPS.

\begin{definizione}[Regola di Produzione]
Una regola di produzione ha la forma:
\begin{equation}
\text{IF condizione THEN azione}
\end{equation}
dove:
\begin{itemize}
\item \textbf{condizione} (LHS) è un pattern sui fatti
\item \textbf{azione} (RHS) modifica la working memory
\end{itemize}
\end{definizione}

\textbf{Caratteristiche}:
\begin{itemize}
\item Modularità: ogni regola è indipendente
\item Dichiaratività: esprime "cosa" non "come"
\item Forward chaining naturale
\item Pattern matching efficiente (RETE)
\end{itemize}

\section{Rappresentazione in CLIPS}

\subsection{Fatti}

\subsubsection{Fatti Ordinati}

Sequenze di campi senza struttura esplicita:

\begin{lstlisting}[language=CLIPS]
(temperatura 25)
(colore rosso verde blu)
(coordina 10.5 20.3)
\end{lstlisting}

\textbf{Pro}: Semplici e compatti \\
\textbf{Contro}: Manca semantica esplicita dei campi

\subsubsection{Fatti Non Ordinati (Deftemplate)}

Strutture con slot nominati:

\begin{lstlisting}[language=CLIPS]
(deftemplate persona
  (slot nome (type STRING))
  (slot eta (type INTEGER) (range 0 150))
  (slot professione (default "disoccupato"))
  (multislot hobby))

(persona 
  (nome "Mario Rossi") 
  (eta 35) 
  (professione "ingegnere")
  (hobby tennis lettura programmazione))
\end{lstlisting}

\textbf{Vantaggi}:
\begin{itemize}
\item Leggibilità e manutenibilità
\item Type checking
\item Valori di default
\item Validazione (range, allowed-values)
\end{itemize}

\subsection{Regole}

\begin{lstlisting}[language=CLIPS]
(defrule diagnosi-influenza
  "Diagnostica influenza in base ai sintomi"
  (declare (salience 100))
  
  ;; Pattern matching (LHS)
  (paziente (id ?id) (nome ?nome))
  (sintomo (paziente-id ?id) (tipo febbre) (valore ?temp&:(> ?temp 38)))
  (sintomo (paziente-id ?id) (tipo tosse))
  (not (diagnosi (paziente-id ?id)))
  
  =>
  
  ;; Azioni (RHS)
  (printout t "Paziente " ?nome " probabile influenza" crlf)
  (assert (diagnosi 
            (paziente-id ?id) 
            (malattia influenza)
            (confidenza 0.8))))
\end{lstlisting}

\textbf{Elementi LHS}:
\begin{itemize}
\item Pattern positivi: \texttt{(paziente ...)}
\item Negazione: \texttt{(not ...)}
\item Congiunzione: \texttt{(and ...)}
\item Disgiunzione: \texttt{(or ...)}
\item Esistenziale: \texttt{(exists ...)}
\item Test: \texttt{(test (> ?x 10))}
\end{itemize}

\subsection{Moduli}

Organizzazione della base di conoscenza in namespace separati:

\begin{lstlisting}[language=CLIPS]
(defmodule ACQUISIZIONE
  "Raccolta dati dal paziente"
  (export deftemplate sintomo paziente))

(defmodule DIAGNOSI
  "Inferenza diagnostica"
  (import ACQUISIZIONE deftemplate sintomo paziente)
  (export deftemplate diagnosi))

(defmodule TERAPIA
  "Prescrizione cura"
  (import DIAGNOSI deftemplate diagnosi))
\end{lstlisting}

\textbf{Benefici}:
\begin{itemize}
\item Incapsulamento
\item Controllo delle dipendenze
\item Scalabilità a grandi sistemi
\item Focus selettivo (focus stack)
\end{itemize}

\section{Pattern e Variabili}

\subsection{Variabili}

\textbf{Variabili singole}:
\begin{lstlisting}[language=CLIPS]
?x        ; Qualsiasi singolo valore
?nome     ; Variabile nominata
?         ; Variabile anonima (wildcard)
\end{lstlisting}

\textbf{Variabili multifield}:
\begin{lstlisting}[language=CLIPS]
$?resto   ; Zero o piu valori
$?        ; Multifield anonimo
\end{lstlisting}

\subsection{Constraint sui Pattern}

\textbf{Predicati}:
\begin{lstlisting}[language=CLIPS]
?x&:(> ?x 10)           ; Valore > 10
?nome&:(eq ?nome "Mario")  ; Valore specifico
?y&:(numberp ?y)        ; Test di tipo
\end{lstlisting}

\textbf{Connettivi}:
\begin{lstlisting}[language=CLIPS]
?x&~nil                 ; Diverso da nil
?x&blue|red|green       ; Uno dei valori
?x&~?y                  ; Diverso da ?y
\end{lstlisting}

\subsection{Binding e Unificazione}

Quando un pattern matcha un fatto:

\begin{enumerate}
\item \textbf{Unificazione}: trovare sostituzioni $\theta$ per variabili
\item \textbf{Binding}: assegnare valori alle variabili
\item \textbf{Consistenza}: verificare constraint
\end{enumerate}

\textbf{Esempio}:
\begin{lstlisting}[language=CLIPS]
;; Pattern
(persona (nome ?n) (eta ?e&:(> ?e 18)) (citta "Roma"))

;; Fatto
(persona (nome "Giulia") (eta 25) (citta "Roma"))

;; Binding risultante
{?n -> "Giulia", ?e -> 25}
\end{lstlisting}

\section{Semantica Dichiarativa vs Procedurale}

\subsection{Interpretazione Dichiarativa}

Le regole esprimono \textit{conoscenza} generale del dominio:

\begin{lstlisting}[language=CLIPS]
(defrule sconti-anziani
  (persona (eta ?e&:(>= ?e 65)))
  =>
  (assert (sconto 20)))
\end{lstlisting}

Significato: "Le persone con 65+ anni hanno diritto a uno sconto del 20\%"

\subsection{Interpretazione Procedurale}

Le stesse regole definiscono un \textit{algoritmo} implicito:

\begin{enumerate}
\item Match delle regole applicabili
\item Conflict resolution (strategia)
\item Esecuzione (firing)
\item Ripeti fino a quiescenza
\end{enumerate}

\subsection{Dualità}

Questa dualità è una forza dei sistemi esperti basati su regole:
\begin{itemize}
\item \textbf{Esperti del dominio} vedono conoscenza dichiarativa
\item \textbf{Il sistema} esegue proceduralmente
\item \textbf{Modifiche} facili: aggiungere/rimuovere regole
\end{itemize}

\section{Chiusura del Mondo e Negazione}

\subsection{Open World Assumption (OWA)}

Nella logica classica, l'assenza di informazione significa \textit{sconosciuto}:
\begin{equation}
\Gamma \not\vdash \varphi \not\Rightarrow \Gamma \vdash \neg \varphi
\end{equation}

\subsection{Closed World Assumption (CWA)}

In CLIPS (e database), l'assenza significa \textit{falso}:
\begin{equation}
\Gamma \not\vdash \varphi \Rightarrow \Gamma \vdash \neg \varphi
\end{equation}

\subsection{Negazione in CLIPS}

\begin{lstlisting}[language=CLIPS]
(defrule nuovi-clienti
  (cliente (id ?id))
  (not (ordine (cliente-id ?id)))  ; CWA: nessun ordine = falso
  =>
  (printout t "Nuovo cliente: " ?id crlf))
\end{lstlisting}

\textbf{Attenzione}: La negazione è \textit{non monotona}:
\begin{itemize}
\item Inizialmente: nessun ordine $\Rightarrow$ regola applicabile
\item Dopo assert di ordine: regola non più applicabile
\item Truth Maintenance necessario in alcuni casi
\end{itemize}

\section{Gerarche e Ereditarietà}

\subsection{Ereditarietà via Regole}

CLIPS non ha ereditarietà built-in, ma si può simulare:

\begin{lstlisting}[language=CLIPS]
;; Gerarchia esplicita
(deffacts tassonomia
  (is-a cane mammifero)
  (is-a gatto mammifero)
  (is-a mammifero animale)
  (is-a animale essere-vivente))

;; Propagazione proprieta
(defrule eredita-proprieta
  (is-a ?figlio ?genitore)
  (proprieta (classe ?genitore) (attributo ?attr) (valore ?val))
  (not (proprieta (classe ?figlio) (attributo ?attr)))
  =>
  (assert (proprieta (classe ?figlio) (attributo ?attr) (valore ?val))))
\end{lstlisting}

\subsection{Overriding ed Eccezioni}

Gestione delle eccezioni tramite salience:

\begin{lstlisting}[language=CLIPS]
(defrule uccelli-volano
  (declare (salience 10))
  (animale (tipo uccello) (nome ?n))
  =>
  (assert (puo-volare ?n)))

(defrule pinguini-non-volano
  (declare (salience 20))  ; Priorita maggiore!
  (animale (tipo pinguino) (nome ?n))
  =>
  (assert (non-puo-volare ?n)))
\end{lstlisting}

\section{Conoscenza Temporale}

\subsection{Rappresentazione dello Stato}

\textbf{Approccio 1: Stato Implicito} (Working Memory = stato corrente)

\begin{lstlisting}[language=CLIPS]
(temperatura 25)
(ora 14:30)
\end{lstlisting}

\textbf{Approccio 2: Stato Esplicito con Timestamp}

\begin{lstlisting}[language=CLIPS]
(deftemplate misura
  (slot parametro)
  (slot valore)
  (slot timestamp))

(misura (parametro temperatura) (valore 25) (timestamp 1445))
(misura (parametro temperatura) (valore 26) (timestamp 1450))
\end{lstlisting}

\subsection{Eventi e Transizioni}

\begin{lstlisting}[language=CLIPS]
(deftemplate evento
  (slot tipo)
  (slot tempo)
  (multislot dati))

(defrule rileva-anomalia
  (evento (tipo misura) (tempo ?t1) (dati temperatura ?temp1))
  (evento (tipo misura) (tempo ?t2&:(> ?t2 ?t1)) (dati temperatura ?temp2))
  (test (> (abs (- ?temp2 ?temp1)) 10))
  =>
  (assert (allarme (tipo variazione-rapida) (tempo ?t2))))
\end{lstlisting}

\section{Conoscenza Incerta}

\subsection{Fattori di Certezza (Certainty Factors)}

Approccio MYCIN:

\begin{lstlisting}[language=CLIPS]
(deftemplate ipotesi
  (slot diagnosi)
  (slot cf (type FLOAT) (range -1.0 1.0)))

(defrule combina-evidenze
  (sintomo (tipo ?s1) (cf ?cf1))
  (regola (se ?s1) (allora ?diagnosi) (cf-regola ?cfr))
  =>
  (bind ?cf-combinato (* ?cf1 ?cfr))
  (assert (ipotesi (diagnosi ?diagnosi) (cf ?cf-combinato))))
\end{lstlisting}

\subsection{Logica Fuzzy}

CLIPS supporta FuzzyCLIPS per logica sfumata:

\begin{verbatim}
(deftemplate temperatura
  0 100 gradi
  ((fredda (z 10 20))
   (mite (pi 15 25))
   (calda (s 20 30))))
\end{verbatim}

\section{Meta-Conoscenza}

\subsection{Conoscenza sulla Conoscenza}

\begin{lstlisting}[language=CLIPS]
(deftemplate regola-meta
  (slot id-regola)
  (slot applicabilita (allowed-values alta media bassa))
  (slot confidenza (type FLOAT))
  (slot fonte))

;; Decidere quando applicare una regola
(defrule usa-regola-affidabile
  (regola-meta (id-regola ?r) (confidenza ?c&:(> ?c 0.8)))
  (agenda ?r ...)
  =>
  (fire ?r))
\end{lstlisting}

\subsection{Strategia Dinamica}

\begin{lstlisting}[language=CLIPS]
(defrule cambia-strategia
  (fase iniziale)
  (num-fatti ?n&:(> ?n 1000))
  =>
  (set-strategy complexity)  ; Passa a strategia per complessita
  (retract-string "(fase iniziale)")
  (assert (fase ottimizzazione)))
\end{lstlisting}

\section{Design Pattern per la Conoscenza}

\subsection{Pattern: State Machine}

\begin{lstlisting}[language=CLIPS]
(deftemplate stato
  (slot nome)
  (slot attivo (default no)))

(defrule transizione
  ?s1 <- (stato (nome ?da) (attivo yes))
  (evento (trigger ?trigger))
  (regola-transizione (da ?da) (evento ?trigger) (a ?a))
  =>
  (modify ?s1 (attivo no))
  (assert (stato (nome ?a) (attivo yes))))
\end{lstlisting}

\subsection{Pattern: Blackboard}

Spazio condiviso per cooperazione tra moduli:

\begin{lstlisting}[language=CLIPS]
(deftemplate ipotesi-blackboard
  (slot livello (allowed-values basso medio alto))
  (slot contenuto)
  (slot fonte))

;; Modulo basso livello
(defrule rileva-feature
  (segnale (dati ?d))
  =>
  (assert (ipotesi-blackboard (livello basso) (contenuto ?d))))

;; Modulo alto livello
(defrule integra-ipotesi
  (ipotesi-blackboard (livello basso) (contenuto ?c1))
  (ipotesi-blackboard (livello basso) (contenuto ?c2))
  =>
  (assert (ipotesi-blackboard (livello alto) (contenuto ...))))
\end{lstlisting}

\subsection{Pattern: Case-Based Reasoning}

\begin{lstlisting}[language=CLIPS]
(deftemplate caso
  (slot problema)
  (slot soluzione)
  (slot similarita))

(defrule recupera-caso-simile
  (problema-corrente ?p)
  (caso (problema ?pc) (soluzione ?s))
  (test (> (calcola-similarita ?p ?pc) 0.8))
  =>
  (assert (candidato-soluzione ?s)))
\end{lstlisting}

\section{Limiti e Trade-off}

\subsection{Espressività vs Efficienza}

\begin{table}[h]
\centering
\begin{tabular}{@{}lcc@{}}
\toprule
\textbf{Formalismo} & \textbf{Espressività} & \textbf{Complessità} \\
\midrule
Logica proposizionale & Bassa & P (SAT: NP-completo) \\
Clausole di Horn & Media & P (lineare) \\
FOL & Alta & Indecidibile \\
Regole produzione & Media-Alta & Efficiente con RETE \\
\bottomrule
\end{tabular}
\caption{Trade-off espressività-efficienza}
\end{table}

\subsection{Limitazioni di CLIPS}

\begin{itemize}
\item No quantificazione universale esplicita in LHS
\item No funzioni higher-order
\item No backtracking (ricerca non esaustiva)
\item No constraint propagation automatica
\item Gestione limitata dell'incertezza
\end{itemize}

\subsection{Quando Usare Altri Formalismi}

\begin{itemize}
\item \textbf{Description Logic} (OWL): ontologie, ragionamento subsumption
\item \textbf{Answer Set Programming}: ottimizzazione combinatoria
\item \textbf{Probabilistic Graphical Models}: incertezza, apprendimento
\item \textbf{Constraint Programming}: scheduling, planning
\end{itemize}

\section{Best Practices}

\subsection{Principi di Buona Modellazione}

\begin{infobox}[Linee Guida]
\begin{enumerate}
\item \textbf{Atomicità}: Un fatto = una informazione atomica
\item \textbf{Normalizzazione}: Evitare ridondanza
\item \textbf{Naming conventions}: Nomi descrittivi e consistenti
\item \textbf{Documentazione}: Commenti per regole complesse
\item \textbf{Modularità}: Usare defmodule per organizzazione
\item \textbf{Testing}: Verificare regole indipendentemente
\end{enumerate}
\end{infobox}

\subsection{Antipattern da Evitare}

\begin{warningbox}[Errori Comuni]
\begin{itemize}
\item \textbf{God rules}: Regole che fanno troppe cose
\item \textbf{Hardcoding}: Valori letterali invece di parametri
\item \textbf{Negazione imprudente}: Può causare loop
\item \textbf{Salience abuse}: Troppa dipendenza da priorità esplicite
\item \textbf{Global state nascosto}: Effetti collaterali non dichiarati
\end{itemize}
\end{warningbox}

\section{Conclusioni del Capitolo}

\subsection{Punti Chiave}

\begin{enumerate}
\item La rappresentazione della conoscenza è cruciale per sistemi efficaci
\item CLIPS offre un buon bilanciamento tra espressività ed efficienza
\item I deftemplate forniscono struttura e validazione
\item I moduli permettono scalabilità
\item Pattern matching unifica dichiaratività e computazione
\end{enumerate}

\subsection{Implicazioni per SLIPS}

SLIPS deve preservare fedelmente:
\begin{itemize}
\item Semantica dei deftemplate e dei fatti
\item Comportamento dell'unificazione e binding
\item Gestione della negazione (CWA)
\item Modularità e namespace
\item Interazione tra rappresentazione e inferenza
\end{itemize}

\subsection{Prossimi Passi}

Il Capitolo~\ref{cap:rete_introduzione} mostrerà come il pattern matching efficiente rende possibile lavorare con grandi basi di conoscenza rappresentate come regole.

\subsection{Letture Consigliate}

\begin{itemize}
\item \textit{Knowledge Representation and Reasoning} - Brachman \& Levesque (2004)
\item \textit{Principles of Knowledge Representation} - Sowa (1999)
\item \textit{Semantic Web for the Working Ontologist} - Allemang \& Hendler (2011)
\item CLIPS Reference Manual - Capitoli 2-5
\item \textit{Expert Systems: Principles and Programming} - Giarratano \& Riley (2004)
\end{itemize}


% PARTE II: L'ALGORITMO RETE
\part{L'Algoritmo RETE: Teoria e Analisi}

% Capitolo 5: Pattern Matching - Fondamenti

\chapter{Pattern Matching: Problemi e Soluzioni}
\label{cap:pattern_matching}

\section{Introduzione}

Il pattern matching è l'operazione fondamentale nei sistemi esperti basati su regole: determinare quali regole sono applicabili dato un certo stato della working memory. L'efficienza di questa operazione determina le prestazioni dell'intero sistema.

\subsection{Il Problema Centrale}

Dato:
\begin{itemize}
\item Un insieme di regole $R = \{r_1, r_2, \ldots, r_n\}$
\item Una working memory $WM = \{f_1, f_2, \ldots, f_m\}$ di fatti
\end{itemize}

\textbf{Obiettivo}: Trovare tutte le \textit{istanziazioni} (binding di variabili) che soddisfano le condizioni LHS di ogni regola.

\begin{definizione}[Istanziazione]
Un'istanziazione $\iota$ di una regola $r$ è un assegnamento di valori alle variabili di $r$ tale che tutti i pattern della LHS matchano fatti in $WM$.
\end{definizione}

\section{Approccio Naïve}

\subsection{Algoritmo di Base}

\begin{algorithm}
\caption{Pattern Matching Naïve}
\begin{algorithmic}[1]
\Require Regole $R$, Working Memory $WM$
\Ensure Conflict Set $CS$
\Function{NaiveMatch}{$R, WM$}
  \State $CS \gets \emptyset$
  \For{each rule $r \in R$}
    \For{each combinazione di fatti $(f_1, \ldots, f_k) \in WM^k$}
      \If{$(f_1, \ldots, f_k)$ soddisfa LHS di $r$}
        \State $CS \gets CS \cup \{(r, f_1, \ldots, f_k)\}$
      \EndIf
    \EndFor
  \EndFor
  \State \Return $CS$
\EndFunction
\end{algorithmic}
\end{algorithm}

\subsection{Complessità}

Per ogni regola con $k$ condizioni e $m$ fatti in WM:
\begin{equation}
O(m^k)
\end{equation}

Con $n$ regole:
\begin{equation}
O(n \cdot m^k)
\end{equation}

\begin{warningbox}[Esplosione Combinatoria]
Con 100 regole, 1000 fatti, e media di 3 condizioni per regola:
\begin{equation}
100 \cdot 1000^3 = 10^{11} \text{ operazioni per ciclo}
\end{equation}
Assolutamente impraticabile!
\end{warningbox}

\section{Principio di Temporalità}

\subsection{Osservazione Chiave}

Tra un ciclo recognize-act e il successivo:
\begin{itemize}
\item La maggior parte dei fatti \textbf{non cambia}
\item Solo pochi fatti vengono aggiunti/rimossi
\item La maggior parte dei match \textbf{rimane valida}
\end{itemize}

\begin{definizione}[Principio di Temporalità]
In un sistema esperto basato su regole, tra cicli consecutivi:
\begin{equation}
|WM_{t+1} \triangle WM_t| \ll |WM_t|
\end{equation}
dove $\triangle$ indica la differenza simmetrica.
\end{definizione}

\textbf{Implicazione}: Ricalcolare tutto da zero spreca lavoro. Dobbiamo \textit{incrementare} il risultato.

\subsection{Approccio Incrementale}

Idea: Memorizzare i match parziali e aggiornarli solo quando necessario.

\begin{infobox}[State Saving]
\begin{itemize}
\item \textbf{Salvare}: Match intermedi tra cicli
\item \textbf{Riutilizzare}: Risultati precedenti
\item \textbf{Aggiornare}: Solo quando fatti cambiano
\item \textbf{Guadagno}: Evitare ricalcoli ridondanti
\end{itemize}
\end{infobox}

\section{Discriminazione}

\subsection{Pattern Simili}

Molte regole condividono parti delle condizioni:

\begin{lstlisting}[language=CLIPS]
;; Regola 1
(defrule r1
  (persona (eta ?e&:(> ?e 18)))
  =>
  ...)

;; Regola 2  
(defrule r2
  (persona (eta ?e&:(> ?e 18)))
  (studente (id ?id))
  =>
  ...)

;; Regola 3
(defrule r3
  (persona (eta ?e&:(> ?e 65)))
  =>
  ...)
\end{lstlisting}

Tutte e tre testano \texttt{persona} con constraint sull'età.

\subsection{Condivisione dei Test}

\begin{definizione}[Discriminazione]
La discriminazione è il processo di \textit{condividere} test comuni tra regole diverse per evitare duplicazione di lavoro.
\end{definizione}

\textbf{Beneficio}: Un test effettuato una volta serve multiple regole.

\begin{figure}[h]
\centering
\begin{tikzpicture}[
  node distance=2cm,
  test/.style={rectangle, draw, minimum width=2cm, minimum height=0.8cm},
  result/.style={ellipse, draw, minimum width=1.5cm}
]
  \node[test] (root) {tipo = persona};
  \node[test, below left of=root] (left) {età > 18};
  \node[test, below right of=root] (right) {età > 65};
  \node[result, below of=left] (r1r2) {R1, R2};
  \node[result, below of=right] (r3) {R3};
  
  \draw[->] (root) -- (left);
  \draw[->] (root) -- (right);
  \draw[->] (left) -- (r1r2);
  \draw[->] (right) -- (r3);
\end{tikzpicture}
\caption{Albero di discriminazione per test comuni}
\label{fig:discriminazione}
\end{figure}

\section{Confronto tra Approcci}

\subsection{Tabella Comparativa}

\begin{table}[h]
\centering
\begin{tabular}{@{}lccc@{}}
\toprule
\textbf{Metodo} & \textbf{Spazio} & \textbf{Tempo/ciclo} & \textbf{Incrementale} \\
\midrule
Naïve & $O(1)$ & $O(n \cdot m^k)$ & No \\
Linear & $O(n)$ & $O(n \cdot m)$ & Parziale \\
RETE & $O(n \cdot m^k)$ & $O(m)$ & Sì \\
\bottomrule
\end{tabular}
\caption{Confronto algoritmi di pattern matching}
\label{tab:confronto_matching}
\end{table}

\subsection{Trade-off Spazio-Tempo}

RETE rappresenta il classico trade-off:
\begin{itemize}
\item \textbf{Più spazio}: Memorizza match parziali
\item \textbf{Meno tempo}: Aggiornamenti incrementali
\end{itemize}

\textbf{Quando conviene}:
\begin{equation}
\text{Costo}(\text{spazio extra}) < \text{Beneficio}(\text{tempo risparmiato})
\end{equation}

Per sistemi con:
\begin{itemize}
\item Molti cicli recognize-act
\item WM moderatamente grande ($m \gg 10$)
\item Cambiamenti piccoli tra cicli
\end{itemize}

RETE è quasi sempre vantaggioso.

\section{Join di Pattern}

\subsection{Il Problema del Join}

Quando due pattern condividono variabili, dobbiamo verificare consistenza:

\begin{lstlisting}[language=CLIPS]
(defrule stesso-reparto
  (impiegato (id ?id1) (reparto ?r))
  (impiegato (id ?id2&~?id1) (reparto ?r))  ; Stessa variabile ?r!
  =>
  ...)
\end{lstlisting}

\subsection{Join in Database}

Analogo al join relazionale:

\begin{equation}
R_1 \bowtie_{\theta} R_2 = \{(t_1, t_2) \mid t_1 \in R_1, t_2 \in R_2, \theta(t_1, t_2)\}
\end{equation}

dove $\theta$ è una condizione di join.

\textbf{Tecniche classiche}:
\begin{itemize}
\item \textbf{Nested loop join}: $O(|R_1| \cdot |R_2|)$
\item \textbf{Hash join}: $O(|R_1| + |R_2|)$ con preprocessing
\item \textbf{Sort-merge join}: $O(|R_1| \log |R_1| + |R_2| \log |R_2|)$
\end{itemize}

\subsection{Join in RETE}

RETE usa hash join incrementale:
\begin{enumerate}
\item Memorizza match parziali in hash table
\item Nuovo fatto $\rightarrow$ lookup nella hash table
\item Crea nuovi match combinando
\end{enumerate}

\textbf{Complessità amortizzata}: $O(1)$ per inserimento.

\section{Tipi di Pattern}

\subsection{Pattern Intra-elemento}

Test su un singolo fatto:

\begin{lstlisting}[language=CLIPS]
(persona (eta ?e&:(> ?e 18)&:(< ?e 65)))
\end{lstlisting}

\textbf{Complessità}: $O(m)$ - scansione lineare dei fatti.

\subsection{Pattern Inter-elemento}

Test che coinvolgono multiple condizioni:

\begin{lstlisting}[language=CLIPS]
(impiegato (id ?id) (stipendio ?s1))
(bonus (impiegato ?id) (importo ?b))
(test (> ?b (* 0.2 ?s1)))  ; Bonus > 20% stipendio
\end{lstlisting}

\textbf{Complessità}: Dipende dal numero di combinazioni.

\subsection{Pattern Negativi}

Negazione (assenza di match):

\begin{lstlisting}[language=CLIPS]
(not (ordine (cliente ?id)))
\end{lstlisting}

\textbf{Semantica}: Vero se \textit{non esiste} un match.

\textbf{Sfida}: Quando invalidare? Quando un fatto che match appare.

\section{Gestione della Negazione}

\subsection{Negation as Failure (NAF)}

\begin{definizione}[NAF]
Un pattern negato $(not~\varphi)$ è soddisfatto se non esiste binding che soddisfa $\varphi$.
\end{definizione}

\textbf{Problema}: Non monotonia.

\begin{esempio}[Non Monotonia della Negazione]
\begin{enumerate}
\item Stato iniziale: WM = $\{(persona~\text{"Mario"})\}$
\item Regola: \texttt{(persona ?x) (not (ordine ?x)) => ...}
\item Match esiste: Mario non ha ordini
\item Aggiungi: \texttt{(ordine "Mario")}
\item Match \textbf{scompare}!
\end{enumerate}
\end{esempio}

\subsection{Implementazione in RETE}

RETE gestisce la negazione con \textit{nodi beta negativi}:

\begin{itemize}
\item Mantengono count di match nel pattern negato
\item Count = 0 $\Rightarrow$ pattern negato soddisfatto
\item Aggiornano count incrementalmente
\end{itemize}

\section{Variabili Multifield}

\subsection{Challenge}

Le variabili multifield matchano zero o più valori:

\begin{lstlisting}[language=CLIPS]
(lista $?inizio 10 $?fine)
\end{lstlisting}

Match possibili per \texttt{(lista 1 2 10 3 4)}:
\begin{itemize}
\item $\texttt{\$?inizio} = [1,2]$, $\texttt{\$?fine} = [3,4]$
\item $\texttt{\$?inizio} = [1,2,10]$, $\texttt{\$?fine} = []$ (no, 10 deve essere letterale)
\end{itemize}

\subsection{Complessità}

Con $n$ multifield variables e fatto di lunghezza $k$:
\begin{equation}
O\left(\binom{k}{n}\right) \text{ possibili partizioni}
\end{equation}

\textbf{Esplosione combinatoria}: Evitare multifield non vincolate.

\begin{warningbox}[Best Practice]
Specificare sempre constraint o ancoraggi per multifield:
\begin{lstlisting}[language=CLIPS]
;; Meglio
(lista primo $?resto&:(> (length$ ?resto) 0))

;; Peggio (troppe possibilita)
(lista $?a $?b $?c)
\end{lstlisting}
\end{warningbox}

\section{Ordinamento dei Pattern}

\subsection{Selettività}

\begin{definizione}[Selettività]
La selettività di un pattern è la frazione di fatti in WM che lo soddisfano:
\begin{equation}
\text{sel}(p) = \frac{|\{f \in WM \mid f \text{ matches } p\}|}{|WM|}
\end{equation}
\end{definizione}

\subsection{Ordinamento Ottimale}

\textbf{Principio}: Valutare prima i pattern più selettivi.

\begin{teorema}[Ordinamento Ottimale]
Data una regola con pattern $p_1, \ldots, p_n$, l'ordinamento che minimizza il costo atteso è quello per selettività crescente:
\begin{equation}
\text{sel}(p_1) \leq \text{sel}(p_2) \leq \ldots \leq \text{sel}(p_n)
\end{equation}
\end{teorema}

\textbf{Intuizione}: Eliminare candidati presto riduce il lavoro successivo.

\subsection{Stima della Selettività}

\textbf{Tecniche}:
\begin{itemize}
\item \textbf{Statistiche}: Raccogliere durante esecuzione
\item \textbf{Euristica}: Pattern con più constraint $\Rightarrow$ più selettivo
\item \textbf{Profiling}: Analizzare run precedenti
\end{itemize}

\section{Indici e Strutture Dati}

\subsection{Hash Index}

Per pattern del tipo \texttt{(persona (id 123))}:

\begin{itemize}
\item Indicizzare fatti per tipo e slot
\item Hash su valore per accesso $O(1)$
\end{itemize}

\subsection{Trie per Pattern}

Per pattern complessi, trie discrimina su prefissi comuni:

\begin{figure}[h]
\centering
\begin{tikzpicture}[
  level distance=1.5cm,
  level 1/.style={sibling distance=3cm},
  level 2/.style={sibling distance=1.5cm}
]
  \node {root}
    child {node {tipo=persona}
      child {node {età>18}}
      child {node {età>65}}
    }
    child {node {tipo=ordine}
      child {node {stato=pending}}
    };
\end{tikzpicture}
\caption{Trie per discriminazione pattern}
\end{figure}

\section{Confronto con Altri Paradigmi}

\subsection{Query in Database}

\textbf{Somiglianze}:
\begin{itemize}
\item Pattern = query SQL
\item Working Memory = tabelle
\item Join di pattern = join relazionali
\end{itemize}

\textbf{Differenze}:
\begin{itemize}
\item DB: query singola su snapshot
\item CLIPS: query continue su stream di aggiornamenti
\item RETE: "standing queries" materializzate
\end{itemize}

\subsection{Pattern Matching Funzionale}

In linguaggi come Haskell/ML:

\begin{verbatim}
fib 0 = 0
fib 1 = 1  
fib n = fib (n-1) + fib (n-2)
\end{verbatim}

\textbf{Differenze}:
\begin{itemize}
\item Matching su struttura dati (non DB)
\item Sequenziale (non parallelo)
\item Deterministico (primo match vince)
\end{itemize}

\section{Metriche di Performance}

\subsection{Metriche Chiave}

\begin{table}[h]
\centering
\begin{tabular}{@{}ll@{}}
\toprule
\textbf{Metrica} & \textbf{Descrizione} \\
\midrule
Tempo per ciclo & Latenza recognize-act \\
Throughput & Cicli/secondo \\
Utilizzo memoria & Spazio per match parziali \\
Hit rate & Frazione match riutilizzati \\
Conflict set size & Numero attivazioni per ciclo \\
\bottomrule
\end{tabular}
\caption{Metriche di performance pattern matching}
\end{table}

\subsection{Profiling}

Strumenti per analizzare bottleneck:
\begin{itemize}
\item Tempo per nodo RETE
\item Distribuzione di firing
\item Crescita memoria nel tempo
\item Pattern più/meno selettivi
\end{itemize}

\section{Limiti e Problemi Aperti}

\subsection{Worst Case}

Anche RETE ha worst case $O(n \cdot m^k)$ quando:
\begin{itemize}
\item Tutti i fatti cambiano ogni ciclo
\item Pattern molto generici (bassa selettività)
\item Join con Cartesian product
\end{itemize}

\subsection{Problemi Aperti}

\begin{itemize}
\item \textbf{Adaptive indexing}: Riottimizzare indici dinamicamente
\item \textbf{Parallel matching}: RETE su GPU/multicore
\item \textbf{Distributed matching}: RETE su cluster
\item \textbf{Approximate matching}: Tolleranza a errori/rumore
\item \textbf{Learning}: Apprendere ordinamento pattern ottimale
\end{itemize}

\section{Conclusioni del Capitolo}

\subsection{Punti Chiave}

\begin{enumerate}
\item Pattern matching naïve è impraticabile per sistemi reali
\item Il \textbf{principio di temporalità} giustifica approcci incrementali
\item La \textbf{discriminazione} permette condivisione di lavoro
\item Join e negazione richiedono tecniche speciali
\item Trade-off spazio-tempo favorisce RETE in pratica
\end{enumerate}

\subsection{Prossimi Capitoli}

\begin{itemize}
\item Capitolo~\ref{cap:rete_introduzione}: Introduzione dettagliata a RETE
\item Capitolo~\ref{cap:rete_alpha}: Rete Alpha (discriminazione)
\item Capitolo~\ref{cap:rete_beta}: Rete Beta (join e negazione)
\item Capitolo~\ref{cap:rete_complessita}: Analisi di complessità formale
\item Capitolo~\ref{cap:rete_ottimizzazioni}: Ottimizzazioni e varianti
\end{itemize}

\subsection{Letture Consigliate}

\begin{itemize}
\item Forgy, C. (1982). "Rete: A Fast Algorithm for the Many Pattern/Many Object Pattern Match Problem"
\item Forgy, C. (1984). "The Rete Algorithm Detailed"
\item Giarratano \& Riley (2004). "Expert Systems" - Cap. 8
\item Brownston et al. (1985). "Programming Expert Systems in OPS5"
\item Doorenbos, R. (1995). "Production Matching for Large Learning Systems" (RETE/UL)
\end{itemize}

% Capitolo 6: Introduzione all'Algoritmo RETE

\chapter{L'Algoritmo RETE: Introduzione}
\label{cap:rete_intro}

\section{Il Problema del Pattern Matching Efficiente}

\subsection{Analisi del Problema}

Consideriamo un sistema a produzione con:
\begin{itemize}
\item $n$ regole nella production memory
\item $m$ fatti nella working memory
\item $k$ condizioni medie per regola
\end{itemize}

\subsubsection{Approccio Naïve}

L'approccio naïve ricalcola da zero ad ogni ciclo:

\begin{algorithm}[H]
\caption{Match Naïve}
\begin{algorithmic}[1]
\Function{MatchNaive}{$PM, WM$}
    \State $CS \gets \emptyset$
    \For{each $r \in PM$}
        \For{each combination $\langle w_1, \ldots, w_k \rangle$ of $k$ facts from $WM$}
            \If{$\langle w_1, \ldots, w_k \rangle$ matches $r.LHS$}
                \State $\theta \gets \text{extract\_bindings}(r.LHS, \langle w_1, \ldots, w_k \rangle)$
                \State $CS \gets CS \cup \{(r, \theta)\}$
            \EndIf
        \EndFor
    \EndFor
    \State \Return $CS$
\EndFunction
\end{algorithmic}
\end{algorithm}

\textbf{Complessità}:
\begin{equation}
T_{\text{naïve}} = O\left(n \cdot \binom{m}{k}\right) = O(n \cdot m^k)
\end{equation}

\begin{esempio}[Costo Computazionale]
Per un sistema realistico:
\begin{align*}
n &= 1000 \text{ regole}\\
m &= 10000 \text{ fatti}\\
k &= 3 \text{ condizioni/regola}
\end{align*}

Otteniamo:
\begin{equation}
T = 1000 \cdot 10000^3 = 10^{15} \text{ confronti}
\end{equation}

A 1 GHz (1 confronto/ns), servirebbero:
\begin{equation}
\frac{10^{15}}{10^9} = 10^6 \text{ secondi} \approx 11.6 \text{ giorni!}
\end{equation}
\end{esempio}

\subsection{L'Intuizione di Forgy}

Charles Forgy osservò due invarianti critici:

\begin{osservazione}[Continuità Temporale]
Tra un ciclo recognize-act e il successivo:
\begin{itemize}
\item La maggior parte dei fatti rimane invariata
\item Solo pochi fatti vengono asseriti o ritratti
\item Molti match parziali rimangono validi
\end{itemize}

Formalmente, se $WM_t$ è la working memory al ciclo $t$:
\begin{equation}
\frac{|WM_{t+1} \Delta WM_t|}{|WM_t|} \ll 1
\end{equation}

dove $\Delta$ denota differenza simmetrica.
\end{osservazione}

\begin{osservazione}[Similarità Strutturale]
Molte regole condividono pattern comuni:

\begin{lstlisting}[language=CLIPS]
(defrule r1
  (persona (nome ?n) (eta ?e))
  ...
  =>
  ...)

(defrule r2
  (persona (nome ?n) (eta ?e))
  ...
  =>
  ...)
\end{lstlisting}

Il pattern \texttt{(persona (nome ?n) (eta ?e))} è condiviso.
\end{osservazione}

\subsection{Idea Centrale di RETE}

L'algoritmo RETE sfrutta queste osservazioni costruendo una \textit{rete di nodi} che:

\begin{enumerate}
\item \textbf{Condivide} risultati di match parziali tra regole
\item \textbf{Memorizza} risultati intermedi per riuso
\item \textbf{Propaga} solo cambiamenti incrementali (delta)
\end{enumerate}

\begin{definizione}[Rete RETE]
Una rete RETE è un grafo diretto aciclico $G = (V, E)$ dove:
\begin{itemize}
\item $V$ è l'insieme dei nodi (alpha, beta, join, production)
\item $E$ è l'insieme degli archi (collegamenti parent-child)
\item Ogni nodo mantiene \textit{memoria locale} di match parziali
\item La propagazione è \textit{incrementale}: solo delta vengono processati
\end{itemize}
\end{definizione}

\section{Architettura della Rete}

\subsection{Tipologia di Nodi}

La rete RETE comprende quattro tipi principali di nodi:

\subsubsection{Nodi Alpha (Alpha Network)}

\begin{definizione}[Nodo Alpha]
Un nodo alpha $\alpha_i$ è associato a un singolo pattern $P_i$ e mantiene:
\begin{equation}
\text{memory}(\alpha_i) = \{w \in WM \mid w \text{ matcha } P_i\}
\end{equation}
\end{definizione}

Funzione: \textit{filtering} --- seleziona fatti che soddisfano un pattern.

\subsubsection{Nodi Beta (Beta Network)}

\begin{definizione}[Nodo Beta Memory]
Un nodo beta memory $\beta_i$ mantiene \textit{token}:
\begin{equation}
\text{memory}(\beta_i) = \{(w_1, \ldots, w_j) \mid \text{combinazione valida fino al pattern } j\}
\end{equation}
\end{definizione}

Funzione: \textit{memorizzazione} di match parziali multi-pattern.

\subsubsection{Nodi Join}

\begin{definizione}[Nodo Join]
Un nodo join $J_{i,j}$ combina:
\begin{itemize}
\item Input sinistro: token da beta memory $\beta_{i-1}$
\item Input destro: fatti da alpha node $\alpha_j$
\item Output: token estesi se join ha successo
\end{itemize}
\end{definizione}

Funzione: \textit{combinazione} di match parziali con nuovi fatti.

\subsubsection{Nodi Production}

\begin{definizione}[Nodo Production]
Un nodo production $\pi_r$ per la regola $r$:
\begin{itemize}
\item Riceve token completi (matchano tutto LHS)
\item Crea istanziazioni $(r, \theta)$ da aggiungere all'agenda
\item Non ha figli (nodo foglia)
\end{itemize}
\end{definizione}

\subsection{Struttura della Rete}

\begin{figure}[h]
\centering
\begin{tikzpicture}[
  scale=0.8,
  node distance=1.5cm,
  every node/.style={font=\small},
  alpha/.style={rectangle, draw=blue!50, fill=blue!10, thick, minimum width=2cm, minimum height=0.8cm},
  beta/.style={rectangle, draw=green!50, fill=green!10, thick, minimum width=2cm, minimum height=0.8cm},
  joinnode/.style={diamond, draw=orange!50, fill=orange!10, thick, minimum width=1.5cm, minimum height=1.5cm},
  prod/.style={ellipse, draw=red!50, fill=red!10, thick, minimum width=2cm, minimum height=0.8cm}
]

% Alpha network
\node[alpha] (a1) {$\alpha_1$: P1};
\node[alpha, right=of a1] (a2) {$\alpha_2$: P2};
\node[alpha, right=of a2] (a3) {$\alpha_3$: P3};

% Beta network - level 1
\node[joinnode, below=of a1] (j1) {$J_1$};
\node[beta, below=of j1] (b1) {$\beta_1$};

% Beta network - level 2
\node[joinnode, below=of b1] (j2) {$J_2$};
\node[beta, below=of j2] (b2) {$\beta_2$};

% Production node
\node[prod, below=of b2] (p1) {$\pi_r$: Rule R};

% Connections
\draw[->, thick] (a1) -- (j1);
\draw[->, thick] (a2) -- (j1);
\draw[->, thick] (j1) -- (b1);
\draw[->, thick] (b1) -- (j2);
\draw[->, thick] (a3) -- (j2);
\draw[->, thick] (j2) -- (b2);
\draw[->, thick] (b2) -- (p1);

% Labels
\node[left=0.5cm of a1, align=right] {\footnotesize Alpha\\Network};
\node[left=0.5cm of j1, align=right] {\footnotesize Beta\\Network};
\node[left=0.5cm of p1] {\footnotesize Production};

% WM connection
\node[above=0.5cm of a2] (wm) {\textbf{Working Memory}};
\draw[->, dashed, thick] (wm) -- (a1);
\draw[->, dashed, thick] (wm) -- (a2);
\draw[->, dashed, thick] (wm) -- (a3);

\end{tikzpicture}
\caption{Architettura generale rete RETE per regola con 3 pattern}
\label{fig:rete_overview}
\end{figure}

\section{Operazioni Fondamentali}

\subsection{Costruzione della Rete}

La rete viene costruita \textit{una volta} all'inizio, quando le regole vengono caricate:

\begin{algorithm}[H]
\caption{Costruzione Rete RETE}
\begin{algorithmic}[1]
\Function{BuildNetwork}{$PM$}
    \State $\alpha \gets \emptyset$ \Comment{Alpha nodes}
    \State $\beta \gets \emptyset$ \Comment{Beta nodes}
    \For{each rule $r \in PM$}
        \State $\text{prev} \gets \text{null}$
        \For{each pattern $P_i$ in $r.LHS$}
            \State $\alpha_i \gets \text{FindOrCreateAlpha}(P_i, \alpha)$
            \If{$\text{prev} = \text{null}$}
                \State $\text{prev} \gets \alpha_i$ \Comment{Primo pattern}
            \Else
                \State $J \gets \text{CreateJoin}(\text{prev}, \alpha_i)$
                \State $\beta_i \gets \text{CreateBetaMemory}()$
                \State $\text{prev} \gets \beta_i$
            \EndIf
        \EndFor
        \State $\pi_r \gets \text{CreateProduction}(r, \text{prev})$
    \EndFor
\EndFunction
\end{algorithmic}
\end{algorithm}

\subsection{Propagazione Assert}

Quando un fatto $w$ viene asserito:

\begin{algorithm}[H]
\caption{Propagazione Assert}
\begin{algorithmic}[1]
\Function{PropagateAssert}{$w, G$}
    \For{each alpha node $\alpha$ that matches $w$}
        \State $\alpha.\text{memory} \gets \alpha.\text{memory} \cup \{w\}$
        \State $\tau \gets \text{CreateToken}(w)$ \Comment{Token iniziale}
        \For{each child join $J$ of $\alpha$}
            \State $\text{PropagateTo}(J, \tau, \text{from-right})$
        \EndFor
    \EndFor
\EndFunction
\end{algorithmic}
\end{algorithm}

\subsection{Propagazione Retract}

Quando un fatto $w$ viene ritratto:

\begin{algorithm}[H]
\caption{Propagazione Retract}
\begin{algorithmic}[1]
\Function{PropagateRetract}{$w, G$}
    \For{each alpha node $\alpha$ containing $w$}
        \State $\alpha.\text{memory} \gets \alpha.\text{memory} \setminus \{w\}$
    \EndFor
    \For{each beta memory $\beta$ in $G$}
        \State $\text{affected} \gets \{\tau \in \beta.\text{memory} \mid w \in \tau\}$
        \State $\beta.\text{memory} \gets \beta.\text{memory} \setminus \text{affected}$
    \EndFor
    \State $A \gets A \setminus \{(r, \theta) \mid w \in \text{support}(r, \theta)\}$
\EndFunction
\end{algorithmic}
\end{algorithm}

\section{Analisi Preliminare di Complessità}

\subsection{Complessità Spaziale}

\subsubsection{Alpha Memory}

Ogni alpha node memorizza fatti:
\begin{equation}
\text{Space}(\alpha_i) = O(|WM_i|)
\end{equation}

dove $|WM_i|$ è il numero di fatti che matchano $P_i$.

Nel caso peggiore (pattern senza costanti): $|WM_i| = |WM|$

Totale alpha memory:
\begin{equation}
\text{Space}_\alpha = O(|\alpha| \cdot |WM|)
\end{equation}

\subsubsection{Beta Memory}

Ogni beta node al livello $j$ memorizza token di lunghezza $j$:
\begin{equation}
\text{Space}(\beta_j) = O(|WM|^j)
\end{equation}

Nel caso peggiore (tutti i fatti matchano):
\begin{equation}
\text{Space}_\beta = O\left(\sum_{j=1}^{k} |WM|^j\right) = O(|WM|^k)
\end{equation}

\begin{warningbox}[Esplosione Combinatoria]
La beta memory può crescere esponenzialmente! Questo è il \textit{beta memory blowup problem}.

In pratica, pattern ben progettati con costanti e join constraints limitano drasticamente la crescita.
\end{warningbox}

\subsection{Complessità Temporale}

\subsubsection{Ciclo Singolo}

Per un singolo ciclo recognize-act:

\textbf{Caso medio} (con $c$ fatti cambiati):
\begin{equation}
T_{\text{RETE}} = O(c \cdot n)
\end{equation}

\textbf{Caso peggiore} (tutti i fatti cambiano):
\begin{equation}
T_{\text{worst}} = O(m \cdot n)
\end{equation}

\subsubsection{Confronto Asintotic

o}

\begin{table}[h]
\centering
\begin{tabular}{@{}lcc@{}}
\toprule
\textbf{Approccio} & \textbf{Caso Medio} & \textbf{Caso Peggiore} \\
\midrule
Naïve & $O(n \cdot m^k)$ & $O(n \cdot m^k)$ \\
RETE & $O(c \cdot n)$ & $O(m \cdot n)$ \\
\midrule
\textbf{Speedup} & $\mathbf{\frac{m^k}{c}}$ & $\mathbf{m^{k-1}}$ \\
\bottomrule
\end{tabular}
\caption{Confronto complessità Naïve vs RETE}
\label{tab:complexity_comparison}
\end{table}

Con $m=10000$, $k=3$, $c=10$:
\begin{equation}
\text{Speedup} = \frac{10000^3}{10} = 10^{11} \text{ volte più veloce!}
\end{equation}

\section{Invarianti Fondamentali}

\subsection{Invariante di Correttezza}

\begin{teorema}[Correttezza RETE]
\label{thm:rete_correctness}
Sia $CS_{\text{naïve}}$ il conflict set calcolato con approccio naïve e $CS_{\text{RETE}}$ quello calcolato con RETE. Allora:
\begin{equation}
CS_{\text{RETE}} = CS_{\text{naïve}}
\end{equation}

per ogni stato della working memory.
\end{teorema}

La dimostrazione verrà fornita nel Capitolo \ref{cap:rete_beta} dopo aver definito formalmente tutti i nodi.

\subsection{Invariante di Consistenza}

\begin{definizione}[Consistenza Alpha]
Per ogni alpha node $\alpha_i$ e working memory $WM$:
\begin{equation}
\alpha_i.\text{memory} = \{w \in WM \mid w \text{ matcha } \alpha_i.\text{pattern}\}
\end{equation}
\end{definizione}

\begin{definizione}[Consistenza Beta]
Per ogni beta node $\beta_j$ al livello $j$:
\begin{equation}
\beta_j.\text{memory} = \{\tau \mid \tau \text{ è un match valido dei primi } j \text{ pattern}\}
\end{equation}
\end{definizione}

Questi invarianti devono essere mantenuti dopo ogni operazione (assert/retract).

\section{Token e Partial Matches}

\subsection{Definizione di Token}

\begin{definizione}[Token]
Un token $\tau$ al livello $j$ è una sequenza:
\begin{equation}
\tau = \langle w_1, w_2, \ldots, w_j \rangle
\end{equation}

dove ogni $w_i \in WM$ e la sequenza matcha i primi $j$ pattern della regola.
\end{definizione}

\subsection{Binding e Consistenza}

Un token $\tau$ ha associato un environment di binding $\theta_\tau$:

\begin{equation}
\theta_\tau: \text{Var}(P_1, \ldots, P_j) \to \text{Val}(WM)
\end{equation}

\textbf{Condizione di consistenza}: tutte le occorrenze della stessa variabile devono avere lo stesso valore.

\begin{esempio}[Binding Consistency]
Pattern:
\begin{lstlisting}[language=CLIPS]
(persona (nome ?n) (età ?e))
(esame (studente ?n) (voto ?v))
\end{lstlisting}

Se $\theta(?n) = \text{"Mario"}$ nel primo pattern, deve essere $\theta(?n) = \text{"Mario"}$ anche nel secondo.
\end{esempio}

\subsection{Join Keys}

\begin{definizione}[Join Keys]
Le \textit{join keys} per un nodo join sono le variabili condivise tra:
\begin{itemize}
\item Token del ramo sinistro (beta memory precedente)
\item Fatto del ramo destro (alpha node corrente)
\end{itemize}
\end{definizione}

Formalmente:
\begin{equation}
\text{JoinKeys}(J) = \text{Var}(\text{left}) \cap \text{Var}(\text{right})
\end{equation}

\section{Propagazione Incrementale}

\subsection{Assert Incrementale}

Quando viene asserito $w_{\text{new}}$:

\begin{enumerate}
\item Trova alpha nodes che matchano: $A = \{\alpha \mid \alpha.\text{pattern} \text{ matcha } w_{\text{new}}\}$
\item Per ogni $\alpha \in A$:
   \begin{enumerate}
   \item Aggiungi $w_{\text{new}}$ a $\alpha.\text{memory}$
   \item Crea token iniziale $\tau_0 = \langle w_{\text{new}} \rangle$
   \item Propaga $\tau_0$ ai join children di $\alpha$
   \end{enumerate}
\item I join tentano combinazioni con token esistenti nel ramo opposto
\item Token validi vengono propagati verso production nodes
\end{enumerate}

\textbf{Chiave}: solo il \textit{nuovo} fatto viene processato, non tutti i fatti.

\subsection{Retract Incrementale}

Quando viene ritratto $w_{\text{old}}$:

\begin{enumerate}
\item Rimuovi $w_{\text{old}}$ da tutti gli alpha nodes che lo contenevano
\item Trova tutti i token che includono $w_{\text{old}}$:
   \begin{equation}
   T_{\text{affected}} = \{\tau \in \bigcup_\beta \beta.\text{memory} \mid w_{\text{old}} \in \tau\}
   \end{equation}
\item Rimuovi $T_{\text{affected}}$ dalle beta memories
\item Rimuovi istanziazioni dipendenti dall'agenda:
   \begin{equation}
   A' = A \setminus \{(r, \theta) \mid w_{\text{old}} \in \text{support}(r, \theta)\}
   \end{equation}
\end{enumerate}

\section{Esempio Completo}

\subsection{Scenario}

Consideriamo un sistema semplice per rilevare coppie di amici:

\textbf{Regola}:
\begin{lstlisting}[language=CLIPS]
(defrule trova-coppia-amici
  (persona (nome ?n1) (hobby ?h))
  (persona (nome ?n2&~?n1) (hobby ?h))
  =>
  (printout t ?n1 " e " ?n2 " condividono hobby: " ?h crlf))
\end{lstlisting}

\textbf{Working Memory Iniziale}:
\begin{align*}
w_1 &= \text{persona}(\text{nome}: \text{"Alice"}, \text{hobby}: \text{"tennis"})\\
w_2 &= \text{persona}(\text{nome}: \text{"Bob"}, \text{hobby}: \text{"tennis"})\\
w_3 &= \text{persona}(\text{nome}: \text{"Carol"}, \text{hobby}: \text{"golf"})
\end{align*}

\subsection{Costruzione Rete}

\begin{enumerate}
\item \textbf{Alpha node} $\alpha_1$ per pattern \texttt{(persona (nome ?n1) (hobby ?h))}:
\begin{equation}
\alpha_1.\text{memory} = \{w_1, w_2, w_3\} \quad \text{(tutti matchano)}
\end{equation}

\item \textbf{Alpha node} $\alpha_2$ per pattern \texttt{(persona (nome ?n2) (hobby ?h))}:
\begin{equation}
\alpha_2.\text{memory} = \{w_1, w_2, w_3\}
\end{equation}

\item \textbf{Join node} $J$ con:
\begin{itemize}
\item Join key: \texttt{?h} (hobby condiviso)
\item Test: \texttt{?n2 \~{} ?n1} (nomi diversi)
\end{itemize}

\item \textbf{Beta memory} $\beta$ memorizza coppie valide

\item \textbf{Production node} $\pi$ crea istanziazioni per agenda
\end{enumerate}

\subsection{Esecuzione Passo-Passo}

\textbf{Inizializzazione}: Asserisci $w_1$, $w_2$, $w_3$

\begin{enumerate}
\item $\alpha_1.\text{memory} = \{w_1, w_2, w_3\}$
\item $\alpha_2.\text{memory} = \{w_1, w_2, w_3\}$
\item Join $J$ combina:
   \begin{itemize}
   \item $w_1$ con $w_2$: $\theta_1 = \{?n1 \mapsto \text{"Alice"}, ?n2 \mapsto \text{"Bob"}, ?h \mapsto \text{"tennis"}\}$ ✓
   \item $w_1$ con $w_3$: hobby diversi ✗
   \item $w_2$ con $w_1$: $\theta_2 = \{?n1 \mapsto \text{"Bob"}, ?n2 \mapsto \text{"Alice"}, ?h \mapsto \text{"tennis"}\}$ ✓
   \item Altri: falliscono per test $?n2 \neq ?n1$ o hobby diversi
   \end{itemize}
\item $\beta.\text{memory} = \{\langle w_1, w_2 \rangle, \langle w_2, w_1 \rangle\}$
\item Agenda: 2 istanziazioni
\end{enumerate}

\textbf{Assert} $w_4 = \text{persona}(\text{nome}: \text{"David"}, \text{hobby}: \text{"tennis"})$:

\begin{enumerate}
\item $\alpha_1.\text{memory} \gets \alpha_1.\text{memory} \cup \{w_4\}$
\item $\alpha_2.\text{memory} \gets \alpha_2.\text{memory} \cup \{w_4\}$
\item Join propaga solo combinazioni con $w_4$:
   \begin{itemize}
   \item $w_4$ con $w_1$: ✓
   \item $w_4$ con $w_2$: ✓
   \item $w_1$ con $w_4$: ✓
   \item $w_2$ con $w_4$: ✓
   \end{itemize}
\item $\beta.\text{memory}$ cresce da 2 a 6 token
\item Agenda: +4 nuove istanziazioni
\end{enumerate}

\textbf{Nota critica}: Solo le \textit{nuove} combinazioni vengono calcolate, non tutte da capo!

\section{Ottimizzazioni Fondamentali}

\subsection{Alpha Node Sharing}

Pattern identici condividono lo stesso alpha node:

\begin{lstlisting}[language=CLIPS]
(defrule r1
  (persona (età ?e))
  ...
  =>
  ...)

(defrule r2
  (persona (età ?e))
  ...
  =>
  ...)
\end{lstlisting}

Entrambe le regole usano lo stesso $\alpha_{\text{persona}}$.

\textbf{Beneficio}: Spazio e tempo di match risparmiati.

\subsection{Join Test Inlining}

Test semplici vengono eseguiti inline durante il join:

\begin{equation}
\text{JoinTest}(\tau_{\text{left}}, w_{\text{right}}) = \bigwedge_{v \in \text{JoinKeys}} \tau_{\text{left}}[v] = w_{\text{right}}[v]
\end{equation}

\textbf{Beneficio}: Fallimento rapido senza creazione token.

\subsection{Hash Indexing}

Beta memories usano hash table per lookup efficiente:

\begin{equation}
H(\tau) = \text{hash}\left(\bigoplus_{v \in \text{JoinKeys}} \tau[v]\right)
\end{equation}

Join diventa:
\begin{enumerate}
\item Calcola $h = H(w_{\text{right}})$
\item Cerca bucket $\beta.\text{hashTable}[h]$
\item Testa solo token in quel bucket
\end{enumerate}

\textbf{Complessità}: $O(1)$ atteso invece di $O(|\beta.\text{memory}|)$

\section{Varianti dell'Algoritmo}

\subsection{TREAT (Miranker, 1987)}

TREAT (Temporal RETE) elimina le beta memories:

\begin{itemize}
\item \textbf{Pro}: Spazio $O(|WM|)$ invece di $O(|WM|^k)$
\item \textbf{Contro}: Tempo peggiore, ricalcola join ad ogni ciclo
\end{itemize}

\textbf{Trade-off}: Spazio vs Tempo

\subsection{RETE-II}

Estensioni moderne includono:
\begin{itemize}
\item Parallel matching su multi-core
\item Incremental delete più efficiente
\item Garbage collection di nodi inutilizzati
\item Adaptive heuristics
\end{itemize}

\section{Conclusioni del Capitolo}

Abbiamo introdotto:

\begin{itemize}
\item L'architettura generale di RETE
\item I quattro tipi di nodi (alpha, beta, join, production)
\item Il meccanismo di propagazione incrementale
\item Analisi preliminare di complessità
\item Ottimizzazioni fondamentali
\end{itemize}

Nei prossimi capitoli approfondiremo:

\begin{itemize}
\item \textbf{Capitolo 7}: Alpha network in dettaglio
\item \textbf{Capitolo 8}: Beta network e join algorithm
\item \textbf{Capitolo 9}: Dimostrazione formale di correttezza e complessità
\item \textbf{Capitolo 10}: Ottimizzazioni avanzate
\end{itemize}

\begin{successbox}[Punti Chiave]
\begin{itemize}
\item RETE riduce complessità da $O(n \cdot m^k)$ a $O(c \cdot n)$ sfruttando continuità e condivisione
\item La rete è costruita una volta, poi usata incrementalmente
\item Invarianti di correttezza garantiscono equivalenza con match naïve
\item Trade-off spazio/tempo gestiti con ottimizzazioni (hashing, sharing)
\end{itemize}
\end{successbox}


% Capitolo 7: Rete Alpha - Discriminazione e Filtraggio

\chapter{Rete Alpha: Discriminazione dei Pattern}
\label{cap:rete_alpha}

\section{Introduzione}

La rete alpha è la prima componente dell'algoritmo RETE, responsabile della \textit{discriminazione} dei fatti: determinare quali fatti soddisfano i test intra-elemento dei pattern.

\subsection{Obiettivi della Rete Alpha}

\begin{infobox}[Funzioni Principali]
\begin{enumerate}
\item \textbf{Filtraggio}: Eliminare fatti che non soddisfano i constraint
\item \textbf{Condivisione}: Riutilizzare test comuni tra regole
\item \textbf{Incrementalità}: Aggiornare solo quando cambiano i fatti
\item \textbf{Indicizzazione}: Accesso rapido ai fatti rilevanti
\end{enumerate}
\end{infobox}

\section{Struttura della Rete Alpha}

\subsection{Tipo di Nodi}

\begin{definizione}[Nodi Alpha]
La rete alpha è un DAG (Directed Acyclic Graph) con nodi di tipo:
\begin{itemize}
\item \textbf{Root node}: Nodo iniziale, riceve tutti i fatti
\item \textbf{Type nodes}: Discriminano per tipo di fatto
\item \textbf{Test nodes}: Applicano constraint specifici
\item \textbf{Alpha memory}: Memorizzano fatti che passano i test
\end{itemize}
\end{definizione}

\begin{figure}[h]
\centering
\begin{tikzpicture}[
  node distance=2cm and 1.5cm,
  root/.style={circle, draw, thick, minimum size=1cm},
  type/.style={rectangle, draw, minimum width=2cm, minimum height=0.8cm},
  test/.style={rectangle, draw, rounded corners, minimum width=2cm, minimum height=0.8cm},
  memory/.style={ellipse, draw, thick, minimum width=1.5cm}
]
  \node[root] (root) {Root};
  \node[type, below left of=root] (persona) {type=persona};
  \node[type, below right of=root] (ordine) {type=ordine};
  \node[test, below of=persona] (eta18) {età>18};
  \node[test, below of=ordine] (pending) {stato=pending};
  \node[memory, below of=eta18] (mem1) {AM1};
  \node[memory, below of=pending] (mem2) {AM2};
  
  \draw[->] (root) -- (persona);
  \draw[->] (root) -- (ordine);
  \draw[->] (persona) -- (eta18);
  \draw[->] (ordine) -- (pending);
  \draw[->] (eta18) -- (mem1);
  \draw[->] (pending) -- (mem2);
\end{tikzpicture}
\caption{Esempio di rete alpha}
\label{fig:rete_alpha_esempio}
\end{figure}

\subsection{Alpha Memory}

\begin{definizione}[Alpha Memory]
Un nodo alpha memory mantiene l'insieme di tutti i fatti che hanno superato tutti i test nel cammino dalla root:
\begin{equation}
AM = \{f \in WM \mid f \text{ passa tutti i test}\}
\end{equation}
\end{definizione}

\textbf{Proprietà}:
\begin{itemize}
\item Aggiornate incrementalmente
\item Condivise tra regole con pattern identici
\item Fonte di input per rete beta
\end{itemize}

\section{Costruzione della Rete Alpha}

\subsection{Algoritmo di Compilazione}

\begin{algorithm}
\caption{Compila Pattern in Rete Alpha}
\begin{algorithmic}[1]
\Require Pattern $p = (tipo~(slot_1~v_1)~\ldots~(slot_n~v_n))$
\Ensure Alpha memory node per $p$
\Function{CompileAlphaPattern}{$p$}
  \State $node \gets root$
  \State $node \gets $ \Call{GetOrCreateTypeNode}{tipo}
  \For{each constraint $c$ in $p$}
    \If{$c$ è test intra-elemento}
      \State $node \gets $ \Call{GetOrCreateTestNode}{$node, c$}
    \EndIf
  \EndFor
  \State $am \gets $ \Call{GetOrCreateAlphaMemory}{$node$}
  \State \Return $am$
\EndFunction
\end{algorithmic}
\end{algorithm}

\subsection{Condivisione dei Nodi}

Regole con pattern simili condividono nodi:

\begin{lstlisting}[language=CLIPS]
;; Regola 1
(defrule r1
  (persona (eta ?e&:(> ?e 18)))
  =>
  ...)

;; Regola 2  
(defrule r2
  (persona (eta ?e&:(> ?e 18)) (citta "Roma"))
  =>
  ...)
\end{lstlisting}

Entrambe condividono:
\begin{itemize}
\item Type node per \texttt{persona}
\item Test node per \texttt{età > 18}
\end{itemize}

Ma R2 ha un test aggiuntivo per \texttt{città}.

\section{Tipi di Test}

\subsection{Test di Tipo}

Il test più comune: verificare il tipo del fatto.

\textbf{Implementazione}:
\begin{itemize}
\item Hash table: tipo $\rightarrow$ nodo
\item Lookup $O(1)$
\item Condivisione automatica
\end{itemize}

\subsection{Test su Costanti}

Test di uguaglianza con valore costante:

\begin{lstlisting}[language=CLIPS]
(persona (citta "Roma"))
\end{lstlisting}

\textbf{Ottimizzazione}: Indicizzare per valore.

\subsection{Test con Predicati}

Test arbitrari usando funzioni:

\begin{lstlisting}[language=CLIPS]
(persona (eta ?e&:(> ?e 18)&:(< ?e 65)))
\end{lstlisting}

\textbf{Implementazione}:
\begin{itemize}
\item Eseguire funzione su valore estratto
\item Caching di risultati quando possibile
\item Attenzione a side-effects!
\end{itemize}

\subsection{Test su Multifield}

Match parziale di sequenze:

\begin{lstlisting}[language=CLIPS]
(lista $?inizio 10 $?fine)
\end{lstlisting}

\textbf{Complessità}: Può richiedere backtracking.

\section{Propagazione dei Fatti}

\subsection{Assertion}

Quando un fatto viene asserito:

\begin{algorithm}
\caption{Propagazione Assert in Alpha}
\begin{algorithmic}[1]
\Require Fatto $f$
\Function{AlphaAssert}{$f$}
  \State $tipo \gets f.\text{type}$
  \State $typeNode \gets $ \Call{GetTypeNode}{$tipo$}
  \If{$typeNode = $ null}
    \State \Return \Comment{Nessuna regola per questo tipo}
  \EndIf
  \State \Call{PropagateAssert}{$typeNode, f$}
\EndFunction
\\
\Function{PropagateAssert}{$node, f$}
  \If{$node$ è test node}
    \If{not \Call{EvaluateTest}{$node.test, f$}}
      \State \Return \Comment{Test fallito}
    \EndIf
  \EndIf
  \If{$node$ è alpha memory}
    \State $node.facts \gets node.facts \cup \{f\}$
    \State \Call{NotifyBetaNetwork}{$node, f$}
  \EndIf
  \For{each child in $node.children$}
    \State \Call{PropagateAssert}{$child, f$}
  \EndFor
\EndFunction
\end{algorithmic}
\end{algorithm}

\textbf{Complessità}: $O(d)$ dove $d$ è la profondità del cammino.

\subsection{Retraction}

Quando un fatto viene retratto:

\begin{algorithm}
\caption{Propagazione Retract in Alpha}
\begin{algorithmic}[1]
\Require Fatto $f$
\Function{AlphaRetract}{$f$}
  \State $memories \gets $ \Call{FindAlphaMemories}{$f$}
  \For{each $am$ in $memories$}
    \State $am.facts \gets am.facts \setminus \{f\}$
    \State \Call{NotifyBetaNetwork}{$am, f, \text{retract}$}
  \EndFor
\EndFunction
\end{algorithmic}
\end{algorithm}

\textbf{Ottimizzazione}: Mantenere back-pointers dai fatti alle alpha memory.

\section{Ottimizzazioni}

\subsection{Hashing}

\textbf{Per tipo}:
\begin{itemize}
\item Hash table: tipo $\rightarrow$ type node
\item Evita scansione lineare
\end{itemize}

\textbf{Per valore}:
\begin{itemize}
\item Hash table: (slot, valore) $\rightarrow$ fatti
\item Per test di uguaglianza costante
\end{itemize}

\subsection{Indexing Multilivello}

Per pattern con più constraint:

\begin{lstlisting}[language=CLIPS]
(persona (eta 30) (citta "Roma"))
\end{lstlisting}

Indice composto: (età, città) $\rightarrow$ fatti.

\subsection{Lazy Evaluation}

Non valutare test finché necessario:

\begin{itemize}
\item Test costosi posticipati
\item Short-circuit evaluation
\item Caching di risultati
\end{itemize}

\section{Gestione della Memoria}

\subsection{Footprint della Rete Alpha}

\textbf{Nodi}:
\begin{equation}
O(n \cdot k)
\end{equation}
dove $n$ = numero regole, $k$ = condizioni medie.

\textbf{Alpha memories}:
\begin{equation}
O(m) \text{ per memory}
\end{equation}
dove $m$ = fatti in WM.

\textbf{Totale}:
\begin{equation}
O(n \cdot k + a \cdot m)
\end{equation}
dove $a$ = numero alpha memories.

\subsection{Garbage Collection}

\begin{itemize}
\item Rimuovere nodi non più referenziati
\item Compattare alpha memories
\item Deallocare quando regole rimosse
\end{itemize}

\section{Alpha Network in CLIPS}

\subsection{Strutture Dati C}

Dal codice CLIPS (\texttt{network.c, factmngr.c}):

\begin{lstlisting}[language=C]
struct patternNode {
    struct patternNode *nextLevel;
    struct patternNode *lastLevel;
    int networkTest;
    void *rightNode;
};

struct alphaMemory {
    struct partialMatch *beta;
    struct alphaMemory *next;
};
\end{lstlisting}

\subsection{Traduzione in Swift (SLIPS)}

\begin{lstlisting}[language=Swift]
class AlphaNode {
    var children: [AlphaNode] = []
    var test: AlphaTest?
    var memory: AlphaMemory?
}

class AlphaMemory {
    var facts: Set<Fact> = []
    var betaSubscribers: [BetaNode] = []
    
    func add(_ fact: Fact) {
        facts.insert(fact)
        notifyBeta(fact, operation: .assert)
    }
    
    func remove(_ fact: Fact) {
        facts.remove(fact)
        notifyBeta(fact, operation: .retract)
    }
}
\end{lstlisting}

\section{Analisi delle Prestazioni}

\subsection{Caso Medio}

\textbf{Assert}:
\begin{itemize}
\item Lookup tipo: $O(1)$
\item Traversal cammino: $O(d)$ dove $d \approx 3-5$
\item Test: $O(1)$ per test
\item Inserimento in memory: $O(1)$
\end{itemize}

\textbf{Totale}: $O(d) \approx O(1)$ con $d$ piccolo.

\subsection{Caso Pessimo}

\textbf{Quando}:
\begin{itemize}
\item Tipo molto comune (molti fatti)
\item Pattern molto generici (pochi test)
\item Multifield con backtracking
\end{itemize}

\textbf{Complessità}: Può degradare a $O(m)$.

\section{Varianti e Estensioni}

\subsection{TREAT}

Alternativa a RETE che non memorizza fatti in alpha memories:

\textbf{Pro}:
\begin{itemize}
\item Meno memoria
\item Adatto a WM volatile
\end{itemize}

\textbf{Contro}:
\begin{itemize}
\item Più tempo per ciclo (re-matching)
\end{itemize}

\subsection{Lazy RETE}

Calcola alpha memories on-demand:

\begin{itemize}
\item Costruisce rete dinamicamente
\item Risparmia memoria per regole rare
\item Trade-off: primo match lento
\end{itemize}

\section{Testing e Debugging}

\subsection{Visualizzazione}

Strumenti per ispezionare rete alpha:

\begin{itemize}
\item Dump struttura grafo
\item Statistiche per nodo (hit rate, num fatti)
\item Cammini attivi vs inattivi
\end{itemize}

\subsection{Profiling}

Metriche utili:

\begin{table}[h]
\centering
\begin{tabular}{@{}ll@{}}
\toprule
\textbf{Metrica} & \textbf{Significato} \\
\midrule
Nodi visitati/assert & Efficienza traversal \\
Test falliti & Selettività pattern \\
Alpha memory size & Utilizzo memoria \\
Sharing factor & Riuso nodi \\
\bottomrule
\end{tabular}
\caption{Metriche profiling rete alpha}
\end{table}

\section{Conclusioni del Capitolo}

\subsection{Punti Chiave}

\begin{enumerate}
\item La rete alpha implementa \textbf{discriminazione efficiente} dei fatti
\item La \textbf{condivisione} dei nodi riduce duplicazione
\item L'\textbf{incrementalità} è chiave per le prestazioni
\item Le \textbf{alpha memories} interfacciano con la rete beta
\item Trade-off memoria-tempo generalmente favorevole
\end{enumerate}

\subsection{Collegamento con Rete Beta}

Le alpha memories forniscono input alla rete beta per:
\begin{itemize}
\item Join tra pattern
\item Gestione della negazione
\item Combinazione di condizioni
\end{itemize}

Vedi Capitolo~\ref{cap:rete_beta} per dettagli.

\subsection{Letture Consigliate}

\begin{itemize}
\item Forgy, C. (1982). "Rete: A Fast Algorithm..." - Sezione 2-3
\item CLIPS Architecture Manual - Capitolo "Pattern Network"
\item Doorenbos, R. (1995). "Production Matching..." - RETE/UL alpha network
\item Miranker, D. (1990). "TREAT: A New Efficient Match Algorithm"
\end{itemize}

% Capitolo 8: Rete Beta - Join e Combinazione di Pattern

\chapter{Rete Beta: Join e Negazione}
\label{cap:rete_beta}

\section{Introduzione}

La rete beta è la seconda componente dell'algoritmo RETE, responsabile della \textit{combinazione} di pattern: verificare constraint inter-elemento e costruire match completi per le regole.

\subsection{Responsabilità della Rete Beta}

\begin{infobox}[Funzioni Principali]
\begin{enumerate}
\item \textbf{Join}: Combinare match di pattern diversi
\item \textbf{Negazione}: Gestire pattern negativi (NOT)
\item \textbf{Test inter-elemento}: Verificare constraint tra fatti
\item \textbf{Partial match storage}: Memorizzare risultati intermedi
\item \textbf{Conflict set generation}: Produrre attivazioni complete
\end{enumerate}
\end{infobox}

\section{Struttura della Rete Beta}

\subsection{Tipi di Nodi Beta}

\begin{definizione}[Nodi Beta]
La rete beta è un albero binario con nodi di tipo:
\begin{itemize}
\item \textbf{Join node}: Combina due stream di partial match
\item \textbf{Negative node}: Implementa negazione (NOT)
\item \textbf{Beta memory}: Memorizza partial match intermedi
\item \textbf{Production node}: Terminale, genera attivazioni
\end{itemize}
\end{definizione}

\begin{figure}[h]
\centering
\begin{tikzpicture}[
  node distance=2cm and 2cm,
  alpha/.style={ellipse, draw, fill=blue!20, minimum width=1.5cm},
  joinnode/.style={trapezium, draw, fill=green!20, minimum width=1.5cm, trapezium left angle=70, trapezium right angle=110},
  beta/.style={rectangle, draw, fill=yellow!20, minimum width=1.5cm},
  prod/.style={rectangle, draw, fill=red!20, minimum width=1.5cm, thick}
]
  \node[alpha] (am1) {AM1};
  \node[alpha, right=of am1] (am2) {AM2};
  \node[joinnode, below=of am1, xshift=2cm] (j1) {Join};
  \node[beta, below=of j1] (bm1) {BM1};
  \node[alpha, below=of am2, xshift=2cm] (am3) {AM3};
  \node[joinnode, below=of bm1, xshift=1.5cm] (j2) {Join};
  \node[prod, below=of j2] (p1) {Rule R1};
  
  \draw[->] (am1) -- (j1);
  \draw[->] (am2) -- (j1);
  \draw[->] (j1) -- (bm1);
  \draw[->] (bm1) -- (j2);
  \draw[->] (am3) -- (j2);
  \draw[->] (j2) -- (p1);
\end{tikzpicture}
\caption{Esempio di rete beta con join nodes}
\label{fig:rete_beta_esempio}
\end{figure}

\subsection{Partial Match}

\begin{definizione}[Partial Match]
Un partial match (token) è una tupla ordinata di fatti che soddisfano i primi $k$ pattern di una regola:
\begin{equation}
t = (f_1, f_2, \ldots, f_k) \quad \text{con binding } \theta
\end{equation}
\end{definizione}

\textbf{Esempio}:
\begin{lstlisting}[language=CLIPS]
(defrule r1
  (persona (id ?id) (nome ?n))       ; Pattern 1
  (ordine (cliente ?id) (totale ?t)) ; Pattern 2
  =>
  ...)

;; Partial match dopo pattern 1:
;; token1 = ([persona id=123 nome="Mario"], {?id=123, ?n="Mario"})

;; Partial match completo (dopo pattern 2):
;; token2 = ([persona id=123 ...], [ordine cliente=123 totale=100], 
;;           {?id=123, ?n="Mario", ?t=100})
\end{lstlisting}

\section{Join Nodes}

\subsection{Funzionamento}

Un join node combina:
\begin{itemize}
\item \textbf{Left input}: Stream di partial match (da beta memory)
\item \textbf{Right input}: Stream di fatti (da alpha memory)
\end{itemize}

\textbf{Output}: Nuovi partial match che soddisfano i join tests.

\subsection{Join Tests}

\begin{definizione}[Join Test]
Un join test è una condizione che deve essere soddisfatta per combinare un partial match con un nuovo fatto:
\begin{equation}
\text{test}(token, fatto) \in \{\text{true}, \text{false}\}
\end{equation}
\end{definizione}

\textbf{Tipi comuni}:
\begin{itemize}
\item Uguaglianza di variabili: \texttt{?id} nel pattern 1 = \texttt{?id} nel pattern 2
\item Predicati: \texttt{(test (> ?x ?y))}
\item Binding consistency
\end{itemize}

\subsection{Algoritmo di Join}

\begin{algorithm}
\caption{Right Activation (nuovo fatto)}
\begin{algorithmic}[1]
\Require Join node $j$, Fatto $f$
\Function{RightActivate}{$j, f$}
  \State $leftMemory \gets j.leftParent.memory$
  \For{each token $t$ in $leftMemory$}
    \If{\Call{EvaluateJoinTests}{$j.tests, t, f$}}
      \State $newToken \gets $ \Call{Extend}{$t, f$}
      \State \Call{Propagate}{$j.children, newToken$}
    \EndIf
  \EndFor
\EndFunction
\end{algorithmic}
\end{algorithm}

\begin{algorithm}
\caption{Left Activation (nuovo partial match)}
\begin{algorithmic}[1]
\Require Join node $j$, Token $t$
\Function{LeftActivate}{$j, t$}
  \State $rightMemory \gets j.rightParent.memory$
  \For{each fact $f$ in $rightMemory$}
    \If{\Call{EvaluateJoinTests}{$j.tests, t, f$}}
      \State $newToken \gets $ \Call{Extend}{$t, f$}
      \State \Call{Propagate}{$j.children, newToken$}
    \EndIf
  \EndFor
\EndFunction
\end{algorithmic}
\end{algorithm}

\subsection{Complessità}

**Worst case** (senza hash join):
\begin{equation}
O(|left| \cdot |right|)
\end{equation}

**Con hash join** (quando possibile):
\begin{equation}
O(|left| + |right|)
\end{equation}

\section{Hash Join Optimization}

\subsection{Principio}

Quando il join test è un'uguaglianza su variabile:

\begin{lstlisting}[language=CLIPS]
(pattern1 ... ?x ...)
(pattern2 ... ?x ...)  ; Stesso ?x
\end{lstlisting}

Possiamo indicizzare per valore di \texttt{?x}.

\subsection{Implementazione}

\begin{lstlisting}[language=Swift]
class BetaMemory {
    var tokens: Set<Token> = []
    var hashIndex: [Int: Set<Token>] = [:]  // valore -> tokens
    
    func add(_ token: Token, hashOn variable: String) {
        tokens.insert(token)
        if let value = token.binding[variable] {
            let hash = value.hashValue
            hashIndex[hash, default: []].insert(token)
        }
    }
    
    func lookup(value: Value) -> Set<Token> {
        return hashIndex[value.hashValue] ?? []
    }
}
\end{lstlisting}

\textbf{Complessità lookup}: $O(1)$ attesa.

\section{Negative Nodes}

\subsection{Semantica}

Un pattern negato è soddisfatto quando \textit{nessun} fatto match esiste:

\begin{lstlisting}[language=CLIPS]
(defrule no-orders
  (cliente (id ?id))
  (not (ordine (cliente ?id)))  ; Negazione!
  =>
  (printout t "Cliente " ?id " senza ordini" crlf))
\end{lstlisting}

\subsection{Implementazione con Counter}

\begin{definizione}[Negative Node]
Un negative node mantiene per ogni partial match un \textit{counter} di quanti fatti matchano il pattern negato:
\begin{equation}
\text{count}(token) = |\{f \mid \text{match}(token, f)\}|
\end{equation}
Token con $\text{count} = 0$ sono propagati.
\end{definizione}

\begin{algorithm}
\caption{Negative Node - Right Activation}
\begin{algorithmic}[1]
\Require Negative node $n$, Fatto $f$
\Function{NegativeRightActivate}{$n, f$}
  \For{each token $t$ in $n.leftMemory$}
    \If{\Call{MatchJoinTests}{$n.tests, t, f$}}
      \State $t.negCount \gets t.negCount + 1$
      \If{$t.negCount = 1$}  \Comment{Era 0, ora non più}
        \State \Call{RemoveFromChildren}{$n, t$}
      \EndIf
    \EndIf
  \EndFor
\EndFunction
\end{algorithmic}
\end{algorithm}

\begin{algorithm}
\caption{Negative Node - Retract}
\begin{algorithmic}[1]
\Require Negative node $n$, Fatto $f$
\Function{NegativeRetract}{$n, f$}
  \For{each token $t$ in $n.leftMemory$}
    \If{\Call{MatchJoinTests}{$n.tests, t, f$}}
      \State $t.negCount \gets t.negCount - 1$
      \If{$t.negCount = 0$}  \Comment{Ora soddisfatto!}
        \State \Call{PropagateToChildren}{$n, t$}
      \EndIf
    \EndIf
  \EndFor
\EndFunction
\end{algorithmic}
\end{algorithm}

\subsection{Esempio Dettagliato}

\textbf{Stato iniziale}:
\begin{itemize}
\item WM = \{(cliente id=1), (cliente id=2)\}
\item Token $t_1$ per cliente 1: count = 0 $\Rightarrow$ propagato
\item Token $t_2$ per cliente 2: count = 0 $\Rightarrow$ propagato
\end{itemize}

\textbf{Assert} \texttt{(ordine cliente=1)}:
\begin{itemize}
\item Match $t_1$: count diventa 1
\item $t_1$ ritirato dai children
\item $t_2$ rimane (count ancora 0)
\end{itemize}

\textbf{Retract} \texttt{(ordine cliente=1)}:
\begin{itemize}
\item $t_1$: count torna a 0
\item $t_1$ ripropagato ai children
\end{itemize}

\section{Beta Memories}

\subsection{Scopo}

Le beta memories memorizzano partial match intermedi per:
\begin{itemize}
\item Evitare ri-costruzione
\item Fornire left input ai join successivi
\item Implementare propagazione incrementale
\end{itemize}

\subsection{Strutture Dati}

\textbf{Opzione 1: Set}
\begin{lstlisting}[language=Swift]
class BetaMemory {
    var tokens: Set<Token> = []
}
\end{lstlisting}

**Pro**: Semplice, no duplicati \\
**Contro**: Overhead di hashing

\textbf{Opzione 2: List}
\begin{lstlisting}[language=Swift]
class BetaMemory {
    var tokens: [Token] = []
}
\end{lstlisting}

**Pro**: Cache-friendly, iteration veloce \\
**Contro**: Possibili duplicati, rimozione $O(n)$

\subsection{Garbage Collection}

\textbf{Problema}: Token obsoleti accumulano memoria.

\textbf{Soluzione}:
\begin{itemize}
\item Reference counting da figli
\item Periodic cleanup
\item Compattazione quando memoria critica
\end{itemize}

\section{Production Nodes}

\subsection{Funzione}

I production nodes sono le foglie della rete beta:
\begin{itemize}
\item Ricevono match completi
\item Generano attivazioni
\item Popolano il conflict set
\end{itemize}

\subsection{Attivazione}

\begin{lstlisting}[language=Swift]
class ProductionNode {
    let rule: Rule
    var activations: Set<Activation> = []
    
    func activate(token: Token) {
        let activation = Activation(
            rule: rule, 
            token: token,
            salience: rule.salience
        )
        activations.insert(activation)
        agenda.add(activation)
    }
    
    func deactivate(token: Token) {
        if let act = activations.first(where: { $0.token == token }) {
            activations.remove(act)
            agenda.remove(act)
        }
    }
}
\end{lstlisting}

\section{Propagazione Token}

\subsection{Assert di Fatto}

Percorso di propagazione:
\begin{enumerate}
\item Fatto entra in alpha memory
\item Right-activate tutti i join che dipendono da quella AM
\item Per ogni join match:
  \begin{enumerate}
  \item Crea nuovo token
  \item Aggiungi a beta memory figlio
  \item Left-activate join successivo
  \end{enumerate}
\item Continua fino a production node
\end{enumerate}

\subsection{Retract di Fatto}

\textbf{Sfida}: Trovare tutti i token che dipendono dal fatto retratto.

\textbf{Soluzione 1: Top-down deletion}
\begin{itemize}
\item Rimuovi fatto da alpha memory
\item Propaga delete attraverso rete
\item Rimuovi token che contengono il fatto
\end{itemize}

\textbf{Soluzione 2: Token tagging}
\begin{itemize}
\item Ogni token referenzia i fatti costituenti
\item Al retract, cerca token con quel fatto
\item Rimuovi direttamente
\end{itemize}

\section{Ottimizzazioni}

\subsection{Node Sharing}

Regole con prefissi comuni condividono nodi beta:

\begin{lstlisting}[language=CLIPS]
(defrule r1
  (a) (b) (c1)
  => ...)

(defrule r2
  (a) (b) (c2)
  => ...)
\end{lstlisting}

Condividono i join per (a) e (b), divergono su (c1) vs (c2).

\subsection{Right Unlinking}

Se alpha memory è vuota, disattiva temporaneamente join:

\begin{itemize}
\item No right input $\Rightarrow$ no match possibili
\item Risparmia left activations inutili
\item Riattiva quando arriva primo fatto
\end{itemize}

\subsection{Left Unlinking}

Dualmente, se beta memory sinistra è vuota:

\begin{itemize}
\item No left input $\Rightarrow$ no match
\item Risparmia right activations
\end{itemize}

\section{Analisi di Complessità}

\subsection{Spazio}

\textbf{Numero di token}:
\begin{equation}
O\left(\prod_{i=1}^{k} |AM_i|\right) \approx O(m^k)
\end{equation}

nel worst case (pattern generici, cross-product).

\textbf{In pratica}: Molto minore grazie a:
\begin{itemize}
\item Selettività dei pattern
\item Join tests stringenti
\item Condivisione dei nodi
\end{itemize}

\subsection{Tempo per Ciclo}

**Assert**:
\begin{equation}
O(\text{\# activations} \cdot \text{costo join}) \approx O(a)
\end{equation}

dove $a$ = numero di join attivati.

**Tipicamente** $a \ll m$ grazie a selettività.

\section{Implementazione in CLIPS}

\subsection{Codice C Rilevante}

Dal file \texttt{reteutil.c}:

\begin{lstlisting}[language=C]
struct joinNode {
    struct betaMemory *leftMemory;
    struct alphaMemoryHash *rightMemory;
    struct expr *networkTest;
    struct joinNode *nextLevel;
};

struct partialMatch {
    unsigned int count;
    struct partialMatch *next;
    struct fact **binds;
};
\end{lstlisting}

\subsection{Traduzione SLIPS}

\begin{lstlisting}[language=Swift]
class JoinNode: BetaNode {
    weak var leftParent: BetaMemory?
    weak var rightParent: AlphaMemory?
    var joinTests: [JoinTest] = []
    
    override func rightActivate(fact: Fact) {
        guard let left = leftParent else { return }
        for token in left.tokens {
            if evaluateTests(token, fact) {
                let newToken = token.extend(with: fact)
                propagateLeft(newToken)
            }
        }
    }
    
    override func leftActivate(token: Token) {
        guard let right = rightParent else { return }
        for fact in right.facts {
            if evaluateTests(token, fact) {
                let newToken = token.extend(with: fact)
                propagateLeft(newToken)
            }
        }
    }
}
\end{lstlisting}

\section{Testing e Debugging}

\subsection{Invarianti da Verificare}

\begin{enumerate}
\item Token in beta memory devono essere consistenti
\item Counter nei negative nodes mai negativo
\item Token in production node hanno tutte le variabili bound
\item No token duplicati (senza semantica bag)
\item Activations corrispondono a token validi
\end{enumerate}

\subsection{Strumenti di Debug}

\begin{itemize}
\item \textbf{Token tracer}: Segue propagazione di specifici token
\item \textbf{Memory dump}: Snapshot di beta memories
\item \textbf{Join profiler}: Statistiche su hit/miss di join
\item \textbf{Activation logger}: Log di aggiunta/rimozione attivazioni
\end{itemize}

\section{Conclusioni del Capitolo}

\subsection{Punti Chiave}

\begin{enumerate}
\item La rete beta \textbf{combina pattern} tramite join incrementali
\item I \textbf{partial match} (token) memorizzano risultati intermedi
\item La \textbf{negazione} usa counter per test di assenza
\item L'\textbf{hash join} ottimizza join su variabili comuni
\item Trade-off spazio-tempo cruciale per prestazioni
\end{enumerate}

\subsection{Integrazione Alpha-Beta}

\begin{itemize}
\item Alpha filtra, Beta combina
\item Alpha memories = input destro per join
\item Beta memories = input sinistro per join
\item Propagazione bidirezionale (left/right activation)
\end{itemize}

\subsection{Prossimi Passi}

\begin{itemize}
\item Capitolo~\ref{cap:rete_complessita}: Analisi formale della complessità
\item Capitolo~\ref{cap:rete_ottimizzazioni}: Tecniche avanzate di ottimizzazione
\end{itemize}

\subsection{Letture Consigliate}

\begin{itemize}
\item Forgy, C. (1982). "Rete: A Fast Algorithm..." - Sezione 4-6
\item Doorenbos, R. (1995). "Production Matching..." - Beta network e RETE/UL
\item CLIPS Architecture Manual - "Join Network"
\item Brant, D. et al. (1991). "A Fast Algorithm for Production System Execution"
\end{itemize}

% Capitolo 9: Analisi di Complessità dell'Algoritmo RETE

\chapter{Complessità Computazionale di RETE}
\label{cap:rete_complessita}

\section{Introduzione}

L'analisi formale della complessità dell'algoritmo RETE è fondamentale per comprenderne i limiti teorici e le prestazioni attese in scenari reali.

\section{Parametri del Modello}

\subsection{Notazione}

\begin{table}[h]
\centering
\begin{tabular}{@{}cl@{}}
\toprule
\textbf{Simbolo} & \textbf{Significato} \\
\midrule
$n$ & Numero di regole (productions) \\
$m$ & Numero di fatti in working memory \\
$k$ & Numero medio di condizioni per regola \\
$a$ & Numero di alpha memories \\
$d$ & Profondità media rete alpha \\
$s$ & Selettività media dei pattern \\
$c$ & Dimensione media conflict set \\
\bottomrule
\end{tabular}
\caption{Parametri di complessità}
\end{table}

\subsection{Assunzioni}

\begin{itemize}
\item Distribuzione uniforme dei tipi di fatti
\item Pattern indipendenti (no correlazioni forti)
\item Join tests eseguibili in tempo $O(1)$
\item Hash table con lookup $O(1)$ atteso
\end{itemize}

\section{Complessità Spaziale}

\subsection{Rete Alpha}

\textbf{Nodi}:
\begin{equation}
O(n \cdot k \cdot d)
\end{equation}

dove $d$ è la profondità media dei cammini (tipicamente $d \leq 5$).

\textbf{Alpha memories}:
\begin{equation}
O(a \cdot \bar{m}_\alpha)
\end{equation}

dove $\bar{m}_\alpha$ è il numero medio di fatti per alpha memory.

Nel worst case $\bar{m}_\alpha = m$, ma tipicamente $\bar{m}_\alpha \ll m$ grazie alla selettività.

\subsection{Rete Beta}

\textbf{Nodi}:
\begin{equation}
O(n \cdot k)
\end{equation}

\textbf{Beta memories}:

Nel worst case (pattern molto generici):
\begin{equation}
O(m^k)
\end{equation}

\textbf{Esempio pessimo}:
\begin{lstlisting}[language=CLIPS]
(defrule cross-product
  (a ?x) (b ?y) (c ?z)  ; Nessun join test!
  =>
  ...)
\end{lstlisting}

Con 100 fatti di tipo a, b, c:
\begin{equation}
100 \times 100 \times 100 = 10^6 \text{ token}
\end{equation}

\textbf{Caso medio}:

Con selettività $s$ e join tests che riducono combinazioni di un fattore $r$:
\begin{equation}
O\left(\left(\frac{m \cdot s}{r}\right)^k\right)
\end{equation}

In pratica, con $s \approx 0.1$ e $r \approx 10$:
\begin{equation}
O((m \cdot 0.01)^k) \approx O(m) \text{ per } k \text{ piccolo}
\end{equation}

\subsection{Totale}

\begin{equation}
\text{Space}_{\text{RETE}} = O(n \cdot k) + O(a \cdot \bar{m}_\alpha) + O(\bar{t})
\end{equation}

dove $\bar{t}$ è il numero medio di token nelle beta memories.

\section{Complessità Temporale}

\subsection{Compilazione (Una Tantum)}

Costruire la rete RETE:
\begin{equation}
O(n \cdot k \cdot d)
\end{equation}

Operazione eseguita una sola volta all'inizio.

\subsection{Recognize Phase}

\subsubsection{Assert}

\textbf{Alpha network traversal}:
\begin{equation}
O(d) \approx O(1)
\end{equation}

\textbf{Right activations}: Per ogni alpha memory toccata, attiva join nodes.

Numero di join attivati:
\begin{equation}
O(a_f)
\end{equation}

dove $a_f$ = alpha memories che contengono il fatto.

\textbf{Join execution}:

Per ogni join:
\begin{itemize}
\item Con hash join: $O(h)$ dove $h$ = size dell'altra memory
\item Senza hash: $O(|left| \cdot |right|)$
\end{itemize}

\textbf{Totale per assert}:
\begin{equation}
O\left(\sum_{j \in J_f} \text{cost}(j)\right)
\end{equation}

dove $J_f$ = join attivati dal fatto $f$.

**Caso medio**: $O(a_f \cdot \bar{h})$ con $\bar{h}$ = dimensione media memory.

**Con buona selettività**: $O(c)$ dove $c$ = nuove attivazioni generate.

\subsubsection{Retract}

Simile ad assert, ma rimuove token e attivazioni.

\textbf{Con reference tracking}: $O(t_f)$ dove $t_f$ = token che contengono $f$.

\textbf{Senza tracking}: Potenzialmente $O(\bar{t})$ (scan tutte le memories).

\subsection{Act Phase}

Esecuzione RHS della regola scelta:
\begin{equation}
O(\text{azioni})
\end{equation}

Tipicamente $O(1)$ per regola semplice, ma può essere arbitrario.

\subsection{Ciclo Recognize-Act}

Un ciclo completo:
\begin{equation}
T_{\text{cycle}} = T_{\text{recognize}} + T_{\text{act}}
\end{equation}

Con $\Delta_m$ fatti modificati per ciclo:
\begin{equation}
T_{\text{recognize}} = O(\Delta_m \cdot c)
\end{equation}

\textbf{Chiave}: Se $\Delta_m \ll m$ (principio di temporalità), allora:
\begin{equation}
T_{\text{recognize}} \ll T_{\text{naive}}
\end{equation}

\section{Confronto con Algoritmo Naïve}

\begin{table}[h]
\centering
\begin{tabular}{@{}lcc@{}}
\toprule
\textbf{Metrica} & \textbf{Naïve} & \textbf{RETE} \\
\midrule
Spazio & $O(n)$ & $O(n \cdot k + \bar{t})$ \\
Tempo/ciclo & $O(n \cdot m^k)$ & $O(\Delta_m \cdot c)$ \\
Setup & $O(1)$ & $O(n \cdot k)$ \\
Incrementale & No & Sì \\
\bottomrule
\end{tabular}
\caption{Confronto complessità Naïve vs RETE}
\end{table}

\subsection{Breakeven Point}

RETE conviene quando:
\begin{equation}
\text{num cicli} \cdot (T_{\text{naive}} - T_{\text{RETE}}) > \text{Space}_{\text{RETE}} \cdot \text{cost}_{\text{mem}}
\end{equation}

In pratica, quasi sempre dopo pochi cicli.

\section{Worst Case vs Caso Medio}

\subsection{Scenari Worst Case}

\begin{enumerate}
\item \textbf{Pattern generici}:
\begin{lstlisting}[language=CLIPS]
(defrule any-fact
  (?) ; Matcha tutto!
  =>
  ...)
\end{lstlisting}

\item \textbf{Cross-product join}:
\begin{lstlisting}[language=CLIPS]
(defrule cartesian
  (a) (b) (c) ; Nessun join test
  =>
  ...)
\end{lstlisting}

\item \textbf{WM completamente rinnovata ogni ciclo}:
\begin{itemize}
\item $\Delta_m = m$
\item Nessun riuso di match
\end{itemize}

\end{enumerate}

\textbf{Complessità worst case}:
\begin{equation}
O(n \cdot m^k) \text{ per ciclo}
\end{equation}

Uguale al naïve!

\subsection{Caso Medio Realistico}

\textbf{Assunzioni tipiche}:
\begin{itemize}
\item Selettività pattern: $s \approx 0.1$
\item Fatti modificati: $\Delta_m \approx 0.01 \cdot m$
\item Join reduction: $r \approx 10$
\item Profondità regole: $k \leq 5$
\end{itemize}

\textbf{Complessità risultante}:
\begin{equation}
O(m) \text{ per ciclo}
\end{equation}

\textbf{Miglioramento}:
\begin{equation}
\text{speedup} \approx \frac{m^{k-1}}{\Delta_m \cdot c} \approx 10^3 \text{--} 10^6
\end{equation}

\section{Analisi Empirica}

\subsection{Studi Sperimentali}

\textbf{Forgy (1982)}:
\begin{itemize}
\item Test su sistemi reali (OPS5)
\item Speedup 100-1000x vs naïve
\item Overhead memoria accettabile (2-5x)
\end{itemize}

\textbf{Miranker (1990)}:
\begin{itemize}
\item Confronto RETE vs TREAT
\item RETE migliore per WM stabile
\item TREAT migliore per WM volatile
\end{itemize}

\textbf{Doorenbos (1995)}:
\begin{itemize}
\item RETE/UL (con unlinking)
\item Riduce overhead fino a 50\%
\item Memoria più efficiente
\end{itemize}

\subsection{Benchmark CLIPS}

Dati tipici da CLIPS su sistemi medium (1000 regole, 10000 fatti):

\begin{table}[h]
\centering
\begin{tabular}{@{}lrr@{}}
\toprule
\textbf{Operazione} & \textbf{Tempo} & \textbf{Note} \\
\midrule
Assert & 50 $\mu$s & Con aggiornamenti \\
Retract & 30 $\mu$s & Cleanup token \\
Fire & 100 $\mu$s & RHS semplice \\
Ciclo completo & 5 ms & 20 regole fired \\
\bottomrule
\end{tabular}
\caption{Benchmark CLIPS (ordini di grandezza)}
\end{table}

\section{Lower Bounds}

\subsection{Limiti Teorici}

\begin{teorema}[Lower Bound Pattern Matching]
Qualsiasi algoritmo per pattern matching incrementale richiede:
\begin{equation}
\Omega(\Delta_m + c)
\end{equation}
nel caso medio, dove $c$ = cambiamenti nel conflict set.
\end{teorema}

\textbf{Dimostrazione (sketch)}:
\begin{itemize}
\item Dobbiamo almeno "vedere" i $\Delta_m$ fatti modificati
\item Dobbiamo generare le $c$ nuove attivazioni
\item Quindi $\Omega(\Delta_m + c)$ è inevitabile
\end{itemize}

\textbf{Corollario}: RETE è \textit{quasi ottimo} nel caso medio!

\subsection{Trade-off Fondamentale}

\begin{teorema}[Space-Time Trade-off]
Per algoritmi di pattern matching:
\begin{equation}
\text{Space} \times \text{Time} \geq \Omega(m \cdot c)
\end{equation}
\end{teorema}

**Intuizione**:
\begin{itemize}
\item Poco spazio $\Rightarrow$ ricalcolo frequente
\item Molto spazio $\Rightarrow$ fast update
\item RETE sceglie il secondo estremo
\end{itemize}

\section{Varianti e Ottimizzazioni}

\subsection{TREAT (Miranker)}

\textbf{Complessità}:
\begin{itemize}
\item Spazio: $O(n \cdot k)$ (no beta memories!)
\item Tempo: $O(m \cdot c)$ per ciclo
\end{itemize}

\textbf{Trade-off}: Meno spazio, più tempo. Meglio per WM volatile.

\subsection{RETE/UL (Doorenbos)}

Con unlinking ottimizzato:
\begin{itemize}
\item Spazio: $O(t_{\text{active}})$ dove $t_{\text{active}} \ll t_{\text{total}}$
\item Tempo: Simile a RETE standard
\end{itemize}

\textbf{Beneficio}: Risparmio memoria significativo.

\subsection{Collection-Oriented Match (LEAPS)}

\textbf{Complessità}:
\begin{itemize}
\item Lazy evaluation di join
\item Spazio: $O(n \cdot k)$
\item Tempo: $O(m \cdot \log m)$ con indici
\end{itemize}

\textbf{Adatto per}: Query-driven execution.

\section{Conclusioni del Capitolo}

\subsection{Punti Chiave}

\begin{enumerate}
\item RETE ha \textbf{worst case} $O(m^k)$ spazio e $O(n \cdot m^k)$ tempo
\item Nel \textbf{caso medio}, complessità ridotta a $O(m)$ per ciclo
\item Il \textbf{principio di temporalità} è cruciale per efficienza
\item Trade-off spazio-tempo favorevole in pratica
\item RETE è \textbf{quasi ottimo} nel caso medio (lower bound)
\end{enumerate}

\subsection{Implicazioni Pratiche}

\begin{infobox}[Linee Guida]
\begin{itemize}
\item Pattern specifici riducono complessità esponenzialmente
\item Join tests sono essenziali per evitare cross-product
\item Monitorare crescita beta memories
\item Preferire selettività precoce nei pattern
\item Considerare TREAT se WM molto volatile
\end{itemize}
\end{infobox}

\subsection{Prossimi Passi}

Il Capitolo~\ref{cap:rete_ottimizzazioni} presenterà tecniche concrete per migliorare ulteriormente le prestazioni di RETE in scenari reali.

\subsection{Letture Consigliate}

\begin{itemize}
\item Forgy, C. (1982). "Rete: A Fast Algorithm..." - Analisi originale
\item Miranker, D. (1990). "TREAT: A New Efficient Match Algorithm"
\item Doorenbos, R. (1995). "Production Matching..." - Analisi RETE/UL
\item Perlin, M. (1990). "The RETE Algorithm, Theory and Practice"
\item Batory, D. (1994). "The LEAPS Algorithm"
\end{itemize}

% Capitolo 10: Ottimizzazioni dell'Algoritmo RETE

\chapter{Ottimizzazioni e Varianti di RETE}
\label{cap:rete_ottimizzazioni}

\section{Introduzione}

Sebbene RETE sia già altamente efficiente, esistono numerose ottimizzazioni e varianti che possono migliorarne ulteriormente le prestazioni in scenari specifici.

\section{Node Sharing}

\subsection{Condivisione tra Regole}

\textbf{Principio}: Regole con pattern comuni condividono nodi.

\begin{lstlisting}[language=CLIPS]
;; Tre regole con prefisso comune
(defrule r1 (a) (b) (c1) => ...)
(defrule r2 (a) (b) (c2) => ...)
(defrule r3 (a) (b) (c3) => ...)
\end{lstlisting}

\textbf{Struttura condivisa}:
\begin{itemize}
\item Un solo join per (a)
\item Un solo join per (b)
\item Tre join diversi per (c1), (c2), (c3)
\end{itemize}

\textbf{Benefici}:
\begin{itemize}
\item Riduzione nodi: da $3 \times 3 = 9$ a $2 + 3 = 5$
\item Riduzione beta memories
\item Join eseguiti una sola volta
\end{itemize}

\subsection{Implementazione}

\begin{algorithm}
\caption{Trova o Crea Nodo Condiviso}
\begin{algorithmic}[1]
\Function{GetOrCreateJoinNode}{$parent, pattern$}
  \For{each child in $parent.children$}
    \If{child.pattern $\equiv$ pattern}
      \State \Return child \Comment{Riusa esistente}
    \EndIf
  \EndFor
  \State $newNode \gets $ \Call{CreateJoinNode}{pattern}
  \State $parent.children$.append($newNode$)
  \State \Return newNode
\EndFunction
\end{algorithmic}
\end{algorithm}

\section{Right/Left Unlinking}

\subsection{Problema}

Join con input vuoto sprecano tempo:

\begin{itemize}
\item Alpha memory vuota $\Rightarrow$ no match possibili
\item Beta memory vuota $\Rightarrow$ no match possibili
\end{itemize}

\subsection{Soluzione: Unlinking}

\textbf{Right unlinking}:
\begin{itemize}
\item Se alpha memory diventa vuota, "scollega" join
\item Non processa left activations
\item Ricollega al primo assert
\end{itemize}

\textbf{Left unlinking}:
\begin{itemize}
\item Se beta memory sinistra vuota, scollega
\item Non processa right activations
\item Ricollega quando arriva primo token
\end{itemize}

\begin{lstlisting}[language=Swift]
class JoinNode {
    var linked: Bool = false
    
    func checkLinking() {
        let shouldLink = !leftMemory.isEmpty && !rightMemory.isEmpty
        if shouldLink && !linked {
            linked = true
            // Processa tutti i match accumulati
        } else if !shouldLink && linked {
            linked = false
        }
    }
    
    override func leftActivate(token: Token) {
        guard linked else { return }  // Skip se unlinked
        // ... normale processing
    }
}
\end{lstlisting}

\textbf{Speedup}: Fino a 50% in scenari con molti join inattivi.

\section{Hashing e Indexing}

\subsection{Hash Join}

Per join su uguaglianza di variabili:

\begin{lstlisting}[language=CLIPS]
(pattern1 ?x ...)
(pattern2 ?x ...)  ; Join test: ?x = ?x
\end{lstlisting}

\textbf{Ottimizzazione}:
\begin{itemize}
\item Indicizza token per valore di \texttt{?x}
\item Lookup $O(1)$ invece di scan $O(n)$
\end{itemize}

\begin{lstlisting}[language=Swift]
class HashJoinNode: JoinNode {
    var hashIndex: [Value: Set<Token>] = [:]
    
    override func leftActivate(token: Token) {
        let key = token.binding[joinVariable]!
        if let rightMatches = rightMemory.lookup(key) {
            for fact in rightMatches {
                let newToken = token.extend(with: fact)
                propagate(newToken)
            }
        }
    }
}
\end{lstlisting}

\subsection{Indexing Multilivello}

Per pattern con più constraint costanti:

\begin{lstlisting}[language=CLIPS]
(persona (eta 30) (citta "Roma") (professione "ingegnere"))
\end{lstlisting}

\textbf{Indice composto}: (eta, citta, professione) $\rightarrow$ fatti.

\textbf{Beneficio}: Da $O(m)$ a $O(1)$ per fatti specifici.

\section{Pattern Reordering}

\subsection{Ordinamento Ottimale}

\textbf{Euristiche per ordinare pattern}:

\begin{enumerate}
\item \textbf{Selettività}: Pattern più selettivi prima
\item \textbf{Costanti}: Pattern con costanti prima
\item \textbf{Variabili condivise}: Massimizzare early join pruning
\end{enumerate}

\begin{esempio}[Riordinamento]
\textbf{Originale}:
\begin{lstlisting}[language=CLIPS]
(defrule example
  (persona (citta ?c))         ; Generico: 1000 match
  (citta (nome ?c) (paese "IT")) ; Selettivo: 100 match
  (meteo (citta ?c) (temp ?t&:(> ?t 30))) ; Molto selettivo: 10 match
  =>
  ...)
\end{lstlisting}

\textbf{Ottimizzato}:
\begin{lstlisting}[language=CLIPS]
(defrule example-opt
  (meteo (citta ?c) (temp ?t&:(> ?t 30))) ; 10 match
  (citta (nome ?c) (paese "IT"))          ; Filter a 10
  (persona (citta ?c))                    ; Final join
  =>
  ...)
\end{lstlisting}

\textbf{Beneficio}:
\begin{itemize}
\item Originale: $1000 \times 100 \times 10 = 10^6$ combinazioni considerate
\item Ottimizzato: $10 \times 100 \times 1000 = 10^6$ ma con early pruning, praticamente $\approx 1000$
\end{itemize}
\end{esempio}

\subsection{Analisi Dinamica}

Raccogliere statistiche a runtime:

\begin{lstlisting}[language=Swift]
class PatternStatistics {
    var matchCount: Int = 0
    var totalFacts: Int = 0
    
    var selectivity: Double {
        guard totalFacts > 0 else { return 1.0 }
        return Double(matchCount) / Double(totalFacts)
    }
}

// Riordina pattern prima di compilare
func optimizeRulePatterns(_ rule: Rule) {
    rule.patterns.sort { p1, p2 in
        stats[p1]!.selectivity < stats[p2]!.selectivity
    }
}
\end{lstlisting}

\section{Partial Evaluation}

\subsection{Costanti Compile-Time}

Pre-calcolare test quando possibile:

\begin{lstlisting}[language=CLIPS]
;; Invece di
(test (> (* 10 5) 40))

;; Valutare a compile-time
(test TRUE)  ; Sempre vero
\end{lstlisting}

\subsection{Inlining}

Sostituire funzioni semplici con codice inline:

\begin{lstlisting}[language=Swift]
// Invece di call dinamica
if evaluatePredicate(">", value, 18) { ... }

// Inline diretto
if value > 18 { ... }
\end{lstlisting}

\section{Memory Management}

\subsection{Token Pooling}

\textbf{Problema}: Allocazione/deallocazione continua di token.

\textbf{Soluzione}: Object pool.

\begin{lstlisting}[language=Swift]
class TokenPool {
    private var pool: [Token] = []
    private let maxPoolSize = 1000
    
    func acquire(facts: [Fact]) -> Token {
        if let token = pool.popLast() {
            token.reset(with: facts)
            return token
        }
        return Token(facts: facts)
    }
    
    func release(_ token: Token) {
        guard pool.count < maxPoolSize else { return }
        pool.append(token)
    }
}
\end{lstlisting}

\textbf{Beneficio}: Riduzione garbage collection, locality migliore.

\subsection{Compact Token Representation}

Invece di:
\begin{lstlisting}[language=Swift]
struct Token {
    var facts: [Fact]          // Array completo
    var bindings: [String: Value]  // Dictionary
}
\end{lstlisting}

Usare:
\begin{lstlisting}[language=Swift]
struct CompactToken {
    var factIDs: [Int32]       // Solo ID (4 byte ciascuno)
    var bindingArray: [Value]  // Array flat, no hash overhead
    var bindingKeys: UInt64    // Bitmap per chiavi
}
\end{lstlisting}

\textbf{Beneficio}: 50-70% riduzione memoria per token.

\section{Parallel RETE}

\subsection{Parallelizzazione Join}

\textbf{Opportunità}:
\begin{itemize}
\item Join indipendenti processabili in parallelo
\item Alpha network intrinsecamente parallelizzabile
\end{itemize}

\begin{lstlisting}[language=Swift]
func rightActivateParallel(fact: Fact) {
    let affectedJoins = findAffectedJoins(fact)
    
    DispatchQueue.concurrentPerform(iterations: affectedJoins.count) { i in
        let join = affectedJoins[i]
        join.process(fact)
    }
}
\end{lstlisting}

\textbf{Sfida}: Sincronizzazione accesso a beta memories.

\subsection{Lock-Free Data Structures}

Per beta memories concorrenti:

\begin{lstlisting}[language=Swift]
class LockFreeBetaMemory {
    private var tokens = Atomic<Set<Token>>()
    
    func add(_ token: Token) -> Bool {
        tokens.modify { set in
            set.insert(token).inserted
        }
    }
}
\end{lstlisting}

\section{Incremental Compilation}

\subsection{Dynamic Rule Addition}

Aggiungere regole senza ricostruire l'intera rete:

\begin{algorithm}
\caption{Aggiungi Regola Incrementalmente}
\begin{algorithmic}[1]
\Function{AddRule}{$rule$}
  \For{each pattern in $rule.patterns$}
    \State $alphaNode \gets $ \Call{CompileAlpha}{pattern}
    \State $betaNode \gets $ \Call{IntegrateInBeta}{pattern, alphaNode}
  \EndFor
  \State $prodNode \gets $ \Call{CreateProductionNode}{rule}
  \State \Call{PropagateExistingFacts}{$prodNode$}
\EndFunction
\end{algorithmic}
\end{algorithm}

\textbf{Complessità}: $O(k \cdot d + m \cdot c_{\text{new}})$ invece di $O(n \cdot k)$.

\section{Specializzazioni}

\subsection{Fast Path per Pattern Semplici}

\begin{lstlisting}[language=Swift]
protocol PatternMatcher {
    func match(_ fact: Fact) -> Bool
}

// Fast path: test singolo
class SimpleEqualityMatcher: PatternMatcher {
    let slot: String
    let value: Value
    
    func match(_ fact: Fact) -> Bool {
        return fact[slot] == value  // Direct comparison
    }
}

// General path: test multipli
class ComplexMatcher: PatternMatcher {
    let tests: [Test]
    
    func match(_ fact: Fact) -> Bool {
        return tests.allSatisfy { $0.evaluate(fact) }
    }
}
\end{lstlisting}

\textbf{Beneficio}: Evitare overhead per casi comuni.

\subsection{Template Specialization}

Per tipi di fatti noti a compile-time:

\begin{lstlisting}[language=Swift]
// Invece di generic access
let age = fact.getValue(slot: "età") as! Int

// Specializzato
struct PersonaFact {
    let id: Int
    let nome: String
    let età: Int
}

// Accesso diretto
let age = personaFact.età
\end{lstlisting}

\section{Profiling e Tuning}

\subsection{Metriche da Monitorare}

\begin{table}[h]
\centering
\small
\begin{tabular}{@{}lll@{}}
\toprule
\textbf{Metrica} & \textbf{Target} & \textbf{Azione se Fuori} \\
\midrule
Token/Beta memory & < 1000 & Rivedere pattern \\
Conflict set size & 10-100 & Adjustare salience \\
Join hit rate & > 0.1 & Verificare selettività \\
Memory growth & Linear & Check memory leak \\
Avg cycle time & < 10 ms & Profiling dettagliato \\
\bottomrule
\end{tabular}
\caption{Metriche e target}
\end{table}

\subsection{Bottleneck Identification}

\begin{lstlisting}[language=Swift]
class ReteProfiler {
    var nodeExecutionTime: [Node: TimeInterval] = [:]
    var nodeActivationCount: [Node: Int] = [:]
    
    func profile<T>(_ node: Node, _ block: () -> T) -> T {
        let start = Date()
        defer {
            let elapsed = Date().timeIntervalSince(start)
            nodeExecutionTime[node, default: 0] += elapsed
            nodeActivationCount[node, default: 0] += 1
        }
        return block()
    }
    
    func topBottlenecks(n: Int) -> [(Node, TimeInterval)] {
        nodeExecutionTime.sorted { $0.value > $1.value }.prefix(n)
    }
}
\end{lstlisting}

\section{Varianti Algoritmiche}

\subsection{TREAT}

\textbf{Differenze da RETE}:
\begin{itemize}
\item No beta memories
\item Re-matching ad ogni ciclo
\item Meno memoria, più tempo
\end{itemize}

\textbf{Quando usare}: WM molto volatile, poche regole.

\subsection{LEAPS}

\textbf{Collection-Oriented Match}:
\begin{itemize}
\item Lazy evaluation
\item Query-driven
\item Ottimo per reasoning backward-chaining
\end{itemize}

\subsection{Gator/A-RETE}

\textbf{Adaptive RETE}:
\begin{itemize}
\item Switch tra RETE e TREAT dinamicamente
\item Monitoring di volatilità WM
\item Selezione automatica strategia
\end{itemize}

\section{Conclusioni del Capitolo}

\subsection{Punti Chiave}

\begin{enumerate}
\item \textbf{Node sharing} riduce nodi fino a 50\%
\item \textbf{Unlinking} elimina join inutili
\item \textbf{Hashing} accelera join su variabili
\item \textbf{Pattern reordering} cruciale per selettività
\item \textbf{Memory management} impatta prestazioni
\item \textbf{Profiling} essenziale per tuning
\end{enumerate}

\subsection{Linee Guida Pratiche}

\begin{successbox}[Best Practices]
\begin{enumerate}
\item Inizia con RETE standard
\item Aggiungi profiling
\item Identifica bottleneck
\item Applica ottimizzazioni mirate
\item Misura impatto
\item Itera
\end{enumerate}
\end{successbox}

\subsection{Trade-off}

\begin{table}[h]
\centering
\begin{tabular}{@{}lll@{}}
\toprule
\textbf{Ottimizzazione} & \textbf{Pro} & \textbf{Contro} \\
\midrule
Unlinking & -50\% activations & Complexity \\
Hashing & $O(1)$ join & Memory overhead \\
Reordering & -90\% combinations & Static analysis \\
Parallelization & Speedup & Synchronization \\
Specialization & +2x speed & Code duplication \\
\bottomrule
\end{tabular}
\caption{Trade-off ottimizzazioni}
\end{table}

\subsection{Completamento Parte II}

Con questo capitolo si conclude la Parte II sull'algoritmo RETE. Abbiamo visto:
\begin{itemize}
\item Fondamenti teorici (Cap. 5, 6)
\item Rete Alpha (Cap. 7)
\item Rete Beta (Cap. 8)
\item Analisi di complessità (Cap. 9)
\item Ottimizzazioni pratiche (Cap. 10)
\end{itemize}

La Parte III esplorerà l'architettura completa di CLIPS.

\subsection{Letture Consigliate}

\begin{itemize}
\item Doorenbos, R. (1995). "Production Matching..." - RETE/UL dettagliato
\item Brant, D. et al. (1991). "Incremental RETE" - Parallelization
\item Wright, I. et al. (1998). "Parallel Pattern Matching in RETE"
\item Miranker, D. (1990). "TREAT Algorithm"
\item Schmolze, J. (1991). "Guaranteeing Serializable Results in RETE"
\end{itemize}


% PARTE III: ARCHITETTURA CLIPS
\part{CLIPS: Architettura e Design}

% Capitolo 11: CLIPS - Panoramica e Architettura

\chapter{CLIPS: Panoramica del Sistema}
\label{cap:clips_overview}

\section{Introduzione a CLIPS}

\textbf{CLIPS} (C Language Integrated Production System) è un sistema esperto sviluppato dalla NASA nel 1985, diventato standard de facto per sistemi a produzione.

\subsection{Storia e Evoluzione}

\begin{itemize}
\item \textbf{1985}: Sviluppo iniziale presso NASA Johnson Space Center
\item \textbf{1986}: Prima release pubblica
\item \textbf{1991}: CLIPS 5.0 - Moduli e object-oriented
\item \textbf{2002}: CLIPS 6.0 - Architettura moderna
\item \textbf{2015}: CLIPS 6.3 - Miglioramenti e bugfix
\item \textbf{2020}: CLIPS 6.4 - Performance e stabilità
\end{itemize}

\subsection{Caratteristiche Principali}

\begin{infobox}[Punti di Forza]
\begin{itemize}
\item \textbf{Portabilità}: C standard, multi-platform
\item \textbf{Efficienza}: Algoritmo RETE ottimizzato
\item \textbf{Integrazione}: Embed in applicazioni C/C++
\item \textbf{Estensibilità}: User-defined functions
\item \textbf{Maturità}: 35+ anni di sviluppo
\item \textbf{Open Source}: Dominio pubblico
\end{itemize}
\end{infobox}

\section{Architettura Complessiva}

\subsection{Componenti Principali}

\begin{figure}[h]
\centering
\begin{tikzpicture}[
  node distance=1.5cm and 2cm,
  component/.style={rectangle, draw, fill=blue!20, minimum width=3cm, minimum height=1cm, align=center}
]
  \node[component] (parser) {Parser \\ Compiler};
  \node[component, below of=parser] (rete) {RETE Engine};
  \node[component, below of=rete] (wm) {Working Memory};
  \node[component, right of=rete] (agenda) {Agenda};
  \node[component, above of=agenda] (modules) {Module System};
  \node[component, below of=agenda] (functions) {Built-in \\ Functions};
  
  \draw[<->] (parser) -- (rete);
  \draw[<->] (rete) -- (wm);
  \draw[<->] (rete) -- (agenda);
  \draw[<->] (modules) -- (rete);
  \draw[<->] (functions) -- (rete);
\end{tikzpicture}
\caption{Architettura CLIPS ad alto livello}
\end{figure}

\subsection{Flusso di Esecuzione}

\begin{algorithm}
\caption{Ciclo Principale CLIPS}
\begin{algorithmic}[1]
\Function{CLIPSMainLoop}{}
  \State \Call{InitializeEnvironment}{}
  \State \Call{LoadRules}{rulefile}
  \State \Call{Reset}{}
  \While{not halted}
    \State \Comment{Recognize phase}
    \State $conflictSet \gets $ \Call{UpdateRETENetwork}{}
    \If{$conflictSet = \emptyset$}
      \State break \Comment{Quiescence}
    \EndIf
    \State \Comment{Act phase}
    \State $activation \gets $ \Call{SelectFromAgenda}{$conflictSet$}
    \State \Call{FireRule}{$activation$}
  \EndWhile
\EndFunction
\end{algorithmic}
\end{algorithm}

\section{Costrutti del Linguaggio}

\subsection{Deftemplate}

Definiscono struttura dei fatti:

\begin{lstlisting}[language=CLIPS]
(deftemplate persona
  "Template per rappresentare una persona"
  (slot nome (type STRING))
  (slot eta (type INTEGER) (range 0 150))
  (slot professione (default "disoccupato"))
  (multislot hobby (allowed-values sport lettura cinema)))
\end{lstlisting}

\subsection{Defrule}

Regole di produzione:

\begin{lstlisting}[language=CLIPS]
(defrule promuovi-senior
  "Promuove impiegati con esperienza"
  (declare (salience 10))
  (impiegato (id ?id) (anni-servizio ?a&:(>= ?a 10)))
  (not (promosso ?id))
  =>
  (assert (promosso ?id))
  (printout t "Promosso impiegato " ?id crlf))
\end{lstlisting}

\subsection{Deffacts}

Fatti iniziali:

\begin{lstlisting}[language=CLIPS]
(deffacts stato-iniziale
  "Popolazione iniziale working memory"
  (impiegato (id 1) (nome "Mario") (anni-servizio 12))
  (impiegato (id 2) (nome "Giulia") (anni-servizio 5)))
\end{lstlisting}

\subsection{Defmodule}

Organizzazione in moduli:

\begin{lstlisting}[language=CLIPS]
(defmodule ACQUISIZIONE
  "Modulo per input dati"
  (export deftemplate persona ordine))

(defmodule ELABORAZIONE
  "Modulo per business logic"
  (import ACQUISIZIONE deftemplate persona ordine))
\end{lstlisting}

\section{Struttura Interna}

\subsection{Environment}

Contesto isolato di esecuzione:

\begin{lstlisting}[language=C]
struct environment {
    struct fact *factList;
    struct defrule *ruleList;
    struct defmodule *moduleList;
    struct agenda *currentAgenda;
    struct partialMatch *betaMemory;
    // ... molti altri campi
};
\end{lstlisting}

\textbf{Supporto multi-environment}: Permette instanze CLIPS separate.

\subsection{Fact Management}

\begin{lstlisting}[language=C]
struct fact {
    struct factHeader header;
    struct deftemplate *whichDeftemplate;
    unsigned long factIndex;
    struct multifield *theProposition;
    struct fact *previousFact;
    struct fact *nextFact;
};
\end{lstlisting}

\subsection{Rule Structure}

\begin{lstlisting}[language=C]
struct defrule {
    struct constructHeader header;
    int salience;
    int localVarCnt;
    struct expr *dynamicSalience;
    struct defruleModule *header.whichModule;
    struct joinNode *lastJoin;
    struct expr *actions;
    // ...
};
\end{lstlisting}

\section{Gestione della Memoria}

\subsection{Memory Pool}

CLIPS usa pool di memoria per efficienza:

\begin{lstlisting}[language=C]
struct memoryPtr {
    struct memoryPtr *next;
};

void *RequestChunk(unsigned int size) {
    if (TopMemoryBlock != NULL) {
        struct memoryPtr *theMemory = TopMemoryBlock;
        TopMemoryBlock = theMemory->next;
        return (void *) theMemory;
    }
    return malloc(size);
}

void ReturnChunk(void *ptr, unsigned int size) {
    struct memoryPtr *theMemory = (struct memoryPtr *) ptr;
    theMemory->next = TopMemoryBlock;
    TopMemoryBlock = theMemory;
}
\end{lstlisting}

\textbf{Beneficio}: Riduzione chiamate malloc/free, meno frammentazione.

\section{I/O e Router System}

\subsection{Router}

Meccanismo flessibile per I/O:

\begin{lstlisting}[language=C]
struct router {
    char *name;
    int priority;
    int (*query)(char *, char *);
    int (*print)(char *, char *);
    int (*getc)(char *);
    int (*ungetc)(int, char *);
    int (*exit)(int);
    struct router *next;
};
\end{lstlisting}

\textbf{Usi}:
\begin{itemize}
\item Redirigere output a file/GUI
\item Interceptare comandi
\item Logging e debugging
\item Integrazione con applicazioni
\end{itemize}

\section{Estensibilità}

\subsection{User-Defined Functions (UDF)}

\begin{lstlisting}[language=C]
#include "clips.h"

void MyFunction(Environment *env, UDFContext *context, UDFValue *ret) {
    UDFValue arg1, arg2;
    
    UDFNthArgument(context, 1, NUMBER_TYPES, &arg1);
    UDFNthArgument(context, 2, NUMBER_TYPES, &arg2);
    
    ret->integerValue = CreateInteger(env, 
        arg1.integerValue->contents + arg2.integerValue->contents);
}

int main() {
    Environment *env = CreateEnvironment();
    AddUDF(env, "my-add", "l", 2, 2, "ll", MyFunction, "MyFunction", NULL);
    // ...
}
\end{lstlisting}

\subsection{External Calls}

Chiamare funzioni esterne da regole:

\begin{lstlisting}[language=CLIPS]
(defrule call-external
  (trigger)
  =>
  (bind ?result (my-add 10 20))
  (printout t "Result: " ?result crlf))
\end{lstlisting}

\section{Debugging e Profiling}

\subsection{Watch Facilities}

\begin{lstlisting}[language=CLIPS]
(watch facts)           ; Trace assert/retract
(watch rules)           ; Trace rule firing
(watch activations)     ; Trace agenda changes
(watch compilations)    ; Trace parsing
\end{lstlisting}

\subsection{Comandi Diagnostici}

\begin{lstlisting}[language=CLIPS]
(facts)                 ; Lista tutti i fatti
(rules)                 ; Lista tutte le regole
(agenda)                ; Mostra conflict set
(matches rule-name)     ; Mostra partial match
\end{lstlisting}

\section{Performance}

\subsection{Ottimizzazioni Interne}

\begin{itemize}
\item \textbf{Incremental reset}: Reset parziale
\item \textbf{Dynamic salience}: Calcolo lazy
\item \textbf{Pattern indexing}: Hash su pattern comuni
\item \textbf{Join network sharing}: Riuso nodi
\item \textbf{Memory compaction}: Garbage collection periodica
\end{itemize}

\subsection{Benchmark Tipici}

\begin{table}[h]
\centering
\begin{tabular}{@{}lrr@{}}
\toprule
\textbf{Sistema} & \textbf{Regole} & \textbf{Cicli/sec} \\
\midrule
Piccolo & 10-100 & 10000+ \\
Medio & 100-1000 & 1000-5000 \\
Grande & 1000+ & 100-1000 \\
\bottomrule
\end{tabular}
\caption{Performance indicative CLIPS}
\end{table}

\section{Integrazione con Applicazioni}

\subsection{Embed CLIPS}

\begin{lstlisting}[language=C]
#include "clips.h"

int main(int argc, char *argv[]) {
    Environment *env = CreateEnvironment();
    
    // Carica regole
    Load(env, "rules.clp");
    Reset(env);
    
    // Assert fatti da applicazione
    AssertString(env, "(temperatura 25)");
    
    // Esegui inference
    Run(env, -1);
    
    // Interroga risultati
    Eval(env, "(find-all-facts ((?f risultato)) TRUE)", &result);
    
    DestroyEnvironment(env);
    return 0;
}
\end{lstlisting}

\subsection{Callback}

Notifiche da CLIPS ad applicazione:

\begin{lstlisting}[language=C]
bool RuleFireCallback(
    Environment *env,
    Defrule *rule,
    void *context
) {
    printf("Fired: %s\n", DefruleName(rule));
    return true;  // Continue execution
}

AddRunFunction(env, "my-callback", RuleFireCallback, 0, NULL);
\end{lstlisting}

\section{Conclusioni del Capitolo}

\subsection{Punti Chiave}

\begin{enumerate}
\item CLIPS è un sistema \textbf{maturo e collaudato} (35+ anni)
\item Architettura \textbf{modulare ed estensibile}
\item Efficienza grazie a \textbf{RETE ottimizzato}
\item \textbf{Portabilità} eccellente (C standard)
\item Supporto completo per \textbf{sviluppo enterprise}
\end{enumerate}

\subsection{Prossimi Capitoli}

\begin{itemize}
\item Cap.~\ref{cap:clips_strutture}: Strutture dati interne dettagliate
\item Cap.~\ref{cap:clips_memoria}: Gestione memoria
\item Cap.~\ref{cap:clips_agenda}: Sistema di agenda e conflict resolution
\item Cap.~\ref{cap:clips_moduli}: Sistema di moduli e namespace
\end{itemize}

\subsection{Letture Consigliate}

\begin{itemize}
\item CLIPS Reference Manual (6.4)
\item CLIPS Architecture Manual
\item Giarratano \& Riley (2004). "Expert Systems: Principles and Programming"
\item Riley, G. (2016). "CLIPS: A Tool for Building Expert Systems"
\end{itemize}

% Capitolo 12: Strutture Dati CLIPS

\chapter{Strutture Dati Interne di CLIPS}
\label{cap:clips_strutture}

\section{Introduzione}

Le strutture dati di CLIPS sono progettate per efficienza e flessibilità. Questo capitolo esplora le implementazioni C che SLIPS deve replicare fedelmente.

\section{Simboli e Atomi}

\subsection{Symbol Table}

CLIPS mantiene una tabella globale di simboli per interning:

\begin{lstlisting}[language=C]
#define SYMBOL_HASH_SIZE 63559

struct symbolHashNode {
    struct symbolHashNode *next;
    long count;
    unsigned int depth;
    unsigned short type;
    char *contents;
};

static struct symbolHashNode **SymbolTable = NULL;
\end{lstlisting}

**Benefici**:
\begin{itemize}
\item Confronto $O(1)$ per uguaglianza (pointer comparison)
\item Risparmio memoria (no duplicati)
\item String immutabili garantite
\end{itemize}

\section{Multifield Values}

\subsection{Struttura}

\begin{lstlisting}[language=C]
struct multifield {
    unsigned short busyCount;
    short multifieldLength;
    struct multifieldMarker *multifields;
};

struct field {
    unsigned short type;
    union {
        void *value;
        CLIPSLexeme *lexemeValue;
        CLIPSFloat *floatValue;
        CLIPSInteger *integerValue;
    };
};
\end{lstlisting}

**Gestione**:
\begin{itemize}
\item Reference counting per sharing
\item Copy-on-write quando modificati
\item Pool per riuso
\end{itemize}

\section{Fatti}

\subsection{Fact Structure}

\begin{lstlisting}[language=C]
struct fact {
    struct patternEntity patternHeader;
    struct deftemplate *whichDeftemplate;
    void *list;
    long long factIndex;
    unsigned long depth;
    struct fact *previousFact;
    struct fact *nextFact;
    struct patternMatch *list;
    struct partialMatch *list;
};
\end{lstlisting}

\subsection{Fact List}

Lista doppiamente linkata per iterazione efficiente:

\begin{lstlisting}[language=C]
struct fact *FactList = NULL;
struct fact *LastFact = NULL;
long long NextFactIndex = 0;

struct fact *Assert(struct fact *theFact) {
    theFact->factIndex = NextFactIndex++;
    theFact->nextFact = NULL;
    theFact->previousFact = LastFact;
    
    if (LastFact == NULL)
        FactList = theFact;
    else
        LastFact->nextFact = theFact;
    
    LastFact = theFact;
    return theFact;
}
\end{lstlisting}

\section{Deftemplate}

\subsection{Struttura}

\begin{lstlisting}[language=C]
struct deftemplate {
    struct constructHeader header;
    struct templateSlot *slotList;
    unsigned int implied : 1;
    unsigned int watch : 1;
    unsigned int inScope : 1;
    unsigned short numberOfSlots;
    long busyCount;
    struct factPatternNode *patternNetwork;
};

struct templateSlot {
    struct symbolHashNode *slotName;
    unsigned int multislot : 1;
    unsigned int noDefault : 1;
    unsigned int defaultPresent : 1;
    unsigned int defaultDynamic : 1;
    struct expr *constraints;
    struct expr *defaultList;
    struct expr *facetList;
    struct templateSlot *next;
};
\end{lstlisting}

\section{Regole}

\subsection{Defrule Structure}

\begin{lstlisting}[language=C]
struct defrule {
    struct constructHeader header;
    int salience;
    unsigned int afterBreakpoint : 1;
    unsigned int watchActivation : 1;
    unsigned int watchFiring : 1;
    unsigned int autoFocus : 1;
    struct expr *dynamicSalience;
    struct expr *actions;
    struct joinNode *lastJoin;
    struct joinNode *disjunct;
};
\end{lstlisting}

\section{Pattern Network}

\subsection{Pattern Nodes}

\begin{lstlisting}[language=C]
struct factPatternNode {
    unsigned short whichField;
    unsigned short whichSlot;
    unsigned short leaveFields;
    struct lhsParseNode *networkTest;
    struct factPatternNode *nextLevel;
    struct factPatternNode *lastLevel;
    struct factPatternNode *leftNode;
    struct factPatternNode *rightNode;
    struct alphaMemoryHash *alphaMemory;
    long bsaveID;
};
\end{lstlisting}

\subsection{Join Network}

\begin{lstlisting}[language=C]
struct joinNode {
    unsigned int firstJoin : 1;
    unsigned int logicalJoin : 1;
    unsigned int joinFromTheRight : 1;
    unsigned int patternIsNegated : 1;
    long long memoryLeftAdds;
    long long memoryRightAdds;
    long long memoryLeftDeletes;
    long long memoryRightDeletes;
    struct expr *networkTest;
    struct joinNode *lastLevel;
    struct joinNode *nextLinks;
    void *rightSideEntryStructure;
    struct betaMemory *leftMemory;
    struct betaMemory *rightMemory;
    long bsaveID;
};
\end{lstlisting}

\section{Token e Partial Match}

\subsection{Struttura}

\begin{lstlisting}[language=C]
struct partialMatch {
    unsigned int betaMemory : 1;
    unsigned int busy : 1;
    unsigned int deleting : 1;
    unsigned int activationf : 1;
    unsigned short bcount;
    struct partialMatch *next;
    struct multifield *binds;
    struct alphaMatch *markers;
    struct partialMatch *children;
    struct partialMatch *rightParent;
    struct partialMatch *nextInMemory;
    struct partialMatch *prevInMemory;
    struct joinNode *owner;
};
\end{lstlisting}

**Gestione complessità**:
\begin{itemize}
\item Reference counting per cleanup
\item Lazy deletion per efficienza
\item Children list per propagazione retract
\end{itemize}

\section{Agenda e Attivazioni}

\subsection{Activation}

\begin{lstlisting}[language=C]
struct activation {
    struct defrule *theRule;
    struct partialMatch *basis;
    int salience;
    unsigned long long timetag;
    struct activation *prev;
    struct activation *next;
    struct patternEntity *sortedBasis;
};
\end{lstlisting}

\subsection{Agenda Structure}

\begin{lstlisting}[language=C]
struct agenda {
    struct activation *first;
    struct activation *last;
    struct defruleModule *whichModule;
};
\end{lstlisting}

\section{Traduzione Swift per SLIPS}

\subsection{Approccio}

\textbf{Principi}:
\begin{enumerate}
\item Preservare semantica esatta
\item Usare Swift idioms dove possibile
\item Mantenere performance comparabili
\item Type safety dove vantaggioso
\end{enumerate}

\subsection{Esempio: Symbol Table}

\textbf{C}:
\begin{lstlisting}[language=C]
struct symbolHashNode *FindSymbol(char *str);
\end{lstlisting}

\textbf{Swift}:
\begin{lstlisting}[language=Swift]
class SymbolTable {
    private var table: [String: Symbol] = [:]
    
    func intern(_ string: String) -> Symbol {
        if let existing = table[string] {
            return existing
        }
        let newSymbol = Symbol(contents: string)
        table[string] = newSymbol
        return newSymbol
    }
}

struct Symbol: Hashable {
    let contents: String
    let id: Int  // For fast comparison
}
\end{lstlisting}

\subsection{Esempio: Fact}

\begin{lstlisting}[language=Swift]
class Fact {
    let template: Deftemplate
    let slots: [String: Value]
    let index: Int
    var next: Fact?
    weak var previous: Fact?
    
    // Pattern matching state
    var alphaMatches: Set<AlphaMemory> = []
    var tokens: Set<Token> = []
}
\end{lstlisting}

\section{Gestione Memoria in Swift}

\subsection{ARC vs Manual}

**Differenze da C**:
\begin{itemize}
\item Swift usa ARC (Automatic Reference Counting)
\item No malloc/free espliciti
\item Weak references per evitare cicli
\item Copy-on-write per collections
\end{itemize}

**Vantaggi**:
\begin{itemize}
\item Meno memory leak
\item Code più sicuro
\item Integrazione con Swift ecosystem
\end{itemize}

**Sfide**:
\begin{itemize}
\item Performance overhead di ARC
\item Cicli di reference da gestire attentamente
\item Pooling più complesso
\end{itemize}

\section{Conclusioni del Capitolo}

\subsection{Punti Chiave}

\begin{enumerate}
\item CLIPS usa strutture C efficienti e compatte
\item Symbol interning cruciale per performance
\item Reference counting pervasivo
\item Liste linkate per fatti e attivazioni
\item Pattern network come DAG di nodi
\end{enumerate}

\subsection{Implicazioni per SLIPS}

\begin{itemize}
\item Preservare semantica esatta delle strutture
\item Adattare a paradigmi Swift dove appropriato
\item Mantenere efficienza comparabile
\item Sfruttare type safety di Swift
\end{itemize}

\subsection{Letture Consigliate}

\begin{itemize}
\item CLIPS Architecture Manual - Capitolo 3
\item CLIPS Source Code - \texttt{factmngr.c}, \texttt{ruledef.c}
\item Swift Programming Language - Memory Management
\end{itemize}

% Capitolo 13: Gestione della Memoria in CLIPS

\chapter{Gestione della Memoria}
\label{cap:clips_memoria}

\section{Introduzione}

La gestione efficiente della memoria è critica per le prestazioni di CLIPS. Il sistema implementa diverse strategie di ottimizzazione per ridurre overhead e frammentazione.

\section{Memory Pools}

\subsection{Implementazione}

CLIPS usa pool segregati per tipi comuni:

\begin{lstlisting}[language=C]
struct chunkInfo {
    unsigned int size;
    struct chunkInfo *prevChunk;
    struct chunkInfo *nextFree;
    long int lastCall;
};

#define STRICT_ALIGN_SIZE sizeof(double)
#define ChunkInfoSize sizeof(struct chunkInfo)

void InitializeMemory() {
    for (int i = 0; i < MAXIMUM_SIZE; i++) {
        MemoryTable[i] = NULL;
    }
}
\end{lstlisting}

\textbf{Pool per dimensione}:
\begin{itemize}
\item 8 bytes
\item 16 bytes
\item 32 bytes
\item 64 bytes
\item ...fino a soglia
\end{itemize}

**Sopra soglia**: Usa `malloc` diretto.

\subsection{Request/Return}

\begin{lstlisting}[language=C]
void *genmalloc(unsigned int size) {
    struct chunkInfo *memptr;
    unsigned int actualSize = size + ChunkInfoSize;
    
    // Arrotonda a multiplo di alignment
    actualSize = (actualSize + (STRICT_ALIGN_SIZE - 1)) 
                 & ~(STRICT_ALIGN_SIZE - 1);
    
    if (actualSize >= MAXIMUM_SIZE) {
        memptr = malloc(actualSize);
    } else {
        memptr = MemoryTable[actualSize];
        if (memptr != NULL) {
            MemoryTable[actualSize] = memptr->nextFree;
        } else {
            memptr = malloc(actualSize);
        }
    }
    
    memptr->size = actualSize;
    return (void *) (((char *) memptr) + ChunkInfoSize);
}

void genfree(void *ptr, unsigned int size) {
    struct chunkInfo *memptr = (struct chunkInfo *)
        (((char *) ptr) - ChunkInfoSize);
    
    if (memptr->size >= MAXIMUM_SIZE) {
        free(memptr);
    } else {
        memptr->nextFree = MemoryTable[memptr->size];
        MemoryTable[memptr->size] = memptr;
    }
}
\end{lstlisting}

\textbf{Benefici}:
\begin{itemize}
\item Riduzione chiamate `malloc`/`free`: 10-100x
\item Meno frammentazione
\item Cache-friendly (oggetti simili vicini)
\end{itemize}

\section{Reference Counting}

\subsection{Shared Values}

Per simboli e multifield:

\begin{lstlisting}[language=C]
void IncrementSymbolCount(SYMBOL_HN *theSymbol) {
    theSymbol->count++;
}

void DecrementSymbolCount(Environment *env, SYMBOL_HN *theSymbol) {
    theSymbol->count--;
    if (theSymbol->count == 0) {
        RemoveSymbol(env, theSymbol);
    }
}
\end{lstlisting}

**Pattern idiomatico CLIPS**:
\begin{lstlisting}[language=C]
SYMBOL_HN *sym = FindSymbol("example");
IncrementSymbolCount(sym);
// ... uso ...
DecrementSymbolCount(env, sym);
\end{lstlisting}

\subsection{Copy-on-Write}

Per multifield values:

\begin{lstlisting}[language=C]
struct multifield *CopyMultifield(Environment *env, struct multifield *src) {
    if (src->busyCount == 0) {
        return src;  // Può riusare
    }
    
    struct multifield *dst = CreateMultifield(env, src->length);
    for (int i = 0; i < src->length; i++) {
        dst->contents[i] = src->contents[i];
        if (dst->contents[i].header->type == MULTIFIELD_TYPE) {
            IncrementMultifieldReferenceCount(
                dst->contents[i].multifieldValue);
        }
    }
    return dst;
}
\end{lstlisting}

\section{Garbage Collection}

\subsection{Periodic Cleanup}

CLIPS non ha GC automatico ma cleanup periodica:

\begin{lstlisting}[language=C]
void PeriodicCleanup(Environment *env) {
    static long lastCall = 0;
    long currentTime = GetTickCount();
    
    if ((currentTime - lastCall) > CLEANUP_INTERVAL) {
        CleanupSymbols(env);
        CleanupFloats(env);
        CleanupIntegers(env);
        CompactMemory(env);
        lastCall = currentTime;
    }
}
\end{lstlisting}

\subsection{Symbol Cleanup}

Rimuove simboli non riferiti:

\begin{lstlisting}[language=C]
void CleanupSymbols(Environment *env) {
    for (int i = 0; i < SYMBOL_HASH_SIZE; i++) {
        SYMBOL_HN **prevPtr = &SymbolTable[i];
        SYMBOL_HN *sym = SymbolTable[i];
        
        while (sym != NULL) {
            if (sym->count == 0 && sym->depth == 0) {
                *prevPtr = sym->next;
                free(sym->contents);
                free(sym);
                sym = *prevPtr;
            } else {
                prevPtr = &sym->next;
                sym = sym->next;
            }
        }
    }
}
\end{lstlisting}

\section{Reset e Clear}

\subsection{Reset}

Ripristina working memory mantenendo regole:

\begin{lstlisting}[language=C]
void Reset(Environment *env) {
    // Rimuovi tutti i fatti
    while (FactList != NULL) {
        Retract(env, FactList);
    }
    
    // Pulisci agenda
    ClearAgenda(env);
    
    // Re-assert deffacts
    for (Deffacts *df = GetFirstDeffacts(); 
         df != NULL; 
         df = GetNextDeffacts(df)) {
        AssertDeffacts(env, df);
    }
}
\end{lstlisting}

\subsection{Clear}

Rimuove tutto:

\begin{lstlisting}[language=C]
void Clear(Environment *env) {
    // Rimuovi fatti
    while (FactList != NULL) {
        Retract(env, FactList);
    }
    
    // Rimuovi regole
    while (RuleList != NULL) {
        Undefrule(env, RuleList);
    }
    
    // Rimuovi deftemplate
    while (DeftemplateList != NULL) {
        Undeftemplate(env, DeftemplateList);
    }
    
    // Reset network
    DestroyRETENetwork(env);
    
    // Cleanup memoria
    PeriodicCleanup(env);
}
\end{lstlisting}

\section{Traduzione per SLIPS}

\subsection{Swift Memory Management}

**ARC invece di manual**:

\begin{lstlisting}[language=Swift]
class Symbol {
    let contents: String
    // ARC gestisce count automaticamente
}

// Invece di manual increment/decrement
let sym = Symbol(contents: "example")
// ARC incrementa automaticamente
// ARC decrementa quando esce da scope
\end{lstlisting}

\subsection{Object Pooling}

Comunque utile per performance:

\begin{lstlisting}[language=Swift]
class TokenPool {
    private var pool: [Token] = []
    private let maxSize = 1000
    
    func acquire(facts: [Fact]) -> Token {
        if let token = pool.popLast() {
            token.reset(with: facts)
            return token
        }
        return Token(facts: facts)
    }
    
    func release(_ token: Token) {
        guard pool.count < maxSize else { return }
        pool.append(token)
    }
}

// Uso
func processMatch() {
    let token = pool.acquire(facts: [...])
    defer { pool.release(token) }
    // ... lavoro ...
}
\end{lstlisting}

\subsection{Weak References per Cicli}

\begin{lstlisting}[language=Swift]
class BetaMemory {
    var tokens: Set<Token> = []
    weak var parent: JoinNode?  // Evita cicli
}

class JoinNode {
    var leftMemory: BetaMemory?
    var rightMemory: AlphaMemory?
}
\end{lstlisting}

\section{Profiling Memoria}

\subsection{Metriche}

\begin{table}[h]
\centering
\begin{tabular}{@{}ll@{}}
\toprule
\textbf{Metrica} & \textbf{Comando} \\
\midrule
Memoria totale & \texttt{(mem-used)} \\
Memoria richieste & \texttt{(mem-requests)} \\
Hit rate pool & Ratio riuso/allocazioni \\
Frammentazione & Memoria richiesta vs usata \\
\bottomrule
\end{tabular}
\end{table}

\subsection{Memory Leak Detection}

\begin{lstlisting}[language=C]
void EnableMemoryTracking(Environment *env) {
    env->trackAllocation = TRUE;
}

void ReportMemoryStatus(Environment *env) {
    printf("Total allocations: %ld\n", TotalAllocations);
    printf("Total frees: %ld\n", TotalFrees);
    printf("Net: %ld\n", TotalAllocations - TotalFrees);
    
    if (TotalAllocations != TotalFrees) {
        printf("WARNING: Possible memory leak!\n");
    }
}
\end{lstlisting}

\section{Conclusioni del Capitolo}

\subsection{Punti Chiave}

\begin{enumerate}
\item CLIPS usa \textbf{memory pools} per efficienza
\item \textbf{Reference counting} per shared values
\item \textbf{Periodic cleanup} invece di GC continua
\item \textbf{Copy-on-write} per multifield
\item Trade-off complessità vs performance
\end{enumerate}

\subsection{SLIPS Adaptations}

\begin{itemize}
\item Sfruttare ARC di Swift dove possibile
\item Mantenere pooling per hot paths
\item Weak references per evitare cicli
\item Profiling per identificare leak
\end{itemize}

\subsection{Letture Consigliate}

\begin{itemize}
\item CLIPS Source - \texttt{memalloc.c}
\item Swift Memory Management Guide
\item "Modern Memory Management" - Apple
\end{itemize}

% Capitolo 14: Sistema di Agenda in CLIPS

\chapter{Sistema di Agenda e Conflict Resolution}
\label{cap:clips_agenda}

\section{Introduzione}

L'agenda in CLIPS gestisce il conflict set e determina l'ordine di esecuzione delle regole tramite strategie di conflict resolution.

\section{Struttura dell'Agenda}

\subsection{Activation}

\begin{lstlisting}[language=C]
struct activation {
    struct defrule *theRule;
    struct partialMatch *basis;
    int salience;
    unsigned long long timetag;
    unsigned long randomID;
    struct activation *prev;
    struct activation *next;
};
\end{lstlisting}

\textbf{Campi chiave}:
\begin{itemize}
\item \texttt{theRule}: Regola da eseguire
\item \texttt{basis}: Partial match che ha attivato la regola
\item \texttt{salience}: Priorità dichiarata
\item \texttt{timetag}: Timestamp di creazione
\item \texttt{randomID}: Per strategia random
\end{itemize}

\subsection{Agenda per Modulo}

\begin{lstlisting}[language=C]
struct defmodule {
    // ... altri campi ...
    struct activation *agenda;
};
\end{lstlisting}

Ogni modulo ha la propria agenda, gestita tramite focus stack.

\section{Conflict Resolution Strategies}

\subsection{Depth Strategy}

\textbf{Ordine}:
\begin{enumerate}
\item Salience (maggiore = priorità)
\item Recency (fatti più recenti = priorità)
\item Rule specificity (più condizioni = priorità)
\item Rule order (definizione)
\end{enumerate}

\begin{lstlisting}[language=C]
int CompareActivations_Depth(
    struct activation *a1,
    struct activation *a2
) {
    // 1. Salience
    if (a1->salience > a2->salience) return -1;
    if (a1->salience < a2->salience) return 1;
    
    // 2. Recency (timetag più alto = più recente)
    if (a1->timetag > a2->timetag) return -1;
    if (a1->timetag < a2->timetag) return 1;
    
    // 3. Specificity
    int spec1 = RuleSpecificity(a1->theRule);
    int spec2 = RuleSpecificity(a2->theRule);
    if (spec1 > spec2) return -1;
    if (spec1 < spec2) return 1;
    
    // 4. Rule order
    return (a1->theRule->header.timeTag - 
            a2->theRule->header.timeTag);
}
\end{lstlisting}

\subsection{Breadth Strategy}

Come depth, ma recency invertita (fatti vecchi prima):

\begin{lstlisting}[language=C]
// In CompareActivations_Breadth:
// Recency check invertito
if (a1->timetag < a2->timetag) return -1;  // Opposto!
if (a1->timetag > a2->timetag) return 1;
\end{lstlisting}

\subsection{LEX e MEA}

\textbf{LEX} (Least Recently Used):
\begin{itemize}
\item Ordina per recency di ogni fatto nel match
\item Lessicografico sui timetag
\end{itemize}

\textbf{MEA} (Most Recently Used):
\begin{itemize}
\item Opposto di LEX
\item Fatti recenti prima
\end{itemize}

\subsection{Complexity Strategy}

Ordina per complessità della regola (numero di condizioni e test):

\begin{lstlisting}[language=C]
int RuleComplexity(struct defrule *rule) {
    int complexity = 0;
    struct joinNode *join = rule->lastJoin;
    
    while (join != NULL) {
        complexity++;
        if (join->networkTest != NULL) {
            complexity += CountTests(join->networkTest);
        }
        join = join->lastLevel;
    }
    
    return complexity;
}
\end{lstlisting}

\subsection{Simplicity Strategy}

Opposto di complexity: regole semplici prima.

\subsection{Random Strategy}

Selezione casuale:

\begin{lstlisting}[language=C]
struct activation *SelectRandom(struct activation *agenda) {
    int count = 0;
    for (struct activation *a = agenda; a != NULL; a = a->next) {
        count++;
    }
    
    if (count == 0) return NULL;
    
    int selected = rand() % count;
    struct activation *result = agenda;
    for (int i = 0; i < selected; i++) {
        result = result->next;
    }
    
    return result;
}
\end{lstlisting}

**Uso**: Testing, simulazioni, evitare bias.

\section{Salience}

\subsection{Static Salience}

Dichiarata nella regola:

\begin{lstlisting}[language=CLIPS]
(defrule emergency
  (declare (salience 100))  ; Alta priorita
  (alarm)
  =>
  (shutdown-system))

(defrule routine
  (declare (salience 0))    ; Priorita normale
  (tick)
  =>
  (log-event))
\end{lstlisting}

\subsection{Dynamic Salience}

Calcolata a runtime:

\begin{lstlisting}[language=CLIPS]
(defrule dynamic-priority
  (declare (salience (+ ?priority (* 10 ?urgency))))
  (task (priority ?priority) (urgency ?urgency))
  =>
  (process-task))
\end{lstlisting}

\textbf{Implementazione}:

\begin{lstlisting}[language=C]
int EvaluateSalience(
    Environment *env,
    struct activation *activation
) {
    if (activation->theRule->dynamicSalience == NULL) {
        return activation->theRule->salience;
    }
    
    UDFValue result;
    EvaluateExpression(env, 
                      activation->theRule->dynamicSalience,
                      &result);
    
    return result.integerValue->contents;
}
\end{lstlisting}

\subsection{Salience Evaluation}

Quando ricalcolare:

\begin{lstlisting}[language=CLIPS]
(set-salience-evaluation when-defined)   ; Default: al build
(set-salience-evaluation when-activated) ; Ad ogni attivazione
(set-salience-evaluation every-cycle)    ; Ogni ciclo
\end{lstlisting}

\section{Agenda Management}

\subsection{Inserimento}

Inserisce attivazione mantenendo ordine:

\begin{lstlisting}[language=C]
void AddActivation(
    Environment *env,
    struct activation *newActivation
) {
    struct activation **current = &(env->currentModule->agenda);
    
    while (*current != NULL) {
        if (CompareActivations(newActivation, *current) < 0) {
            break;  // Posizione trovata
        }
        current = &((*current)->next);
    }
    
    newActivation->next = *current;
    if (*current != NULL) {
        (*current)->prev = newActivation;
    }
    *current = newActivation;
    newActivation->prev = (current == &(env->currentModule->agenda)) 
                          ? NULL 
                          : container_of(current, struct activation, next);
}
\end{lstlisting}

\subsection{Rimozione}

Quando un fatto che supporta l'attivazione viene retratto:

\begin{lstlisting}[language=C]
void RemoveActivation(
    Environment *env,
    struct activation *activation
) {
    if (activation->prev != NULL) {
        activation->prev->next = activation->next;
    } else {
        env->currentModule->agenda = activation->next;
    }
    
    if (activation->next != NULL) {
        activation->next->prev = activation->prev;
    }
    
    ReturnActivation(env, activation);
}
\end{lstlisting}

\section{Refresh e Reorder}

\subsection{Refresh}

Ricalcola tutte le attivazioni:

\begin{lstlisting}[language=CLIPS]
(refresh rule-name)
\end{lstlisting}

\textbf{Uso}: Dopo modifica dinamica di salience o priorità.

\subsection{Reorder}

Riordina agenda con nuova strategia:

\begin{lstlisting}[language=C]
void RefreshAgenda(Environment *env, struct defrule *rule) {
    // 1. Rimuovi attivazioni esistenti
    RemoveActivationsForRule(env, rule);
    
    // 2. Rigenera da partial matches
    for (struct partialMatch *pm = rule->lastJoin->betaMemory;
         pm != NULL;
         pm = pm->nextInMemory) {
        AddActivation(env, CreateActivation(env, rule, pm));
    }
}
\end{lstlisting}

\section{Focus Stack}

\subsection{Struttura}

\begin{lstlisting}[language=C]
struct focus {
    struct defmodule *theModule;
    struct focus *next;
};

struct focus *CurrentFocus = NULL;
\end{lstlisting}

\subsection{Operazioni}

\begin{lstlisting}[language=CLIPS]
(focus MODULE-NAME)      ; Push modulo su stack
(return)                 ; Pop modulo corrente
(get-focus)              ; Query modulo corrente
(list-focus-stack)       ; Visualizza stack
\end{lstlisting}

\textbf{Implementazione}:

\begin{lstlisting}[language=C]
void Focus(Environment *env, struct defmodule *module) {
    struct focus *newFocus = get_struct(env, focus);
    newFocus->theModule = module;
    newFocus->next = env->CurrentFocus;
    env->CurrentFocus = newFocus;
}

struct defmodule *GetCurrentModule(Environment *env) {
    if (env->CurrentFocus != NULL) {
        return env->CurrentFocus->theModule;
    }
    return env->FindDefmodule(env, "MAIN");
}
\end{lstlisting}

\section{Conclusioni del Capitolo}

\subsection{Punti Chiave}

\begin{enumerate}
\item Agenda organizza il \textbf{conflict set}
\item \textbf{Strategie} multiple per ordinamento
\item \textbf{Salience} permette priorità esplicite
\item \textbf{Dynamic salience} per priorità calcolate
\item \textbf{Focus stack} gestisce moduli
\end{enumerate}

\subsection{Per SLIPS}

\begin{itemize}
\item Implementare tutte le strategie standard
\item Supportare salience dinamica
\item Gestire focus stack correttamente
\item Mantenere efficienza in inserimento/rimozione
\end{itemize}

\subsection{Letture Consigliate}

\begin{itemize}
\item CLIPS Reference - Capitolo "Agenda"
\item CLIPS Source - \texttt{agenda.c}
\item Brownston et al. (1985). "OPS5 Conflict Resolution"
\end{itemize}

% Capitolo 15: Sistema di Moduli CLIPS

\chapter{Sistema di Moduli e Namespace}
\label{cap:clips_moduli}

\section{Introduzione}

Il sistema di moduli in CLIPS fornisce namespace separati per organizzare grandi basi di conoscenza, simile ai package in linguaggi moderni.

\section{Struttura dei Moduli}

\subsection{Defmodule}

\begin{lstlisting}[language=C]
struct defmodule {
    struct constructHeader header;
    struct portItem *importList;
    struct portItem *exportList;
    unsigned int visitedFlag : 1;
    struct defmoduleItemHeader **itemsArray;
    struct activation *agenda;
};

struct portItem {
    struct defmodule *theModule;
    struct constructHeader *constructType;
    char *constructName;
    struct portItem *next;
};
\end{lstlisting}

\subsection{Dichiarazione}

\begin{lstlisting}[language=CLIPS]
(defmodule DIAGNOSTICS
  "Sistema diagnostico principale"
  (import SENSORS deftemplate reading)
  (export deftemplate diagnosis))
\end{lstlisting}

\section{Import/Export}

\subsection{Export}

Rende costrutti visibili ad altri moduli:

\begin{lstlisting}[language=CLIPS]
(defmodule A
  (export deftemplate ?ALL)     ; Tutti i deftemplate
  (export defrule specific-rule)) ; Regola specifica
\end{lstlisting}

\textbf{Wildcards}:
\begin{itemize}
\item \texttt{?ALL}: Tutti i costrutti di quel tipo
\item \texttt{?NONE}: Nessun costrutto (default)
\end{itemize}

\subsection{Import}

Importa costrutti da altri moduli:

\begin{lstlisting}[language=CLIPS]
(defmodule B
  (import A deftemplate sensor-data)  ; Specifico
  (import A deftemplate ?ALL))        ; Tutti
\end{lstlisting}

\textbf{Implementazione}:

\begin{lstlisting}[language=C]
bool IsConstructExported(
    struct defmodule *fromModule,
    const char *constructType,
    const char *constructName
) {
    for (struct portItem *port = fromModule->exportList;
         port != NULL;
         port = port->next) {
        if (strcmp(port->constructType->name, constructType) != 0)
            continue;
        
        if (strcmp(port->constructName, "?ALL") == 0)
            return true;
        
        if (strcmp(port->constructName, constructName) == 0)
            return true;
    }
    return false;
}
\end{lstlisting}

\section{Visibility e Scope}

\subsection{Regole di Scope}

\begin{enumerate}
\item Regole vedono solo fatti dei template nel loro modulo o importati
\item Ogni regola appartiene a un solo modulo
\item Il modulo corrente determina quali regole possono fired
\end{enumerate}

\begin{lstlisting}[language=CLIPS]
(defmodule MAIN
  (export deftemplate sensor))

(deftemplate sensor
  (slot value))

;; Questa regola e in MAIN
(defrule process-sensor
  (sensor (value ?v))
  =>
  ...)

(defmodule PROCESSOR
  (import MAIN deftemplate sensor))

;; Questa regola e in PROCESSOR
(defrule PROCESSOR::analyze
  (sensor (value ?v))  ; OK: sensor e importato
  =>
  ...)
\end{lstlisting}

\subsection{Qualified Names}

Accesso esplicito a costrutti in altri moduli:

\begin{lstlisting}[language=CLIPS]
(MAIN::sensor (value 10))        ; Fatto qualificato
(MAIN::process-data)             ; Chiamata funzione
\end{lstlisting}

\section{Focus e Esecuzione}

\subsection{Focus Stack}

Determina quale modulo è attivo:

\begin{lstlisting}[language=CLIPS]
(focus DIAGNOSTICS)    ; Rendi DIAGNOSTICS corrente
(focus SENSORS PROCESSOR MAIN)  ; Push multipli
\end{lstlisting}

\textbf{Comportamento}:
\begin{itemize}
\item Solo regole del modulo in focus possono fired
\item Quando agenda modulo vuota, pop automatico
\item MAIN è sempre in fondo allo stack
\end{itemize}

\begin{lstlisting}[language=C]
struct defmodule *PopFocus(Environment *env) {
    if (env->CurrentFocus == NULL) {
        return env->FindDefmodule(env, "MAIN");
    }
    
    struct focus *oldFocus = env->CurrentFocus;
    struct defmodule *module = oldFocus->theModule;
    env->CurrentFocus = oldFocus->next;
    
    rtn_struct(env, focus, oldFocus);
    return env->CurrentFocus ? 
           env->CurrentFocus->theModule : 
           env->FindDefmodule(env, "MAIN");
}
\end{lstlisting}

\subsection{Auto-Focus}

Regole possono auto-focus su firing:

\begin{lstlisting}[language=CLIPS]
(defrule trigger-diagnostics
  (declare (auto-focus TRUE))
  (alarm)
  =>
  (printout t "Running diagnostics..." crlf))
\end{lstlisting}

Quando matcha, automaticamente fa `(focus modulo-della-regola)`.

\section{Modularità e Design}

\subsection{Pattern di Uso}

\textbf{Layering}:
\begin{lstlisting}[language=CLIPS]
(defmodule INPUT
  (export deftemplate raw-data))

(defmodule PROCESSING
  (import INPUT deftemplate raw-data)
  (export deftemplate processed-data))

(defmodule OUTPUT
  (import PROCESSING deftemplate processed-data))
\end{lstlisting}

\textbf{Separation of Concerns}:
\begin{itemize}
\item Modulo per acquisizione dati
\item Modulo per elaborazione
\item Modulo per output/azioni
\end{itemize}

\subsection{Best Practices}

\begin{infobox}[Linee Guida]
\begin{enumerate}
\item Un modulo = una responsabilità
\item Export solo l'interfaccia pubblica
\item Documentare dipendenze tra moduli
\item Usare focus esplicitamente quando necessario
\item Evitare cicli nelle dipendenze
\end{enumerate}
\end{infobox}

\section{Implementazione SLIPS}

\subsection{Module Structure}

\begin{lstlisting}[language=Swift]
class Defmodule {
    let name: String
    var importList: [PortItem] = []
    var exportList: [PortItem] = []
    var templates: [String: Deftemplate] = [:]
    var rules: [String: Defrule] = []
    var agenda: Agenda
    
    func canAccess(template: String, from: Defmodule) -> Bool {
        // Check se template è locale o importato
        if templates[template] != nil {
            return true
        }
        
        for import in importList {
            if import.constructName == template || 
               import.constructName == "?ALL" {
                if import.module.isExported(template: template) {
                    return true
                }
            }
        }
        
        return false
    }
}

struct PortItem {
    let module: Defmodule
    let constructType: ConstructType
    let constructName: String
}
\end{lstlisting}

\subsection{Focus Stack}

\begin{lstlisting}[language=Swift]
class FocusStack {
    private var stack: [Defmodule] = []
    private let mainModule: Defmodule
    
    init(mainModule: Defmodule) {
        self.mainModule = mainModule
    }
    
    var current: Defmodule {
        return stack.last ?? mainModule
    }
    
    func push(_ module: Defmodule) {
        stack.append(module)
    }
    
    @discardableResult
    func pop() -> Defmodule? {
        return stack.popLast()
    }
    
    func clear() {
        stack.removeAll()
    }
}
\end{lstlisting}

\section{Testing e Debug}

\subsection{Comandi Diagnostici}

\begin{lstlisting}[language=CLIPS]
(list-defmodules)           ; Lista tutti i moduli
(ppdefmodule MODULE-NAME)   ; Pretty-print modulo
(get-current-module)        ; Modulo corrente
(set-current-module NAME)   ; Cambia modulo corrente
(list-focus-stack)          ; Mostra stack
\end{lstlisting}

\subsection{Dependency Analysis}

\begin{lstlisting}[language=Swift]
func analyzeDependencies(environment: Environment) -> DependencyGraph {
    var graph = DependencyGraph()
    
    for module in environment.modules {
        for import in module.importList {
            graph.addEdge(from: module, to: import.module)
        }
    }
    
    // Check cicli
    if graph.hasCycle() {
        print("Warning: Circular dependencies detected!")
    }
    
    return graph
}
\end{lstlisting}

\section{Conclusioni del Capitolo}

\subsection{Punti Chiave}

\begin{enumerate}
\item Moduli forniscono \textbf{namespace} e organizzazione
\item \textbf{Import/Export} controllano visibilità
\item \textbf{Focus stack} determina esecuzione
\item \textbf{Auto-focus} permette context switching
\item Design modulare migliora manutenibilità
\end{enumerate}

\subsection{Fine Parte III}

Con questo capitolo si conclude la Parte III sull'architettura di CLIPS. Abbiamo esplorato:
\begin{itemize}
\item Overview e architettura generale
\item Strutture dati interne
\item Gestione della memoria
\item Sistema di agenda
\item Sistema di moduli
\end{itemize}

La Parte IV analizzerà l'implementazione specifica di SLIPS in Swift.

\subsection{Letture Consigliate}

\begin{itemize}
\item CLIPS Reference Manual - Capitolo "Defmodule Construct"
\item CLIPS Source - \texttt{moduldef.c}, \texttt{modulpsr.c}
\item Giarratano \& Riley - Capitolo "Modular Design"
\end{itemize}


% PARTE IV: IMPLEMENTAZIONE SLIPS
\part{SLIPS: Traduzione in Swift}

% Capitolo 16: Architettura di SLIPS

\chapter{Architettura di SLIPS}
\label{cap:slips_arch}

\section{Principi di Progettazione}

L'architettura di SLIPS si fonda su tre pilastri:

\subsection{Fedeltà Semantica}

\begin{definizione}[Equivalenza Comportamentale]
Per ogni programma CLIPS valido $P$ e input $I$:
\begin{equation}
\text{output}_{\text{CLIPS}}(P, I) = \text{output}_{\text{SLIPS}}(P, I)
\end{equation}
\end{definizione}

Questo implica:
\begin{itemize}
\item Stesso ordine di firing (con stessa strategia)
\item Stessi fatti asseriti/ritratti
\item Stessi valori calcolati
\item Stesso comportamento di watch/trace
\end{itemize}

\subsection{Sicurezza del Tipo}

Swift 6.2 offre garanzie che C non può fornire:

\begin{table}[h]
\centering
\begin{tabular}{@{}lll@{}}
\toprule
\textbf{Problema in C} & \textbf{Soluzione Swift} & \textbf{Garanzia} \\
\midrule
Buffer overflow & Array bounds checking & Runtime safety \\
Use-after-free & ARC + ownership & Compile-time \\
Null pointer deref & Optional types & Compile-time \\
Type confusion & Strong typing & Compile-time \\
Data races & Sendable \& actor & Compile-time \\
\bottomrule
\end{tabular}
\caption{Garanzie di sicurezza Swift vs C}
\label{tab:safety}
\end{table}

\subsection{Manutenibilità}

Obiettivi di manutenibilità:
\begin{itemize}
\item File < 1000 LOC (limite soft)
\item Funzioni < 50 LOC
\item Complessità ciclomatica < 15
\item Coverage test > 85\%
\item Documentazione inline con riferimenti C
\end{itemize}

\section{Mapping C → Swift}

\subsection{Regole di Traduzione}

\subsubsection{Strutture Dati}

\begin{table}[h]
\centering
\small
\begin{tabular}{@{}p{4cm}p{5cm}p{4cm}@{}}
\toprule
\textbf{Costrutto C} & \textbf{Equivalente Swift} & \textbf{Esempio} \\
\midrule
\texttt{struct} semplice & \texttt{struct} value type & 
\begin{minipage}{4cm}\begin{lstlisting}[language=C,basicstyle=\tiny\ttfamily]
struct Point {
  int x, y;
};
\end{lstlisting}\end{minipage}
$\to$
\begin{minipage}{4cm}\begin{lstlisting}[language=Swift,basicstyle=\tiny\ttfamily]
struct Point {
  var x: Int
  var y: Int
}
\end{lstlisting}\end{minipage} \\
\midrule

\texttt{struct} con puntatori & \texttt{class} reference type & 
\begin{minipage}{4cm}\begin{lstlisting}[language=C,basicstyle=\tiny\ttfamily]
struct Node {
  int data;
  struct Node *next;
};
\end{lstlisting}\end{minipage}
$\to$
\begin{minipage}{4cm}\begin{lstlisting}[language=Swift,basicstyle=\tiny\ttfamily]
class Node {
  var data: Int
  var next: Node?
}
\end{lstlisting}\end{minipage} \\
\midrule

\texttt{union} + tag & \texttt{enum} + associated values & 
\begin{minipage}{4cm}\begin{lstlisting}[language=C,basicstyle=\tiny\ttfamily]
enum Type {INT, STR};
union {
  int i;
  char *s;
} value;
\end{lstlisting}\end{minipage}
$\to$
\begin{minipage}{4cm}\begin{lstlisting}[language=Swift,basicstyle=\tiny\ttfamily]
enum Value {
  case int(Int)
  case string(String)
}
\end{lstlisting}\end{minipage} \\
\bottomrule
\end{tabular}
\caption{Mappatura strutture dati C $\to$ Swift}
\label{tab:struct_mapping}
\end{table}

\subsubsection{Gestione Memoria}

\begin{table}[h]
\centering
\begin{tabular}{@{}lll@{}}
\toprule
\textbf{Operazione C} & \textbf{Equivalente Swift} & \textbf{Note} \\
\midrule
\texttt{malloc(size)} & \texttt{Array(repeating:count:)} & ARC gestisce dealloc \\
\texttt{calloc(n, size)} & \texttt{Array<T>()} & Inizializzato a default \\
\texttt{realloc(ptr, new\_size)} & \texttt{array.append(\_:)} & Espansione automatica \\
\texttt{free(ptr)} & — & ARC libera automaticamente \\
\texttt{memcpy(dst, src, n)} & \texttt{Array slicing} & Copy-on-write \\
\bottomrule
\end{tabular}
\caption{Mappatura gestione memoria}
\label{tab:memory_mapping}
\end{table}

\subsection{Pattern di Traduzione Comuni}

\subsubsection{Linked List}

\textbf{C}:
\begin{lstlisting}[language=C]
struct Node {
    void *data;
    struct Node *next;
};

void append(struct Node **head, void *data) {
    struct Node *new_node = malloc(sizeof(struct Node));
    new_node->data = data;
    new_node->next = NULL;
    
    if (*head == NULL) {
        *head = new_node;
    } else {
        struct Node *curr = *head;
        while (curr->next != NULL) curr = curr->next;
        curr->next = new_node;
    }
}
\end{lstlisting}

\textbf{Swift}:
\begin{lstlisting}[language=Swift]
class Node {
    var data: AnyObject
    var next: Node?
    
    init(data: AnyObject) {
        self.data = data
        self.next = nil
    }
}

func append(_ head: inout Node?, _ data: AnyObject) {
    let newNode = Node(data: data)
    
    guard var current = head else {
        head = newNode
        return
    }
    
    while let next = current.next {
        current = next
    }
    current.next = newNode
}
\end{lstlisting}

\textbf{Miglioramento}: reference semantics automatica, no manual dealloc.

\subsubsection{Function Pointer}

\textbf{C}:
\begin{lstlisting}[language=C]
typedef void (*Callback)(void *data);

struct Handler {
    Callback func;
    void *context;
};

void invoke(struct Handler *h) {
    h->func(h->context);
}
\end{lstlisting}

\textbf{Swift}:
\begin{lstlisting}[language=Swift]
struct Handler {
    let callback: (AnyObject?) -> Void
    let context: AnyObject?
    
    func invoke() {
        callback(context)
    }
}
\end{lstlisting}

\textbf{Miglioramento}: closures con capture automatico.

\section{Architettura Modulare di SLIPS}

\subsection{Organizzazione in Pacchetti}

\begin{verbatim}
SLIPS/
|-- Sources/SLIPS/
|   |-- CLIPS.swift           [Facade]
|   |-- Core/                 [22 file]
|   |   |-- Environment       [State management]
|   |   |-- Evaluator         [Expression evaluation]
|   |   |-- Parser            [Lexing & parsing]
|   |   |-- Functions         [Built-ins]
|   |   |-- Router            [I/O system]
|   |   +-- Modules           [Module system]
|   |-- Rete/                 [12 file]
|   |   |-- Alpha             [Pattern filtering]
|   |   |-- Beta              [Join & memory]
|   |   |-- Drive             [Propagation]
|   |   +-- Builder           [Network construction]
|   +-- Agenda/               [1 file]
|       +-- Conflict resolution
+-- Tests/SLIPSTests/         [39 file]
    +-- 91 test (97.8% pass)
\end{verbatim}

\subsection{Dipendenze tra Moduli}

\begin{figure}[h]
\centering
\begin{tikzpicture}[
  node distance=2cm,
  module/.style={rectangle, draw, fill=blue!10, thick, minimum width=2.5cm, minimum height=1cm, font=\small},
  arrow/.style={->, >=stealth, thick}
]

% Nodes
\node[module] (facade) {CLIPS Facade};
\node[module, below=of facade] (env) {Environment};
\node[module, below left=1.5cm and 1cm of env] (eval) {Evaluator};
\node[module, below right=1.5cm and 1cm of env] (func) {Functions};
\node[module, below=of eval] (rete) {RETE Engine};
\node[module, below=of func] (agenda) {Agenda};
\node[module, below=of rete] (modules) {Modules};

% Arrows
\draw[arrow] (facade) -- (env);
\draw[arrow] (facade) -- (eval);
\draw[arrow] (facade) -- (func);
\draw[arrow] (eval) -- (env);
\draw[arrow] (func) -- (env);
\draw[arrow] (rete) -- (env);
\draw[arrow] (rete) -- (agenda);
\draw[arrow] (agenda) -- (env);
\draw[arrow] (modules) -- (env);

\end{tikzpicture}
\caption{Grafo delle dipendenze tra moduli SLIPS}
\label{fig:dependencies}
\end{figure}

\textbf{Regole di dipendenza}:
\begin{itemize}
\item \textbf{Permesse}: Core $\to$ Rete, Rete $\to$ Agenda
\item \textbf{Vietate}: Rete $\to$ Functions (ciclica), Agenda $\to$ Rete (ciclica)
\end{itemize}

\section{Environment: Il Cuore di SLIPS}

\subsection{Struttura dell'Environment}

\texttt{Environment} è il contesto globale di esecuzione:

\begin{lstlisting}[language=Swift]
public final class Environment {
    // Facts management
    public var facts: [Int: FactRec] = [:]
    public var nextFactId: Int = 1
    
    // Rules management
    public var rules: [Rule] = []
    
    // Templates
    public var templates: [String: Template] = [:]
    
    // RETE network
    public var rete: ReteNetwork = ReteNetwork()
    
    // Agenda
    public var agendaQueue: Agenda = Agenda()
    
    // Modules (Fase 3)
    internal var _currentModule: Defmodule?
    internal var _moduleStack: ModuleStackItem?
    
    // Bindings
    public var localBindings: [String: Value] = [:]
    public var globalBindings: [String: Value] = [:]
    
    // Watch flags
    public var watchFacts: Bool = false
    public var watchRules: Bool = false
    public var watchRete: Bool = false
    
    // ... ~100 campi totali
}
\end{lstlisting}

\subsection{Design Pattern: God Object}

\texttt{Environment} è intenzionalmente un \textit{God Object}:

\begin{itemize}
\item \textbf{Pro}:
  \begin{itemize}
  \item Compatibile con design C di CLIPS
  \item Passaggio singolo parametro (\texttt{inout})
  \item Stato centralizzato
  \end{itemize}
\item \textbf{Contro}:
  \begin{itemize}
  \item Violazione Single Responsibility
  \item Testing più complesso
  \item Accoppiamento elevato
  \end{itemize}
\end{itemize}

\textbf{Decisione}: Manteniamo pattern C per fedeltà, ma organizziamo in extension logiche.

\subsection{Extension per Dominio}

\begin{lstlisting}[language=Swift]
// envrnmnt.swift
public final class Environment { ... }

// Modules.swift
extension Environment {
    func initializeModules() { ... }
    func createDefmodule(...) -> Defmodule? { ... }
}

// ruleengine.swift
extension Environment {
    func addRule(_ rule: Rule) { ... }
    func findRule(_ name: String) -> Rule? { ... }
}
\end{lstlisting}

\section{Value Type: Rappresentazione Dati}

\subsection{Enum per Valori Eterogenei}

CLIPS supporta tipi multipli (int, float, string, symbol, multifield). In C:

\begin{lstlisting}[language=C]
enum TypeCode { INTEGER, FLOAT, STRING, SYMBOL, MULTIFIELD };

struct UDFValue {
    enum TypeCode type;
    union {
        long long int_value;
        double float_value;
        char *string_value;
        struct multifield *mf_value;
    } value;
};
\end{lstlisting}

In Swift, usiamo enum con associated values:

\begin{lstlisting}[language=Swift]
public enum Value: Codable, Equatable {
    case int(Int64)
    case float(Double)
    case string(String)
    case symbol(String)
    case boolean(Bool)
    case multifield([Value])
    case none
}
\end{lstlisting}

\textbf{Vantaggi}:
\begin{itemize}
\item Type-safe: impossibile accedere al campo sbagliato
\item Pattern matching exhaustive: compilatore verifica tutti i casi
\item Codable: serializzazione automatica
\item Equatable: confronto strutturale
\end{itemize}

\subsection{Pattern Matching su Value}

\begin{lstlisting}[language=Swift]
func eval(_ value: Value) throws -> Double {
    switch value {
    case .int(let i):
        return Double(i)
    case .float(let d):
        return d
    case .string, .symbol, .boolean, .multifield, .none:
        throw EvaluationError.typeError("Expected number")
    }
}
\end{lstlisting}

Il compilatore garantisce che tutti i casi siano gestiti.

\section{Facciata Pubblica}

\subsection{Design Pattern: Facade}

\texttt{CLIPS.swift} fornisce API semplificata:

\begin{lstlisting}[language=Swift]
@MainActor
public enum CLIPS {
    private static var currentEnv: Environment? = nil
    
    public static func createEnvironment() -> Environment {
        var env = Environment()
        Functions.registerBuiltins(&env)
        ExpressionEnv.InitExpressionData(&env)
        env.initializeModules()
        // ...
        currentEnv = env
        return env
    }
    
    @discardableResult
    public static func eval(expr: String) -> Value {
        guard var env = currentEnv else { return .none }
        // Parse and evaluate
        return evaluateExpression(&env, expr)
    }
    
    public static func run(limit: Int?) -> Int {
        guard var env = currentEnv else { return 0 }
        return RuleEngine.run(&env, limit: limit)
    }
    
    // ... altre 10+ funzioni pubbliche
}
\end{lstlisting}

\subsection{Thread Safety con @MainActor}

Swift 6 introduce \textit{strict concurrency checking}:

\begin{lstlisting}[language=Swift]
@MainActor
public enum CLIPS {
    // Tutte le operazioni sono confinate al main thread
    // Impossibile chiamare da thread secondari senza await
}
\end{lstlisting}

\textbf{Garanzia}: Zero data races, verificato a compile-time.

\section{Architettura RETE in SLIPS}

\subsection{Dual Implementation}

SLIPS offre DUE implementazioni RETE:

\begin{enumerate}
\item \textbf{Legacy RETE} (BetaEngine.swift):
   \begin{itemize}
   \item Basato su compilazione pattern → IR
   \item Beta memory con hash indexing
   \item Backtracking + incremental
   \end{itemize}

\item \textbf{Explicit RETE} (Nodes.swift + DriveEngine.swift):
   \begin{itemize}
   \item Nodi espliciti (class-based)
   \item Fedele a \texttt{drive.c} CLIPS
   \item Propagazione C-like
   \end{itemize}
\end{enumerate}

\textbf{Flag di controllo}:
\begin{lstlisting}[language=Swift]
env.useExplicitReteNodes = true  // Usa nodi espliciti
\end{lstlisting}

\subsection{Nodi Espliciti}

\subsubsection{Protocollo ReteNode}

\begin{lstlisting}[language=Swift]
public protocol ReteNode: AnyObject {
    var id: UUID { get }
    var level: Int { get }
    func activate(token: BetaToken, env: inout Environment)
}
\end{lstlisting}

\subsubsection{Implementazioni}

\begin{lstlisting}[language=Swift]
public final class AlphaNodeClass: ReteNode {
    public let id: UUID
    public let level: Int
    public let pattern: Pattern
    public var memory: Set<Int> = []  // Fact IDs
    public var successors: [JoinNodeClass] = []
    public var rightJoinListeners: [JoinNodeClass] = []
    
    public func activate(token: BetaToken, env: inout Environment) {
        for join in successors {
            join.activateFromLeft(token: token, env: &env)
        }
    }
}

public final class JoinNodeClass: ReteNode {
    public let id: UUID
    public let level: Int
    public var leftInput: ReteNode?
    public var rightInput: AlphaNodeClass?
    public var joinKeys: Set<String>
    public var tests: [ExpressionNode]
    public var successors: [ReteNode] = []
    public var firstJoin: Bool = false
    
    public func activate(token: BetaToken, env: inout Environment) {
        // Logica join complessa
    }
    
    func activateFromLeft(token: BetaToken, env: inout Environment) {
        // Match con fatti in rightInput.memory
    }
    
    func activateFromRight(fact: FactRec, env: inout Environment) {
        if firstJoin {
            DriveEngine.EmptyDrive(join: self, fact: fact, env: &env)
        } else {
            DriveEngine.NetworkAssertRight(join: self, fact: fact, env: &env)
        }
    }
}

public final class ProductionNode: ReteNode {
    public let id: UUID
    public let level: Int
    public let ruleName: String
    public let rhs: [ExpressionNode]
    public let salience: Int
    
    public func activate(token: BetaToken, env: inout Environment) {
        // Crea attivazione in agenda
        var activation = Activation(
            priority: salience,
            ruleName: ruleName,
            bindings: token.bindings
        )
        activation.factIDs = token.usedFacts
        
        if !env.agendaQueue.contains(activation) {
            env.agendaQueue.add(activation)
        }
    }
}
\end{lstlisting}

\section{DriveEngine: Port Fedele di drive.c}

\subsection{Strutture C-Faithful}

\texttt{DriveEngine.swift} traduce fedelmente \texttt{drive.c} di CLIPS:

\begin{lstlisting}[language=Swift]
public enum DriveEngine {
    /// Port di NetworkAssert (drive.c)
    public static func NetworkAssertRight(
        join: JoinNodeClass,
        fact: FactRec,
        env: inout Environment
    ) {
        // Ottieni beta memory sinistra
        guard let leftMemory = GetLeftBetaMemory(join, env: env) else {
            return
        }
        
        // Per ogni partial match a sinistra
        for pm in leftMemory.allMatches {
            // Verifica compatibilita'
            if isCompatible(pm, fact, join, env: &env) {
                // Merge in nuovo partial match
                let newPM = mergePartialMatches(pm, fact, join)
                // Propaga ai successori
                propagatePartialMatch(newPM, join, env: &env)
            }
        }
    }
    
    /// Port di EmptyDrive (drive.c)
    public static func EmptyDrive(
        join: JoinNodeClass,
        fact: FactRec,
        env: inout Environment
    ) {
        // Caso speciale: primo join senza predecessori
        let alphMatch = createAlphaMatch(fact)
        let initialPM = PartialMatch()
        initialPM.binds = [GenericMatch(theMatch: alphaMatch)]
        
        // Propaga attraverso nextLinks
        propagateEmptyDrive(initialPM, join, env: &env)
    }
}
\end{lstlisting}

\subsection{Partial Match Structure}

Port fedele di \texttt{struct partialMatch} (match.h):

\begin{lstlisting}[language=Swift]
/// Port fedele di struct partialMatch (match.h linee 74-98)
public final class PartialMatch {
    // Flags (bitfield in C)
    public var betaMemory: Bool = false
    public var busy: Bool = false
    public var rhsMemory: Bool = false
    
    // Count e hash
    public var bcount: UInt16 = 0
    public var hashValue: UInt = 0
    
    // Parent-child relationships
    public var children: PartialMatch? = nil
    public var rightParent: PartialMatch? = nil
    public var leftParent: PartialMatch? = nil
    
    // Bindings array (flexible array in C)
    public var binds: [GenericMatch] = []
    
    // MultifieldMarker
    public var marker: MultifieldMarker? = nil
}
\end{lstlisting}

Ogni campo corrisponde esattamente al C, preservando semantica.

\section{NetworkBuilder: Costruzione Rete}

\subsection{Algoritmo di Build}

\begin{lstlisting}[language=Swift]
public enum NetworkBuilder {
    public static func buildNetwork(
        for rule: Rule,
        env: inout Environment
    ) -> ProductionNode {
        var currentLevel = 0
        var currentNode: ReteNode? = nil
        
        for (index, pattern) in rule.patterns.enumerated() {
            // 1. Trova o crea alpha node
            let alphaNode = findOrCreateAlphaNode(
                pattern: pattern,
                env: &env
            )
            
            if index == 0 {
                // Primo pattern: alpha e' root
                currentNode = alphaNode
            } else {
                // Pattern successivi: crea join
                let joinKeys = extractJoinKeys(
                    pattern,
                    previousPatterns: Array(rule.patterns[..<index])
                )
                
                let joinNode = JoinNodeClass(
                    left: currentNode!,
                    right: alphaNode,
                    keys: joinKeys,
                    level: currentLevel + 1
                )
                
                // Marca primo join
                if index == 1 {
                    joinNode.firstJoin = true
                }
                
                // Beta memory per persistenza
                let betaMemory = BetaMemoryNode(level: currentLevel + 1)
                
                linkNodes(from: joinNode, to: betaMemory)
                currentNode = betaMemory
            }
            
            currentLevel += 1
        }
        
        // Production node terminale
        let productionNode = ProductionNode(
            ruleName: rule.name,
            rhs: rule.rhs,
            salience: rule.salience,
            level: currentLevel + 1
        )
        
        linkNodes(from: currentNode!, to: productionNode)
        
        return productionNode
    }
}
\end{lstlisting}

\subsection{Alpha Node Sharing}

\begin{lstlisting}[language=Swift]
private static func findOrCreateAlphaNode(
    pattern: Pattern,
    env: inout Environment
) -> AlphaNodeClass {
    // Genera chiave basata su signature pattern
    let key = alphaNodeKey(pattern)
    
    // Cerca esistente
    if let existing = env.rete.alphaNodes[key] {
        return existing  // CONDIVISIONE!
    }
    
    // Crea nuovo
    let alphaNode = AlphaNodeClass(
        pattern: pattern,
        level: 0
    )
    
    env.rete.alphaNodes[key] = alphaNode
    return alphaNode
}

private static func alphaNodeKey(_ pattern: Pattern) -> String {
    var key = pattern.name
    
    // Includi costanti nella signature
    for (slot, test) in pattern.slots.sorted(by: { $0.key < $1.key }) {
        if case .constant(let value) = test.kind {
            key += ":\(slot)=\(value)"
        }
    }
    
    return key
}
\end{lstlisting}

\textbf{Invariante}: Pattern identici condividono stesso alpha node.

\section{Gestione della Memoria}

\subsection{Automatic Reference Counting}

Swift usa ARC invece di malloc/free:

\begin{lstlisting}[language=Swift]
class Node {
    var data: Int
    var next: Node?  // Strong reference
    
    init(data: Int) {
        self.data = data
    }
    
    // Deinit chiamato automaticamente quando refcount = 0
    deinit {
        print("Node deallocato")
    }
}

var head: Node? = Node(data: 1)
head?.next = Node(data: 2)
head = nil  // Entrambi i nodi deallocati automaticamente
\end{lstlisting}

\subsection{Cicli di Riferimento}

Problema: parent-child con riferimenti bidirezionali.

\textbf{In C}: Gestito manualmente con careful dealloc order.

\textbf{In Swift}: Uso di \texttt{weak} references:

\begin{lstlisting}[language=Swift]
class PartialMatch {
    var children: PartialMatch?           // Strong
    weak var leftParent: PartialMatch?    // Weak!
    weak var rightParent: PartialMatch?   // Weak!
}
\end{lstlisting}

\textbf{Regola}: parent $\to$ child strong, child $\to$ parent weak.

\section{Pattern di Traduzione Avanzati}

\subsection{Flexible Array Member}

\textbf{C} usa flexible array:
\begin{lstlisting}[language=C]
struct PartialMatch {
    // ... campi fissi ...
    struct GenericMatch binds[1];  // Flexible array
};

// Allocazione
struct PartialMatch *pm = malloc(
    sizeof(struct PartialMatch) + 
    (n - 1) * sizeof(struct GenericMatch)
);
\end{lstlisting}

\textbf{Swift} usa Array:
\begin{lstlisting}[language=Swift]
class PartialMatch {
    // ... campi fissi ...
    var binds: [GenericMatch] = []  // Array dinamico
}

// Allocazione
let pm = PartialMatch()
pm.binds = Array(repeating: GenericMatch(), count: n)
\end{lstlisting}

\textbf{Vantaggio}: bounds checking automatico, crescita dinamica.

\subsection{Macro Preprocessing}

\textbf{C} usa macro pesantemente:
\begin{lstlisting}[language=C]
#define GetEnvironmentData(env, pos) \
    ((env)->theData[pos])

#define PatternData(env) \
    ((struct patternData *) GetEnvironmentData(env, PATTERN_DATA))
\end{lstlisting}

\textbf{Swift} usa computed properties o funzioni static:
\begin{lstlisting}[language=Swift]
extension Environment {
    func getEnvironmentData<T>(_ position: Int) -> T? {
        return theData[position] as? T
    }
    
    var patternData: PatternData? {
        return getEnvironmentData(PATTERN_DATA)
    }
}
\end{lstlisting}

\subsection{Callback e Function Pointers}

\textbf{C}:
\begin{lstlisting}[language=C]
typedef int (*RouterQueryFunction)(void *env, const char *name);
typedef void (*RouterWriteFunction)(void *env, const char *name, const char *str);

struct Router {
    RouterQueryFunction query;
    RouterWriteFunction write;
    void *context;
};
\end{lstlisting}

\textbf{Swift}:
\begin{lstlisting}[language=Swift]
public struct RouterCallbacks {
    public let query: (Environment, String) -> Bool
    public let write: (Environment, String, String) -> Void
}

// Uso con closure
let router = RouterCallbacks(
    query: { env, name in name == "stdout" },
    write: { env, name, str in print(str, terminator: "") }
)
\end{lstlisting}

\section{Testing e Validazione}

\subsection{Architettura dei Test}

\begin{verbatim}
Tests/SLIPSTests/
|-- Equivalence Tests      [Confronto CLIPS output]
|-- Unit Tests             [Singoli moduli]
|-- Integration Tests      [Flussi completi]
|-- Performance Tests      [Benchmark]
+-- Regression Tests       [Bug fixes]
\end{verbatim}

\subsection{Strategia di Testing}

\begin{enumerate}
\item \textbf{Golden Files}: Output CLIPS C come riferimento
\item \textbf{Property-Based}: Invarianti verificati
\item \textbf{Mutation Testing}: Robustezza modifiche
\item \textbf{Coverage}: Target 85\%+
\end{enumerate}

\begin{lstlisting}[language=Swift]
final class CLIPSEquivalenceTests: XCTestCase {
    func testRuleExecutionOrder() {
        let env = CLIPS.createEnvironment()
        
        // Carica stesse regole di CLIPS C
        _ = CLIPS.eval(expr: "(deftemplate person (slot name))")
        _ = CLIPS.eval(expr: "(defrule r1 (person) => (printout t \"R1\"))")
        _ = CLIPS.eval(expr: "(defrule r2 (person) => (printout t \"R2\"))")
        
        _ = CLIPS.eval(expr: "(assert (person (name \"Mario\")))")
        
        let fired = CLIPS.run(limit: nil)
        
        // Verifica equivalenza
        XCTAssertEqual(fired, 2)
        // Verifica ordine di firing (depth strategy)
        // ... confronto con output CLIPS C
    }
}
\end{lstlisting}

\section{Metriche e Qualità}

\subsection{Metriche Statiche}

\begin{table}[h]
\centering
\begin{tabular}{@{}lrr@{}}
\toprule
\textbf{Metrica} & \textbf{Valore} & \textbf{Target} \\
\midrule
Linee codice totali & 8.046 & --- \\
File Swift & 35 & < 50 \\
LOC/file medio & 230 & < 300 \\
File > 1000 LOC & 1 & 0 \\
Funzioni > 50 LOC & 12 & < 20 \\
Unsafe code files & 1 & < 3 \\
Force unwraps pubblici & 0 & 0 \\
\bottomrule
\end{tabular}
\caption{Metriche statiche del codice}
\label{tab:static_metrics}
\end{table}

\subsection{Metriche di Test}

\begin{table}[h]
\centering
\begin{tabular}{@{}lrr@{}}
\toprule
\textbf{Metrica} & \textbf{Valore} & \textbf{Target} \\
\midrule
Test totali & 91 & > 50 \\
Test passanti & 89 & 100\% \\
Pass rate & 97.8\% & > 90\% \\
LOC test & 2.004 & --- \\
Ratio test/code & 1:4 & 1:3--1:5 \\
Coverage stimata & 85\% & > 80\% \\
\bottomrule
\end{tabular}
\caption{Metriche di testing}
\label{tab:test_metrics}
\end{table}

\section{Decisioni Architetturali Chiave}

\subsection{Scelta 1: Class vs Struct per Nodi}

\textbf{Decisione}: \texttt{class} (reference semantics)

\textbf{Motivazione}:
\begin{itemize}
\item Nodi formano grafo con cicli potenziali
\item Identità di nodi è importante (non solo valore)
\item Mutabilità condivisa necessaria
\item Allineamento con puntatori C
\end{itemize}

\subsection{Scelta 2: Dual RETE Implementation}

\textbf{Decisione}: Mantenere entrambe le implementazioni

\textbf{Motivazione}:
\begin{itemize}
\item Legacy RETE: stabile, testato, performante
\item Explicit RETE: fedele a C, manutenibile, comprensibile
\item Permettere confronti e validazione incrociata
\item Transizione graduale
\end{itemize}

\subsection{Scelta 3: Environment Mutability}

\textbf{Decisione}: \texttt{inout} parameter pattern

\textbf{Motivazione}:
\begin{itemize}
\item Compatibile con C (pass pointer)
\item Esplicita la mutazione
\item Evita copy implicite
\item Facilita refactoring
\end{itemize}

\begin{lstlisting}[language=Swift]
// Invece di metodi mutanti su oggetto:
// env.eval(expr)

// Usiamo funzioni con inout:
Evaluator.eval(&env, expr)
\end{lstlisting}

\section{Performance Preliminari}

\subsection{Benchmark Sintetici}

\begin{table}[h]
\centering
\begin{tabular}{@{}lrr@{}}
\toprule
\textbf{Operazione} & \textbf{Tempo} & \textbf{Note} \\
\midrule
Assert 1000 fatti & 15 ms & Regola semplice \\
Join 2 pattern (10k fatti) & 45 ms & Hash join \\
Retract 1000 fatti & 8 ms & Beta cleanup \\
Build network (100 regole) & 5 ms & Una tantum \\
\bottomrule
\end{tabular}
\caption{Performance preliminari (Apple M1)}
\label{tab:perf_prelim}
\end{table}

\subsection{Confronto con CLIPS C}

\begin{table}[h]
\centering
\begin{tabular}{@{}lrrr@{}}
\toprule
\textbf{Benchmark} & \textbf{CLIPS C} & \textbf{SLIPS} & \textbf{Overhead} \\
\midrule
Assert (1k facts) & 10 ms & 15 ms & 1.5x \\
Join (10k facts) & 30 ms & 45 ms & 1.5x \\
Fire rules (100) & 5 ms & 8 ms & 1.6x \\
\bottomrule
\end{tabular}
\caption{Confronto performance CLIPS C vs SLIPS (stimato)}
\label{tab:perf_comparison}
\end{table}

\textbf{Overhead accettabile} considerando:
\begin{itemize}
\item Safety garantita (bounds checking, type safety)
\item ARC overhead vs manual memory
\item Swift non ottimizzato come C puro
\end{itemize}

\section{Conclusioni del Capitolo}

In questo capitolo abbiamo:

\begin{itemize}
\item Definito l'architettura generale di SLIPS
\item Presentato le regole di mappatura C $\to$ Swift
\item Descritto l'implementazione dual RETE
\item Analizzato decisioni architetturali chiave
\item Mostrato pattern di traduzione comuni
\end{itemize}

Nei prossimi capitoli approfondiremo l'implementazione specifica di ciascun componente.

\begin{successbox}[Punti Chiave]
\begin{itemize}
\item SLIPS preserva architettura CLIPS ma con type safety Swift
\item Dual implementation RETE: legacy (stabile) + explicit (C-faithful)
\item Environment è God Object intenzionale per compatibilità
\item ARC + value types eliminano gestione manuale memoria
\item 97.8\% test pass rate garantisce equivalenza comportamentale
\end{itemize}
\end{successbox}


% Capitolo 17: Implementazione Core di SLIPS

\chapter{SLIPS Core: Fondamenta Swift}
\label{cap:slips_core}

\section{Introduzione}

Questo capitolo presenta l'implementazione core di SLIPS, mostrando come le strutture C di CLIPS vengono tradotte idiomaticamente in Swift preservando la semantica.

\section{Environment}

\subsection{Struttura Principale}

\begin{lstlisting}[language=Swift]
@MainActor
public class Environment {
    // Fatti
    private(set) var factList: [Fact] = []
    private var nextFactID: Int = 0
    
    // Regole e template
    private(set) var rules: [String: Rule] = [:]
    private(set) var templates: [String: Deftemplate] = [:]
    
    // Moduli
    private(set) var modules: [String: Defmodule]
    private(set) var currentModule: Defmodule
    private var focusStack: FocusStack
    
    // RETE Network
    private(set) var alphaNetwork: AlphaNetwork
    private(set) var betaNetwork: BetaNetwork
    
    // Agenda
    private(set) var agenda: Agenda
    
    // Router system
    private var routers: [Router] = []
    
    // State
    private var isRunning: Bool = false
    private var haltFlag: Bool = false
}
\end{lstlisting}

\subsection{Isolamento}

Ogni `Environment` è isolato:

\begin{lstlisting}[language=Swift]
let env1 = Environment()
let env2 = Environment()

env1.load("rules1.clp")
env2.load("rules2.clp")

// Completamente indipendenti
env1.run()  
env2.run()
\end{lstlisting}

\textbf{Benefici}:
\begin{itemize}
\item Testing parallelo
\item Multi-tenancy
\item Sandbox per sperimentazione
\end{itemize}

\section{Value System}

\subsection{Value Enum}

\begin{lstlisting}[language=Swift]
public enum Value: Hashable {
    case symbol(String)
    case string(String)
    case integer(Int)
    case float(Double)
    case fact(Int)  // Fact ID
    case multifield([Value])
    case external(AnyHashable)  // User-defined
    
    var type: ValueType {
        switch self {
        case .symbol: return .symbol
        case .string: return .string
        case .integer: return .integer
        case .float: return .float
        case .fact: return .factAddress
        case .multifield: return .multifield
        case .external: return .external
        }
    }
}
\end{lstlisting}

\textbf{Vs C}: In C usano tagged union, in Swift enum con associated values è più type-safe.

\subsection{Symbol Interning}

\begin{lstlisting}[language=Swift]
class SymbolTable {
    private var symbols: [String: Symbol] = [:]
    private var nextID: Int = 0
    
    func intern(_ string: String) -> Symbol {
        if let existing = symbols[string] {
            return existing
        }
        let symbol = Symbol(id: nextID, contents: string)
        nextID += 1
        symbols[string] = symbol
        return symbol
    }
}

struct Symbol: Hashable {
    let id: Int
    let contents: String
    
    static func == (lhs: Symbol, rhs: Symbol) -> Bool {
        return lhs.id == rhs.id  // O(1) comparison
    }
}
\end{lstlisting}

\section{Fatti}

\subsection{Fact Structure}

\begin{lstlisting}[language=Swift]
public class Fact: Hashable, Identifiable {
    public let id: Int
    public let template: Deftemplate
    public let slots: [String: Value]
    public let isOrdered: Bool
    
    // RETE state
    var alphaMemories: Set<AlphaMemory> = []
    var tokens: Set<Token> = []
    
    init(id: Int, template: Deftemplate, slots: [String: Value]) {
        self.id = id
        self.template = template
        self.slots = slots
        self.isOrdered = template.isImplied
    }
    
    public func hash(into hasher: inout Hasher) {
        hasher.combine(id)
    }
    
    public static func == (lhs: Fact, rhs: Fact) -> Bool {
        return lhs.id == rhs.id
    }
}
\end{lstlisting}

\subsection{Assertion}

\begin{lstlisting}[language=Swift]
@discardableResult
public func assert(template: String, slots: [String: Value]) -> Fact? {
    guard let deftemplate = templates[template] else {
        print("Error: Template '\(template)' not found")
        return nil
    }
    
    // Validate
    guard deftemplate.validate(slots: slots) else {
        print("Error: Invalid slots for template '\(template)'")
        return nil
    }
    
    // Create fact
    let fact = Fact(id: nextFactID, template: deftemplate, slots: slots)
    nextFactID += 1
    factList.append(fact)
    
    // Propagate through RETE
    alphaNetwork.assertFact(fact)
    
    return fact
}
\end{lstlisting>

\subsection{Retraction}

\begin{lstlisting}[language=Swift]
public func retract(fact: Fact) {
    // Remove from list
    factList.removeAll { $0.id == fact.id }
    
    // Propagate retraction through RETE
    alphaNetwork.retractFact(fact)
}

public func retract(id: Int) {
    guard let fact = factList.first(where: { $0.id == id }) else {
        print("Error: Fact \(id) not found")
        return
    }
    retract(fact: fact)
}
\end{lstlisting>

\section{Deftemplate}

\subsection{Structure}

\begin{lstlisting}[language=Swift]
public struct Deftemplate {
    public let name: String
    public let isImplied: Bool  // Ordered fact
    public let slots: [Slot]
    public let module: Defmodule
    
    public struct Slot {
        public let name: String
        public let isMultifield: Bool
        public let type: ValueType?
        public let defaultValue: Value?
        public let range: ClosedRange<Double>?
        public let allowedValues: Set<Value>?
        
        func validate(_ value: Value) -> Bool {
            // Type check
            if let type = type, value.type != type {
                return false
            }
            
            // Range check
            if let range = range, case .float(let f) = value {
                return range.contains(f)
            }
            
            // Allowed values
            if let allowed = allowedValues {
                return allowed.contains(value)
            }
            
            return true
        }
    }
    
    func validate(slots: [String: Value]) -> Bool {
        for slot in self.slots {
            if let value = slots[slot.name] {
                if !slot.validate(value) {
                    return false
                }
            } else if slot.defaultValue == nil {
                // Required slot missing
                return false
            }
        }
        return true
    }
}
\end{lstlisting>

\section{Defrule}

\subsection{Structure}

\begin{lstlisting}[language=Swift]
public class Defrule {
    public let name: String
    public let module: Defmodule
    public let patterns: [Pattern]
    public let actions: [Action]
    public let salience: Int
    public let autoFocus: Bool
    public let dynamicSalience: Expression?
    
    // RETE connection
    weak var productionNode: ProductionNode?
}

public struct Pattern {
    public let template: String
    public let constraints: [Constraint]
    public let isNegated: Bool
    
    public struct Constraint {
        public let slot: String
        public let test: Test
        
        public enum Test {
            case equals(Value)
            case variable(String)
            case predicate((Value) -> Bool)
            case compound([Test])
        }
    }
}

public enum Action {
    case assert(template: String, slots: [(String, Expression)])
    case retract(Expression)
    case modify(Expression, slots: [(String, Expression)])
    case printout(router: String, values: [Expression])
    case bind(variable: String, value: Expression)
    case functionCall(name: String, args: [Expression])
}
\end{lstlisting}

\section{Parser e Compiler}

\subsection{S-Expression Parser}

\begin{lstlisting}[language=Swift]
class SExpressionParser {
    func parse(_ input: String) throws -> [SExpr] {
        var tokens = tokenize(input)
        var result: [SExpr] = []
        
        while !tokens.isEmpty {
            result.append(try parseExpr(&tokens))
        }
        
        return result
    }
    
    private func parseExpr(_ tokens: inout [Token]) throws -> SExpr {
        guard let first = tokens.first else {
            throw ParseError.unexpectedEOF
        }
        
        tokens.removeFirst()
        
        switch first {
        case .lparen:
            var list: [SExpr] = []
            while tokens.first != .rparen {
                list.append(try parseExpr(&tokens))
            }
            tokens.removeFirst()  // consume rparen
            return .list(list)
            
        case .symbol(let s):
            return .symbol(s)
            
        case .string(let s):
            return .string(s)
            
        case .number(let n):
            return .number(n)
            
        default:
            throw ParseError.unexpected(first)
        }
    }
}
\end{lstlisting>

\subsection{Rule Compiler}

\begin{lstlisting}[language=Swift]
class RuleCompiler {
    func compile(sexpr: SExpr, env: Environment) throws -> Defrule {
        guard case .list(let items) = sexpr,
              case .symbol("defrule") = items[0],
              case .symbol(let name) = items[1] else {
            throw CompileError.invalidDefrule
        }
        
        var idx = 2
        var salience = 0
        var autoFocus = false
        
        // Parse declare
        if case .list(let declare) = items[idx],
           case .symbol("declare") = declare[0] {
            (salience, autoFocus) = try parseDeclare(declare)
            idx += 1
        }
        
        // Parse patterns (LHS)
        var patterns: [Pattern] = []
        while idx < items.count, 
              case .symbol("=>") = items[idx] {
            break
        }
        while idx < items.count {
            if case .symbol("=>") = items[idx] {
                break
            }
            patterns.append(try compilePattern(items[idx], env: env))
            idx += 1
        }
        
        idx += 1  // skip =>
        
        // Parse actions (RHS)
        var actions: [Action] = []
        while idx < items.count {
            actions.append(try compileAction(items[idx], env: env))
            idx += 1
        }
        
        return Defrule(
            name: name,
            module: env.currentModule,
            patterns: patterns,
            actions: actions,
            salience: salience,
            autoFocus: autoFocus,
            dynamicSalience: nil
        )
    }
}
\end{lstlisting>

\section{Execution Engine}

\subsection{Run Loop}

\begin{lstlisting}[language=Swift]
public func run(limit: Int = -1) {
    isRunning = true
    haltFlag = false
    var fired = 0
    
    while !haltFlag {
        // Check limit
        if limit >= 0 && fired >= limit {
            break
        }
        
        // Get next activation
        guard let activation = agenda.next(from: currentModule) else {
            break  // Quiescence
        }
        
        // Fire rule
        fireRule(activation)
        fired += 1
        
        // Check for module changes (focus)
        if focusStack.needsUpdate {
            currentModule = focusStack.current
        }
    }
    
    isRunning = false
}
\end{lstlisting}

\subsection{Rule Firing}

\begin{lstlisting}[language=Swift]
private func fireRule(_ activation: Activation) {
    let bindings = activation.token.bindings
    
    for action in activation.rule.actions {
        execute(action: action, bindings: bindings)
    }
}

private func execute(action: Action, bindings: [String: Value]) {
    switch action {
    case .assert(let template, let slots):
        let evaluatedSlots = slots.mapValues { expr in
            evaluate(expr, bindings: bindings)
        }
        assert(template: template, slots: evaluatedSlots)
        
    case .retract(let expr):
        if case .fact(let id) = evaluate(expr, bindings: bindings) {
            retract(id: id)
        }
        
    case .printout(let router, let values):
        let output = values.map { evaluate($0, bindings: bindings) }
                          .map { "\($0)" }
                          .joined()
        print(to: router, output)
        
    // ... other actions
    }
}
\end{lstlisting}

\section{Conclusioni del Capitolo}

\subsection{Punti Chiave}

\begin{enumerate}
\item SLIPS usa \textbf{Swift idioms} preservando semantica CLIPS
\item \textbf{@MainActor} garantisce thread-safety
\item \textbf{Enum con associated values} per type-safety
\item \textbf{ARC} semplifica memory management
\item Struttura modulare facilita testing
\end{enumerate}

\subsection{Trade-off}

\begin{itemize}
\item \textbf{Pro}: Type safety, memory safety, modern Swift
\item \textbf{Contro}: Overhead ARC, meno controllo fine-grained
\item \textbf{Risultato}: Codice più sicuro e manutenibile con performance accettabili
\end{itemize}

\subsection{Letture Consigliate}

\begin{itemize}
\item Swift Programming Language - Memory Management
\item Swift Concurrency - MainActor
\item CLIPS Source - Core modules
\end{itemize}

% Capitolo 18: Implementazione RETE in SLIPS

\chapter{SLIPS RETE: Network Implementation}
\label{cap:slips_rete}

\section{Introduzione}

Questo capitolo mostra l'implementazione Swift della rete RETE, cuore del pattern matching di SLIPS.

\section{Node Hierarchy}

\begin{lstlisting}[language=Swift]
protocol ReteNode: AnyObject {
    var id: Int { get }
    var children: [ReteNode] { get set }
    func activate(token: Token)
}

// Alpha Network
class AlphaNode: ReteNode {
    let id: Int
    var children: [ReteNode] = []
    var test: AlphaTest?
    
    func activate(token: Token) {
        guard evaluateTest(token) else { return }
        for child in children {
            child.activate(token: token)
        }
    }
}

class AlphaMemory: ReteNode {
    let id: Int
    var children: [ReteNode] = []
    var facts: Set<Fact> = []
    
    func add(_ fact: Fact) {
        facts.insert(fact)
        notifyBeta(fact)
    }
}

// Beta Network
class JoinNode: ReteNode {
    let id: Int
    var children: [ReteNode] = []
    weak var leftParent: BetaMemory?
    weak var rightParent: AlphaMemory?
    var joinTests: [JoinTest] = []
    
    func leftActivate(token: Token) {
        guard let right = rightParent else { return }
        for fact in right.facts {
            if testPass(token, fact) {
                let newToken = token.extend(with: fact)
                propagate(newToken)
            }
        }
    }
    
    func rightActivate(fact: Fact) {
        guard let left = leftParent else { return }
        for token in left.tokens {
            if testPass(token, fact) {
                let newToken = token.extend(with: fact)
                propagate(newToken)
            }
        }
    }
}

class BetaMemory: ReteNode {
    var tokens: Set<Token> = []
    
    func add(_ token: Token) {
        tokens.insert(token)
        for child in children {
            child.activate(token: token)
        }
    }
}

class ProductionNode: ReteNode {
    let rule: Defrule
    var activations: Set<Activation> = []
    
    func activate(token: Token) {
        let activation = Activation(rule: rule, token: token)
        activations.insert(activation)
        agenda.add(activation)
    }
}
\end{lstlisting}

\section{Network Builder}

\begin{lstlisting}[language=Swift]
class NetworkBuilder {
    private var alphaNodes: [String: AlphaNode] = [:]
    private var nextNodeID = 0
    
    func buildNetwork(for rule: Defrule) -> ProductionNode {
        var currentBeta: ReteNode = dummyTopNode
        
        for pattern in rule.patterns {
            // Build alpha part
            let alphaMemory = buildAlphaNetwork(for: pattern)
            
            // Build join
            let joinNode = createJoinNode(
                left: currentBeta as! BetaMemory,
                right: alphaMemory,
                tests: extractJoinTests(pattern)
            )
            
            // Beta memory after join
            let betaMemory = BetaMemory(id: nextNodeID)
            nextNodeID += 1
            joinNode.children.append(betaMemory)
            
            currentBeta = betaMemory
        }
        
        // Production node
        let prodNode = ProductionNode(id: nextNodeID, rule: rule)
        nextNodeID += 1
        currentBeta.children.append(prodNode)
        
        return prodNode
    }
    
    private func buildAlphaNetwork(for pattern: Pattern) -> AlphaMemory {
        let key = pattern.template
        
        // Get or create type node
        if let existing = alphaNodes[key] {
            return findOrCreateAlphaMemory(under: existing, pattern: pattern)
        }
        
        let typeNode = AlphaNode(id: nextNodeID)
        nextNodeID += 1
        alphaNodes[key] = typeNode
        
        return buildAlphaChain(typeNode, pattern: pattern)
    }
    
    private func buildAlphaChain(_ node: AlphaNode, pattern: Pattern) -> AlphaMemory {
        var current = node
        
        // Add test nodes for each constraint
        for constraint in pattern.constraints where constraint.isIntraElement {
            let testNode = AlphaNode(id: nextNodeID)
            nextNodeID += 1
            testNode.test = AlphaTest(constraint: constraint)
            current.children.append(testNode)
            current = testNode
        }
        
        // Alpha memory at end
        let memory = AlphaMemory(id: nextNodeID)
        nextNodeID += 1
        current.children.append(memory)
        
        return memory
    }
}
\end{lstlisting}

\section{Hash Join Optimization}

\begin{lstlisting}[language=Swift]
class HashJoinNode: JoinNode {
    private var leftIndex: [Value: Set<Token>] = [:]
    private var rightIndex: [Value: Set<Fact>] = [:]
    private let joinVariable: String
    
    override func leftActivate(token: Token) {
        guard let value = token.binding[joinVariable] else { return }
        
        // Add to index
        leftIndex[value, default: []].insert(token)
        
        // Lookup in right index
        if let rightMatches = rightIndex[value] {
            for fact in rightMatches {
                let newToken = token.extend(with: fact)
                propagate(newToken)
            }
        }
    }
    
    override func rightActivate(fact: Fact) {
        guard let value = fact.slots[joinVariable] else { return }
        
        // Add to index
        rightIndex[value, default: []].insert(fact)
        
        // Lookup in left index
        if let leftMatches = leftIndex[value] {
            for token in leftMatches {
                let newToken = token.extend(with: fact)
                propagate(newToken)
            }
        }
    }
}
\end{lstlisting}

\section{Negative Nodes}

\begin{lstlisting}[language=Swift]
class NegativeNode: ReteNode {
    private var counters: [Token: Int] = [:]
    
    func leftActivate(token: Token) {
        counters[token] = 0
        
        // Check right memory
        guard let right = rightParent else { return }
        for fact in right.facts {
            if testPass(token, fact) {
                counters[token]! += 1
            }
        }
        
        // Propagate if count = 0
        if counters[token] == 0 {
            propagate(token)
        }
    }
    
    func rightActivate(fact: Fact) {
        guard let left = leftParent else { return }
        for token in left.tokens {
            if testPass(token, fact) {
                counters[token]! += 1
                if counters[token] == 1 {
                    // Era 0, ora non più - ritira
                    removeFromChildren(token)
                }
            }
        }
    }
    
    func rightRetract(fact: Fact) {
        guard let left = leftParent else { return }
        for token in left.tokens {
            if testPass(token, fact) {
                counters[token]! -= 1
                if counters[token] == 0 {
                    // Ora soddisfatto - propaga
                    propagate(token)
                }
            }
        }
    }
}
\end{lstlisting}

\section{Token Management}

\begin{lstlisting}[language=Swift]
struct Token: Hashable {
    let facts: [Fact]
    let bindings: [String: Value]
    
    func extend(with fact: Fact) -> Token {
        var newFacts = facts
        newFacts.append(fact)
        
        var newBindings = bindings
        // Extract new bindings from fact
        // (logic depends on pattern variables)
        
        return Token(facts: newFacts, bindings: newBindings)
    }
    
    func hash(into hasher: inout Hasher) {
        hasher.combine(facts.map(\.id))
    }
    
    static func == (lhs: Token, rhs: Token) -> Bool {
        return lhs.facts.map(\.id) == rhs.facts.map(\.id)
    }
}

// Token Pool for performance
class TokenPool {
    private var pool: [Token] = []
    private let maxSize = 1000
    
    func acquire(facts: [Fact], bindings: [String: Value]) -> Token {
        if let token = pool.popLast() {
            // Reuse (would need mutable token)
            return Token(facts: facts, bindings: bindings)
        }
        return Token(facts: facts, bindings: bindings)
    }
    
    func release(_ token: Token) {
        guard pool.count < maxSize else { return }
        pool.append(token)
    }
}
\end{lstlisting}

\section{Propagation Engine}

\begin{lstlisting}[language=Swift]
class PropagationEngine {
    func assertFact(_ fact: Fact, in network: AlphaNetwork) {
        let typeNode = network.getTypeNode(for: fact.template.name)
        propagateAssert(fact, through: typeNode)
    }
    
    private func propagateAssert(_ fact: Fact, through node: AlphaNode) {
        // Evaluate test
        if let test = node.test {
            guard test.evaluate(fact) else { return }
        }
        
        // Propagate to children
        for child in node.children {
            if let alphaMemory = child as? AlphaMemory {
                alphaMemory.add(fact)
            } else if let alphaNode = child as? AlphaNode {
                propagateAssert(fact, through: alphaNode)
            }
        }
    }
    
    func retractFact(_ fact: Fact, from network: AlphaNetwork) {
        // Find alpha memories containing fact
        for memory in fact.alphaMemories {
            memory.remove(fact)
        }
        
        // Find and remove tokens containing fact
        for token in fact.tokens {
            removeToken(token)
        }
    }
    
    private func removeToken(_ token: Token) {
        // Traverse beta network removing token
        // and dependent tokens/activations
    }
}
\end{lstlisting}

\section{Performance Optimization}

\subsection{Node Sharing}

\begin{lstlisting}[language=Swift]
class SharedNodeRegistry {
    private var alphaNodes: [AlphaNodeKey: AlphaNode] = [:]
    private var joinNodes: [JoinNodeKey: JoinNode] = [:]
    
    func getOrCreateAlphaNode(
        type: String,
        test: AlphaTest?
    ) -> AlphaNode {
        let key = AlphaNodeKey(type: type, test: test)
        
        if let existing = alphaNodes[key] {
            return existing
        }
        
        let node = AlphaNode(id: nextNodeID)
        nextNodeID += 1
        node.test = test
        alphaNodes[key] = node
        return node
    }
}

struct AlphaNodeKey: Hashable {
    let type: String
    let test: AlphaTest?
}
\end{lstlisting}

\section{Conclusioni del Capitolo}

\subsection{Punti Chiave}

\begin{enumerate}
\item RETE implementato con \textbf{protocol-oriented design}
\item \textbf{Weak references} per evitare cicli
\item \textbf{Hash join} per ottimizzazione
\item \textbf{Node sharing} riduce duplicazione
\item Token pool per performance
\end{enumerate}

\subsection{Prossimi Capitoli}

Capitolo~\ref{cap:slips_agenda} mostra l'implementazione dell'agenda in Swift.

\subsection{Letture Consigliate}

\begin{itemize}
\item CLIPS Source - \texttt{drive.c}, \texttt{reteutil.c}
\item Swift Performance - Protocol-Oriented Programming
\end{itemize}

% Capitolo 19: Implementazione Agenda in SLIPS

\chapter{SLIPS Agenda Implementation}
\label{cap:slips_agenda}

\section{Introduzione}

L'agenda di SLIPS gestisce il conflict set con le strategie di CLIPS tradotte in Swift idiomatico.

\section{Activation Structure}

\begin{lstlisting}[language=Swift]
public struct Activation: Hashable, Identifiable {
    public let id: UUID = UUID()
    public let rule: Defrule
    public let token: Token
    public let salience: Int
    public let timetag: UInt64
    public let randomID: UInt32
    
    init(rule: Defrule, token: Token, timetag: UInt64) {
        self.rule = rule
        self.token = token
        self.timetag = timetag
        self.salience = rule.salience
        self.randomID = UInt32.random(in: 0..<UInt32.max)
    }
    
    public func hash(into hasher: inout Hasher) {
        hasher.combine(id)
    }
    
    public static func == (lhs: Activation, rhs: Activation) -> Bool {
        return lhs.id == rhs.id
    }
}
\end{lstlisting}

\section{Conflict Resolution Strategies}

\begin{lstlisting}[language=Swift]
public enum ConflictStrategy {
    case depth
    case breadth
    case simplicity
    case complexity
    case lex
    case mea
    case random
}

protocol ActivationComparator {
    func compare(_ a1: Activation, _ a2: Activation) -> ComparisonResult
}

class DepthComparator: ActivationComparator {
    func compare(_ a1: Activation, _ a2: Activation) -> ComparisonResult {
        // 1. Salience (higher first)
        if a1.salience != a2.salience {
            return a1.salience > a2.salience ? .orderedAscending : .orderedDescending
        }
        
        // 2. Recency (higher timetag first)
        if a1.timetag != a2.timetag {
            return a1.timetag > a2.timetag ? .orderedAscending : .orderedDescending
        }
        
        // 3. Specificity
        let spec1 = a1.rule.specificity
        let spec2 = a2.rule.specificity
        if spec1 != spec2 {
            return spec1 > spec2 ? .orderedAscending : .orderedDescending
        }
        
        // 4. Rule order
        return a1.rule.definitionOrder < a2.rule.definitionOrder 
            ? .orderedAscending 
            : .orderedDescending
    }
}

class BreadthComparator: ActivationComparator {
    func compare(_ a1: Activation, _ a2: Activation) -> ComparisonResult {
        // Like depth, but recency reversed
        if a1.salience != a2.salience {
            return a1.salience > a2.salience ? .orderedAscending : .orderedDescending
        }
        
        // Older facts first (opposite of depth)
        if a1.timetag != a2.timetag {
            return a1.timetag < a2.timetag ? .orderedAscending : .orderedDescending
        }
        
        // Rest is same
        let spec1 = a1.rule.specificity
        let spec2 = a2.rule.specificity
        if spec1 != spec2 {
            return spec1 > spec2 ? .orderedAscending : .orderedDescending
        }
        
        return a1.rule.definitionOrder < a2.rule.definitionOrder 
            ? .orderedAscending 
            : .orderedDescending
    }
}

class RandomComparator: ActivationComparator {
    func compare(_ a1: Activation, _ a2: Activation) -> ComparisonResult {
        // Salience still matters
        if a1.salience != a2.salience {
            return a1.salience > a2.salience ? .orderedAscending : .orderedDescending
        }
        
        // Random for same salience
        return a1.randomID < a2.randomID ? .orderedAscending : .orderedDescending
    }
}
\end{lstlisting}

\section{Agenda Implementation}

\begin{lstlisting}[language=Swift]
public class Agenda {
    private var activations: [Activation] = []
    private var comparator: ActivationComparator
    private var strategy: ConflictStrategy
    
    init(strategy: ConflictStrategy = .depth) {
        self.strategy = strategy
        self.comparator = Self.createComparator(for: strategy)
    }
    
    func add(_ activation: Activation) {
        // Insert maintaining sorted order
        let insertionIndex = activations.firstIndex { existing in
            comparator.compare(activation, existing) == .orderedAscending
        } ?? activations.endIndex
        
        activations.insert(activation, at: insertionIndex)
    }
    
    func remove(_ activation: Activation) {
        activations.removeAll { $0.id == activation.id }
    }
    
    func next() -> Activation? {
        return activations.first
    }
    
    func removeAndReturnNext() -> Activation? {
        guard !activations.isEmpty else { return nil }
        return activations.removeFirst()
    }
    
    func setStrategy(_ strategy: ConflictStrategy) {
        self.strategy = strategy
        self.comparator = Self.createComparator(for: strategy)
        reorder()
    }
    
    private func reorder() {
        activations.sort { a1, a2 in
            comparator.compare(a1, a2) == .orderedAscending
        }
    }
    
    func clear() {
        activations.removeAll()
    }
    
    var count: Int {
        return activations.count
    }
    
    var all: [Activation] {
        return activations
    }
}
\end{lstlisting}

\section{Module-Aware Agenda}

\begin{lstlisting}[language=Swift]
public class ModuleAgenda {
    private var agendas: [Defmodule: Agenda] = [:]
    private var focusStack: FocusStack
    
    init(focusStack: FocusStack) {
        self.focusStack = focusStack
    }
    
    func add(_ activation: Activation) {
        let module = activation.rule.module
        let agenda = agendas[module, default: Agenda()]
        agenda.add(activation)
        agendas[module] = agenda
    }
    
    func next() -> Activation? {
        // Try current focus
        if let activation = agendas[focusStack.current]?.next() {
            return activation
        }
        
        // Pop and try next
        focusStack.pop()
        
        if focusStack.isEmpty {
            return nil  // Quiescence
        }
        
        return next()  // Recursive
    }
    
    func removeAndReturnNext() -> Activation? {
        guard let activation = next() else { return nil }
        agendas[focusStack.current]?.remove(activation)
        return activation
    }
}
\end{lstlisting}

\section{Dynamic Salience}

\begin{lstlisting}[language=Swift]
extension Defrule {
    func evaluateSalience(with token: Token, env: Environment) -> Int {
        guard let dynamicExpr = dynamicSalience else {
            return salience  // Static
        }
        
        let bindings = token.bindings
        let result = env.evaluate(dynamicExpr, bindings: bindings)
        
        if case .integer(let value) = result {
            return value
        }
        
        return salience  // Fallback
    }
}

// Usage in Activation creation
func createActivation(rule: Defrule, token: Token, env: Environment) -> Activation {
    let evaluatedSalience = rule.evaluateSalience(with: token, env: env)
    
    return Activation(
        rule: rule,
        token: token,
        timetag: env.currentTimetag,
        overrideSalience: evaluatedSalience
    )
}
\end{lstlisting}

\section{Salience Evaluation Modes}

\begin{lstlisting}[language=Swift]
public enum SalienceEvaluation {
    case whenDefined      // At rule compilation
    case whenActivated    // When activation created
    case everyCycle       // Before each rule selection
}

public class SalienceManager {
    var mode: SalienceEvaluation = .whenDefined
    
    func evaluateSalience(
        for activation: Activation,
        env: Environment
    ) -> Int {
        switch mode {
        case .whenDefined:
            return activation.salience  // Already computed
            
        case .whenActivated, .everyCycle:
            return activation.rule.evaluateSalience(
                with: activation.token,
                env: env
            )
        }
    }
}
\end{lstlisting}

\section{Refresh and Reorder}

\begin{lstlisting}[language=Swift]
extension Agenda {
    func refresh(rule: Defrule, env: Environment) {
        // Remove existing activations for this rule
        activations.removeAll { $0.rule.name == rule.name }
        
        // Regenerate from production node
        if let prodNode = rule.productionNode {
            for token in prodNode.tokens {
                let activation = Activation(
                    rule: rule,
                    token: token,
                    timetag: env.currentTimetag
                )
                add(activation)
            }
        }
    }
    
    func refreshAll(env: Environment) {
        let rules = Set(activations.map(\.rule))
        for rule in rules {
            refresh(rule: rule, env: env)
        }
    }
}
\end{lstlisting}

\section{Conclusioni del Capitolo}

\subsection{Punti Chiave}

\begin{enumerate}
\item Agenda usa \textbf{sorted array} per efficienza
\item \textbf{Strategy pattern} per conflict resolution
\item \textbf{Module-aware} agenda con focus stack
\item \textbf{Dynamic salience} supportata
\item Refresh e reorder per flessibilità
\end{enumerate}

\subsection{Prossimi Passi}

Cap.~\ref{cap:slips_pattern_matching} mostra pattern matching avanzato.

\subsection{Letture Consigliate}

\begin{itemize}
\item CLIPS Source - \texttt{agenda.c}
\item Swift Collections - Sorted Arrays
\end{itemize}

% Capitolo 20: Sistema di Moduli in SLIPS

\chapter{Sistema di Moduli in SLIPS}
\label{cap:slips_moduli}

\section{Introduzione ai Moduli}

Il sistema di moduli di CLIPS permette di organizzare la conoscenza in namespace separati, facilitando:

\begin{itemize}
\item \textbf{Modularità}: separazione logica di domini
\item \textbf{Riutilizzo}: import/export di costrutti
\item \textbf{Scalabilità}: gestione di grandi basi di conoscenza
\item \textbf{Focus}: controllo esplicito dell'attenzione del sistema
\end{itemize}

\subsection{Motivazione}

In sistemi complessi con centinaia di regole, l'organizzazione diventa critica:

\begin{esempio}[Sistema Ospedaliero]
\begin{itemize}
\item Modulo \texttt{TRIAGE}: regole per classificazione urgenza
\item Modulo \texttt{DIAGNOSI}: regole per diagnosi
\item Modulo \texttt{TERAPIA}: regole per prescrizioni
\item Modulo \texttt{BILLING}: regole per fatturazione
\end{itemize}

Senza moduli, tutte le regole sarebbero attive contemporaneamente, causando:
\begin{itemize}
\item Conflitti indesiderati
\item Performance degradate
\item Difficoltà di manutenzione
\end{itemize}
\end{esempio}

\section{Formalizzazione}

\subsection{Defmodule}

\begin{definizione}[Modulo]
Un modulo $M$ è una quintupla:
\begin{equation}
M = \langle \text{name}, \text{constructs}, \text{imports}, \text{exports}, \text{focus} \rangle
\end{equation}

dove:
\begin{itemize}
\item $\text{name} \in \Sigma^*$ è il nome univoco
\item $\text{constructs} \subseteq \mathcal{C}$ è l'insieme dei costrutti definiti in $M$
\item $\text{imports} \subseteq M \times \mathcal{C}$ sono gli import da altri moduli
\item $\text{exports} \subseteq \mathcal{C}$ sono i costrutti esportati
\item $\text{focus} \in \mathbb{B}$ indica se ha focus corrente
\end{itemize}
\end{definizione}

\subsection{Visibilità}

\begin{definizione}[Costrutto Visibile]
Un costrutto $c \in \mathcal{C}$ è \textit{visibile} nel modulo $M$ se:
\begin{equation}
c \in M.\text{constructs} \lor \exists M': (M', c) \in M.\text{imports}
\end{equation}
\end{definizione}

\subsection{Focus Stack}

Il focus stack $\mathcal{F}$ è una pila LIFO di moduli:

\begin{equation}
\mathcal{F} = [M_1, M_2, \ldots, M_k]
\end{equation}

dove $M_k$ (top dello stack) ha priorità massima per firing.

\begin{definizione}[Modulo Attivo]
Il modulo attivo è:
\begin{equation}
M_{\text{active}} = \begin{cases}
\text{top}(\mathcal{F}) & \text{se } \mathcal{F} \neq \emptyset\\
M_{\text{current}} & \text{altrimenti}
\end{cases}
\end{equation}
\end{definizione}

\section{Implementazione in Swift}

\subsection{Struttura Defmodule}

Port fedele di \texttt{struct defmodule} (moduldef.h linee 138--145):

\begin{lstlisting}[language=Swift]
/// Defmodule - modulo CLIPS
/// (ref: struct defmodule in moduldef.h linee 138-145)
public class Defmodule {
    public var header: ConstructHeader
    public var itemsArray: [DefmoduleItemHeader?] = []
    public var importList: PortItem?
    public var exportList: PortItem?
    public var visitedFlag: Bool = false
    public var next: Defmodule?
    
    public init(name: String, ppForm: String? = nil) {
        self.header = ConstructHeader(
            type: .defmodule,
            name: name,
            ppForm: ppForm
        )
    }
    
    public var name: String {
        return header.name
    }
}
\end{lstlisting}

\subsection{Port Item per Import/Export}

\begin{lstlisting}[language=Swift]
/// Port item per import/export
/// (ref: struct portItem in moduldef.h linee 147-153)
public class PortItem {
    public var moduleName: String
    public var constructType: String?  // nil = tutti i tipi
    public var constructName: String?  // nil = tutti i nomi
    public var next: PortItem?
    
    public init(
        moduleName: String,
        constructType: String? = nil,
        constructName: String? = nil
    ) {
        self.moduleName = moduleName
        self.constructType = constructType
        self.constructName = constructName
    }
}
\end{lstlisting}

\textbf{Semantica}:
\begin{itemize}
\item \texttt{constructType = nil}: import/export tutti i tipi
\item \texttt{constructName = nil}: import/export tutti i nomi
\item \texttt{next}: linked list di port items
\end{itemize}

\subsection{Focus Stack Implementation}

\begin{lstlisting}[language=Swift]
/// Module stack item per focus
/// (ref: struct moduleStackItem in moduldef.h linee 200-205)
public class ModuleStackItem {
    public var changeFlag: Bool = false
    public var theModule: Defmodule?
    public var next: ModuleStackItem?
    
    public init(module: Defmodule?, changeFlag: Bool = false) {
        self.theModule = module
        self.changeFlag = changeFlag
    }
}

// Environment extensions per focus stack
extension Environment {
    public func focusPush(module: Defmodule) {
        let stackItem = ModuleStackItem(
            module: module,
            changeFlag: true
        )
        stackItem.next = moduleStack
        moduleStack = stackItem
    }
    
    public func focusPop() -> Defmodule? {
        guard let top = moduleStack else { return nil }
        moduleStack = top.next
        return top.theModule
    }
    
    public func focusPeek() -> Defmodule? {
        return moduleStack?.theModule
    }
}
\end{lstlisting}

\section{Parsing di Defmodule}

\subsection{Sintassi CLIPS}

\begin{lstlisting}[language=CLIPS]
(defmodule <module-name>
  [(export <construct-type> <construct-name>*)]
  [(import <module-name> <construct-type> <construct-name>*)])
\end{lstlisting}

\subsection{Implementazione Parser}

In \texttt{evaluator.swift}:

\begin{lstlisting}[language=Swift]
if name == "defmodule" {
    var cur = node.argList
    guard let nameNode = cur else { return .boolean(false) }
    let nameVal = try eval(&env, nameNode)
    
    let moduleName: String
    switch nameVal {
    case .string(let s): moduleName = s
    case .symbol(let s): moduleName = s
    default: moduleName = "UNNAMED"
    }
    
    cur = nameNode.nextArg
    var exportList: PortItem? = nil
    var importList: PortItem? = nil
    
    // Parsing export/import clauses
    while let clause = cur {
        if clause.type == .fcall {
            let clauseName = (clause.value?.value as? String) ?? ""
            
            if clauseName == "export" {
                exportList = parseExportClause(clause, moduleName)
            } else if clauseName == "import" {
                importList = parseImportClause(clause)
            }
        }
        cur = clause.nextArg
    }
    
    // Crea modulo
    if let newModule = env.createDefmodule(
        name: moduleName,
        importList: importList,
        exportList: exportList
    ) {
        _ = env.setCurrentModule(newModule)
        return .symbol(moduleName)
    }
    
    return .boolean(false)
}
\end{lstlisting}

\subsection{Creazione Modulo}

\begin{lstlisting}[language=Swift]
extension Environment {
    public func createDefmodule(
        name: String,
        importList: PortItem? = nil,
        exportList: PortItem? = nil
    ) -> Defmodule? {
        // Verifica che non esista già
        if findDefmodule(name: name) != nil {
            print("[ERROR] Defmodule \(name) already exists")
            return nil
        }
        
        let newModule = Defmodule(
            name: name,
            ppForm: "(defmodule \(name))"
        )
        
        // Alloca array di item headers
        newModule.itemsArray = Array(
            repeating: nil,
            count: Int(numberOfModuleItems)
        )
        
        for i in 0..<Int(numberOfModuleItems) {
            let header = DefmoduleItemHeader()
            header.theModule = newModule
            newModule.itemsArray[i] = header
        }
        
        // Imposta import/export
        newModule.importList = importList
        newModule.exportList = exportList
        
        // Aggiungi alla lista globale
        if let last = lastDefmodule {
            last.next = newModule
        } else {
            listOfDefmodules = newModule
        }
        lastDefmodule = newModule
        
        return newModule
    }
}
\end{lstlisting}

\section{Comandi per Moduli}

\subsection{Comando focus}

\begin{lstlisting}[language=Swift]
/// (focus <module-name>+)
/// Imposta il focus su uno o più moduli
/// (ref: FocusCommand in modulbsc.c)
private func builtin_focus(
    _ env: inout Environment,
    _ args: [Value]
) throws -> Value {
    guard !args.isEmpty else {
        print("[ERROR] focus requires at least one argument")
        return .boolean(false)
    }
    
    // Push moduli nello stack
    for arg in args {
        let moduleName: String
        switch arg {
        case .symbol(let s): moduleName = s
        case .string(let s): moduleName = s
        default:
            print("[ERROR] focus arguments must be symbols")
            return .boolean(false)
        }
        
        guard let module = env.findDefmodule(name: moduleName) else {
            print("[ERROR] Unable to find defmodule \(moduleName)")
            return .boolean(false)
        }
        
        env.focusPush(module: module)
    }
    
    return .boolean(true)
}
\end{lstlisting}

\subsection{Semantica del Focus}

\begin{algorithm}[H]
\caption{Focus-Based Rule Selection}
\begin{algorithmic}[1]
\Function{SelectNextRule}{$A, \mathcal{F}$}
    \If{$\mathcal{F} = \emptyset$}
        \State \Return $\max_\sigma A$ \Comment{Strategia standard}
    \EndIf
    \For{$M$ in $\mathcal{F}$ from top to bottom}
        \State $A_M \gets \{(r, \theta) \in A \mid r \in M.\text{constructs}\}$
        \If{$A_M \neq \emptyset$}
            \State $r^* \gets \max_\sigma A_M$
            \State \Return $r^*$
        \Else
            \State $\mathcal{F}.\text{pop}(M)$ \Comment{Modulo esaurito}
        \EndIf
    \EndFor
    \State \Return $\max_\sigma A$ \Comment{Fallback a current module}
\EndFunction
\end{algorithmic}
\end{algorithm}

\textbf{Comportamento}:
\begin{enumerate}
\item Controlla attivazioni nel modulo top dello stack
\item Se presenti, esegue quella con priorità massima
\item Se assenti, fa pop e controlla modulo successivo
\item Se stack vuoto, usa modulo corrente standard
\end{enumerate}

\section{Import/Export Resolution}

\subsection{Algoritmo di Lookup}

Quando si cerca un costrutto $c$ nel modulo $M$:

\begin{algorithm}[H]
\caption{FindConstruct($M$, $name$, $type$)}
\begin{algorithmic}[1]
\State $c \gets M.\text{constructs}[name, type]$
\If{$c \neq \text{null}$}
    \State \Return $c$ \Comment{Definito localmente}
\EndIf
\For{each $(M', t, n)$ in $M.\text{imports}$}
    \If{$(t = \text{null} \lor t = type) \land (n = \text{null} \lor n = name)$}
        \State $c \gets M'.\text{constructs}[name, type]$
        \If{$c \neq \text{null}$}
            \State \Return $c$ \Comment{Importato}
        \EndIf
    \EndIf
\EndFor
\State \Return $\text{null}$ \Comment{Non trovato}
\end{algorithmic}
\end{algorithm}

\subsection{Validazione Export}

Prima di permettere import, verifica export:

\begin{algorithm}[H]
\caption{ValidateImport($M_{\text{from}}$, $M_{\text{to}}$, $c$)}
\begin{algorithmic}[1]
\State $exports \gets M_{\text{from}}.\text{exports}$
\If{$exports = \text{null}$}
    \State \Return $\text{true}$ \Comment{Nessuna restrizione}
\EndIf
\For{each $(t, n)$ in $exports$}
    \If{$(t = \text{null} \lor t = c.type) \land (n = \text{null} \lor n = c.name)$}
        \State \Return $\text{true}$ \Comment{Esplicitamente esportato}
    \EndIf
\EndFor
\State \Return $\text{false}$ \Comment{Non esportato}
\end{algorithmic}
\end{algorithm}

\section{Implementazione in SLIPS}

\subsection{Gestione Moduli nell'Environment}

\begin{lstlisting}[language=Swift]
extension Environment {
    // Lista globale di moduli
    internal var _listOfDefmodules: Defmodule? = nil
    internal var _currentModule: Defmodule? = nil
    internal var _lastDefmodule: Defmodule? = nil
    
    // Stack di focus
    internal var _moduleStack: ModuleStackItem? = nil
    
    // Registry tipi costrutti
    internal var _listOfModuleItems: ModuleItem? = nil
    internal var _numberOfModuleItems: UInt = 0
    
    // Computed properties per accesso sicuro
    public var currentModule: Defmodule? {
        get { return _currentModule }
        set { _currentModule = newValue }
    }
    
    public var moduleStack: ModuleStackItem? {
        get { return _moduleStack }
        set { _moduleStack = newValue }
    }
}
\end{lstlisting}

\subsection{Inizializzazione Sistema Moduli}

\begin{lstlisting}[language=Swift]
extension Environment {
    /// (ref: InitializeDefmodules in moduldef.c linee 183-200)
    public func initializeModules() {
        // Prima registra tipi di item
        registerModuleItems()
        
        // Poi crea modulo MAIN di default
        createMainModule()
    }
    
    private func registerModuleItems() {
        _ = registerModuleItem(name: "defrule")
        _ = registerModuleItem(name: "deftemplate")
        _ = registerModuleItem(name: "deffacts")
    }
    
    private func createMainModule() {
        let mainModule = Defmodule(
            name: "MAIN",
            ppForm: "(defmodule MAIN)"
        )
        
        // Alloca item headers
        mainModule.itemsArray = Array(
            repeating: nil,
            count: Int(numberOfModuleItems)
        )
        
        for i in 0..<Int(numberOfModuleItems) {
            let header = DefmoduleItemHeader()
            header.theModule = mainModule
            mainModule.itemsArray[i] = header
        }
        
        // Imposta come corrente
        listOfDefmodules = mainModule
        lastDefmodule = mainModule
        currentModule = mainModule
    }
}
\end{lstlisting}

\section{Esempi d'Uso}

\subsection{Esempio 1: Sistema Multi-Modulo}

\begin{lstlisting}[language=CLIPS]
;; Modulo per gestione dati
(defmodule DATA-MANAGEMENT
  (export deftemplate data-record)
  (export deftemplate validation-result))

(deftemplate data-record
  (slot id (type INTEGER))
  (slot value (type NUMBER))
  (slot timestamp))

(deftemplate validation-result
  (slot record-id)
  (slot status (allowed-symbols valid invalid)))

;; Modulo per elaborazione
(defmodule DATA-PROCESSING
  (import DATA-MANAGEMENT deftemplate data-record)
  (import DATA-MANAGEMENT deftemplate validation-result))

(defrule validate-data
  (data-record (id ?id) (value ?v&:(< ?v 0)))
  =>
  (assert (validation-result 
    (record-id ?id) 
    (status invalid))))

;; Modulo per reporting
(defmodule REPORTING
  (import DATA-MANAGEMENT deftemplate validation-result))

(defrule report-invalid
  (validation-result (record-id ?id) (status invalid))
  =>
  (printout t "Record " ?id " è invalido" crlf))
\end{lstlisting}

\subsection{Esempio 2: Focus Dinamico}

\begin{lstlisting}[language=CLIPS]
;; Setup iniziale
(defmodule MAIN)

(defmodule INITIALIZATION
  (export defrule setup-complete))

(defmodule PROCESSING)

(defmodule CLEANUP)

;; In MAIN: orchestra il flusso
(defrule start
  =>
  (focus INITIALIZATION))

;; In INITIALIZATION
(defrule setup-complete
  ?f <- (initialized)
  =>
  (retract ?f)
  (focus PROCESSING))

;; In PROCESSING  
(defrule processing-done
  (all-processed)
  =>
  (focus CLEANUP))
\end{lstlisting}

\textbf{Flusso di esecuzione}:
\begin{equation}
\text{MAIN} \xrightarrow{\text{focus}} \text{INITIALIZATION} \xrightarrow{\text{focus}} \text{PROCESSING} \xrightarrow{\text{focus}} \text{CLEANUP}
\end{equation}

\section{Test del Sistema Moduli}

\subsection{Test Suite}

SLIPS include 22 test specifici per moduli:

\begin{lstlisting}[language=Swift]
final class ModulesTests: XCTestCase {
    // Basic module management (6 test)
    func testMainModuleCreatedByDefault()
    func testGetCurrentModule()
    func testCreateNewModule()
    func testCannotCreateDuplicateModule()
    func testSetCurrentModule()
    func testListDefmodules()
    
    // Focus stack (5 test)
    func testFocusStackInitiallyEmpty()
    func testFocusPushAndPop()
    func testFocusStackMultiplePushes()
    func testGetCurrentFocusModule()
    func testModuleItemsRegistered()
    
    // Defmodule parsing (3 test)
    func testDefmoduleParsing()
    func testDefmoduleWithExport()
    func testDefmoduleWithImport()
    
    // Commands (7 test)
    func testFocusCommand()
    func testFocusMultipleModules()
    func testGetCurrentModuleCommand()
    func testSetCurrentModuleCommand()
    func testListDefmodulesCommand()
    func testGetDefmoduleListCommand()
    func testAgendaWithModule()
    
    // Integration (1 test)
    func testModuleWithRules()
}
\end{lstlisting}

\subsection{Test Case Significativo}

\begin{lstlisting}[language=Swift]
func testFocusMultipleModules() {
    _ = CLIPS.createEnvironment()
    
    // Crea moduli
    _ = CLIPS.eval(expr: "(defmodule MOD-A)")
    _ = CLIPS.eval(expr: "(defmodule MOD-B)")
    _ = CLIPS.eval(expr: "(defmodule MOD-C)")
    
    // Focus su più moduli (sintassi CLIPS)
    _ = CLIPS.eval(expr: "(focus MOD-A MOD-B MOD-C)")
    
    guard let env = CLIPS.currentEnvironment else {
        XCTFail("Environment non disponibile")
        return
    }
    
    // Verifica: top dello stack è MOD-C (ultimo argomento)
    XCTAssertEqual(env.focusPeek()?.name, "MOD-C")
    
    // Pop sequenziale dovrebbe dare C, B, A
    XCTAssertEqual(env.focusPop()?.name, "MOD-C")
    XCTAssertEqual(env.focusPop()?.name, "MOD-B")
    XCTAssertEqual(env.focusPop()?.name, "MOD-A")
    XCTAssertTrue(env.isFocusStackEmpty())
}
\end{lstlisting}

\textbf{Coverage}: 100\% (22/22 test pass)

\section{Integrazione con RETE}

\subsection{Module-Aware Activation}

\begin{osservazione}[Stato Implementazione]
Attualmente, le attivazioni NON contengono informazione sul modulo. Implementazione futura:

\begin{lstlisting}[language=Swift]
public struct Activation {
    public var priority: Int
    public var ruleName: String
    public var bindings: [String: Value]
    public var factIDs: Set<Int>
    public var module: Defmodule?  // <-- DA AGGIUNGERE
}
\end{lstlisting}

Con questo, l'agenda potrebbe filtrare per modulo focus.
\end{osservazione}

\subsection{Costruzione Rete per Modulo}

Ogni modulo mantiene la propria rete RETE:

\begin{lstlisting}[language=Swift]
extension Defmodule {
    var alphaNodes: [String: AlphaNodeClass] = [:]
    var productionNodes: [String: ProductionNode] = [:]
}
\end{lstlisting}

Quando si cambia modulo corrente, le regole vengono aggiunte alla rete di quel modulo.

\section{Performance del Sistema Moduli}

\subsection{Complessità Operazioni}

\begin{table}[h]
\centering
\begin{tabular}{@{}lll@{}}
\toprule
\textbf{Operazione} & \textbf{Complessità} & \textbf{Note} \\
\midrule
Crea modulo & $O(1)$ & Allocazione costante \\
Find modulo & $O(m)$ & Linear search, $m$ = \# moduli \\
Set current & $O(1)$ & Assegnamento puntatore \\
Focus push & $O(1)$ & Linked list prepend \\
Focus pop & $O(1)$ & Linked list head remove \\
Import lookup & $O(k \cdot m)$ & $k$ = import items \\
\bottomrule
\end{tabular}
\caption{Complessità operazioni moduli}
\label{tab:module_complexity}
\end{table}

\subsection{Overhead Focus}

Focus introduce overhead minimo:
\begin{itemize}
\item Push/pop: 2--3 istruzioni
\item Lookup modulo: $O(1)$ se cached
\item Nessun impatto su pattern matching
\end{itemize}

\begin{successbox}[Performance]
In sistemi realistici con < 20 moduli, l'overhead è trascurabile (< 1\% tempo totale).
\end{successbox}

\section{Best Practices per Moduli}

\subsection{Organizzazione Raccomandata}

\begin{enumerate}
\item \textbf{Modulo MAIN}:
   \begin{itemize}
   \item Orchestrazione generale
   \item Import da tutti i moduli necessari
   \item Regole di controllo flusso
   \end{itemize}

\item \textbf{Moduli Dominio}:
   \begin{itemize}
   \item Un modulo per area funzionale
   \item Export solo interfacce pubbliche
   \item Incapsulamento dettagli implementativi
   \end{itemize}

\item \textbf{Moduli Utility}:
   \begin{itemize}
   \item Funzioni e template riutilizzabili
   \item Senza stato globale
   \item Export selettivo
   \end{itemize}
\end{enumerate}

\subsection{Anti-Pattern da Evitare}

\begin{warningbox}[Anti-Pattern Comuni]
\begin{enumerate}
\item \textbf{Moduli Monolitici}
   \begin{itemize}
   \item Sintomo: modulo con 100+ regole
   \item Problema: difficile da mantenere
   \item Soluzione: spezzare in sotto-moduli
   \end{itemize}

\item \textbf{Import Circolari}
   \begin{itemize}
   \item Sintomo: $M_1$ importa da $M_2$ che importa da $M_1$
   \item Problema: accoppiamento stretto
   \item Soluzione: estrarre modulo comune
   \end{itemize}

\item \textbf{Export Indiscriminato}
   \begin{itemize}
   \item Sintomo: \texttt{(export ?ALL)}
   \item Problema: viola incapsulamento
   \item Soluzione: export selettivo
   \end{itemize}
\end{enumerate}
\end{warningbox}

\section{Caso di Studio: Sistema Esperto Medico}

\subsection{Architettura Moduli}

\begin{lstlisting}[language=CLIPS]
;; Modulo: Dati Paziente
(defmodule PATIENT-DATA
  (export deftemplate patient)
  (export deftemplate symptom)
  (export deftemplate test-result))

(deftemplate patient
  (slot id (type INTEGER))
  (slot name (type STRING))
  (slot age (type INTEGER)))

(deftemplate symptom
  (slot patient-id)
  (slot description)
  (slot severity (type INTEGER) (range 1 10)))

;; Modulo: Diagnosi
(defmodule DIAGNOSIS
  (import PATIENT-DATA deftemplate patient)
  (import PATIENT-DATA deftemplate symptom)
  (export deftemplate diagnosis))

(deftemplate diagnosis
  (slot patient-id)
  (slot condition)
  (slot confidence (type FLOAT)))

(defrule diagnose-flu
  (patient (id ?pid) (age ?age&:(> ?age 5)))
  (symptom (patient-id ?pid) (description "fever"))
  (symptom (patient-id ?pid) (description "cough"))
  =>
  (assert (diagnosis 
    (patient-id ?pid)
    (condition "influenza")
    (confidence 0.85))))

;; Modulo: Terapia
(defmodule TREATMENT
  (import DIAGNOSIS deftemplate diagnosis)
  (export deftemplate prescription))

(deftemplate prescription
  (slot patient-id)
  (slot medication)
  (slot dosage))

(defrule prescribe-antiviral
  (diagnosis (patient-id ?pid) (condition "influenza") (confidence ?c&:(> ?c 0.8)))
  =>
  (assert (prescription
    (patient-id ?pid)
    (medication "oseltamivir")
    (dosage "75mg BID x 5 days"))))

;; Orchestrazione in MAIN
(defmodule MAIN
  (import PATIENT-DATA deftemplate patient)
  (import PATIENT-DATA deftemplate symptom)
  (import DIAGNOSIS deftemplate diagnosis)
  (import TREATMENT deftemplate prescription))

(defrule start-diagnosis
  (patient (id ?pid))
  (symptom (patient-id ?pid))
  =>
  (focus DIAGNOSIS))

(defrule start-treatment
  (diagnosis (patient-id ?pid))
  =>
  (focus TREATMENT))
\end{lstlisting}

\subsection{Flusso di Esecuzione}

\begin{enumerate}
\item Assert fatti paziente e sintomi in MAIN
\item Regola \texttt{start-diagnosis} imposta focus su DIAGNOSIS
\item DIAGNOSIS esegue regole di diagnosi
\item Nuova diagnosi attiva \texttt{start-treatment}
\item Focus passa a TREATMENT
\item TREATMENT genera prescrizioni
\item Focus ritorna a MAIN
\end{enumerate}

\begin{figure}[h]
\centering
\begin{tikzpicture}[
  node distance=2.5cm,
  module/.style={rectangle, draw, thick, minimum width=3cm, minimum height=1.2cm, align=center},
  arrow/.style={->, >=stealth, thick}
]

\node[module, fill=blue!20] (main) {MAIN\\Orchestrazione};
\node[module, fill=green!20, below left=of main] (data) {PATIENT-DATA\\Dati};
\node[module, fill=yellow!20, below=of main] (diag) {DIAGNOSIS\\Diagnosi};
\node[module, fill=red!20, below right=of main] (treat) {TREATMENT\\Terapia};

\draw[arrow, dashed] (main) -- node[left, font=\small] {import} (data);
\draw[arrow, dashed] (diag) -- node[left, font=\small] {import} (data);
\draw[arrow, dashed] (treat) -- node[above, font=\small, sloped] {import} (diag);

\draw[arrow, bend right] (main) to node[left, font=\small] {focus} (diag);
\draw[arrow, bend left] (diag) to node[right, font=\small] {focus} (treat);
\draw[arrow, bend left=60] (treat) to node[right, font=\small] {return} (main);

\end{tikzpicture}
\caption{Architettura moduli sistema medico}
\label{fig:medical_modules}
\end{figure}

\section{Validazione Formale}

\subsection{Invarianti del Sistema Moduli}

\begin{proposizione}[Unicità Nomi Moduli]
\begin{equation}
\forall M_1, M_2 \in \text{Modules}: M_1 \neq M_2 \Rightarrow M_1.\text{name} \neq M_2.\text{name}
\end{equation}
\end{proposizione}

\begin{proof}
Per costruzione: \texttt{createDefmodule} verifica con \texttt{findDefmodule} prima di creare.
\end{proof}

\begin{proposizione}[Consistenza Stack]
Il focus stack mantiene invariante LIFO:
\begin{equation}
\text{pop}(\text{push}(\mathcal{F}, M)) = (\mathcal{F}, M)
\end{equation}
\end{proposizione}

\begin{proposizione}[Modulo Corrente Valido]
\begin{equation}
\text{currentModule} \neq \text{nil} \land \text{currentModule} \in \text{listOfDefmodules}
\end{equation}

dopo \texttt{initializeModules()}.
\end{proposizione}

\subsection{Verifica Proprietà}

Test di proprietà con QuickCheck-style:

\begin{lstlisting}[language=Swift]
func testFocusStackProperty() {
    // Proprietà: push(M); pop() deve dare M
    for _ in 1...100 {
        let env = CLIPS.createEnvironment()
        let moduleName = randomString()
        _ = env.createDefmodule(name: moduleName)
        
        guard let module = env.findDefmodule(name: moduleName) else {
            XCTFail()
            return
        }
        
        env.focusPush(module: module)
        let popped = env.focusPop()
        
        XCTAssertEqual(popped?.name, moduleName)
    }
}
\end{lstlisting}

\section{Confronto con CLIPS C}

\subsection{Equivalenza Comportamentale}

Test di equivalenza con CLIPS C 6.4.2:

\begin{table}[h]
\centering
\begin{tabular}{@{}lcc@{}}
\toprule
\textbf{Funzionalità} & \textbf{CLIPS C} & \textbf{SLIPS} \\
\midrule
Defmodule parsing & ✓ & ✓ \\
Import/export & ✓ & ✓ \\
Focus stack & ✓ & ✓ \\
get-current-module & ✓ & ✓ \\
set-current-module & ✓ & ✓ \\
list-defmodules & ✓ & ✓ \\
Module-aware agenda & ✓ & ⏳ Parziale \\
\bottomrule
\end{tabular}
\caption{Copertura funzionalità moduli}
\label{tab:module_coverage}
\end{table}

\subsection{Differenze Minori}

\begin{itemize}
\item \textbf{Module-aware agenda}: SLIPS accetta parametro ma non filtra ancora
\item \textbf{Binary save/load}: Non implementato in SLIPS 1.0
\item \textbf{Module callbacks}: Semplificati in SLIPS
\end{itemize}

\section{Conclusioni del Capitolo}

Il sistema di moduli di SLIPS:

\begin{itemize}
\item È una traduzione fedele di CLIPS C (moduldef.h/c)
\item Supporta defmodule, import/export, focus stack
\item Ha 22 test con 100\% pass rate
\item Permette organizzazione scalabile di grandi KB
\item Facilita riuso e manutenzione
\end{itemize}

\textbf{Stato}: 95\% completo (module-aware agenda rimanente è feature avanzata).

\begin{successbox}[Achievement]
Con il sistema di moduli, SLIPS raggiunge il \textbf{70\% di copertura CLIPS 6.4.2}, posizionandosi come implementazione production-ready per la maggior parte dei casi d'uso.
\end{successbox}


% Capitolo 21: Pattern Matching Avanzato in SLIPS

\chapter{Pattern Matching Avanzato}
\label{cap:slips_pattern_matching}

\section{Introduzione}

SLIPS implementa tutte le funzionalità avanzate di pattern matching di CLIPS, con particolare attenzione a multifield e constraint complessi.

\section{Constraint System}

\begin{lstlisting}[language=Swift]
public enum Constraint {
    case equals(Value)
    case notEquals(Value)
    case variable(String)
    case multifieldVariable(String)
    case predicate(PredicateFunction)
    case conjunction([Constraint])
    case disjunction([Constraint])
    case negation(Constraint)
    
    func evaluate(_ value: Value, bindings: [String: Value]) -> BindingResult {
        switch self {
        case .equals(let expected):
            return value == expected ? .success([:]) : .failure
            
        case .variable(let name):
            if let bound = bindings[name] {
                return value == bound ? .success([:]) : .failure
            }
            return .success([name: value])
            
        case .predicate(let fn):
            return fn(value, bindings) ? .success([:]) : .failure
            
        case .conjunction(let constraints):
            return evaluateConjunction(constraints, value, bindings)
            
        case .disjunction(let constraints):
            return evaluateDisjunction(constraints, value, bindings)
            
        case .negation(let inner):
            let result = inner.evaluate(value, bindings: bindings)
            return result.isSuccess ? .failure : .success([:])
            
        default:
            return .failure
        }
    }
}

public enum BindingResult {
    case success([String: Value])
    case failure
    
    var isSuccess: Bool {
        if case .success = self { return true }
        return false
    }
}
\end{lstlisting}

\section{Multifield Matching}

\subsection{Multifield Pattern}

\begin{lstlisting}[language=Swift]
struct MultifieldPattern {
    var segments: [Segment]
    
    enum Segment {
        case single(Constraint)
        case multifield(String?)  // Variable name or anonymous
    }
    
    func match(_ values: [Value], bindings: [String: Value]) -> [BindingResult] {
        return matchSegments(segments, values: values, bindings: bindings)
    }
    
    private func matchSegments(
        _ segments: [Segment],
        values: [Value],
        bindings: [String: Value],
        offset: Int = 0
    ) -> [BindingResult] {
        guard !segments.isEmpty else {
            return offset == values.count ? [.success(bindings)] : []
        }
        
        let first = segments.first!
        let rest = Array(segments.dropFirst())
        
        switch first {
        case .single(let constraint):
            guard offset < values.count else { return [] }
            let result = constraint.evaluate(values[offset], bindings: bindings)
            guard case .success(let newBindings) = result else { return [] }
            var combined = bindings
            combined.merge(newBindings) { $1 }
            return matchSegments(rest, values: values, bindings: combined, offset: offset + 1)
            
        case .multifield(let varName):
            // Try all possible lengths for multifield
            var results: [BindingResult] = []
            let minRest = minimumLength(rest)
            let maxLength = values.count - offset - minRest
            
            for length in 0...maxLength {
                let segment = Array(values[offset..<(offset + length)])
                var newBindings = bindings
                if let name = varName {
                    newBindings[name] = .multifield(segment)
                }
                results.append(contentsOf: matchSegments(
                    rest,
                    values: values,
                    bindings: newBindings,
                    offset: offset + length
                ))
            }
            
            return results
        }
    }
}
\end{lstlisting}

\subsection{Example Usage}

\begin{lstlisting}[language=CLIPS]
;; Pattern: (lista $?start 10 $?end)
;; Fact: (lista 1 2 10 3 4)
\end{lstlisting}

\begin{lstlisting}[language=Swift]
let pattern = MultifieldPattern(segments: [
    .multifield("start"),
    .single(.equals(.integer(10))),
    .multifield("end")
])

let fact = [.integer(1), .integer(2), .integer(10), .integer(3), .integer(4)]

let matches = pattern.match(fact, bindings: [:])
// Result: [
//   .success(["start": .multifield([1, 2]), "end": .multifield([3, 4])])
// ]
\end{lstlisting}

\section{Unification}

\begin{lstlisting}[language=Swift]
class Unifier {
    func unify(
        pattern: Pattern,
        fact: Fact,
        bindings: [String: Value] = [:]
    ) -> BindingResult {
        var currentBindings = bindings
        
        for (slotName, constraint) in pattern.constraints {
            guard let factValue = fact.slots[slotName] else {
                return .failure  // Missing slot
            }
            
            let result = constraint.evaluate(factValue, bindings: currentBindings)
            
            guard case .success(let newBindings) = result else {
                return .failure
            }
            
            // Merge bindings
            for (key, value) in newBindings {
                if let existing = currentBindings[key] {
                    if existing != value {
                        return .failure  // Conflict
                    }
                } else {
                    currentBindings[key] = value
                }
            }
        }
        
        return .success(currentBindings)
    }
}
\end{lstlisting}

\section{Test Nodes}

\subsection{Test CE}

\begin{lstlisting}[language=CLIPS]
(defrule example
  (a ?x)
  (b ?y)
  (test (> ?x ?y))  ; Inter-element test
  =>
  ...)
\end{lstlisting}

\begin{lstlisting}[language=Swift]
class TestNode: ReteNode {
    let testExpression: Expression
    
    func activate(token: Token) {
        let result = evaluate(testExpression, bindings: token.bindings)
        
        if case .symbol("TRUE") = result {
            // Pass through unchanged
            for child in children {
                child.activate(token: token)
            }
        }
        // Else filter out
    }
}
\end{lstlisting}

\section{Function Calls in Patterns}

\begin{lstlisting}[language=CLIPS]
(defrule check-range
  (value ?v&:(numberp ?v)&:(> ?v 10)&:(< ?v 100))
  =>
  ...)
\end{lstlisting}

\begin{lstlisting}[language=Swift]
typealias PredicateFunction = (Value, [String: Value]) -> Bool

let constraint = Constraint.conjunction([
    .predicate { value, _ in
        if case .integer = value { return true }
        if case .float = value { return true }
        return false
    },
    .predicate { value, _ in
        guard case .integer(let i) = value else { return false }
        return i > 10
    },
    .predicate { value, _ in
        guard case .integer(let i) = value else { return false }
        return i < 100
    }
])
\end{lstlisting}

\section{Exists and Forall}

\subsection{Exists}

\begin{lstlisting}[language=CLIPS]
(defrule has-order
  (exists (order (status pending)))
  =>
  (printout t "Has pending orders" crlf))
\end{lstlisting}

\begin{lstlisting}[language=Swift]
class ExistsNode: ReteNode {
    private var counters: [Token: Int] = [:]
    
    func leftActivate(token: Token) {
        counters[token] = 0
        checkAndPropagate(token)
    }
    
    func rightActivate(fact: Fact) {
        for token in leftParent!.tokens {
            if testPass(token, fact) {
                let wasZero = (counters[token] == 0)
                counters[token]! += 1
                
                if wasZero {
                    // Now exists - propagate
                    propagate(token)
                }
            }
        }
    }
    
    func rightRetract(fact: Fact) {
        for token in leftParent!.tokens {
            if testPass(token, fact) {
                counters[token]! -= 1
                
                if counters[token] == 0 {
                    // No longer exists - remove
                    removeFromChildren(token)
                }
            }
        }
    }
}
\end{lstlisting}

\section{Pattern Optimization}

\subsection{Compile-Time Optimization}

\begin{lstlisting}[language=Swift]
class PatternOptimizer {
    func optimize(_ pattern: Pattern) -> Pattern {
        var optimized = pattern
        
        // 1. Constant folding
        optimized = foldConstants(optimized)
        
        // 2. Predicate simplification
        optimized = simplifyPredicates(optimized)
        
        // 3. Reorder constraints by selectivity
        optimized = reorderConstraints(optimized)
        
        return optimized
    }
    
    private func foldConstants(_ pattern: Pattern) -> Pattern {
        // Replace (test (> 10 5)) with (test TRUE)
        // ...
    }
    
    private func reorderConstraints(_ pattern: Pattern) -> Pattern {
        // Put most selective constraints first
        let sorted = pattern.constraints.sorted { c1, c2 in
            estimateSelectivity(c1) < estimateSelectivity(c2)
        }
        return Pattern(template: pattern.template, constraints: sorted)
    }
}
\end{lstlisting}

\section{Conclusioni del Capitolo}

\subsection{Punti Chiave}

\begin{enumerate}
\item \textbf{Constraint system} flessibile con enum
\item \textbf{Multifield matching} con backtracking
\item \textbf{Unification} preserva semantica CLIPS
\item \textbf{Test nodes} per constraint inter-elemento
\item \textbf{Exists/forall} con counter-based logic
\end{enumerate}

\subsection{Fine Parte IV}

Con questo si conclude la Parte IV sull'implementazione SLIPS. La Parte V copre sviluppo, performance e futuro.

\subsection{Letture Consigliate}

\begin{itemize}
\item CLIPS Source - \texttt{pattern.c}, \texttt{prcdrpsr.c}
\item Swift Pattern Matching
\end{itemize}

% Capitolo 22: Testing e Validazione in SLIPS

\chapter{Testing e Validazione}
\label{cap:testing}

\section{Filosofia del Testing}

\subsection{Test-Driven Translation}

La traduzione di SLIPS segue approccio test-driven:

\begin{enumerate}
\item \textbf{Scrivi test} basati su comportamento CLIPS C
\item \textbf{Traduci} modulo C in Swift
\item \textbf{Verifica} che test passino
\item \textbf{Refactor} mantenendo test verdi
\end{enumerate}

\textbf{Benefici}:
\begin{itemize}
\item Specifica comportamento atteso prima di implementare
\item Confidence nel refactoring
\item Documentazione eseguibile
\item Regression prevention
\end{itemize}

\subsection{Gerarchia di Test}

\begin{verbatim}
Test Pyramid (SLIPS):

         /\
        /  \     Unit Tests (60%)
       /____\    - Singole funzioni
      /      \   - Strutture dati
     /________\  - Algoritmi isolati
    /          \
   /   Integr.  \ Integration Tests (30%)
  /______________\- Flussi completi
 /                - Interazione moduli
/___Equivalence___\ Equivalence Tests (10%)
                    - vs CLIPS C output
\end{verbatim}

\section{Test Suite di SLIPS}

\subsection{Organizzazione Test}

\begin{verbatim}
Tests/SLIPSTests/
|-- Core Tests
|   |-- ScannerTests.swift          (Lexer/tokenizer)
|   |-- EvalTests.swift             (Expression evaluation)
|   |-- ConstructsTests.swift       (Deftemplate, defrule, deffacts)
|   +-- VariablesTests.swift        (Binding resolution)
|
|-- RETE Tests
|   |-- ReteAlphaTests.swift        (Alpha network)
|   |-- ReteJoinTests.swift         (Join operations)
|   |-- ReteBetaTests.swift         (Beta memory)
|   |-- ReteExplicitNodesTests.swift(Explicit nodes)
|   |-- RetePerformanceTests.swift  (Benchmarks)
|   +-- ...                         (10+ file)
|
|-- Rules Tests
|   |-- RuleEngineTests.swift       (Rule management)
|   |-- RuleJoinTests.swift         (Multi-pattern rules)
|   |-- RuleNotExistsTests.swift    (NOT CE)
|   |-- RuleExistsTests.swift       (EXISTS CE)
|   |-- RuleOrAndTests.swift        (OR/AND CE)
|   +-- ...                         (8+ file)
|
|-- Pattern Matching Tests
|   |-- MultifieldAdvancedTests.swift (Multifield $?x)
|   |-- TemplateConstraintsTests.swift(Constraints)
|   +-- PatternTests.swift          (Pattern syntax)
|
|-- Agenda Tests
|   |-- AgendaStrategyTests.swift   (4 strategie)
|   +-- SalienceTests.swift         (Priorita')
|
|-- Modules Tests
|   +-- ModulesTests.swift          (22 test)
|
|-- Router Tests
|   |-- RouterRegistryTests.swift   (I/O routing)
|   +-- RouterCallbackTests.swift   (Custom routers)
|
+-- Equivalence Tests
    +-- CLIPSEquivalenceTests.swift (Golden tests)
\end{verbatim}

\textbf{Totale}: 39 file, 91 test, 2004 LOC

\subsection{Distribuzione per Categoria}

\begin{table}[h]
\centering
\begin{tabular}{@{}lrrrr@{}}
\toprule
\textbf{Categoria} & \textbf{Test} & \textbf{Pass} & \textbf{Fail} & \textbf{Rate} \\
\midrule
Modules & 22 & 22 & 0 & 100\% \\
RETE & 15+ & 13 & 2 & 87\% \\
Rules & 12+ & 12 & 0 & 100\% \\
Multifield & 7 & 7 & 0 & 100\% \\
Agenda & 8 & 8 & 0 & 100\% \\
Templates & 7+ & 7 & 0 & 100\% \\
Core & 10+ & 10 & 0 & 100\% \\
Router & 5 & 5 & 0 & 100\% \\
Misc & 5 & 5 & 0 & 100\% \\
\midrule
\textbf{Totale} & \textbf{91} & \textbf{89} & \textbf{2} & \textbf{97.8\%} \\
\bottomrule
\end{tabular}
\caption{Distribuzione test per categoria}
\label{tab:test_distribution}
\end{table}

\section{Test Unitari}

\subsection{Esempio: Scanner Tests}

\begin{lstlisting}[language=Swift]
import XCTest
@testable import SLIPS

final class ScannerTests: XCTestCase {
    func testTokenizeInteger() {
        var env = Environment()
        RouterEnvData.setup(&env, inputString: "42")
        
        var token = Token(.STOP_TOKEN)
        Scanner.GetToken(&env, "test", &token)
        
        XCTAssertEqual(token.tknType, .INTEGER_TOKEN)
        XCTAssertEqual(token.intValue, 42)
    }
    
    func testTokenizeMultifieldVariable() {
        var env = Environment()
        RouterEnvData.setup(&env, inputString: "$?items")
        
        var token = Token(.STOP_TOKEN)
        Scanner.GetToken(&env, "test", &token)
        
        XCTAssertEqual(token.tknType, .MF_VARIABLE_TOKEN)
        XCTAssertEqual(token.text, "items")
    }
    
    func testTokenizeString() {
        var env = Environment()
        RouterEnvData.setup(&env, inputString: "\"hello world\"")
        
        var token = Token(.STOP_TOKEN)
        Scanner.GetToken(&env, "test", &token)
        
        XCTAssertEqual(token.tknType, .STRING_TOKEN)
        XCTAssertEqual(token.text, "hello world")
    }
}
\end{lstlisting}

\textbf{Pattern}: Test singola responsabilità, no dipendenze esterne.

\section{Test di Integrazione}

\subsection{Esempio: Rule Execution Flow}

\begin{lstlisting}[language=Swift]
final class RuleEngineTests: XCTestCase {
    func testCompleteRuleFlow() {
        // Setup environment
        var env = CLIPS.createEnvironment()
        
        // 1. Define template
        _ = CLIPS.eval(expr: """
        (deftemplate person
          (slot name (type STRING))
          (slot age (type INTEGER)))
        """)
        
        // 2. Define rule
        _ = CLIPS.eval(expr: """
        (defrule find-adult
          (person (name ?n) (age ?a&:(>= ?a 18)))
          =>
          (printout t ?n " e' maggiorenne" crlf))
        """)
        
        // 3. Assert facts
        _ = CLIPS.eval(expr: "(assert (person (name \"Mario\") (age 25)))")
        _ = CLIPS.eval(expr: "(assert (person (name \"Luigi\") (age 16)))")
        
        // 4. Verify agenda
        guard let env2 = CLIPS.currentEnvironment else {
            XCTFail()
            return
        }
        XCTAssertEqual(env2.agendaQueue.count, 1) // Solo Mario
        
        // 5. Run
        let fired = CLIPS.run(limit: nil)
        XCTAssertEqual(fired, 1)
        
        // 6. Verify side effects
        // (output capture con custom router)
    }
}
\end{lstlisting}

\section{Test di Equivalenza}

\subsection{Golden File Testing}

\begin{lstlisting}[language=Swift]
final class CLIPSEquivalenceTests: XCTestCase {
    func testAgainstGoldenFile() {
        // 1. Carica file .clp di test
        let clpPath = Bundle.module.path(
            forResource: "test_case_001",
            ofType: "clp"
        )!
        
        // 2. Carica file .out atteso (da CLIPS C)
        let goldenPath = Bundle.module.path(
            forResource: "test_case_001",
            ofType: "out"
        )!
        let expectedOutput = try! String(
            contentsOfFile: goldenPath
        )
        
        // 3. Esegui in SLIPS con output capture
        var actualOutput = ""
        var env = CLIPS.createEnvironment()
        _ = RouterRegistry.AddRouter(
            &env,
            "capture",
            100,
            query: { _, name in name == "t" },
            write: { _, _, s in actualOutput += s }
        )
        
        try! CLIPS.load(clpPath)
        _ = CLIPS.run(limit: nil)
        
        // 4. Confronta output
        XCTAssertEqual(
            normalizeOutput(actualOutput),
            normalizeOutput(expectedOutput),
            "Output differs from CLIPS C"
        )
    }
    
    func normalizeOutput(_ s: String) -> String {
        // Normalizza whitespace, ordine non deterministico, etc.
        return s.trimmingCharacters(in: .whitespacesAndNewlines)
    }
}
\end{lstlisting}

\subsection{Property-Based Testing}

\begin{lstlisting}[language=Swift]
import XCTest
@testable import SLIPS

final class PropertyTests: XCTestCase {
    func testMatchIdempotence() {
        // Proprieta': match(WM, r) = match(match(WM, r), r)
        for _ in 1...100 {
            let env = generateRandomEnvironment()
            let rule = generateRandomRule()
            
            let cs1 = computeConflictSet(env, [rule])
            let cs2 = computeConflictSet(env, [rule])
            
            XCTAssertEqual(cs1, cs2, "Match non idempotente")
        }
    }
    
    func testAssertRetractInverse() {
        // Proprieta': retract(assert(WM, f), f) = WM
        for _ in 1...100 {
            var env = CLIPS.createEnvironment()
            
            let factsBefore = env.facts.count
            
            let id = CLIPS.eval(expr: "(assert (test-fact))")
            guard case .int(let fid) = id else {
                XCTFail()
                continue
            }
            
            XCTAssertEqual(env.facts.count, factsBefore + 1)
            
            CLIPS.retract(id: Int(fid))
            
            guard let env2 = CLIPS.currentEnvironment else {
                XCTFail()
                continue
            }
            
            XCTAssertEqual(env2.facts.count, factsBefore)
        }
    }
}
\end{lstlisting}

\section{Test di Performance}

\subsection{Benchmark Suite}

\begin{lstlisting}[language=Swift]
import XCTest
@testable import SLIPS

final class RetePerformanceTests: XCTestCase {
    func testAssert1000Facts() {
        measure {
            var env = CLIPS.createEnvironment()
            env.useExplicitReteNodes = true
            
            _ = CLIPS.eval(expr: "(deftemplate item (slot id))")
            _ = CLIPS.eval(expr: "(defrule check (item (id ?i)) => (printout t ?i))")
            
            for i in 1...1000 {
                _ = CLIPS.eval(expr: "(assert (item (id \(i))))")
            }
        }
        
        // Metrics: average, std dev, min, max
    }
    
    func testJoin10kFacts() {
        measure {
            var env = CLIPS.createEnvironment()
            
            _ = CLIPS.eval(expr: "(deftemplate a (slot x))")
            _ = CLIPS.eval(expr: "(deftemplate b (slot x))")
            _ = CLIPS.eval(expr: "(defrule join (a (x ?v)) (b (x ?v)) => (printout t ?v))")
            
            for i in 1...10000 {
                _ = CLIPS.eval(expr: "(assert (a (x \(i))))")
                _ = CLIPS.eval(expr: "(assert (b (x \(i))))")
            }
        }
    }
}
\end{lstlisting}

\subsection{Profiling con Instruments}

Swift offre eccellente integrazione con Instruments:

\begin{enumerate}
\item \textbf{Time Profiler}: identifica hot paths
   \begin{itemize}
   \item Self time per funzione
   \item Call tree con percentuali
   \item Source-level annotation
   \end{itemize}

\item \textbf{Allocations}: traccia memoria
   \begin{itemize}
   \item Object allocations
   \item Retain/release events
   \item Memory leaks
   \end{itemize}

\item \textbf{Leaks}: rileva memory leaks
   \begin{itemize}
   \item Reference cycles
   \item Abandoned objects
   \end{itemize}
\end{enumerate}

\begin{infobox}[Strumentazione]
Eseguire con profiling:
\begin{verbatim}
swift test --enable-code-coverage
xcodebuild -scheme SLIPS -enableCodeCoverage YES test
\end{verbatim}
\end{infobox}

\section{Coverage Analysis}

\subsection{Line Coverage}

Obiettivo: > 85\% line coverage

\begin{table}[h]
\centering
\begin{tabular}{@{}lrrr@{}}
\toprule
\textbf{Modulo} & \textbf{Linee} & \textbf{Coperte} & \textbf{Coverage} \\
\midrule
Core/evaluator.swift & 528 & 475 & 90\% \\
Core/functions.swift & 948 & 820 & 87\% \\
Rete/BetaEngine.swift & 1050 & 890 & 85\% \\
Rete/NetworkBuilder.swift & 374 & 350 & 94\% \\
Agenda/Agenda.swift & 92 & 92 & 100\% \\
Core/Modules.swift & 363 & 363 & 100\% \\
\midrule
\textbf{Totale stimato} & \textbf{8046} & \textbf{~6800} & \textbf{~85\%} \\
\bottomrule
\end{tabular}
\caption{Coverage stimata per modulo}
\label{tab:coverage}
\end{table}

\subsection{Branch Coverage}

Coverage dei branch decisionali:

\begin{lstlisting}[language=Swift]
// Esempio di branch coverage
switch value {
case .int(let i):      // Branch 1: testato OK
    return Double(i)
case .float(let d):    // Branch 2: testato OK
    return d
case .string:          // Branch 3: testato OK
    throw TypeError()
case .symbol:          // Branch 4: testato NO
    throw TypeError()
// ... altri casi
}
\end{lstlisting}

Obiettivo: > 80\% branch coverage.

\section{Mutation Testing}

\subsection{Concetto}

Il mutation testing valuta la \textit{qualità} dei test:

\begin{enumerate}
\item Introduce mutazioni nel codice (bug artificiali)
\item Esegue test suite
\item Verifica che test falliscano (rilevano mutazione)
\end{enumerate}

\textbf{Mutation score}:
\begin{equation}
\text{Score} = \frac{\text{Mutazioni rilevate}}{\text{Mutazioni totali}}
\end{equation}

\subsection{Esempio di Mutazioni}

\begin{table}[h]
\centering
\small
\begin{tabular}{@{}lll@{}}
\toprule
\textbf{Tipo} & \textbf{Originale} & \textbf{Mutazione} \\
\midrule
Operatore & \texttt{if x > 0} & \texttt{if x >= 0} \\
Costante & \texttt{return 42} & \texttt{return 43} \\
Booleano & \texttt{if cond} & \texttt{if !cond} \\
Statement & \texttt{x = y + z} & \texttt{x = y - z} \\
Return & \texttt{return value} & \texttt{return nil} \\
\bottomrule
\end{tabular}
\caption{Tipi di mutazioni comuni}
\label{tab:mutations}
\end{table}

\section{Continuous Integration}

\subsection{GitHub Actions Workflow}

\begin{lstlisting}
name: CI

on: [push, pull_request]

jobs:
  test:
    runs-on: macos-latest
    steps:
      - uses: actions/checkout@v3
      
      - name: Setup Swift
        uses: swift-actions/setup-swift@v1
        with:
          swift-version: "6.2"
      
      - name: Build
        run: swift build -c release
      
      - name: Test
        run: swift test --enable-code-coverage
      
      - name: Coverage Report
        run: |
          xcrun llvm-cov export \
            -format=lcov \
            .build/debug/SLIPSPackageTests.xctest/Contents/MacOS/SLIPSPackageTests \
            -instr-profile .build/debug/codecov/default.profdata \
            > coverage.lcov
      
      - name: Upload Coverage
        uses: codecov/codecov-action@v3
\end{lstlisting}

\subsection{Quality Gates}

Gates che devono passare per merge:

\begin{itemize}
\item $\checkmark$ Build successful
\item $\checkmark$ Tutti i test passano (100\%)
\item $\checkmark$ Coverage > 80\%
\item $\checkmark$ No new warnings
\item $\checkmark$ Lint checks pass
\item $\checkmark$ Code review approved
\end{itemize}

\section{Test Failures Analysis}

\subsection{Test Correntemente Falliti}

\textbf{Test 1}: \texttt{ReteExplicitNodesTests.testJoinNodeWithMultiplePatterns}

\begin{lstlisting}[language=Swift]
func testJoinNodeWithMultiplePatterns() {
    // Setup: regola con 3 pattern
    _ = createEnv()
    _ = CLIPS.eval(expr: "(deftemplate node (slot id) (slot next))")
    _ = CLIPS.eval(expr: """
    (defrule chain
      (node (id ?a) (next ?b))
      (node (id ?b) (next ?c))
      (node (id ?c))
      =>
      (printout t "Chain: " ?a " -> " ?b " -> " ?c crlf))
    """)
    
    // Assert fatti
    _ = CLIPS.eval(expr: "(assert (node (id 1) (next 2)))")
    _ = CLIPS.eval(expr: "(assert (node (id 2) (next 3)))")
    _ = CLIPS.eval(expr: "(assert (node (id 3)))")
    
    // Atteso: 1 attivazione per catena 1->2->3
    guard let env = CLIPS.currentEnvironment else {
        XCTFail()
        return
    }
    
    XCTAssertGreaterThan(
        env.agendaQueue.count,
        0,
        "Dovrebbe esserci almeno un'attivazione"
    )
    // FALLISCE: agenda vuota (0 attivazioni)
}
\end{lstlisting}

\textbf{Causa}: Helper \texttt{isCompatible} in \texttt{DriveEngine.swift} è stub:

\begin{lstlisting}[language=Swift]
private static func isCompatible(...) -> Bool {
    // TODO: Implementare check completo con join tests
    return true  // Ottimistico - SBAGLIATO per casi complessi!
}
\end{lstlisting}

\textbf{Fix pianificato}: Implementare verifica completa compatibilità bindings.

\subsection{Root Cause Analysis}

Processo di analisi:

\begin{enumerate}
\item \textbf{Riprodurre}: Isolare test in environment minimale
\item \textbf{Debuggare}: Breakpoint e watch su variabili chiave
\item \textbf{Tracciare}: Abilitare \texttt{watchRete} per vedere propagazione
\item \textbf{Confrontare}: Eseguire stesso test in CLIPS C
\item \textbf{Identificare}: Pinpoint della divergenza
\item \textbf{Fixare}: Correggere mantenendo equivalenza
\end{enumerate}

\begin{lstlisting}[language=Swift]
// Debug session
var env = CLIPS.createEnvironment()
env.watchRete = true  // Abilita trace RETE

// Esegui test case
// Output mostrera' propagazione step-by-step
\end{lstlisting}

\section{Regression Testing}

\subsection{Test per Bug Fixes}

Ogni bug fixato ottiene un test di regressione:

\begin{lstlisting}[language=Swift]
// Issue #42: multifield binding non preservato in join
func testIssue42_MultifieldBindingInJoin() {
    var env = CLIPS.createEnvironment()
    
    _ = CLIPS.eval(expr: "(deftemplate item (multislot tags))")
    _ = CLIPS.eval(expr: """
    (defrule test
      (item (tags $?x))
      (item (tags $?x))  ; Stesso binding
      =>
      (printout t "Match: " $?x crlf))
    """)
    
    _ = CLIPS.eval(expr: "(assert (item (tags a b c)))")
    
    let fired = CLIPS.run(limit: nil)
    XCTAssertEqual(fired, 1, "Bug #42: multifield non matchato")
}
\end{lstlisting}

\subsection{Non-Regression Suite}

\begin{verbatim}
Tests/Regression/
|-- Issue_042_multifield.swift
|-- Issue_087_retract_cascade.swift
|-- Issue_103_salience_tie.swift
+-- ...
\end{verbatim}

\textbf{Politica}: Ogni PR deve includere regression test se fixa bug.

\section{Test-Driven Development Workflow}

\subsection{Ciclo Red-Green-Refactor}

\begin{enumerate}
\item \textbf{RED}: Scrivi test che fallisce
\begin{lstlisting}[language=Swift]
func testNewFeature() {
    let result = newFeature(input)
    XCTAssertEqual(result, expected) // FAIL
}
\end{lstlisting}

\item \textbf{GREEN}: Implementa minimo per passare test
\begin{lstlisting}[language=Swift]
func newFeature(_ input: Input) -> Output {
    return expected  // Hardcoded - ma test passa!
}
\end{lstlisting}

\item \textbf{REFACTOR}: Generalizza mantenendo test verdi
\begin{lstlisting}[language=Swift]
func newFeature(_ input: Input) -> Output {
    // Implementazione vera
    return compute(input)
}
\end{lstlisting}
\end{enumerate}

\subsection{Esempio Reale: Implementazione Moduli}

\textbf{Fase 1 - RED}: Scrivi test prima di implementare

\begin{lstlisting}[language=Swift]
func testDefmoduleParsing() {
    _ = CLIPS.createEnvironment()
    let result = CLIPS.eval(expr: "(defmodule TEST-MODULE)")
    
    // Test fallisce: defmodule non implementato
    guard let env = CLIPS.currentEnvironment else {
        XCTFail()
        return
    }
    
    let module = env.findDefmodule(name: "TEST-MODULE")
    XCTAssertNotNil(module)  // FAIL: nil
}
\end{lstlisting}

\textbf{Fase 2 - GREEN}: Implementa defmodule

\begin{lstlisting}[language=Swift]
// In evaluator.swift
if name == "defmodule" {
    // Parse name
    let moduleName = extractName(node)
    
    // Create module
    let module = env.createDefmodule(name: moduleName)
    
    return .symbol(moduleName)
}
\end{lstlisting}

Test ora passa! $\checkmark$

\textbf{Fase 3 - REFACTOR}: Aggiungi import/export

\begin{lstlisting}[language=Swift]
if name == "defmodule" {
    let moduleName = extractName(node)
    let imports = parseImports(node)    // NEW
    let exports = parseExports(node)    // NEW
    
    let module = env.createDefmodule(
        name: moduleName,
        importList: imports,             // NEW
        exportList: exports              // NEW
    )
    
    return .symbol(moduleName)
}
\end{lstlisting}

Test ancora verde! Aggiungi nuovi test per import/export.

\section{Metodologia di Validazione}

\subsection{Multi-Level Validation}

\begin{enumerate}
\item \textbf{Livello 1: Sintassi}
   \begin{itemize}
   \item Parsing corretto di costrutti CLIPS
   \item Gestione errori sintattici
   \end{itemize}

\item \textbf{Livello 2: Semantica}
   \begin{itemize}
   \item Type checking (template constraints)
   \item Binding consistency
   \item Scope resolution
   \end{itemize}

\item \textbf{Livello 3: Comportamento}
   \begin{itemize}
   \item Ordine firing rules
   \item Fatti asseriti/ritratti
   \item Side effects (I/O)
   \end{itemize}

\item \textbf{Livello 4: Performance}
   \begin{itemize}
   \item Tempi di esecuzione accettabili
   \item Uso memoria ragionevole
   \item Scalabilità verificata
   \end{itemize}
\end{enumerate}

\subsection{Acceptance Criteria}

Per considerare un modulo "completo":

\begin{itemize}
\item $\checkmark$ Tutti i test unitari passano
\item $\checkmark$ Test di integrazione passano
\item $\checkmark$ Almeno 1 golden test vs CLIPS C passa
\item $\checkmark$ Coverage > 80\%
\item $\checkmark$ No memory leaks rilevati
\item $\checkmark$ Performance entro 2x di CLIPS C
\item $\checkmark$ Documentazione completa
\end{itemize}

\section{Debugging Techniques}

\subsection{Watch System}

SLIPS eredita il sistema watch da CLIPS:

\begin{lstlisting}[language=Swift]
// Abilita watch
CLIPS.eval(expr: "(watch facts)")
CLIPS.eval(expr: "(watch rules)")
CLIPS.eval(expr: "(watch activations)")
CLIPS.eval(expr: "(watch rete)")

// Output:
// ==> f-1 (person (nome "Mario") (eta 25))
// ==> Activation 0: find-adult (salience 0)
// <== f-1 (person (nome "Mario") (eta 25))
\end{lstlisting}

\subsection{RETE Tracing}

Per debugging propagazione:

\begin{lstlisting}[language=Swift]
env.watchRete = true
env.watchReteProfile = true

// Output dettagliato:
// [RETE Assert] Propagating fact 1: (person ...)
// [RETE Assert]   Matched 1 alpha node(s)
// [RETE Assert]   Alpha 'person': memory size = 1
// [RETE Join] Attempting join: left=<token>, right=<fact-1>
// [RETE Join]   Join keys: {?n}
// [RETE Join]   Join SUCCESS
// [RETE Profile] Assert propagation: 0.15ms
\end{lstlisting}

\subsection{Breakpoint Debugging}

Con Xcode:

\begin{lstlisting}[language=Swift]
// Conditional breakpoint
func propagateAssert(fact: FactRec, env: inout Environment) {
    if env.watchRete {
        print("[RETE] Assert fact \(fact.id)")
        // BREAKPOINT QUI con condition: fact.id == 42
    }
    
    // ...
}
\end{lstlisting}

\section{Best Practices per Testing}

\subsection{Test Naming Convention}

\begin{lstlisting}[language=Swift]
// Pattern: test<What><Scenario>[Expected]
func testAssert_WhenFactValid_ShouldAddToWorkingMemory()
func testRetract_WhenFactNotExists_ShouldNotThrow()
func testJoin_WithEmptyLeftMemory_ShouldProduceNoTokens()
\end{lstlisting}

\subsection{Test Organization}

\begin{lstlisting}[language=Swift]
final class RuleEngineTests: XCTestCase {
    // MARK: - Setup
    override func setUp() {
        // Inizializzazione comune
    }
    
    // MARK: - Basic Functionality
    func testAddRule() { ... }
    func testFindRule() { ... }
    
    // MARK: - Edge Cases
    func testAddDuplicateRule() { ... }
    func testAddRuleWithInvalidPattern() { ... }
    
    // MARK: - Integration
    func testRuleWithTemplateConstraints() { ... }
    
    // MARK: - Performance
    func testAdd1000Rules() { ... }
}
\end{lstlisting}

\subsection{Assertion Messages}

\begin{lstlisting}[language=Swift]
// BAD: messaggio generico
XCTAssertEqual(result, 42)

// GOOD: messaggio descrittivo
XCTAssertEqual(
    result,
    42,
    "La regola dovrebbe generare esattamente 1 attivazione per il fatto (person (eta 25))"
)
\end{lstlisting}

\section{Statistiche Test Suite SLIPS}

\subsection{Metriche Quantitative}

\begin{table}[h]
\centering
\begin{tabular}{@{}lrl@{}}
\toprule
\textbf{Metrica} & \textbf{Valore} & \textbf{Commento} \\
\midrule
Test totali & 91 & Copertura estensiva \\
Test passanti & 89 & 97.8\% pass rate \\
Test falliti & 2 & DriveEngine helpers stub \\
Linee codice test & 2004 & Ratio 1:4 con codice \\
Tempo esecuzione & < 2 sec & Suite completa \\
File test & 39 & Ben organizzati \\
Assertions totali & ~500 & Media 5.5/test \\
\bottomrule
\end{tabular}
\caption{Metriche test suite SLIPS 1.0}
\label{tab:test_stats}
\end{table}

\subsection{Copertura Funzionalità CLIPS}

\begin{table}[h]
\centering
\begin{tabular}{@{}lcc@{}}
\toprule
\textbf{Funzionalità} & \textbf{Implementata} & \textbf{Testata} \\
\midrule
Deftemplate & $\checkmark$ & $\checkmark$ (7 test) \\
Defrule & $\checkmark$ & $\checkmark$ (12 test) \\
Deffacts & $\checkmark$ & $\checkmark$ (3 test) \\
Defmodule & $\checkmark$ & $\checkmark$ (22 test) \\
Assert/Retract & $\checkmark$ & $\checkmark$ (8 test) \\
Pattern matching (SF) & $\checkmark$ & $\checkmark$ (10 test) \\
Pattern matching (MF) & $\checkmark$ & $\checkmark$ (7 test) \\
NOT CE & $\checkmark$ & $\checkmark$ (5 test) \\
EXISTS CE & $\checkmark$ & $\checkmark$ (3 test) \\
OR CE & $\checkmark$ & $\checkmark$ (2 test) \\
Agenda strategies & $\checkmark$ & $\checkmark$ (8 test) \\
RETE propagation & $\checkmark$ & $\checkmark$ (10 test) \\
Module system & $\checkmark$ & $\checkmark$ (22 test) \\
\bottomrule
\end{tabular}
\caption{Copertura funzionalità con test}
\label{tab:feature_coverage}
\end{table}

\section{Conclusioni del Capitolo}

In questo capitolo abbiamo:

\begin{itemize}
\item Presentato la strategia di testing di SLIPS
\item Analizzato la suite di 91 test (97.8\% pass rate)
\item Mostrato esempi di test unitari, integrazione, equivalenza
\item Descritto tecniche di debugging e profiling
\item Illustrato workflow di continuous integration
\item Analizzato test failures e root causes
\end{itemize}

Il testing rigoroso è fondamentale per garantire:
\begin{itemize}
\item Equivalenza comportamentale con CLIPS C
\item Assenza di regressioni
\item Confidence nel refactoring
\item Qualità production-ready
\end{itemize}

\begin{successbox}[Quality Assurance]
Con 97.8\% test pass rate e coverage stimata > 85\%, SLIPS soddisfa standard industriali per software critico.
\end{successbox}



% PARTE V: SVILUPPO E MANUTENZIONE
\part{Guida allo Sviluppo}

% Capitolo 23: Estendere SLIPS

\chapter{Estendere SLIPS con Nuove Funzionalità}
\label{cap:estendere_slips}

\section{Introduzione}

SLIPS è progettato per essere estensibile. Questo capitolo mostra come aggiungere funzioni, tipi e funzionalità custom.

\section{User-Defined Functions}

\subsection{Registrazione Funzioni}

\begin{lstlisting}[language=Swift]
public typealias UserFunction = ([Value], Environment) throws -> Value

public class FunctionRegistry {
    private var functions: [String: UserFunction] = [:]
    
    public func register(
        name: String,
        function: @escaping UserFunction
    ) {
        functions[name] = function
    }
    
    public func call(
        name: String,
        args: [Value],
        env: Environment
    ) throws -> Value {
        guard let fn = functions[name] else {
            throw RuntimeError.undefinedFunction(name)
        }
        return try fn(args, env)
    }
}

// Uso
let env = Environment()
env.functions.register(name: "square") { args, _ in
    guard args.count == 1,
          case .integer(let n) = args[0] else {
        throw RuntimeError.invalidArguments
    }
    return .integer(n * n)
}
\end{lstlisting}

\subsection{Funzioni Swift Native}

\begin{lstlisting}[language=Swift]
extension Environment {
    public func registerSwiftFunction<T: Numeric>(
        _ name: String,
        _ fn: @escaping (T, T) -> T
    ) {
        functions.register(name: name) { args, _ in
            guard args.count == 2 else {
                throw RuntimeError.wrongArity(expected: 2, got: args.count)
            }
            
            let a = try Self.extractNumber(args[0]) as T
            let b = try Self.extractNumber(args[1]) as T
            
            return .integer(Int(fn(a, b)))
        }
    }
}

// Esempio
env.registerSwiftFunction("add", +)
env.registerSwiftFunction("multiply", *)
\end{lstlisting>

\section{Custom Value Types}

\subsection{External Values}

\begin{lstlisting}[language=Swift]
public struct ExternalValue: Hashable {
    public let type: String
    public let data: AnyHashable
    
    public init<T: Hashable>(type: String, data: T) {
        self.type = type
        self.data = AnyHashable(data)
    }
}

// Esempio: Date
extension Environment {
    func registerDateType() {
        functions.register(name: "create-date") { args, _ in
            guard args.count == 3,
                  case .integer(let y) = args[0],
                  case .integer(let m) = args[1],
                  case .integer(let d) = args[2] else {
                throw RuntimeError.invalidArguments
            }
            
            let date = DateComponents(year: y, month: m, day: d)
            let calendar = Calendar.current
            let realDate = calendar.date(from: date)!
            
            return .external(ExternalValue(type: "date", data: realDate))
        }
        
        functions.register(name: "date-year") { args, _ in
            guard args.count == 1,
                  case .external(let ext) = args[0],
                  ext.type == "date",
                  let date = ext.data.base as? Date else {
                throw RuntimeError.invalidArguments
            }
            
            let year = Calendar.current.component(.year, from: date)
            return .integer(year)
        }
    }
}
\end{lstlisting}

\section{Router Extensions}

\subsection{Custom Router}

\begin{lstlisting}[language=Swift]
public protocol Router {
    var name: String { get }
    func query(logicalName: String) -> Bool
    func print(_ string: String)
    func getChar() -> Character?
    func ungetChar(_ char: Character)
}

public class FileRouter: Router {
    public let name = "file"
    private let fileHandle: FileHandle
    
    public init(path: String) throws {
        guard let handle = FileHandle(forWritingAtPath: path) else {
            throw RouterError.cannotOpenFile(path)
        }
        self.fileHandle = handle
    }
    
    public func query(logicalName: String) -> Bool {
        return logicalName == "file"
    }
    
    public func print(_ string: String) {
        if let data = string.data(using: .utf8) {
            fileHandle.write(data)
        }
    }
    
    public func getChar() -> Character? {
        return nil  // Not supported for file output
    }
    
    public func ungetChar(_ char: Character) {
        // Not supported
    }
}

// Uso
let fileRouter = try FileRouter(path: "/tmp/output.txt")
env.routers.add(fileRouter)

// In CLIPS
// (printout file "Hello world" crlf)
\end{lstlisting}

\section{Pattern Extensions}

\subsection{Custom Predicates}

\begin{lstlisting}[language=Swift]
public class PredicateRegistry {
    private var predicates: [String: (Value) -> Bool] = [:]
    
    public func register(name: String, predicate: @escaping (Value) -> Bool) {
        predicates[name] = predicate
    }
    
    public func evaluate(name: String, value: Value) -> Bool {
        return predicates[name]?(value) ?? false
    }
}

// Esempio
env.predicates.register(name: "is-email") { value in
    guard case .string(let s) = value else { return false }
    return s.contains("@") && s.contains(".")
}

// Uso in pattern:
// (utente (email ?e&:(is-email ?e)))
\end{lstlisting>

\section{Agenda Hooks}

\subsection{Rule Firing Callbacks}

\begin{lstlisting}[language=Swift]
public typealias RuleFiringCallback = (Defrule, Token) -> Void

extension Environment {
    public func onRuleFiring(_ callback: @escaping RuleFiringCallback) {
        self.ruleFiringCallbacks.append(callback)
    }
    
    private func fireRule(_ activation: Activation) {
        // Notify observers
        for callback in ruleFiringCallbacks {
            callback(activation.rule, activation.token)
        }
        
        // Execute rule
        // ...
    }
}

// Uso
env.onRuleFiring { rule, token in
    print("Firing: \(rule.name) with bindings: \(token.bindings)")
}
\end{lstlisting>

\section{Module Plugins}

\subsection{Plugin Architecture}

\begin{lstlisting}[language=Swift]
public protocol SLIPSPlugin {
    var name: String { get }
    func initialize(environment: Environment)
    func cleanup(environment: Environment)
}

public class HTTPPlugin: SLIPSPlugin {
    public let name = "HTTP"
    
    public func initialize(environment: Environment) {
        environment.functions.register(name: "http-get") { args, _ in
            guard args.count == 1,
                  case .string(let url) = args[0] else {
                throw RuntimeError.invalidArguments
            }
            
            let data = try await URLSession.shared.data(from: URL(string: url)!)
            let string = String(data: data.0, encoding: .utf8)!
            
            return .string(string)
        }
    }
    
    public func cleanup(environment: Environment) {
        // Cleanup resources
    }
}

// Uso
let plugin = HTTPPlugin()
env.loadPlugin(plugin)
\end{lstlisting>

\section{Testing Extensions}

\subsection{Test Utilities}

\begin{lstlisting}[language=Swift]
public class SLIPSTestCase {
    let env: Environment
    
    public init() {
        env = Environment()
    }
    
    public func load(_ rules: String) throws {
        try env.loadString(rules)
    }
    
    public func assertFact(_ template: String, _ slots: [String: Value]) {
        env.assert(template: template, slots: slots)
    }
    
    public func run() {
        env.run()
    }
    
    public func assertFactExists(_ template: String) -> Bool {
        return env.factList.contains { $0.template.name == template }
    }
    
    public func assertRuleFired(_ ruleName: String) -> Bool {
        return env.firedRules.contains(ruleName)
    }
}

// Uso
let test = SLIPSTestCase()
try test.load("""
    (defrule test
      (trigger)
      =>
      (assert (result)))
    """)

test.assertFact("trigger", [:])
test.run()

XCTAssertTrue(test.assertFactExists("result"))
XCTAssertTrue(test.assertRuleFired("test"))
\end{lstlisting}

\section{Conclusioni del Capitolo}

\subsection{Punti Chiave}

\begin{enumerate}
\item SLIPS \textbf{estensibile} tramite UDF
\item \textbf{External values} per tipi custom
\item \textbf{Router} per I/O flessibile
\item \textbf{Callbacks} per observability
\item \textbf{Plugin architecture} per modularità
\end{enumerate}

\subsection{Letture Consigliate}

\begin{itemize}
\item CLIPS Advanced Programming Guide
\item Swift Package Manager
\end{itemize}

% Capitolo 24: Best Practices

\chapter{Best Practices per Sviluppo con SLIPS}
\label{cap:best_practices}

\section{Progettazione di Regole Efficienti}

\subsection{Principio della Specificità}

\begin{infobox}[Regola d'Oro]
Pattern più specifici riducono il conflict set e migliorano performance.
\end{infobox}

\textbf{Cattivo esempio}:
\begin{lstlisting}[language=CLIPS]
(defrule troppo-generica
  (persona)  ; Matcha TUTTE le persone!
  =>
  ...)
\end{lstlisting}

\textbf{Buon esempio}:
\begin{lstlisting}[language=CLIPS]
(defrule specifica
  (persona (eta ?e&:(>= ?e 18)) (citta "Roma"))  ; Molto selettiva
  =>
  ...)
\end{lstlisting}

\subsection{Ordinamento Pattern}

\textbf{Euristica}: Pattern più selettivi prima.

\begin{lstlisting}[language=CLIPS]
; BAD: pattern generico prima
(defrule bad-order
  (persona (nome ?n))           ; 10000 match
  (vip (nome ?n))               ; 10 match
  =>
  ...)

; GOOD: pattern selettivo prima
(defrule good-order
  (vip (nome ?n))               ; 10 match
  (persona (nome ?n))           ; Ridotto a 10 check
  =>
  ...)
\end{lstlisting}

\textbf{Risparmio}: da $10000 \times 10 = 100k$ a $10 \times 1 = 10$ confronti!

\subsection{Uso delle Costanti}

Costanti nei pattern attivano ottimizzazioni RETE:

\begin{lstlisting}[language=CLIPS]
; Senza costanti: alpha node generale
(defrule generic
  (item (type ?t) (value ?v))
  =>
  ...)

; Con costanti: alpha node specializzato
(defrule specific
  (item (type "premium") (value ?v&:(> ?v 1000)))
  =>
  ...)
\end{lstlisting}

Alpha node con costanti può usare hash index per match $O(1)$.

\section{Gestione della Working Memory}

\subsection{Minimizzare Fatti Ridondanti}

\textbf{Problema}: WM cresce senza controllo.

\begin{lstlisting}[language=CLIPS]
; BAD: accumula fatti
(defrule process-item
  (item ?i)
  =>
  (assert (processed ?i))
  (assert (timestamp (now)))   ; Nuovo fatto ogni volta!
  ...)
\end{lstlisting}

\textbf{Soluzione}: Retract fatti temporanei.

\begin{lstlisting}[language=CLIPS]
; GOOD: pulizia esplicita
(defrule process-item
  ?f <- (item ?i)
  =>
  (retract ?f)                  ; Rimuovi item processato
  (assert (processed ?i))
  ...)
\end{lstlisting}

\subsection{Fact Temporal Validity}

Per fatti temporanei, usa pattern:

\begin{lstlisting}[language=CLIPS]
(deftemplate event
  (slot type)
  (slot data)
  (slot timestamp)
  (slot ttl (default 100)))  ; Time-to-live

(defrule expire-events
  ?e <- (event (timestamp ?ts) (ttl ?ttl))
  (test (> (- (now) ?ts) ?ttl))
  =>
  (retract ?e))
\end{lstlisting}

\section{Modularizzazione e Riuso}

\subsection{Design Pattern: Utility Modules}

\begin{lstlisting}[language=CLIPS]
(defmodule STRING-UTILS
  (export deffunction starts-with)
  (export deffunction ends-with)
  (export deffunction contains))

(deffunction starts-with (?str ?prefix)
  (eq (sub-string 1 (str-length ?prefix) ?str) ?prefix))

(deffunction ends-with (?str ?suffix)
  (bind ?len (str-length ?str))
  (bind ?slen (str-length ?suffix))
  (eq (sub-string (- ?len ?slen -1) ?len ?str) ?suffix))

;; Altri moduli importano
(defmodule VALIDATION
  (import STRING-UTILS deffunction starts-with))
\end{lstlisting}

\subsection{Design Pattern: Pipeline}

Moduli organizzati in pipeline:

\begin{lstlisting}[language=CLIPS]
(defmodule INPUT-PROCESSING
  (export deftemplate cleaned-data))

(defmodule VALIDATION
  (import INPUT-PROCESSING deftemplate cleaned-data)
  (export deftemplate validated-data))

(defmodule ENRICHMENT
  (import VALIDATION deftemplate validated-data)
  (export deftemplate enriched-data))

(defmodule OUTPUT
  (import ENRICHMENT deftemplate enriched-data))

; Main orchestra il flusso
(defmodule MAIN)
(defrule start
  =>
  (focus INPUT-PROCESSING VALIDATION ENRICHMENT OUTPUT))
\end{lstlisting}

\section{Performance Optimization}

\subsection{Profiling-Guided Optimization}

\begin{enumerate}
\item \textbf{Measure}: Usa Instruments per identificare bottleneck
\item \textbf{Analyze}: Determina causa (algoritmo? struttura dati?)
\item \textbf{Optimize}: Intervento mirato
\item \textbf{Verify}: Conferma miglioramento senza breaking changes
\end{enumerate}

\subsection{Common Hotspots}

\begin{table}[h]
\centering
\begin{tabular}{@{}lll@{}}
\toprule
\textbf{Hotspot} & \textbf{Causa} & \textbf{Fix} \\
\midrule
Join operations & Pattern generici & Aggiungi costanti/constraints \\
Alpha matching & Linear scan fatti & Hash indexing (già implementato) \\
Beta memory growth & Join produttivo & Limita prodotto cartesiano \\
Agenda operations & Troppi tie & Usa salience \\
\bottomrule
\end{tabular}
\caption{Hotspot comuni e soluzioni}
\label{tab:hotspots}
\end{table}

\subsection{Salience Strategy}

Usa salience per controllo esplicito:

\begin{lstlisting}[language=CLIPS]
; Alta priorita': cleanup urgente
(defrule cleanup-critical-error
  (declare (salience 1000))
  (error (severity critical))
  =>
  (halt))

; Priorita' normale: processing
(defrule process-data
  (declare (salience 0))
  (data ?d)
  =>
  ...)

; Bassa priorita': logging
(defrule log-completion
  (declare (salience -1000))
  (all-done)
  =>
  (printout t "Completato" crlf))
\end{lstlisting}

\section{Memory Management}

\subsection{Evitare Memory Leaks}

In Swift con ARC, leaks tipicamente da:

\begin{enumerate}
\item \textbf{Reference Cycles}

\begin{lstlisting}[language=Swift]
// BAD: ciclo forte
class Node {
    var parent: Node?  // Strong reference
    var children: [Node] = []  // Strong references
}

// GOOD: weak parent
class Node {
    weak var parent: Node?  // Weak!
    var children: [Node] = []
}
\end{lstlisting}

\item \textbf{Closures Capturing Self}

\begin{lstlisting}[language=Swift]
// BAD: self captured strongly
class Handler {
    var callback: (() -> Void)?
    
    func setup() {
        callback = {
            self.doSomething()  // Strong capture!
        }
    }
}

// GOOD: weak self
func setup() {
    callback = { [weak self] in
        self?.doSomething()
    }
}
\end{lstlisting}
\end{enumerate}

\subsection{Monitoring con Instruments}

Usa \textbf{Leaks} instrument:

\begin{enumerate}
\item Run con Instruments
\item Esegui operazioni (assert/retract ciclo)
\item Verifica che memoria rimane costante
\item Indaga picchi o crescita continua
\end{enumerate}

\section{Code Style e Convenzioni}

\subsection{Swift Style Guide}

Seguiamo convenzioni Swift standard:

\begin{lstlisting}[language=Swift]
// Naming: camelCase per funzioni/variabili
func evaluateExpression(_ env: inout Environment, _ node: ExpressionNode) -> Value

// Naming: PascalCase per tipi
class AlphaNodeClass: ReteNode

// Indentazione: 4 spazi
func example() {
    if condition {
        doSomething()
    }
}

// Line length: < 120 caratteri (soft limit)
// Function length: < 50 linee (soft limit)

// Access control esplicito
public func publicAPI()
internal func internalHelper()
private func implementationDetail()
\end{lstlisting}

\subsection{Documentazione Inline}

\begin{lstlisting}[language=Swift]
/// Propaga assert di un fatto attraverso la rete RETE
///
/// Questa funzione implementa la logica di NetworkAssert da drive.c (CLIPS).
/// Quando un fatto viene asserito, viene propagato attraverso alpha nodes
/// che matchano il pattern, generando token che fluiscono attraverso
/// join nodes fino ai production nodes.
///
/// - Parameters:
///   - fact: Il fatto da propagare
///   - env: Environment (modificato in-place)
///
/// - Complexity: O(alpha * j * beta) dove:
///   - alpha = numero alpha nodes matchanti
///   - j = numero join per alpha
///   - beta = dimensione beta memory
///
/// - SeeAlso: NetworkRetract per operazione inversa
/// - Note: Riferimento C: drive.c linee 450-520
public static func propagateAssert(
    fact: Environment.FactRec,
    env: inout Environment
) {
    // Implementazione...
}
\end{lstlisting}

\section{Error Handling}

\subsection{Swift Error Model}

\begin{lstlisting}[language=Swift]
enum EvaluationError: Error {
    case typeError(String)
    case unboundVariable(String)
    case divisionByZero
    case templateNotFound(String)
    case constraintViolation(String)
}

func divide(_ a: Value, _ b: Value) throws -> Value {
    let x = try asDouble(a)
    let y = try asDouble(b)
    
    guard y != 0 else {
        throw EvaluationError.divisionByZero
    }
    
    return .float(x / y)
}
\end{lstlisting}

\subsection{Graceful Degradation}

Per funzioni built-in, preferire:

\begin{lstlisting}[language=Swift]
// Invece di crash:
func builtin_sqrt(_ env: inout Environment, _ args: [Value]) throws -> Value {
    guard args.count == 1 else {
        print("[ERROR] sqrt requires exactly 1 argument")
        return .none  // Graceful failure
    }
    
    guard case .float(let x) = args[0], x >= 0 else {
        print("[ERROR] sqrt requires non-negative number")
        return .none
    }
    
    return .float(sqrt(x))
}
\end{lstlisting}

\section{Contribuire a SLIPS}

\subsection{Workflow per Contributor}

\begin{enumerate}
\item \textbf{Fork} repository
\item \textbf{Clone} localmente
\item \textbf{Branch} per feature: \texttt{git checkout -b feature/my-feature}
\item \textbf{Implementa} con test
\item \textbf{Commit} con messaggi descrittivi
\item \textbf{Push} e apri Pull Request
\item \textbf{Code Review}
\item \textbf{Merge} dopo approvazione
\end{enumerate}

\subsection{Commit Message Convention}

Seguiamo Conventional Commits:

\begin{verbatim}
<type>(<scope>): <subject>

<body>

<footer>
\end{verbatim}

\textbf{Types}:
\begin{itemize}
\item \texttt{feat}: Nuova funzionalità
\item \texttt{fix}: Bug fix
\item \texttt{docs}: Documentazione
\item \texttt{test}: Aggiunta test
\item \texttt{refactor}: Refactoring
\item \texttt{perf}: Performance improvement
\end{itemize}

\textbf{Esempio}:
\begin{verbatim}
feat(modules): implementa sistema completo di moduli CLIPS

Fase 3 completata (95%):

- Defmodule parsing con import/export
- Focus stack LIFO
- 5 comandi builtin: focus, get/set-current-module, list-defmodules
- 22 test (100% pass)

Riferimenti CLIPS C:
- moduldef.h/c → Modules.swift
- modulbsc.c → functions.swift

Closes #123
\end{verbatim}

\section{Code Review Checklist}

Prima di PR, verifica:

\begin{itemize}
\item[$\square$] Build passa senza warning
\item[$\square$] Tutti i test passano (inclusi i nuovi)
\item[$\square$] Coverage non diminuisce
\item[$\square$] Documentazione aggiornata
\item[$\square$] Riferimenti a file C originali presenti
\item[$\square$] No force unwrap in codice pubblico
\item[$\square$] Error handling appropriato
\item[$\square$] Performance accettabili
\item[$\square$] Commit messages descrittivi
\item[$\square$] AGENTS.md rispettato
\end{itemize}

\section{Antipattern da Evitare}

\subsection{Force Unwrap in Production Code}

\begin{lstlisting}[language=Swift]
// BAD: crash se nil
let module = env.findDefmodule(name: moduleName)!

// GOOD: gestione esplicita
guard let module = env.findDefmodule(name: moduleName) else {
    print("[ERROR] Module \(moduleName) not found")
    return .boolean(false)
}
\end{lstlisting}

\subsection{Premature Optimization}

\begin{warningbox}[Knuth's Law]
"Premature optimization is the root of all evil" --- Donald Knuth
\end{warningbox}

Workflow corretto:
\begin{enumerate}
\item Implementa correttamente (equivalenza con CLIPS)
\item Misura performance (profiler)
\item Identifica bottleneck reali
\item Ottimizza dove necessario
\item Verifica miglioramento
\end{enumerate}

\subsection{God Objects}

\texttt{Environment} è intenzionalmente God Object per compatibilità C, ma:

\begin{itemize}
\item Non aggiungere responsabilità non-essenziali
\item Usa extension per organizzare logicamente
\item Considera refactor se cresce > 150 campi
\end{itemize}

\section{Sicurezza e Robustezza}

\subsection{Input Validation}

\begin{lstlisting}[language=Swift]
func builtin_assert(_ env: inout Environment, _ args: [Value]) throws -> Value {
    // 1. Valida numero argomenti
    guard args.count >= 1 else {
        print("[ERROR] assert requires at least 1 argument")
        return .int(-1)
    }
    
    // 2. Valida tipo argomenti
    guard case .symbol(let templateName) = args[0] else {
        print("[ERROR] assert first argument must be template name")
        return .int(-1)
    }
    
    // 3. Valida template exists
    guard env.templates[templateName] != nil else {
        print("[ERROR] Template \(templateName) not defined")
        return .int(-1)
    }
    
    // 4. Valida constraints
    // ...
    
    // 5. Esegui operazione
    let factID = createFact(template: templateName, slots: slots, env: &env)
    return .int(Int64(factID))
}
\end{lstlisting}

\subsection{Defensive Programming}

\begin{lstlisting}[language=Swift]
func join(leftToken: BetaToken, rightFact: FactRec) -> BetaToken? {
    // Precondizioni
    precondition(!leftToken.usedFacts.isEmpty, "Token deve contenere almeno 1 fatto")
    precondition(rightFact.id > 0, "Fact ID deve essere positivo")
    
    // Verifica no duplicati
    guard !leftToken.usedFacts.contains(rightFact.id) else {
        return nil  // Fact gia' usato in questo token
    }
    
    // Verifica consistenza join keys
    for key in joinKeys {
        guard let leftValue = leftToken.bindings[key],
              let rightValue = rightFact.slots[key] else {
            continue
        }
        
        guard leftValue == rightValue else {
            return nil  // Inconsistente
        }
    }
    
    // Crea nuovo token
    // ... postcondizioni
    return newToken
}
\end{lstlisting}

\section{Documentazione}

\subsection{Inline Documentation}

Ogni file public deve avere:

\begin{lstlisting}[language=Swift]
// File header
// SLIPS - Swift Language Implementation of Production Systems
// Copyright (c) 2025 SLIPS Contributors
// Licensed under the MIT License - see LICENSE file for details

import Foundation

// MARK: - Module Name
// Traduzione fedele da <file.c>, <file.h> (CLIPS 6.4.2)
// Riferimenti C:
// - FunctionName (file.c linee 123-145)
// - StructName (file.h linee 67-89)

/// Brief description of module
///
/// Longer description explaining purpose, usage, and relationship
/// with CLIPS C implementation.
///
/// Example:
/// ```swift
/// var env = Environment()
/// let result = SomeFunction(&env, param)
/// ```
///
/// - SeeAlso: RelatedModule.swift
/// - Note: Port of file.c from CLIPS 6.4.2
\end{lstlisting}

\subsection{README e Guide}

Ogni modulo significativo ha README:

\begin{verbatim}
Sources/SLIPS/Rete/README.md

# RETE Engine

Implementazione algoritmo RETE per pattern matching efficiente.

## Files

- AlphaNetwork.swift: Alpha nodes (pattern filtering)
- BetaEngine.swift: Beta network (join operations)
- DriveEngine.swift: Propagation (assert/retract)
- Nodes.swift: Explicit node classes
- NetworkBuilder.swift: Network construction

## References

CLIPS C files:
- drive.c: Network propagation
- reteutil.c: RETE utilities
- pattern.c: Pattern nodes
- network.c: Network structures

## Usage

See ReteTests.swift for examples.
\end{verbatim}

\section{Conclusioni del Capitolo}

Best practices fondamentali:

\begin{enumerate}
\item \textbf{Testing}: TDD, alta coverage, proprietà verificate
\item \textbf{Performance}: Profiling-guided, pattern specifici
\item \textbf{Safety}: No force unwrap, validation input, error handling
\item \textbf{Style}: Convenzioni Swift, documentazione inline
\item \textbf{Workflow}: Git flow, code review, CI/CD
\end{enumerate}

\begin{successbox}[Quality Standards]
Aderendo a queste best practices, SLIPS mantiene qualità production-ready con:
\begin{itemize}
\item 97.8\% test pass rate
\item Zero unsafe code pubblico
\item Build sempre clean
\item Documentazione completa
\end{itemize}
\end{successbox}


% Capitolo 25: Performance e Ottimizzazione

\chapter{Performance e Ottimizzazione}
\label{cap:performance}

\section{Introduzione}

Questo capitolo presenta tecniche per ottimizzare le prestazioni di sistemi SLIPS in produzione.

\section{Profiling}

\subsection{Time Profiling}

\begin{lstlisting}[language=Swift]
class ReteProfiler {
    struct NodeStats {
        var executionCount: Int = 0
        var totalTime: TimeInterval = 0
        var avgTime: TimeInterval { totalTime / Double(executionCount) }
    }
    
    private var nodeStats: [Int: NodeStats] = [:]
    
    func profile<T>(node: ReteNode, _ block: () -> T) -> T {
        let start = Date()
        defer {
            let elapsed = Date().timeIntervalSince(start)
            var stats = nodeStats[node.id, default: NodeStats()]
            stats.executionCount += 1
            stats.totalTime += elapsed
            nodeStats[node.id] = stats
        }
        return block()
    }
    
    func report(top n: Int = 10) {
        let sorted = nodeStats.sorted { $0.value.totalTime > $1.value.totalTime }
        print("Top \(n) nodes by execution time:")
        for (nodeID, stats) in sorted.prefix(n) {
            print("  Node \(nodeID): \(stats.totalTime)s (\(stats.executionCount) calls, avg: \(stats.avgTime)s)")
        }
    }
}
\end{lstlisting}

\subsection{Memory Profiling}

\begin{lstlisting}[language=Swift]
class MemoryProfiler {
    func snapshot(environment: Environment) -> MemorySnapshot {
        return MemorySnapshot(
            factCount: environment.factList.count,
            ruleCount: environment.rules.count,
            tokenCount: countTokens(environment),
            activationCount: environment.agenda.count
        )
    }
    
    private func countTokens(_ env: Environment) -> Int {
        var total = 0
        // Traverse beta network counting tokens
        return total
    }
}

struct MemorySnapshot {
    let factCount: Int
    let ruleCount: Int
    let tokenCount: Int
    let activationCount: Int
    
    var estimatedMemory: Int {
        factCount * 256 +        // Avg fact size
        tokenCount * 128 +       // Avg token size
        activationCount * 64     // Avg activation size
    }
}
\end{lstlisting>

\section{Pattern Optimization}

\subsection{Pattern Reordering}

\begin{lstlisting}[language=Swift]
class PatternOptimizer {
    func reorderPatterns(_ rule: Defrule, stats: Statistics) -> Defrule {
        let patterns = rule.patterns.sorted { p1, p2 in
            estimateSelectivity(p1, stats) < estimateSelectivity(p2, stats)
        }
        
        return Defrule(
            name: rule.name,
            module: rule.module,
            patterns: patterns,
            actions: rule.actions,
            salience: rule.salience,
            autoFocus: rule.autoFocus
        )
    }
    
    private func estimateSelectivity(_ pattern: Pattern, _ stats: Statistics) -> Double {
        let templateStats = stats.templates[pattern.template] ?? TemplateStats()
        var selectivity = 1.0
        
        for constraint in pattern.constraints {
            selectivity *= estimateConstraintSelectivity(constraint, templateStats)
        }
        
        return selectivity
    }
}
\end{lstlisting}

\section{Memory Optimization}

\subsection{Token Pooling}

\begin{lstlisting}[language=Swift]
class OptimizedTokenPool {
    private var pools: [Int: [Token]] = [:]  // By fact count
    private let maxPoolSize = 100
    
    func acquire(factCount: Int) -> Token? {
        return pools[factCount]?.popLast()
    }
    
    func release(_ token: Token) {
        let factCount = token.facts.count
        var pool = pools[factCount, default: []]
        
        guard pool.count < maxPoolSize else { return }
        pool.append(token)
        pools[factCount] = pool
    }
    
    func clear() {
        pools.removeAll()
    }
}
\end{lstlisting>

\subsection{Compact Representations}

\begin{lstlisting}[language=Swift]
// Instead of full facts in token
struct CompactToken {
    let factIDs: [Int32]  // 4 bytes per ID
    let bindingIndices: [UInt8: UInt8]  // Compact binding map
    
    var memoryFootprint: Int {
        factIDs.count * 4 + bindingIndices.count * 2
    }
}

// vs standard Token
struct StandardToken {
    let facts: [Fact]  // 8 bytes per reference + object overhead
    let bindings: [String: Value]  // Dictionary overhead
    
    var memoryFootprint: Int {
        facts.count * 8 + bindings.count * (24 + 32)  // Approx
    }
}

// Savings: 70-80% for typical tokens
\end{lstlisting}

\section{Execution Optimization}

\subsection{Batch Processing}

\begin{lstlisting}[language=Swift]
extension Environment {
    func assertBatch(_ facts: [(String, [String: Value])]) {
        // Disable intermediate propagation
        alphaNetwork.suspendPropagation()
        
        for (template, slots) in facts {
            assert(template: template, slots: slots)
        }
        
        // Resume and propagate all at once
        alphaNetwork.resumePropagation()
    }
}
\end{lstlisting>

\subsection{Lazy Evaluation}

\begin{lstlisting}[language=Swift]
class LazyJoinNode: JoinNode {
    private var pendingLeft: [Token] = []
    private var pendingRight: [Fact] = []
    
    override func leftActivate(token: Token) {
        if pendingRight.isEmpty {
            pendingLeft.append(token)
        } else {
            processPending()
            super.leftActivate(token)
        }
    }
    
    private func processPending() {
        for token in pendingLeft {
            super.leftActivate(token)
        }
        pendingLeft.removeAll()
    }
}
\end{lstlisting}

\section{Benchmarking}

\subsection{Benchmark Suite}

\begin{lstlisting}[language=Swift]
class SLIPSBenchmark {
    func runBenchmarks() {
        measure("Assert 1000 facts") {
            let env = Environment()
            env.load("benchmark.clp")
            for i in 0..<1000 {
                env.assert(template: "data", slots: ["id": .integer(i)])
            }
        }
        
        measure("Run with 100 rules") {
            let env = Environment()
            env.load("complex-rules.clp")
            env.run()
        }
        
        measure("Pattern matching complex") {
            let env = Environment()
            env.load("complex-patterns.clp")
            env.assert(template: "trigger", slots: [:])
            env.run()
        }
    }
    
    private func measure(_ name: String, iterations: Int = 10, _ block: () -> Void) {
        var times: [TimeInterval] = []
        
        for _ in 0..<iterations {
            let start = Date()
            block()
            times.append(Date().timeIntervalSince(start))
        }
        
        let avg = times.reduce(0, +) / Double(iterations)
        let min = times.min()!
        let max = times.max()!
        
        print("\(name):")
        print("  Avg: \(avg * 1000)ms")
        print("  Min: \(min * 1000)ms")
        print("  Max: \(max * 1000)ms")
    }
}
\end{lstlisting}

\section{Best Practices}

\subsection{Do's}

\begin{infobox}[Raccomandazioni]
\begin{itemize}
\item Pattern specifici con constraint stringenti
\item Salience solo quando necessaria
\item Batch assert quando possibile
\item Monitor memoria con profiler
\item Testare con dati realistici
\end{itemize}
\end{infobox}

\subsection{Don'ts}

\begin{warningbox}[Da Evitare]
\begin{itemize}
\item Pattern troppo generici
\item Troppe regole con alta salience
\item Assert/retract in loop stretti
\item Ignorare memory leak
\item Ottimizzazione prematura
\end{itemize}
\end{warningbox}

\section{Conclusioni del Capitolo}

\subsection{Punti Chiave}

\begin{enumerate}
\item \textbf{Profilare} prima di ottimizzare
\item \textbf{Pattern reordering} impatta significativamente
\item \textbf{Memory management} critico per sistemi grandi
\item \textbf{Batch operations} riducono overhead
\item \textbf{Benchmark regolari} per regression
\end{enumerate}

\subsection{Letture Consigliate}

\begin{itemize}
\item Instruments User Guide (Apple)
\item Swift Performance Tips
\end{itemize}

% Capitolo 26: Debugging e Troubleshooting

\chapter{Debugging e Troubleshooting}
\label{cap:debugging}

\section{Introduzione}

Debugging di sistemi a produzione richiede strumenti e tecniche specifiche. Questo capitolo presenta gli strumenti di debugging di SLIPS.

\section{Watch Facilities}

\subsection{Implementazione}

\begin{lstlisting}[language=Swift]
public enum WatchItem {
    case facts
    case rules
    case activations
    case focus
    case compilations
    case statistics
}

public class WatchManager {
    private var watched: Set<WatchItem> = []
    
    public func watch(_ item: WatchItem) {
        watched.insert(item)
    }
    
    public func unwatch(_ item: WatchItem) {
        watched.remove(item)
    }
    
    public func isWatching(_ item: WatchItem) -> Bool {
        return watched.contains(item)
    }
}

// Uso nell'environment
extension Environment {
    func notifyFactAssert(_ fact: Fact) {
        if watchManager.isWatching(.facts) {
            print("==> f-\(fact.id) (\(fact.template.name) \(formatSlots(fact.slots)))")
        }
    }
    
    func notifyRuleFiring(_ rule: Defrule) {
        if watchManager.isWatching(.rules) {
            print("FIRE \(rule.name)")
        }
    }
}
\end{lstlisting>

\section{Inspection Commands}

\subsection{Facts Inspection}

\begin{lstlisting}[language=Swift]
extension Environment {
    public func facts(module: String? = nil) {
        let filtered = module != nil 
            ? factList.filter { $0.template.module.name == module }
            : factList
        
        for fact in filtered {
            print("f-\(fact.id) (\(fact.template.name)")
            for (slot, value) in fact.slots {
                print("  (\(slot) \(value))")
            }
            print(")")
        }
    }
    
    public func ppfact(_ id: Int) {
        guard let fact = factList.first(where: { $0.id == id }) else {
            print("Error: Fact \(id) not found")
            return
        }
        
        print("(\(fact.template.name)")
        for (slot, value) in fact.slots.sorted(by: { $0.key < $1.key }) {
            print("  (\(slot) \(value))")
        }
        print(")")
    }
}
\end{lstlisting}

\subsection{Agenda Inspection}

\begin{lstlisting}[language=Swift]
extension Environment {
    public func agenda(module: String? = nil) {
        let activations = module != nil
            ? agenda.all.filter { $0.rule.module.name == module }
            : agenda.all
        
        for (index, activation) in activations.enumerated() {
            print("\(index): \(activation.salience) : \(activation.rule.name)")
            print("  Bindings: \(activation.token.bindings)")
        }
    }
}
\end{lstlisting}

\subsection{Matches Inspection}

\begin{lstlisting}[language=Swift]
extension Environment {
    public func matches(ruleName: String) {
        guard let rule = rules[ruleName] else {
            print("Error: Rule \(ruleName) not found")
            return
        }
        
        guard let prodNode = rule.productionNode else {
            print("Rule not compiled")
            return
        }
        
        print("Partial matches for \(ruleName):")
        
        // Show matches at each level
        var currentNode: ReteNode? = prodNode
        var level = rule.patterns.count
        
        while let node = currentNode {
            if let betaMem = node as? BetaMemory {
                print("Pattern \(level):")
                for token in betaMem.tokens {
                    print("  \(token.facts.map { "f-\($0.id)" }.joined(separator: ", "))")
                }
                level -= 1
            }
            
            // Navigate up (simplified)
            currentNode = nil  // Would need parent pointers
        }
    }
}
\end{lstlisting}

\section{Breakpoints}

\subsection{Rule Breakpoints}

\begin{lstlisting}[language=Swift]
public class BreakpointManager {
    private var ruleBreakpoints: Set<String> = []
    private var factBreakpoints: Set<Int> = []
    
    public func setRuleBreakpoint(_ ruleName: String) {
        ruleBreakpoints.insert(ruleName)
    }
    
    public func removeRuleBreakpoint(_ ruleName: String) {
        ruleBreakpoints.remove(ruleName)
    }
    
    public func shouldBreak(beforeFiring rule: Defrule) -> Bool {
        return ruleBreakpoints.contains(rule.name)
    }
}

extension Environment {
    private func fireRule(_ activation: Activation) {
        if breakpointManager.shouldBreak(beforeFiring: activation.rule) {
            print("BREAKPOINT: About to fire \(activation.rule.name)")
            print("Bindings: \(activation.token.bindings)")
            
            // Enter debug REPL
            debugREPL(activation: activation)
        }
        
        // Execute rule
        // ...
    }
    
    private func debugREPL(activation: Activation) {
        print("Debug> (c)ontinue, (s)tep, (i)nspect, (q)uit")
        // Interactive debugging
    }
}
\end{lstlisting}

\section{Tracing}

\subsection{Execution Trace}

\begin{lstlisting}[language=Swift]
public class ExecutionTracer {
    private var trace: [TraceEvent] = []
    private var isTracing = false
    
    public enum TraceEvent {
        case factAssert(Fact)
        case factRetract(Int)
        case ruleFire(Defrule, Token)
        case agendaAdd(Activation)
        case agendaRemove(Activation)
    }
    
    public func startTracing() {
        isTracing = true
        trace.removeAll()
    }
    
    public func stopTracing() {
        isTracing = false
    }
    
    public func record(_ event: TraceEvent) {
        guard isTracing else { return }
        trace.append(event)
    }
    
    public func printTrace() {
        for (index, event) in trace.enumerated() {
            print("\(index): \(formatEvent(event))")
        }
    }
    
    private func formatEvent(_ event: TraceEvent) -> String {
        switch event {
        case .factAssert(let f):
            return "ASSERT f-\(f.id)"
        case .factRetract(let id):
            return "RETRACT f-\(id)"
        case .ruleFire(let rule, _):
            return "FIRE \(rule.name)"
        case .agendaAdd(let act):
            return "AGENDA+ \(act.rule.name)"
        case .agendaRemove(let act):
            return "AGENDA- \(act.rule.name)"
        }
    }
}
\end{lstlisting>

\section{Common Issues}

\subsection{Infinite Loops}

\begin{warningbox}[Loop Detection]
\begin{lstlisting}[language=CLIPS]
;; BAD: Creates infinite loop
(defrule loop
  (counter ?n)
  =>
  (assert (counter (+ ?n 1))))
\end{lstlisting}

\textbf{Detection}:
\begin{lstlisting}[language=Swift]
class LoopDetector {
    private var firedRules: [String] = []
    private let maxConsecutiveFirings = 100
    
    func checkLoop(rule: String) -> Bool {
        firedRules.append(rule)
        
        if firedRules.count > maxConsecutiveFirings {
            let recent = firedRules.suffix(maxConsecutiveFirings)
            if Set(recent).count < 10 {
                print("WARNING: Possible infinite loop detected!")
                print("Recent firings: \(recent.suffix(10))")
                return true
            }
        }
        
        return false
    }
}
\end{lstlisting}
\end{warningbox}

\subsection{Memory Leaks}

\begin{lstlisting}[language=Swift]
class LeakDetector {
    private var baselineSnapshot: MemorySnapshot?
    
    func setBaseline(_ env: Environment) {
        baselineSnapshot = MemorySnapshot(env)
    }
    
    func checkLeaks(_ env: Environment) {
        guard let baseline = baselineSnapshot else { return }
        let current = MemorySnapshot(env)
        
        let factGrowth = current.factCount - baseline.factCount
        let tokenGrowth = current.tokenCount - baseline.tokenCount
        
        if factGrowth > 1000 {
            print("WARNING: Fact count grew by \(factGrowth)")
        }
        
        if tokenGrowth > 10000 {
            print("WARNING: Token count grew by \(tokenGrowth)")
        }
    }
}
\end{lstlisting}

\section{Conclusioni del Capitolo}

\subsection{Punti Chiave}

\begin{enumerate}
\item \textbf{Watch facilities} per osservare esecuzione
\item \textbf{Inspection commands} per interrogare stato
\item \textbf{Breakpoints} per debugging interattivo
\item \textbf{Tracing} per analisi post-mortem
\item \textbf{Loop/leak detection} per robustezza
\end{enumerate}

\subsection{Letture Consigliate}

\begin{itemize}
\item CLIPS User Guide - Debugging
\item Xcode Debugging Guide
\end{itemize}

% Capitolo 27: Sviluppi Futuri

\chapter{Sviluppi Futuri e Roadmap}
\label{cap:futuro}

\section{Introduzione}

SLIPS è un progetto in evoluzione. Questo capitolo delinea gli sviluppi futuri pianificati e le direzioni di ricerca.

\section{Roadmap Tecnica}

\subsection{Fase 1 (COMPLETATA): Core Foundation}

\begin{successbox}[Achievements]
\begin{itemize}
\item[$\checkmark$] Environment e strutture base
\item[$\checkmark$] Parser e compiler CLIPS
\item[$\checkmark$] RETE network base
\item[$\checkmark$] Agenda e conflict resolution
\item[$\checkmark$] Module system
\item[$\checkmark$] Basic testing suite
\end{itemize}
\end{successbox}

\subsection{Fase 2 (IN CORSO): Advanced Features}

\begin{infobox}[Current Work]
\begin{itemize}
\item Multifield avanzati (\$?x completo)
\item Pattern composti (AND/OR/NOT nidificati)
\item Dynamic salience completo
\item Object-oriented extensions (COOL)
\item Incremental reset
\end{itemize}
\end{infobox}

\subsection{Fase 3 (PIANIFICATA): Performance}

\begin{itemize}
\item Parallel RETE (multi-core)
\item RETE/UL (unlinking ottimizzato)
\item JIT compilation per RHS
\item Memory optimization (compact tokens)
\item Profiler integrato
\end{itemize}

\subsection{Fase 4 (RICERCA): Extensions}

\begin{itemize}
\item Distributed RETE (cluster)
\item Probabilistic reasoning
\item Machine learning integration
\item Fuzzy logic support
\item Temporal reasoning
\end{itemize}

\section{Estensioni Linguaggio}

\subsection{Swift DSL}

\textbf{Visione}: DSL Swift type-safe per regole.

\begin{lstlisting}[language=Swift]
@RuleBuilder
var rules: [Rule] {
    Rule("discount") {
        Pattern("customer") { customer in
            customer.age >= 65
        }
    } action: { bindings in
        assert("discount", amount: 20)
    }
    
    Rule("vip-treatment") {
        Pattern("customer") { c in
            c.purchaseTotal > 10000
        }
        Pattern("order") { o in
            o.customerID == c.id
        }
    } action: {
        assert("priority-shipping")
    }
}
\end{lstlisting}

\subsection{Swift Concurrency Integration}

\begin{lstlisting}[language=Swift]
extension Environment {
    public func runAsync(limit: Int = -1) async {
        await withTaskGroup(of: Void.self) { group in
            group.addTask {
                await self.runRecognizePhase()
            }
            group.addTask {
                await self.runActPhase()
            }
        }
    }
}
\end{lstlisting}

\section{Interoperabilità}

\subsection{C Interop}

\begin{lstlisting}[language=Swift]
// Chiamare librerie C esistenti
@_cdecl("clips_compatible_assert")
public func clipsCompatibleAssert(
    _ env: OpaquePointer,
    _ factString: UnsafePointer<CChar>
) -> Int {
    let swift Env = Unmanaged<Environment>.fromOpaque(env).takeUnretainedValue()
    let string = String(cString: factString)
    
    // Parse and assert
    return swiftEnv.assertString(string) != nil ? 1 : 0
}
\end{lstlisting}

\subsection{Python Bindings}

\begin{lstlisting}[language=Python]
# Potenziale binding Python
import slips

env = slips.Environment()
env.load("rules.clp")
env.assert_fact("person", {"name": "Mario", "age": 30})
env.run()

facts = env.get_facts()
for fact in facts:
    print(f"f-{fact.id}: {fact.template} {fact.slots}")
\end{lstlisting}

\section{Community e Contributi}

\subsection{Open Source Development}

\begin{itemize}
\item \textbf{GitHub}: \url{https://github.com/gpicchiarelli/SLIPS}
\item \textbf{Issues}: Bug reports e feature requests
\item \textbf{Pull Requests}: Contributi della community
\item \textbf{Discussions}: Design discussions
\item \textbf{Documentation}: Continua evoluzione
\end{itemize}

\subsection{Areas for Contribution}

\begin{infobox}[Contribute]
\begin{itemize}
\item \textbf{Testing}: Aggiungere test cases
\item \textbf{Documentation}: Migliorare documentazione
\item \textbf{Examples}: Esempi pratici
\item \textbf{Optimization}: Performance improvements
\item \textbf{Platforms}: Support per Linux, Windows
\item \textbf{Tooling}: IDE support, syntax highlighting
\end{itemize}
\end{infobox}

\section{Ricerca Futura}

\subsection{Parallel Pattern Matching}

\textbf{Sfide}:
\begin{itemize}
\item Sincronizzazione beta memories
\item Ordering activations cross-thread
\item Load balancing tra core
\end{itemize}

\textbf{Approcci}:
\begin{itemize}
\item Partitioning del pattern network
\item Lock-free data structures
\item Actor model per nodi
\end{itemize}

\subsection{Machine Learning Integration}

\textbf{Opportunità}:
\begin{itemize}
\item Apprendere salience da dati
\item Ottimizzare pattern ordering
\item Suggerire nuove regole
\item Anomaly detection
\end{itemize}

\textbf{Esempio concettuale}:
\begin{lstlisting}[language=Swift]
class MLSalienceOptimizer {
    let model: MLModel
    
    func optimizeSalience(rule: Defrule, context: Context) -> Int {
        let features = extractFeatures(rule, context)
        let prediction = try! model.prediction(from: features)
        return Int(prediction.salience)
    }
}
\end{lstlisting}

\subsection{Distributed Production Systems}

\textbf{Architettura proposta}:

\begin{verbatim}
Node 1: Alpha Network + Local WM
Node 2: Alpha Network + Local WM
...
Coordinator: Beta Network + Agenda
\end{verbatim}

\textbf{Sfide}:
\begin{itemize}
\item Network latency
\item Consistency models
\item Fault tolerance
\item State synchronization
\end{itemize}

\section{Standardization}

\subsection{CLIPS Compatibility}

\textbf{Obiettivo}: 100\% compatibilità CLIPS 6.4.

\textbf{Metriche}:
\begin{itemize}
\item Test suite CLIPS ufficiale: pass rate
\item Performance parity: $\pm 20\%$ di CLIPS C
\item Feature completeness: tutte le built-in
\end{itemize}

\subsection{Swift Package Ecosystem}

\textbf{Visione}: SLIPS come package Swift standard per expert systems.

\begin{itemize}
\item SwiftPM integration
\item Documentation hosting
\item CI/CD pipeline
\item Version management
\item Semantic versioning
\end{itemize}

\section{Educational Use}

\subsection{Teaching Tool}

SLIPS come strumento didattico:

\begin{itemize}
\item \textbf{Leggibilità}: Codice Swift più accessibile di C
\item \textbf{Type Safety}: Errori catturati a compile-time
\item \textbf{Playground}: Xcode Playgrounds per sperimentazione
\item \textbf{Debugging}: Strumenti moderni
\end{itemize}

\subsection{Course Material}

\begin{itemize}
\item Tutorial interattivi
\item Video lectures
\item Esercizi progressivi
\item Projects per studenti
\item Integration con curricula universitari
\end{itemize}

\section{Industrial Applications}

\subsection{Use Cases}

\textbf{Potenziali applicazioni}:

\begin{itemize}
\item \textbf{iOS/macOS Apps}: Expert systems nativi
\item \textbf{Server-Side Swift}: Business rules engine
\item \textbf{IoT}: Edge computing con regole
\item \textbf{Healthcare}: Decision support systems
\item \textbf{Finance}: Trading rules, compliance
\item \textbf{Automation}: Smart home, robotics
\end{itemize}

\subsection{Enterprise Features}

\textbf{Necessari per adoption aziendale}:

\begin{itemize}
\item Persistence (save/restore state)
\item Audit logging
\item Security (sandboxing)
\item Monitoring e metrics
\item High availability
\item Scalability horizontale
\end{itemize}

\section{Long-Term Vision}

\subsection{SLIPS 2.0}

\textbf{Possibili evoluzioni}:

\begin{itemize}
\item \textbf{Query language}: SQL-like per fatti
\item \textbf{Reactive streams}: Integration con Combine
\item \textbf{SwiftUI integration}: Visualizzazione rete
\item \textbf{Cloud integration}: Distributed execution
\item \textbf{ML-augmented}: Hybrid symbolic/subsymbolic
\end{itemize}

\subsection{Research Directions}

\begin{itemize}
\item Quantum-inspired algorithms per pattern matching
\item Probabilistic production systems
\item Neuro-symbolic integration
\item Explainable AI basato su regole
\item Verification formale con SMT solvers
\end{itemize}

\section{Call to Action}

\subsection{Come Contribuire}

\begin{enumerate}
\item \textbf{Fork} il repository
\item \textbf{Scegli} un'area di interesse
\item \textbf{Leggi} CONTRIBUTING.md
\item \textbf{Sviluppa} con TDD
\item \textbf{Sottometti} PR con test
\item \textbf{Collabora} con maintainers
\end{enumerate}

\subsection{Join the Community}

\begin{itemize}
\item GitHub Discussions
\item Discord server (futuro)
\item Stack Overflow tag
\item Conferenze e meetup
\item Paper e pubblicazioni
\end{itemize}

\section{Conclusioni del Libro}

\subsection{Riepilogo Generale}

In questo libro abbiamo coperto:

\begin{itemize}
\item \textbf{Parte I}: Fondamenti teorici (logica, rappresentazione)
\item \textbf{Parte II}: Algoritmo RETE (alpha, beta, complessità, ottimizzazioni)
\item \textbf{Parte III}: Architettura CLIPS (strutture, memoria, agenda, moduli)
\item \textbf{Parte IV}: Implementazione SLIPS (core, RETE, agenda, pattern)
\item \textbf{Parte V}: Sviluppo pratico (testing, estensioni, performance, debug)
\item \textbf{Appendici}: API, built-in, esempi, benchmark
\end{itemize}

\subsection{Messaggiofinale}

SLIPS rappresenta un ponte tra passato e futuro dell'intelligenza artificiale simbolica:

\begin{itemize}
\item \textbf{Passato}: L'eredità robusta di CLIPS e dei sistemi esperti
\item \textbf{Presente}: Le garanzie di sicurezza di Swift moderno
\item \textbf{Futuro}: Nuove possibilità di integrazione e innovazione
\end{itemize}

Il progetto è aperto, la community è accogliente, e il futuro è da scrivere insieme.

\vspace{1cm}

\begin{center}
\textit{Buon coding con SLIPS!}

\bigskip

I Contributori SLIPS \\
Ottobre 2025
\end{center}

\subsection{Letture Consigliate}

\begin{itemize}
\item CLIPS 6.4 Documentation
\item Swift Evolution Proposals
\item Expert Systems Research Papers
\item Symbolic AI Renaissance (2020+)
\end{itemize}


% Appendici
\appendix
% Appendice A: Riferimento API Completo

\chapter{Riferimento API Completo}
\label{app:api}

\section{API Pubblica CLIPS}

\subsection{Gestione Environment}

\begin{lstlisting}[language=Swift]
@MainActor
public enum CLIPS {
    /// Crea nuovo environment
    /// - Returns: Environment inizializzato
    public static func createEnvironment() -> Environment
    
    /// Carica file .clp
    /// - Parameter path: Percorso file
    /// - Throws: IOError se file non trovato
    public static func load(_ path: String) throws
    
    /// Reset environment (clear + deffacts)
    public static func reset()
    
    /// Accesso environment corrente
    public static var currentEnvironment: Environment? { get }
}
\end{lstlisting}

\subsection{Esecuzione Regole}

\begin{lstlisting}[language=Swift]
extension CLIPS {
    /// Esegue regole fino a esaurimento o limite
    /// - Parameter limit: Numero massimo regole (nil = infinito)
    /// - Returns: Numero regole eseguite
    @discardableResult
    public static func run(limit: Int?) -> Int
}
\end{lstlisting}

\subsection{Gestione Fatti}

\begin{lstlisting}[language=Swift]
extension CLIPS {
    /// Asserisce fatto
    /// - Parameter fact: Espressione CLIPS
    /// - Returns: ID del fatto (-1 se errore)
    @discardableResult
    public static func assert(fact: String) -> Int
    
    /// Ritrae fatto
    /// - Parameter id: ID del fatto da ritrarre
    public static func retract(id: Int)
}
\end{lstlisting}

\subsection{Valutazione Espressioni}

\begin{lstlisting}[language=Swift]
extension CLIPS {
    /// Valuta espressione CLIPS
    /// - Parameter expr: Espressione S-expression
    /// - Returns: Valore risultante
    @discardableResult
    public static func eval(expr: String) -> Value
}
\end{lstlisting}

\section{Built-in Functions}

\subsection{Matematica}

\begin{table}[h]
\centering
\small
\begin{tabular}{@{}llp{6cm}@{}}
\toprule
\textbf{Funzione} & \textbf{Args} & \textbf{Descrizione} \\
\midrule
\texttt{+} & $n \geq 1$ & Somma argomenti \\
\texttt{-} & $n \geq 1$ & Sottrazione (unario: negazione) \\
\texttt{*} & $n \geq 1$ & Prodotto \\
\texttt{/} & $n \geq 1$ & Divisione \\
\texttt{div} & 2 & Divisione intera \\
\texttt{mod} & 2 & Modulo \\
\texttt{abs} & 1 & Valore assoluto \\
\texttt{min} & $n \geq 1$ & Minimo \\
\texttt{max} & $n \geq 1$ & Massimo \\
\texttt{sqrt} & 1 & Radice quadrata \\
\texttt{pow} & 2 & Potenza \\
\texttt{exp} & 1 & Esponenziale \\
\texttt{log} & 1 & Logaritmo naturale \\
\texttt{log10} & 1 & Logaritmo base 10 \\
\bottomrule
\end{tabular}
\caption{Funzioni matematiche}
\label{tab:api_math}
\end{table}

\subsection{Logiche}

\begin{table}[h]
\centering
\small
\begin{tabular}{@{}llp{6cm}@{}}
\toprule
\textbf{Funzione} & \textbf{Args} & \textbf{Descrizione} \\
\midrule
\texttt{and} & $n \geq 1$ & AND logico \\
\texttt{or} & $n \geq 1$ & OR logico \\
\texttt{not} & 1 & NOT logico \\
\texttt{eq} & 2+ & Uguaglianza valore \\
\texttt{neq} & 2+ & Disuguaglianza \\
\texttt{=} & 2+ & Uguaglianza numerica \\
\texttt{<>} & 2+ & Disuguaglianza numerica \\
\texttt{<} & 2+ & Minore \\
\texttt{<=} & 2+ & Minore o uguale \\
\texttt{>} & 2+ & Maggiore \\
\texttt{>=} & 2+ & Maggiore o uguale \\
\bottomrule
\end{tabular}
\caption{Funzioni logiche}
\label{tab:api_logic}
\end{table}

\subsection{Facts e Rules}

\begin{table}[h]
\centering
\small
\begin{tabular}{@{}llp{6cm}@{}}
\toprule
\textbf{Funzione} & \textbf{Args} & \textbf{Descrizione} \\
\midrule
\texttt{assert} & 1+ & Asserisce fatto \\
\texttt{retract} & 1+ & Ritrae fatto (per ID) \\
\texttt{modify} & 2+ & Modifica fatto \\
\texttt{duplicate} & 2+ & Duplica fatto \\
\texttt{facts} & 0-1 & Lista fatti [modulo] \\
\texttt{rules} & 0-1 & Lista regole [modulo] \\
\texttt{agenda} & 0-1 & Lista agenda [modulo] \\
\texttt{clear} & 0 & Pulisce environment \\
\texttt{reset} & 0 & Reset + assert deffacts \\
\texttt{run} & 0-1 & Esegue regole [limit] \\
\bottomrule
\end{tabular}
\caption{Funzioni facts e rules}
\label{tab:api_facts}
\end{table}

\subsection{Moduli}

\begin{table}[h]
\centering
\small
\begin{tabular}{@{}llp{5cm}@{}}
\toprule
\textbf{Funzione} & \textbf{Args} & \textbf{Descrizione} \\
\midrule
\texttt{focus} & 1+ & Imposta focus su moduli \\
\texttt{get-current-module} & 0 & Ritorna modulo corrente \\
\texttt{set-current-module} & 1 & Imposta modulo corrente \\
\texttt{list-defmodules} & 0 & Stampa lista moduli \\
\texttt{get-defmodule-list} & 0 & Ritorna multifield moduli \\
\bottomrule
\end{tabular}
\caption{Funzioni moduli}
\label{tab:api_modules}
\end{table}

\section{Value Type}

\begin{lstlisting}[language=Swift]
public enum Value: Codable, Equatable {
    case int(Int64)
    case float(Double)
    case string(String)
    case symbol(String)
    case boolean(Bool)
    case multifield([Value])
    case none
}
\end{lstlisting}

\section{Template e Pattern}

\subsection{Pattern Test Types}

\begin{lstlisting}[language=Swift]
public struct PatternTest: Codable {
    public enum Kind: Codable {
        case constant(Value)
        case variable(String)
        case mfVariable(String)
        case predicate(ExpressionNode)
        case sequence([PatternTest])
    }
    public let kind: Kind
}
\end{lstlisting}

\section{Pattern Matching API}

\subsection{Constraint Builders}

\begin{lstlisting}[language=Swift]
public class PatternBuilder {
    public func pattern(_ template: String, 
                       @ConstraintBuilder _ constraints: () -> [Constraint]) -> Pattern {
        Pattern(template: template, constraints: constraints())
    }
}

@resultBuilder
public struct ConstraintBuilder {
    public static func buildBlock(_ components: Constraint...) -> [Constraint] {
        Array(components)
    }
}
\end{lstlisting}

\section{Error Handling}

\subsection{Error Types}

\begin{lstlisting}[language=Swift]
public enum CLIPSError: Error {
    case parseError(String, line: Int, column: Int)
    case runtimeError(String)
    case undefinedTemplate(String)
    case undefinedRule(String)
    case invalidSlot(String)
    case typeMismatch(expected: ValueType, got: ValueType)
    case fileNotFound(String)
}
\end{lstlisting}

\section{Esempi Completi}

Vedere Appendice~\ref{app:esempi} per esempi d'uso completi e casi di studio.


% Appendice B: Catalogo Built-in Functions

\chapter{Catalogo Completo Built-in Functions}
\label{app:builtin}

\section{Organizzazione}

SLIPS 1.0 implementa 87+ funzioni built-in, organizzate per categoria.

\section{Lista Completa Funzioni Implementate}

\subsection{Matematiche (20 funzioni)}

\texttt{+}, \texttt{-}, \texttt{*}, \texttt{/}, \texttt{div}, \texttt{mod}, \texttt{abs}, \texttt{min}, \texttt{max}, \texttt{sqrt}, \texttt{pow}, \texttt{exp}, \texttt{log}, \texttt{log10}, \texttt{sin}, \texttt{cos}, \texttt{tan}, \texttt{asin}, \texttt{acos}, \texttt{atan}

\subsection{Logiche e Confronto (15 funzioni)}

\texttt{and}, \texttt{or}, \texttt{not}, \texttt{eq}, \texttt{neq}, \texttt{=}, \texttt{<>}, \texttt{<}, \texttt{<=}, \texttt{>}, \texttt{>=}, \texttt{eq*}, \texttt{neq*}, \texttt{<*}, \texttt{>*}

\subsection{Facts Management (12 funzioni)}

\texttt{assert}, \texttt{retract}, \texttt{modify}, \texttt{duplicate}, \texttt{facts}, \texttt{ppfact}, \texttt{fact-index}, \texttt{fact-relation}, \texttt{fact-slot-value}, \texttt{get-fact-list}, \texttt{fact-existp}, \texttt{save-facts}

\subsection{Rules Management (10 funzioni)}

\texttt{rules}, \texttt{ppdefrule}, \texttt{undefrule}, \texttt{refresh}, \texttt{get-defrule-list}, \texttt{matches}, \texttt{list-focus-stack}, \texttt{pop-focus}, \texttt{get-focus}, \texttt{clear-focus-stack}

\subsection{Templates (8 funzioni)}

\texttt{deftemplate}, \texttt{undeftemplate}, \texttt{ppdeftemplate}, \texttt{list-deftemplates}, \texttt{get-deftemplate-list}, \texttt{deftemplate-slot-names}, \texttt{deftemplate-slot-types}, \texttt{deftemplate-slot-range}

\subsection{Modules (5 funzioni)}

\texttt{defmodule}, \texttt{focus}, \texttt{get-current-module}, \texttt{set-current-module}, \texttt{list-defmodules}, \texttt{get-defmodule-list}

\subsection{Agenda e Strategie (10 funzioni)}

\texttt{agenda}, \texttt{run}, \texttt{halt}, \texttt{set-strategy}, \texttt{get-strategy}, \texttt{refresh-agenda}, \texttt{reorder}, \texttt{get-salience-evaluation}, \texttt{set-salience-evaluation}, \texttt{clear}

\subsection{I/O (7 funzioni)}

\texttt{printout}, \texttt{read}, \texttt{readline}, \texttt{format}, \texttt{open}, \texttt{close}, \texttt{get-char}

\subsection{Multifield (10 funzioni)}

\texttt{create\$}, \texttt{length\$}, \texttt{nth\$}, \texttt{rest\$}, \texttt{first\$}, \texttt{insert\$}, \texttt{delete\$}, \texttt{replace\$}, \texttt{subseq\$}, \texttt{member\$}

\subsection{Altre (10+ funzioni)}

\texttt{bind}, \texttt{progn}, \texttt{if}, \texttt{while}, \texttt{foreach}, \texttt{break}, \texttt{return}, \texttt{switch}, \texttt{gensym}, \texttt{eval}

\textit{Per dettagli completi, consultare la documentazione online.}


% Appendice C: Esempi Completi

\chapter{Esempi Completi e Casi di Studio}
\label{app:esempi}

\section{Esempio 1: Sistema di Raccomandazioni}

\begin{lstlisting}[language=CLIPS]
;; Template
(deftemplate utente
  (slot id (type INTEGER))
  (slot nome (type STRING))
  (multislot interessi))

(deftemplate prodotto
  (slot id (type INTEGER))
  (slot nome (type STRING))
  (slot categoria)
  (slot prezzo (type FLOAT)))

(deftemplate raccomandazione
  (slot utente-id)
  (slot prodotto-id)
  (slot score (type FLOAT)))

;; Regole
(defrule raccomanda-per-interesse
  (utente (id ?uid) (interessi $? ?cat $?))
  (prodotto (id ?pid) (categoria ?cat) (prezzo ?p&:(< ?p 100)))
  =>
  (assert (raccomandazione 
    (utente-id ?uid)
    (prodotto-id ?pid)
    (score 0.8))))

;; Uso
(assert (utente (id 1) (nome "Mario") (interessi sport tecnologia)))
(assert (prodotto (id 101) (nome "Laptop") (categoria tecnologia) (prezzo 899.00)))
(run)
\end{lstlisting}

\section{Esempio 2: Sistema di Workflow}

\begin{lstlisting}[language=CLIPS]
(deftemplate richiesta
  (slot id (type INTEGER))
  (slot tipo)
  (slot importo (type FLOAT))
  (slot stato (allowed-values pending approved rejected)))

(defrule approva-automatica
  (declare (salience 10))
  ?r <- (richiesta (id ?id) (importo ?i&:(< ?i 1000)) (stato pending))
  =>
  (modify ?r (stato approved))
  (printout t "Richiesta " ?id " approvata automaticamente" crlf))

(defrule richiedi-manager
  ?r <- (richiesta (importo ?i&:(>= ?i 1000)) (stato pending))
  =>
  (modify ?r (stato requires-approval))
  (assert (notifica (destinatario manager) (richiesta-id ?r))))
\end{lstlisting}

\section{Esempio 3: Sistema Diagnostico}

\begin{lstlisting}[language=CLIPS]
(deftemplate paziente
  (slot id)
  (slot nome)
  (slot età (type INTEGER)))

(deftemplate sintomo
  (slot paziente-id)
  (slot tipo)
  (slot gravità (allowed-values lieve moderata grave)))

(deftemplate diagnosi
  (slot paziente-id)
  (slot malattia)
  (slot confidenza (type FLOAT)))

(defrule influenza
  (paziente (id ?pid))
  (sintomo (paziente-id ?pid) (tipo febbre) (gravità moderata|grave))
  (sintomo (paziente-id ?pid) (tipo tosse))
  (not (diagnosi (paziente-id ?pid)))
  =>
  (assert (diagnosi (paziente-id ?pid) (malattia influenza) (confidenza 0.7))))

(defrule covid-sospetto
  (declare (salience 20))  ; Priorità alta
  (paziente (id ?pid))
  (sintomo (paziente-id ?pid) (tipo febbre) (gravità grave))
  (sintomo (paziente-id ?pid) (tipo tosse))
  (sintomo (paziente-id ?pid) (tipo difficoltà-respiratoria))
  =>
  (assert (diagnosi (paziente-id ?pid) (malattia covid-19) (confidenza 0.85)))
  (assert (azione-urgente (paziente-id ?pid) (tipo test-pcr))))
\end{lstlisting}

\section{Esempio 4: Regole con Moduli}

\begin{lstlisting}[language=CLIPS]
(defmodule ACQUISITION
  "Acquisizione dati sensori"
  (export deftemplate sensor-reading))

(deftemplate sensor-reading
  (slot sensor-id)
  (slot timestamp)
  (slot value (type FLOAT)))

(defmodule PROCESSING
  "Elaborazione dati"
  (import ACQUISITION deftemplate sensor-reading)
  (export deftemplate alert))

(deftemplate alert
  (slot sensor-id)
  (slot level (allowed-values warning critical))
  (slot message))

(defrule PROCESSING::rileva-anomalia
  (sensor-reading (sensor-id ?id) (value ?v&:(> ?v 100)))
  =>
  (assert (alert (sensor-id ?id) (level critical) 
                 (message "Valore critico rilevato")))
  (focus NOTIFICATION))

(defmodule NOTIFICATION
  "Gestione notifiche"
  (import PROCESSING deftemplate alert))

(defrule NOTIFICATION::invia-alert
  (alert (sensor-id ?id) (level critical) (message ?msg))
  =>
  (printout t "ALERT: Sensore " ?id " - " ?msg crlf))
\end{lstlisting}

\section{Esempio 5: Pattern Matching Avanzato}

\begin{lstlisting}[language=CLIPS]
;; Multifield variables
(defrule analizza-sequenza
  (sequenza $?inizio ?target $?fine)
  (test (> (length$ $?inizio) 2))
  (test (> (length$ $?fine) 1))
  =>
  (printout t "Trovato " ?target " in posizione " 
            (+ (length$ $?inizio) 1) crlf))

;; Pattern composti con AND/OR
(defrule contratto-speciale
  (or (cliente (tipo premium))
      (and (cliente (tipo standard))
           (ordini-totali ?n&:(> ?n 10))))
  =>
  (assert (sconto-disponibile)))

;; Constraint complessi
(defrule valida-range
  (misura (valore ?v&:(>= ?v 0)&:(<= ?v 100)&:(numberp ?v)))
  =>
  (assert (misura-valida)))
\end{lstlisting}

\textit{Per esempi completi e casi di studio più approfonditi, consultare il repository GitHub del progetto SLIPS.}


% Appendice D: Benchmark e Performance

\chapter{Benchmark di Performance}
\label{app:benchmark}

\section{Metodologia}

Benchmark eseguiti su:
\begin{itemize}
\item Hardware: Apple M1 Pro, 16GB RAM
\item OS: macOS 15 Sequoia
\item Swift: 6.2
\item Build: Release (\texttt{-O})
\end{itemize}

\section{Risultati}

\begin{table}[h]
\centering
\begin{tabular}{@{}lrr@{}}
\toprule
\textbf{Operazione} & \textbf{Tempo (ms)} & \textbf{Note} \\
\midrule
Assert 1000 fatti & 15 & Regola semplice \\
Assert 10k fatti & 180 & Linear scaling \\
Join 2 pattern (10k) & 45 & Hash join \\
Retract 1000 fatti & 8 & Beta cleanup \\
Build 100 regole & 5 & Una tantum \\
\bottomrule
\end{tabular}
\caption{Benchmark SLIPS 1.0}
\end{table}

\textit{Vedere documentazione online per benchmark completi.}
\end{lstlisting}



% Bibliografia e indice
\backmatter
\printbibliography[heading=bibintoc]
\printindex

\end{document}

